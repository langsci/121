%%%%%%%%%%%%%%%%%%%%%%%%%%%%%%%%%%%%%%%%%%%%%%%%%%%%
%%%                                              %%%
%%%                 Metadata                     %%%
%%%          fill in as appropriate              %%%
%%%                                              %%%
%%%%%%%%%%%%%%%%%%%%%%%%%%%%%%%%%%%%%%%%%%%%%%%%%%%%

\title{African Language Diversities} 
\subtitle{Selected Papers from the 46th Annual Conference on African Linguistics}
\BackTitle{African Language Diversities} % Change if BackTitle != Title
\BackBody{\textit{African Language Diversities} contains a selection of revised and peer reviewed papers presented at the 46th Annual Conference on African Linguistics, held at the University of Oregon. Most chapters focus on single languages, addressing aspects of their phonology, morphology, semantics, syntax, information structure, or historical development. These chapters represent nine different genera: Mande, Gur, Kwa, Edoid, Bantu, Nilotic, Gumuzic, Cushitic, and Omotic. Other chapters investigate a mix of languages and families, moving from typological issues to sociolinguistic and inter-ethnic factors that affect language and accent switching. Some chapters are primarily descriptive, while others push forward the theoretical understanding of tone, particular semantic domains, discourse related structures, and other linguistic systems. The papers on Bantu languages reflect something of the internal richness and continued fascination of the family for linguists, as well as maturation of research on the family. The distribution of papers on other families highlights the need for intensified investigation of all language families of Africa, including basic documentation. In this regard, the chapter on Daats’íin (Gumuzic) stands out as the first-ever published article on this hitherto unknown and endangered language, found in the Ethiopian-Sudanese border lands.}
%\dedication{Change dedication in localmetadata.tex}
%\typesetter{Change typesetter in localmetadata.tex}
%\proofreader{Change proofreaders in localmetadata.tex}
\author{Doris L. Payne\and Sara Pacchiarotti\lastand  Mokaya Bosire} %use this field for the volume editors

\renewcommand{\lsISBNdigital}{978-0-000000-00-0}    
\renewcommand{\lsISBNhardcover}{000-0-000000-00-0}
\renewcommand{\lsISBNsoftcover}{000-0-000000-00-0}
\renewcommand{\lsISBNsoftcoverus}{000-0-000000-00-0}

\renewcommand{\lsSeries}{calseries}
\renewcommand{\lsSeriesNumber}{2} 
\renewcommand{\lsURL}{http://langsci-press.org/catalog/book/00} % contact the coordinator for the right number