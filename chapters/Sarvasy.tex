\documentclass[output=paper]{langsci/langscibook} 
\title{The future in Logoori oral texts} 
\author{%
Hannah Sarvasy \affiliation{Australian National University} 
}
% \chapterDOI{} %will be filled in at production


\abstract{
Bantu languages are renowned for their complex tense and aspect systems. Tense reference in Bantu languages is also known to have variable application, often depending on information structure or the scale of the time frames involved. With apparently four positive-polarity future tense inflections, the Luyia Bantu language Logoori (JE41) nearly tops the future tense distinction charts for Bantu and other languages. While the existence of these future-related forms is a given in the literature, the semantics and applications of Logoori future tense inflections are as yet undescribed. Logoori speakers also employ several other forms and constructions for denoting future time. This paper examines future time reference in a corpus of Logoori texts.
}

\maketitle
\begin{document}

\section{Introduction} \label{sec:sarvasy:1}

Bantu languages are renowned for their complex tense and aspect systems \citep{Nurse2003,Nurse2008}. Tense reference in Bantu languages is also known to have variable application (\citealt{Besha1989}; \citealt[101]{Nurse2003}; \citealt{Crane2011}), often depending on information structure or the scale of the time frames involved \citep{BotneKershner2008,Botne2013}. With apparently four positive-polarity future tense inflections \citep{Mould1981,Leung1991,Nurse2003}, the Luyia Bantu language Logoori (JE41) nearly tops the future tense distinction charts for Bantu and other languages \citep[89]{Nurse2008}. The existence of these inflections may be established through elicitation, but the semantics and applications of Logoori future tense inflections are as yet undescribed. With so many future tense forms, one might think that Logoori speakers would not need other auxiliary constructions for future time reference. But Logoori speakers do have several other options for denoting future time, beyond their four dedicated future tense forms. This paper examines future time reference in a corpus of Logoori texts.

\citet{BotneKershner2008} and \citet{Botne2013} present examples of tenses in Bantu languages that do not fit the timeline-extending-from-deictic-center model in \citet{Comrie1985}. As \citet{Botne2013} points out, a tense system may include forms that fuse epistemic reality and time reference. This is likely the case for Logoori, as well. Three of the Logoori future tenses seem to be staggered along a two-dimensional timeline, overlapping for most of their ranges but differentiated in the closest time frames. Another one or two “tenses” may differ from these three in epistemic certainty. 

Natural speech is the best context to gauge discourse functions of tense inflections, in consultation with a Logoori native speaker. The data used for this paper is a small corpus including 20 Logoori texts of varying lengths from three sources: a) original narratives recorded by UCLA Field Methods consultant Mwabeni Indire; b) original narratives and religious songs recorded by various female and male Logoori speakers in Kenya on behalf of Michael Diercks; and c) interview clips in Logoori from Sandra Nichols and Joseph Ssennyonga’s \citeyear{NicholsSsennyonga1976} film \textit{Maragoli}. All texts were transcribed in consultation with Mr. Indire using the software ELAN \citep{SloetjesWittenberg2008}. Throughout, texts are cited by speaker’s name and line number. Occasional examples volunteered by Mr. Indire lack this citation.

Logoori is far from monolithic, and at least two major dialect groups are represented in the corpus, judging by binary distinctions in forms. Apparent differences range from phonology $-$ one dialect’s palatal glide corresponds to an interdental consonant in the other $-$ to morphology. Based on Mr. Indire’s description and reports from Michael Diercks and colleagues, the dialect situation is complex; nowadays, at least, dialect differences do not seem to correspond to geography or village locations. Further, while forms generally differ in maximally two ways, individual speakers may employ different combinations of the binary distinctions. Examples here are drawn from all Logoori dialects present in the corpus. Where a dialect difference may contribute to a difference in morphology, this is noted.

\citet[12]{Botne2013} warns that any discussion of Bantu tense must cover temporal “domain, time region, and time scale” for each form. This is beyond the scope of this paper. Instead, this paper aims at exposing the ways in which various future tense inflections, and other constructions with future reference, are used in the texts corpus.

\section{What has been said about Logoori future tenses} \label{sec:sarvasy:2}

Earlier formal descriptions of Logoori future tenses are in \citet[206]{Mould1981}, \citet[174-189, 204-207, 273-284, 285-322]{Leung1991}, and \citet[100]{Nurse2003}.\footnote{Logoori is assumed here, following \citet{Leung1991}, to have seven phonological vowel distinctions. These are written here as: /i ɪ e a o ʊ u/. Leung’s /i/ corresponds to /i/ here, while Leung’s /I/ corresponds to /ɪ/; Leung’s /u/ corresponds to /u/ here, and her /U/ corresponds to /ʊ/.} These accord in the number of future tenses listed: four, including Near, Middle, Far/Remote, and ``Indefinite'' %
%Please use double quotes here only if you are distancing yourself from and rejecting their description.
%
%
%
%
(Mould, Leung) or “Uncertain” (Nurse). All three descriptions emphasize form over meaning or function. \citet{Mould1981} and \citet{Nurse2003} indicate meaning only indirectly through their chosen labels. \citet{Leung1991} briefly describes use of only two of the four forms: her Near Future (\citeyear[174]{Leung1991}) and Middle Future (\citeyear[285]{Leung1991}). 

None of the sources explain the label Indefinite or Uncertain. In general (for instance, in \citealt[20]{Johnson1977}), indefinite %
%Here, should this be in double quotes (if you are distancing yourself from a term they use), or in single quotes as a gloss?
%
%
%
%
future tenses are those that can apply to many discrete segments of the presumed timeline, while uncertain implies epistemic evaluation of the likelihood of the event. These would seem to be two different concepts. As described in \sectref{sec:sarvasy:4}, the relevant form designated by these terms is the rarest in the present corpus, so this paper will not resolve its semantics.

The forms and terminology given by previous sources for the various Logoori future tenses are in \tabref{tab:sarvasy:1}, along with the terminology used here and an additional form in the last column that is introduced in this paper.\footnote{In \tabref{tab:sarvasy:1}, SP stands for the subject-indexing prefix; the underlining marks the location of the verb stem in the template. Mould and Leung differ in tonal analyses. Since a full tonal analysis of Logoori verbal inflections has not yet been completed, marking of tone is omitted here except when critical to tense distinctions.}  

\begin{table}
\begin{tabular}{lllllll} &  SP-{ra-}\_\_{-a} &  {na-}SP-\_\_{-e} &  SP-{raka-}\_\_-{e} &  SP-{rika}-\_\_-{e} &  SP-{ri}-\_\_{-a} &  {naa-}SP-\_\_{-e} \\
\lsptoprule
 Mould &  Near &  Middle &  Far Future &  — &  Indefinite &  — \\ 
 1981 &  Future &  Future & & &  Future & \\ \midrule
 Leung &  Near &  Middle &  Far Future &  Far Future &  Indefinite & \mdseries  — \\ 
 1991 &  Future; &  Future; &  (variant) &  (variant) &  Future & \\
&  hodiernal &  tomorrow & & & &  \\
&  reference &  through & & & & \\
& &  several & & & & \\
& &  years & & & & \\
& &  hence & & & & \\
\midrule 
 Nurse &  Near &  Middle &  — &  Far Future &  Uncertain & \mdseries  — \\ 
 2003 &  Future &  Future & & &  Future & \\ \midrule
 This &  Hodiernal &  Post- &  Post- &  — &  Uncertain &  Post- \\
 paper &  + crastinal; &  crastinal &  crastinal, & &  Future? &  crastinal, \\
&  also & &  more & & &  more \\
&  unspecified & &  distant & & &  distant \\
&  future & &  than & & &  than \\
& & &  Middle & & &  Middle \\
& & &  Future & & &  Future \\
\lspbottomrule
\end{tabular}
\caption{Logoori future tense forms}
\label{tab:sarvasy:1}
\end{table}

These forms are discussed in \sectref{sec:sarvasy:3}-\sectref{sec:sarvasy:6}, and additional constructions with future time reference present in the corpus are explored in \sectref{sec:sarvasy:7}.

\section{The Near Future that moonlights as General Future} \label{sec:sarvasy:3}

According to Mr. Indire, the form /SP-ra-\_\_-a/ may be understood by speakers to describe actions and events limited to a time between deictic center ‘now’ and the end of the following day, i.e. ‘later today’ through the end of ‘tomorrow’ (crastinal). This is counter to the description by \citet[174]{Leung1991}, who describes the Near Future as used only for later the same day, not also ‘tomorrow’. In its ‘today’-‘tomorrow’ function, the Near Future coincides with another inflectional form and with at least two auxiliary constructions; see \sectref{sec:sarvasy:7}.

Unlike these other hodiernal/crastinal forms, however, Logoori texts reveal that the Near Future has an extended function as the most general of the future tenses, being applicable to any future time, near or distant.\footnote{It may always be speculated for a tonal language with unfinished tonal analysis that there could be a grammatical-tone formal distinction between the hodiernal/crastinal and general time functions of the /ra-/ form. This was not perceived by Mr. Indire, however; for him, the form is the same, morphologically and tonally.} One parallel case is found in the Bantu language Kikuyu \citep[19]{Johnson1977}, where the Near Future doubles as Indefinite Future, applicable to any unspecified future time. Farther afield, the Papuan language Nungon shows a similar extended use of the Near Future form for general time \citep{Sarvasy2014}. The Nungon Near Future is either strictly hodiernal, used only for situations that will occur between the present moment and the end of ‘today’, or a generic future, used for situations that could take place at any future time, hodiernal or otherwise. 

The Logoori form /SP-ra-\_\_-a/ is shown with its restricted semantics, referring to only ‘later today’ through ‘tomorrow’, in \sectref{sec:sarvasy:3.1}. Its extended function for general time is discussed in \sectref{sec:sarvasy:3.2}.

\subsection{Strict Near Future /SP-ra-\_\_-a/}
\label{sec:sarvasy:3.1}

\citeauthor{Leung1991} writes that the Near Future /SP-ra-\_\_-a/ is “used when speaking of events or actions that are to take place later during the day” (\citeyear[174]{Leung1991}). As noted above, Mr. Indire considers this form to be primarily applicable to events that will take place later today or tomorrow. Regardless of whether the Near Future is strictly hodiernal, or hodiernal and crastinal, this application of /SP-ra-\_\_-a/ may be understood as the semantically restricted one, akin to the Kikuyu or Nungon Near Future tenses. An example of the restricted function of the Logoori Near Future is in \REF{ex:sarvasy:1}. 

\ea\label{ex:sarvasy:1}
\gll Mu-gamba,   \textbf{a-ra-zj-a}   mʊ  ɪ-skuru. \\
3-tomorrow  1-\textsc{nf}-go-\textsc{fv}  \textsc{loc}  9-school \\
\glt ‘Tomorrow, \textbf{s/he will go} to school.’ [Note that in other dialects, \textit{mu-gamba} refers to ‘morning’.] \z

Other forms are also used for hodiernal/crastinal reference; these include a form that appears to be a linking prefix plus the Subjunctive form, and auxiliary verb constructions. These are discussed in \sectref{sec:sarvasy:7}.

\subsection{General Future /SP-ra-\_\_-a/}
\label{sec:sarvasy:3.2}

Throughout the corpus, it is the Near Future form /SP-ra-\_\_-a/ that occurs with application to maximally ‘general’ time. While the Middle and Remote Future forms may be applied to time periods extending far into the future, they have not yet been found to be applicable to time periods closer to deictic center. The Near Future in its general application can apply to any future time, regardless of distance from deictic center. 

The first two examples of the Near Future form functioning as a general, not hodiernal/crastinal, future come from a funerary song from the Diercks corpus, sung by Ms. Linette Mbone. The song describes what people will encounter on arriving in heaven. The song is largely framed in the Near Future, and includes numerous verses similar to the following:

\ea\label{ex:sarvasy:2}
\gll V-oosi   \textbf{va-ra-ɲor-a}     ɪ-taʤi. \\
2-all 2-\textsc{nf}-find-\textsc{fv}    9-crown \\
\glt ‘All \textbf{will find} a crown (in heaven).’ (Funeral song, ln 5)
\z

\ea\label{ex:sarvasy:3}
\gll Mi-handa   \textbf{ʤi-ra-v-a}   mi-rahɪ. \\
4-road 4-\textsc{nf}-be-\textsc{fv} 4-good \\
\glt ‘The roads \textbf{will be} good (in heaven).’ (Funeral song, ln 7)
\z

Here, clearly, the Near Future form does not restrict time reference to the hodiernal/crastinal period. Rather, Mr. Indire described these instances as set in ‘general future’ time, equally applicable to many years hence, or much sooner. A person could die later today, or more than fifty years after the speech act.

The next examples also show the Near Future form used outside the restricted hodiernal/crastinal context. In another text from the Diercks corpus, Mr. Benjamin Egadwa explains how to care for a cow. Much of the text is framed in the Subjunctive, with most verbs lacking a tense prefix and bearing the Subjunctive suffix \textit{-e.} When Mr. Egadwa does use tense-marked verbs, however, these are most often Near Future forms, as in the following examples: 

\ea\label{ex:sarvasy:4}
\gll Zi-seendi   zi-ra,     \textbf{zi-ra-ku-koɲ-a}   kʊ   vi-indo   vivj-o       vi-ra. \\
10-money  10-\textsc{dem}  10-\textsc{nf-2sg}-help-\textsc{fv}  \textsc{loc}  8-thing  8:\textsc{red-2sg.poss}  8-\textsc{dem} \\
\glt ‘That money, \textbf{it will help you} with those things of yours.’ (B. E. cow care, ln 69)
\z

\ea\label{ex:sarvasy:5}
\gll Kaande,   \textbf{ɪ-ra-ku-h-a}     za   ma-veere. \\
again  9-\textsc{nf-2sg}-give-\textsc{fv} \textsc{link}  6-milk \\
\glt ‘Again, \textbf{it will give you} milk.’ (B. E. cow care, ln 45)
\z

\ea\label{ex:sarvasy:6}
\gll ɪ-v-e   ɪn-dahɪ,   kɪgera   \textbf{ʊ-ra-gur-iz-a}       kʊ. \\
9-be-\textsc{fv}  9-good  because  \textsc{2sg-nf}-sell-\textsc{caus-fv}  \textsc{loc} \\
\glt ‘It is good, because \textbf{you will sell} of it.’ (B. E. cow care, ln 80)
\z 

\ea\label{ex:sarvasy:7}
\gll ʊ-ta-ret-a     kʊ   ɪ-ŋombe   ɪ-v-e     ɪ-mbarava   ha-aŋgo   daave, \\ 
\textsc{2sg-neg}-bring-\textsc{fv}  \textsc{loc}  9-cow    9-be-\textsc{fv}  9-fierce  16-home  not \\ 
\glt ‘Don’t bring home a cow that is fierce, 

\gll \textbf{ɪ-ra-ku-ret-er-a}     ɪ-ʃida. \\
9-\textsc{nf-2sg-}bring-\textsc{applic-fv} 9-trouble \\
\glt \textbf{it will bring} trouble for you.’ (B. E. cow care, ln 130)
\z 

The Near Future used for general time may occur negated, as:

\ea\label{ex:sarvasy:8}
\gll ɪ-ŋombe   si  \textbf{ɪ-ra-kw-em-a}   ma-veere   daave. \\
9-cow  \textsc{neg}  \textsc{9-nf-2sg}-deny-\textsc{fv}  6-milk    not \\
\glt ‘The cow \textbf{will not deny you} milk.’ (B. E. cow care, ln 32)
\z

In examples \REF{ex:sarvasy:2}-\REF{ex:sarvasy:8}, the Near Future form is employed to describe future times that are not fixed, along the lines of the Kikuyu Near/Indefinite Future \citep[20]{Johnson1977}. The Logoori Near Future thus seems to be a good candidate for the secondary label Indefinite Future! In fact, in his Bantu-wide discussion of -\textit{ka-} verbal prefixes, \citet[492]{Botne1999} writes: “Llogoori, itself, has an indefinite future formative -\textit{ra}-.” 

So what of the Logoori /ri-/ future, which the Mould/Leung label implies is the dedicated Indefinite Future form? The next section addresses this question.

\section[The rare /ri-/ future]{The rare /ri-/ future}
\label{sec:sarvasy:4}

The /ri-/ future form $-$ Leung and Mould’s Indefinite; Nurse’s Uncertain $-$ is too rare in the corpus for any revelations here. It is noteworthy that in the same discussion in which he calls the Logoori /ra-/ form a dedicated indefinite future, \citet[491]{Botne1999} also mentions a Logoori “near future formative” -\textit{ri}-. It is unclear from his paper whether Botne means to reverse Leung’s and Mould’s labels. In any case, there is no evidence in the corpus or from Mr. Indire’s intuitions for the /ri-/ future form being more applicable to hodiernal/crastinal times than the /ra-/ future form.

The /ri-/ future co-occurs with the Near Future form in the funerary song sung by Ms. Mbone, source of examples \REF{ex:sarvasy:2} and \REF{ex:sarvasy:3}. There are only two instances of it, however, compared with 24 of the Near Future. Both instances of the /ri-/ form are negated, and the two instances occur in adjacent clauses, shown in \REF{ex:sarvasy:9} and \REF{ex:sarvasy:10}:

\ea\label{ex:sarvasy:9}
\gll Va-ra-ger-iz-a     kw-iŋgir-a   jo,\\ 
2-\textsc{nf}-try\textsc{-caus-fv}  15-enter-\textsc{fv}  9.there\\ 
\glt ‘They will try to enter there,

\gll navuzwa   si   \textbf{va-ri-ɲar-a}     mba. \\
but    \textsc{neg}  2-\textsc{if-}be.able-\textsc{fv}  \textsc{neg} \\
\glt but \textbf{they will not be able to}.’ (Funeral song, lns 11-12)
\z

\ea\label{ex:sarvasy:10}
\gll Si   \textbf{va-ri-ɲor-a}   kʊ   vi-hanwa. \\
\textsc{neg}  2-\textsc{if-}find-\textsc{fv}  \textsc{loc}  8-gift \\
\glt ‘\textbf{They will not find} any presents.’ (Funeral song, ln 13)
\z

Both of these negated instances of the /ri-/ future refer to a time frame pre-established by the Near Future. Mr. Indire further indicated that if \textit{si va-ri-ɲor-a} lacked the negator \textit{si}, it would sound better to use the Near Future prefix /ra-/, not /ri-/.\footnote{When asked, Mr. Indire also expressed doubt that the /ri-/ and /ra-/ future forms differ in degree of certainty.} It is crosslinguistically common for less-certain or irrealis forms to be used as negated counterparts to certain or realis forms. 

Another of the few instances of the /ri-/ future is found in a song embedded in a legend told by Ms. Hellen Makungu. Here, the form occurs with positive polarity, seen in \REF{ex:sarvasy:11}. A woman is quoted as saying that if anyone breaks one of her pots, she will take the culprit to an ogre.

\ea\label{ex:sarvasy:11}
\gll Na-a-vol-a,      mu-ndu  u-rj-ataɲ-a  ka-rara, \\
\textsc{consec-1}-say.\textsc{applic-fv}  1-person  1-5-break-\textsc{fv}  12-one \\
\glt ‘And she said, the person who breaks it just a little,

\gll \textbf{n-di-mu-ʃir-a}     wa   gʊ-nani. \\
\textsc{1sg-if-}1-take-\textsc{fv}  \textsc{1.gen}  20-ogre \\
\glt \textbf{I will take her} to the ogre.’ (Funeral song, lns 11-13)
\z

As will be seen in \sectref{sec:sarvasy:6}, the /ri-/ future form here occurs in a similar context to the /raka-/ future form; it is unclear why /ri-/ is used here and not /raka-/. Although the function of /ri-/ is unclear from these paltry examples, its very scarcity may be of note. Perhaps its use does relate more to epistemic judgment than the more common future forms.

\section{The Middle Future}

The Near Future form is much more common in the corpus than the other future forms, such as the Middle Future /na-SP-\_\_-e/. While \citet[285]{Leung1991} writes that the Middle Future applies to “sometime between tomorrow and several years later,” Mr. Indire describes the Middle Future as simply referring to any time after tomorrow. Accepting either of these, the Middle Future tense would seem to entirely encompass the reach of any Remote Future tense on a timeline. That is, either the Remote Future or Middle Future could apparently be used to refer to distant times, probably depending on discourse context, but only the Middle Future could refer to nearer times. A Remote Future (as in \sectref{sec:sarvasy:6} below) then would have a reduced range within the range of the Middle Future on the timeline. %
%I don’t understand what this means. 
%
The Middle Future co-occurs with the Near Future in reported speech in at least one text. It is conceivable that the Middle Future is used there for discourse-related purposes such as increased politeness. 

The basic time-referencing function of the Middle Future is illustrated in \REF{ex:sarvasy:12}. When Michael Diercks had not yet arrived in the Logoori area, a group of singers (led by Ms. Carolyn Chesi) recorded a praise song for him, in his absence. In the song, the singers promise to bestow flowers upon Diercks on his arrival. This bestowal is framed in the Middle Future tense:

\ea\label{ex:sarvasy:12}
\gll \textbf{Na-kʊ-sjaj-e,}    \textbf{na-kʊ-sjaj-e}    Majki  w-etu    ma-uwa. \\
\textsc{mf-1pl}-award-\textsc{fv}  \textsc{mf-1pl-}award-\textsc{fv}  Mike  1-\textsc{1pl.poss}  6-flower \\
\glt ‘\textbf{We will award, we will award} our Mike flowers.’ (Praise song, ln 1)
\z

Use of the Middle Future here may be justified according to Mr. Indire’s understanding of the Middle Future tense: the singers knew that Diercks would not arrive later that day, nor the next. The Middle Future allows both for his arrival and the giving of the flowers any time beyond the day after the singing act. The Middle Future may also have been used because the singers expected to bestow flowers on Diercks within, perhaps, three years. 

The function of the Middle Future form is not always this easily explained, however. In another legend told by Ms. Grace Otieno, Near Future and Middle Future forms coexist within a short passage, shown in \REF{ex:sarvasy:13}. A woman has just come upon a group of people. She carries wings she was given by birds, and asks the group whether, if she gives them the wings, they will be able to finish ‘knitting’ them.\footnote{The notion of ‘knitting’ wings comes from the broad English translation of this text supplied by Michael Diercks’s Logoori consultant in Kenya. Mr. Indire was unable to confirm the glosses for these terms.}

\ea\label{ex:sarvasy:13}
\gll Va-a-ʤib-a   \textbf{va-ra-mar-a}. \\
2\textsc{-rp}-reply-\textsc{fv}  2-\textsc{nf-}finish-\textsc{fv} \\
\glt ‘They replied (that) \textbf{they will finish}.

\gll Ja-a-va-h-a     nɪ-v-i-ituŋg-a     nɪ-va-a-mar-a. \\
1\textsc{-rp-2}-give-\textsc{fv}    \textsc{consec-2-rp}-?knit-\textsc{fv}  \textsc{consec-2-rp-}finish-\textsc{fv} \\
\glt She gave (the wings) to them and they knitted (them) and finished.

\gll Ja-a-va-vor-er-a     mw-ituŋg-aŋg-i   a-ma-vaha   ga-aŋge \\
1-\textsc{rp-}2-say-\textsc{applic-fv}    \textsc{2pl-}knit-\textsc{progr-perf}  \textsc{aug-}6-wing  6-\textsc{1sg.poss} \\
\glt She told them, (since) you have knitted my wings, 

\gll \textbf{na-mu-m-b-e}       kɪ?  \\
\textsc{mf-2pl-1sg-}give-\textsc{fv}    what \\
\glt what \textbf{will you give me}?

\gll Va-a-vʊgʊr-a    ɪ-mburi  nɪ-va-a-mu-h-a    n-a-a-zj-a. \\
2\textsc{-rp}-take-\textsc{fv}    9-goat    \textsc{consec-2-rp}-1-give-\textsc{fv}  \textsc{consec-1-rp-}go-\textsc{fv} \\
\glt They took a goat and gave (it) to her and she went.’ (G. O. woman and hen, lns 16-17)
\z

The time period in which the knitters complete the knitting of the wings is not specified in the story. The knitters affirm that they ‘will finish’ using the Near Future form. This could be compatible with either hodiernal/crastinal or general time reference: perhaps the wings will be finished on the same or following day, or perhaps later. Once they finish, however, the woman then asks what they will give her using the Middle Future form, which is usually restricted to time beyond ‘tomorrow’. 

The story seems to imply that the giving of the goat occurs quite soon after the woman’s query, although no time frames are explicitly described. If the woman actually expects an immediate gift, it is possible that she uses the Middle Future rather than the Near Future for politeness, suggesting that the giving need not take place by the morrow.\footnote{This conjecture has not been confirmed by Mr. Indire.}

\section[Remote Future forms]{Remote Future forms} \label{sec:sarvasy:6}
\citet{Mould1981}, \citet{Leung1991} and \citet{Nurse2003} accord in describing a Far Future %
%Why is this in double quotes?
%
form /SP-raka\_\_e/ or /SP-rika-\_\_-e/. While Mould only lists /SP-raka-\_\_-e/ and Nurse only /SP-rika-\_\_-e/, Leung’s consultant was familiar with both forms and found them synonymous and equally acceptable (\citeyear[204, fn 10]{Leung1991}). It is thus possible that these are dialectal variants. Neither of these forms was initially supplied by Mr. Indire in discussing the distant future, however. Mr. Indire preferred /naa-SP-\_\_-e/ for occurrences far in the future. This form is clearly similar to the Middle Future form /na-SP-\_\_-e/, but with a long vowel /aa/ in the tense prefix. 

\subsection{The /raka-/ Remote Future}\label{sec:sarvasy:6.1}

Along with \citet{Botne1999}, \citet[85]{Nurse2008} notes that a form /-ka-/ occurs in future tense formatives in a number of Bantu languages, and that it “sometimes forms a composite marker with other morphemes, and refers predominantly to far future\ldots in systems with multiple future reference.” Few instances of the form /SP-raka-\_\_-e/ are found in the present corpus, and no tokens at all occur of the alternative form /SP-rika-\_\_-e/. Mr. Indire indicated that /raka-/ has a further alternative form /aaka-/. Indeed, most speakers in the corpus use only the /aaka-/ form with SPs other than \textsc{1sg}, and /raka-/ only after \textsc{1sg}, as [nda(:)ka].

Most instances of the /raka-/ future form occur in a legend recorded by Ms. Hellen Makungu, source of example \REF{ex:sarvasy:11}. In the legend, a mother initially threatens her eight daughters with death-by-ogre if they should break any of the pots she made for fetching water. This is shown in \REF{ex:sarvasy:14}, where the /raka-/ form occurs in both the protasis and apodosis of a conditional sentence. 

\ea\label{ex:sarvasy:14}
\gll Mu-kana   mu-rara   neva   \textbf{a-raka-ataɲ-e}   kʊ     ɪsjoŋgo   ɪ-jɪ,  \\
1-girl    1-one    if  1-\textsc{rf}-break-\textsc{fv}  of  9pot   9-\textsc{dem} \\
\glt ‘One girl, if \textbf{she will break} a pot,

\gll \textbf{n-daka-mu-ʃir-e}   wa   gʊ-nani. \\
\textsc{1sg-rf-}1-take-\textsc{fv}  1.\textsc{gen}  20-ogre \\
\glt \textbf{I will take her} to the ogre.’ (H. M. woman and eight girls, lns 6-7)
\z

If \citet{Botne1999} is correct in positing that /raka-/ originally comprised the /ra-/ future prefix plus a distal /ka%
%Here you use two dashes, but in the next instance you list it with just a following dash. 
%
-/, the function of this distal /ka-/ in \REF{ex:sarvasy:14} may be epistemic rather than timeline-related. Epistemically certain events predicted to occur at unspecified future times are described using the /ra-/ future form, as in \REF{ex:sarvasy:2}-\REF{ex:sarvasy:8}. In contrast, both the breaking of the pot and the taking to the ogre here are hypothetical future events. 

One daughter does break a pot in the mother’s absence. The other girls urge her to wait for the mother and admit her guilt, but she refuses, saying:

\ea\label{ex:sarvasy:15}
\gll A-a!  Mama    a-m-bol-e      ndɪ   mu-ndo   \textbf{u-raka-ataɲ-e}   kʊ   ɪ-sjoŋgo  \\
no  mother  1-1\textsc{sg}-say\textsc{.applic-perf}  that  1-person  1\textsc{-rf}-break-\textsc{fv}  of  9-pot\\
\glt Uh-uh! Mother told me that a person \textbf{who will break} a pot,

\gll \textbf{a-raka-mu-ʃir-e}   wa   gʊ-nani  gʊ-ra-zj-a  kʊ-mu-rj-a.\\
1\textsc{-rf-}1-take-\textsc{fv}    1.\textsc{gen}  20-ogre  20-\textsc{nf}-go-\textsc{fv}  15-1-eat-\textsc{fv} \\
\glt \textbf{she will take her} to the ogre, who will go to eat her. (H. M. woman and eight girls, lns 11-12)
\z

Here, both the breaking of the pot and the bringing to the ogre are framed with the /raka-/ form, but these are followed by the Near Future /ra-/ form for the ogre’s subsequent predicted action. It is possible that the Near Future is used here because the ogre’s action is expected to follow immediately on receipt of the pot-breaker.

\subsection{Another Remote Future: /naa-SP-\_\_-e/}\label{sec:sarvasy:6.2}

Mr. Indire described yet another Logoori Remote Future form that is missing from extant descriptions. \citet[86, fn c]{Nurse2008} notes that in Bantu E15, E16, and E55, a future formative /naa-/ corresponds to the /raa-/ of closely-related Bantu languages. It may be that the Logoori Remote Future /naa-SP-\_\_-e/ described by Mr. Indire is another instance of this. Its relationship to the /raka-/ future of \sectref{sec:sarvasy:6.1} may be one of dialectal variance, since the /naa-/ formative seems to be the primary one used by Mr. Indire for most distant future time reference.

Mr. Indire indicated that the temporal domain of this tense would fall within the domain of the Middle Future, but begin farther from deictic center. This form is unattested in the corpus.

\ea\label{ex:sarvasy:16}
\gll Na-a-kor-e  ɪ-gaasi. \\
\textsc{mf}-1-do-\textsc{fv}  9-work \\
\glt ‘S/he will do work (after tomorrow).’
\z 

\ea\label{ex:sarvasy:17}
\gll \textbf{Naa-a-kor-e}  ɪ-gaasi. \\
\textsc{rf}-1-do-\textsc{fv}  9-work \\
\glt ‘\textbf{S/he will do} work (at a later date).’
\z

Alternatively to Nurse’s proposal, in which /naa-/ relates to a more-common Bantu form, /raa-/, this Remote Future form may be analyzed in two different ways: either the long vowel in /naa-/ represents iconic lengthening of the Middle Future tense prefix vowel to show temporal distance, or /naa-/ actually comprises the Middle Future tense prefix /na-/ followed by a prefix /a(a)-/ indicating remoteness in time. The Logoori Remote Past form, /SP-aa-\_\_-a/, also features a prefixed long /aa-/, although after, not before, the subject prefix (cf.\ \citealt[206]{Mould1981}; \citealt[323]{Leung1991}). In this respect, this Remote Future form mirrors the Remote Past form. If it is innovated (since none of the previous sources mention it), it could represent a kind of leveling on the part of language learners.

\section[Further forms used with general and near future times]{Further forms used with general and near future times}
\label{sec:sarvasy:7}

Logoori texts show a number of additional forms and constructions used with both unspecified/general and near future times. These go beyond the future tense forms shown in \tabref{tab:sarvasy:1}. These forms may not stand in for specific post-crastinal times. Indeed, \citet[85]{Nurse2008} writes that Bantu future-referencing “forms deriving from ‘come’ and the subjunctive tend to refer to near futures.” The Logoori Near Future tense form has potential substitutes for both its general future function (the Subjunctive) and its hodiernal/crastinal future function (/ma-SP-\_\_-e/ or an auxiliary construction). In contrast, the Middle Future and Remote Future tense forms have no auxiliary or other formal substitutes with similar time reference.

\subsection{Subjunctive for general time} \label{sec:sarvasy:7.1}

The Subjunctive form lacks a tense prefix but has final suffix \textit{-e} instead of unmarked \textit{-a}, Present Progressive \textit{-aa}, or Perfective \textit{-i.} As seen above, several of the future tense forms also employ a final suffix \textit{-e.} 

In the corpus, the Subjunctive form predominates in texts that describe general processes or give instructions/advice. The following example comes from the narrative instructions for bovine care from Mr. Egadwa. 

\ea\label{ex:sarvasy:18}
\gll \textbf{ʊ-gɪ-h-e}     a-ma-aze…   nɪ-ɪ-dook-a   saa   saba, \\  
\textsc{\textup{2sg}}\textsc{-}9-give-\textsc{subj}  \textsc{aug}-6-water  \textsc{foc}-9-arrive-\textsc{fv}  hour  seven \\
\glt ‘\textbf{You (will) give it water} … when (the time) arrives at one o’clock,

\gll \textbf{ʊ-gɪ-ker-e}   a-ma-veere\ldots \\
\textsc{2sg}-9-milk-\textsc{subj}  \textsc{aug}-6-milk \\
\glt \textbf{you milk it}…’ (B. E. cow care, lns 27-29)
\z

The Subjunctive form alone is not found in the corpus with specific future time reference; for reference to a specific time, it apparently must co-occur with auxiliary \textit{kʊman̪a} ‘to know,’ as described in \sectref{sec:sarvasy:7.3}. 

\subsection{Auxiliary construction with \textit{kʊzja} ‘go’} \label{sec:sarvasy:7.2}

This auxiliary construction used to express future time uses the cross-linguistically-common device of a verb of motion to indicate relatively-near future time \citep[161-163]{HeineKuteva2004}. Here, the verb \textit{kʊzja} ‘go’ is inflected for Present Progressive, and followed by the lexical verb in the Infinitive form. Example \REF{ex:sarvasy:1} showed the verb \textit{kʊzja} with full lexical meaning ‘go,’ while \REF{ex:sarvasy:19} shows it serving as an auxiliary indicating imminent future time. 

\ea\label{ex:sarvasy:19}
\gll Karoono,   \textbf{a-zez-a}      kʊ-nw-a  ɪ-kahava. \\
now/today  1-go.\textsc{pres.progr-fv}  15-drink-\textsc{fv}  9-coffee \\
\glt ‘Now, \textbf{s/he is going} to drink coffee.’
\z

The other auxiliary construction used frequently with future time reference in the corpus involves the verb \textit{kʊman̪a} ‘know,’ described in \sectref{sec:sarvasy:7.3}.

\subsection{Future semantics with \textit{k}\textit{ʊman̪a}\textit{} plus Subjunctive} \label{sec:sarvasy:7.3}

The verb \textit{kʊman̪a} ‘to know’ is employed as auxiliary in another construction with future time reference in Logoori. Here, the lexical verb following inflected ‘know’ occurs in the Subjunctive form /SP-\_\_-e/. This auxiliary construction may stand in for the Near Future tense form in either its specified hodiernal/crastinal or general time reference applications. Because the Subjunctive has modal, future-related meaning without \textit{kʊman̪a} (\sectref{sec:sarvasy:7.1}), future meaning here apparently comes from the Subjunctive, rather than from ‘know,’ which occurs unmarked for tense. The verb ‘to know’ also functions as an auxiliary with telic verbs in non-future contexts. It is unclear whether \textit{kʊman̪a} as auxiliary in future constructions may only occur with telic lexical verbs.

Example \REF{ex:sarvasy:20} shows the \textit{kʊman̪a} plus Subjunctive construction used with specified crastinal time reference:

\ea\label{ex:sarvasy:20}
\gll Mu-gamba,   \textbf{man̪-a}    \textbf{n-zj-e}    mʊ  ɪ-skuru. \\
3-tomorrow  \textsc{1sg.}know-\textsc{fv}  \textsc{1sg}-go-\textsc{subj}  \textsc{loc}  9-school \\
\glt ‘Tomorrow, \textbf{I will go} to school.’
\z

Note that the Subjunctive by itself has only general, unspecified future time reference in the corpus; \textit{kʊman̪a} here apparently aids in the specification of time, although \textit{kʊman̪a} constructions may also have general time reference. 

The remaining examples here show this construction with general, unspecified time reference. In Mr. Egadwa’s text on cow care, the /SP-ra-\_\_-a/ future tense form alternates with the \textit{kʊman̪a} auxiliary future construction. After issuing instructions framed in the Subjunctive about keeping timely milking schedules, the speaker says: 

\ea\label{ex:sarvasy:21}
\gll \textbf{ʊ-man̪-a}   \textbf{ʊ-n̪or-e}     mʊ   a-ma-veere   sja   w-eɲ-aŋg-a. \\
\textsc{2sg-}know-\textsc{fv}  \textsc{2sg}-find-\textsc{subj}  \textsc{loc}  \textsc{aug-}6-milk  how  \textsc{2sg}-desire-\textsc{progr}-\textsc{fv} \\
\glt ‘\textbf{You will find} in it milk as you desire.’ (B. E. cow care ln 31)
\z

In another procedural text, Ms. Carolyn Chesi describes how she makes the famous Kenyan cornmeal dish \textit{vuchima.} Ms. Chesi uses the \textit{kʊman̪a} future time construction frequently throughout the text. The following are only a few samples:

\ea\label{ex:sarvasy:22}
\gll \textbf{Man̪-a}     \textbf{n-zj-e}     m̩-mʊreme,   n-zj-e     kw-ah-a     i-ri-kove, \\
\textsc{1sg}.know-\textsc{fv}  \textsc{1sg}-go-\textsc{subj}  \textsc{loc}-farm  \textsc{1sg}-go-\textsc{subj}  15-pick    \textsc{aug}-5-veg.sp \\
\glt ‘\textbf{I will go} to the farm, I will pick rikove,’ (C. C. lunchtime ln 6)
\z

\ea\label{ex:sarvasy:23}
\gll Gw-aaka-ɲaar-a,   \textbf{ɪ-man̪-a}   \textbf{ɪ-dook-e}   ɪ-saa,   ʃ\'{i}mbe   saa   taano, \\
3-\textsc{futperf-}shrivel-\textsc{fv}  9-know-\textsc{fv}  9-arrive-\textsc{subj}    9hour    near  hour  five \\
\glt ‘Once (the greens) have shriveled, \textbf{it will arrive at}, close to eleven o’clock,

\gll saa   siita,   aa-m-bek-e       kʊ   ma-higa. \\
hour  six  \textsc{consec2-1sg}-put-\textsc{subj}  \textsc{loc}  6-stove \\
\glt twelve o'clock, then I will put (it) on the stove.’ (lns 13-16)
\z 

\ea\label{ex:sarvasy:24}
\gll N-daaka-rik-a,         \textbf{man̪-a}     \textbf{n-duk-e}. \\
\textsc{1sg-futperf}-put.water.on-\textsc{fv}  1\textsc{sg.}know-\textsc{fv}    \textsc{1sg}-make.cornmeal-\textsc{subj} \\
\glt ‘Once I've put water on, \textbf{I will make the cornmeal}.’ (lns 37-38)
\z

The verb \textit{kʊman̪a} ‘know’ only functions as an auxiliary if its subject and the subject of the following verb are coreferential. Otherwise, it must be interpreted with full lexical value, as in \REF{ex:sarvasy:25}.

\ea\label{ex:sarvasy:25}
\gll Zja   \textbf{ʊ-man̪-a}     nɪ-ɪ-vet-a     kʊ     mu-ndo, \\
so  2\textsc{sg-}know\textsc{-fv}  \textsc{consec}-9-\textsc{pass}-\textsc{fv}   \textsc{loc}  1-person \\
\glt ‘So \textbf{you know} when (the cow) passes by a person,

\gll ɪ-vor-e     kʊ-mu-duj-a   daave. \\
9-refrain-\textsc{subj}  15-1-hit-\textsc{fv}  not \\
\glt it will not hit him.’ (B. E. cow care, ln 133)
\z

The semantic difference between the \textit{kʊman̪a} constructions with general future time reference and the Subjunctive alone with such reference is as yet unclear. It is tempting to assume this difference is epistemic or relates to the strength of the assertion, but this is pure conjecture.

\citet{Botne2009} describes a counterpart auxiliary construction using ‘to know’ in Lwitaxo (Idakho). He identifies two functions of the construction: a “generic” function, “normally/ordinarily/typically do V” and a “culminative” function, “often suggesting a consequence: ‘end up Ving,’ ‘ultimately V,’ or ‘in the end’” (\citeyear[93]{Botne2009}). He writes that the culminative function “focuses on the event as the culmination of a series of events, while the generic does not refer to any specific event” (\citeyear[95]{Botne2009}).

The formal difference between the two functions identified by Botne is the presence of the ‘focus’ marker /n(i)/:\footnote{This form has been described as marking ‘focus’ \citep{Dalgish1979,Nurse2006,Botne2009}. As Botne writes (2009, fn 5): “There are five different functions associated with the form \textit{ni-}: copula (COP), participial (PRT), sequentive (SEQ) (often equivalent to ‘and’ or ‘then’), temporal ‘when’, and focus (FOC).” With \textit{nɪ}-marked verbs that are not preceded by an auxiliary, Botne’s sequentive use prevails in Logoori narratives. Some speakers use \textit{nɪ-/n-} more than others. (Over-reliance on \textit{nɪ-/n-} seems to be a characteristic of less-proficient speech, with more-proficient speakers varying their narrative devices.)  

\gll Ma,   nɪ-kʊ-tʊr-a     jo,  ma   va-ande     nɪ-va-zj-a…  \\
then  \textsc{consec}-\textsc{1pl-}leave-\textsc{fv}  9.there  then  2-other    \textsc{consec}-2-go-\textsc{fv}\\

‘Then, we left there, then the others went…’ (M. I. today tomorrow lns 21-22)}

\begin{tabular}{ll}
Generic:  & SP\textsubscript{1}-man̪-a   SP\textsubscript{1}-\_\_-a \\
Culminative:  & SP\textsubscript{1}-man̪-a  n(i)-SP\textsubscript{1}-\_\_-a
\end{tabular}

Both of the equivalent constructions $-$ and at least one other related one $-$ are present in the Logoori texts, but Botne’s description of their functions may not fit the Logoori data.

At the end of one legend, Ms. Grace Otieno relates of the two protagonists:

\ea\label{ex:sarvasy:26}
\gll Mu-jaajɪ   na   mu-kaana   \textbf{va-a-man̪-a}   \textbf{va-a-romb-a}     kɪ-serero. \\
1-boy    \textsc{conj}  1-girl    2-\textsc{rp}-know-\textsc{fv}  2-\textsc{rp}-make-\textsc{fv}  7-wedding \\
\glt ‘The boy and the girl \textbf{made a wedding}.’ (G. O. girls who wanted luck, ln 78)
\z

If this were equivalent to Botne’s description of the Idakho forms, we would expect generic meaning, with no specific event indicated. That is clearly not the case. The following lines from the same Logoori story also run counter to the Idakho analysis: 

\ea\label{ex:sarvasy:27}
\gll Ja-a-n̪or-a   ɪ-mbwa   ha-aŋgo. \\
1-\textsc{rp}-find-\textsc{fv}  9-dog    16-home \\
\glt ‘He found the dog at home.

\gll \textbf{Ja-a-man̪-a}   \textbf{n-a-a-vʊgʊr-a}     ɪ-mbwa   jɪjo, \\
1-\textsc{rp}-know-\textsc{fv}  \textsc{consec}-1-\textsc{rp}-take-\textsc{fv}  9-dog    9.\textsc{dem} \\
\glt \textbf{He took} that dog,

\gll n-a-a-taaŋg-a       kʊ-m-ben̪-a     na-jo. \\
\textsc{consec}-1-\textsc{rp}-begin-\textsc{fv}  15-?1-?stay-\textsc{fv}  \textsc{conj}-9 \\
\glt and started to stay with it.’ (G. O. girls who wanted luck lns 39-41)
\z 

Following Botne’s description of Idakho, the presence of /n(ɪ)-/ on the verb following ‘know’ should indicate “culminative” function. But here, only one event, the finding of the dog, leads up to the taking of it $-$ and surely the ‘start to stay with’ is more of an end result than is the ‘taking’.

It seems that $-$ in Logoori texts, at least $-$ the greatest generalization possible about non-future \textit{kʊman̪a} auxiliary constructions has to do with the types of lexical verbs used. These may be considered, without exception, to be telic verbs: ‘make,’ ‘arrive,’ ‘go to the farm,’ ‘take,’ ‘make cornmeal,’ ‘find,’ ‘remove from stove,’ ‘transform into,’ among others.

\subsection{Narrative form /ma-/ or /aa-/ before Subjunctive for near future times} \label{sec:sarvasy:7.4}

Like other Bantu languages, Logoori is rich in ‘narrative’ %
%Why in double quotes? 
%
forms that primarily function in sequences of sentences or clauses. Two of these forms are probably dialectal variants of a single form, comprising the discourse particle \textit{ma} plus the inflected verb; this is realized as \textit{ma-} by some speakers and \textit{aa-} by others.

When the narrative prefix \textit{ma-/aa-} combines with the Subjunctive inflection, Mr. Indire considers the resulting form /ma-SP-\_\_-e/ to have similar hodiernal/crastinal application to the /ra-SP-\_\_-a/ Near Future. Here, a narrative context is not necessary. In fact, Mr. Indire used this form rather than the dedicated Near Future in describing his plan for the next day:

\ea\label{ex:sarvasy:28}
\gll Ma   mu-gamba   \textbf{ma-m-bok-e},     \textbf{ma-n-zj-e}     kʊ   ɪ-gaasi   j-eende. \\
then   3-tomorrow   \textsc{nf2}-\textsc{1sg}-wake-\textsc{subj}  \textsc{nf2}-\textsc{1sg}-go-\textsc{subj}  \textsc{loc}  9-work  9-other \\
\glt ‘Then tomorrow \textbf{I will wake up}, \textbf{then I will go} to another job.’ (M. I. today, tomorrow ln 62)
\z

Although Mr. Indire introduced this form as an actual Near Future tense, in the corpus it most commonly occurs on non-initial verbs in narrative sequences, indicating a temporal or causal connection with the actions denoted by earlier verbs, as in \REF{ex:sarvasy:29}: 

\ea\label{ex:sarvasy:29}
\gll \textbf{Aa-nz-ah-e}       neende   mu-tere,  \textbf{aa-n̪or-e}.\\
\textsc{consec2}-\textsc{1sg}-uproot-\textsc{subj}  with  3-veg.sp  \textsc{consec2}-separate.leaves.from.stems-\textsc{subj} \\
\glt ‘\textbf{I will uproot (it)} together with \textit{umutere}, \textbf{then I will separate the leaves from the stems}.’ (C. C. lunchtime lns 7-8)
\z

In one instance in the corpus, this\textit{ aa-} occurs before a verb lacking the subjunctive suffix \textit{-e}, which clearly refers to a nonfuture event:

\ea\label{ex:sarvasy:30}
\gll Vj-age,     kw-aat-a   m-mba,   a-vi-ivi.\\
8-barn    \textsc{1pl}-put-\textsc{fv}  \textsc{loc}-house  \textsc{aug}-2-thief \\
\glt ‘As for the barns,\footnote{‘Barns’ here apparently refers to grain storage; rather than store food outside, it is now being stored indoors.} we have put (them) in the houses: thieves.

\gll va-ku-n̪o   kʊ-no,   ʊ-kob-e     zi-kwiiri,   na   va-ku-temag-aa. \\
2-\textsc{2sg}-find  \textsc{loc}-\textsc{dem}  \textsc{2sg}-beat-\textsc{subj}  10-cries  \textsc{conj}  2-\textsc{2sg}-chop-\textsc{pres}.\textsc{prog} \\
\glt They find you there; cry out, and they are chopping you. 

\gll \textbf{Aa-kw-aat-a}     ha   vʊ-riri. \\
\textsc{consec2}-\textsc{1pl}-put-\textsc{fv}  \textsc{loc}  14-eating \\
\glt \textbf{(Thus) we have put (them)} in the eating area.’ (\citealt{NicholsSsennyonga1976}, barns lns 1-3)
\z

Here, \textit{aa-kw-aat-a} may be best analyzed as a narrative %
%double quotes?
%
or consecutive form. It may be that pre-SP \textit{aa-} and its variant \textit{ma-} simply mark this narrative form, with relative time understood through other verbal inflections. Mr. Indire did not perceive a difference in meaning between a \textit{ma-\_\_-e} future form uttered as part of a sequence of clauses versus one uttered in isolation.

\section[Conclusion]{Conclusion} \label{sec:sarvasy:8}

This paper has aimed to describe some of the complex usage patterns of recognized future tense markers in Logoori texts, as well as additional constructions used to indicate future time. What can be gleaned from elicitation must be augmented with discourse analysis to yield a fuller picture of the functions of tense forms. 

The Near Future /SP-ra-\_\_-a/ form is the most common future form found in the corpus; it functions both in contexts where a hodiernal or crastinal future time is specified, and in contexts in which epistemically-certain events are predicted to occur at unspecified future times. Scant evidence indicates that the Middle Future form /na-SP-\_\_-e/ may both serve to indicate post-crastinal time periods and possibly for purposes of politeness. The /SP-raka-\_\_-e/ form is called Far Future in the literature, but in at least one text it seems to differ from the unspecified use of the Near Future primarily in epistemic certainty. In the corpus data it seems nearly interchangeable with the /ri-/ form. A second Remote Future form with a longer vowel than the Middle Future form is unattested in the corpus and is not found in previous descriptions of the language, but was produced with consistent duration contrast by Mr. Indire. 

It might seem that speakers of a language with so many future-related inflections would not need to resort to periphrasis to discuss future time. But the Logoori texts corpus shows otherwise; in fact, dedicated tense inflections are supplemented by the Subjunctive, narrative forms and at least two auxiliary constructions to indicate future time.

\section*{Abbreviations}
\begin{tabularx}{.45\textwidth}{lX}
\textsc{1sg}, etc. &  first person singular \\

1, 2, 3, etc. & noun classes \\

\textsc{applic} & applicative \\

\textsc{aug}  &  augment\\

\textsc{caus} &   causative \\

\textsc{conj}   &  conjunction \\

\textsc{consec, consec2} &  consecutive forms \\

\textsc{cont}  &  continuous \\

\textsc{dem}  &  demonstrative \\

\textsc{futperf} & future {\ \ \ \ \ \ \}  perfect \\

\textsc{fv}  &  final vowel \\
\end{tabularx}
\begin{tabularx}{.45\textwidth}{lX}
\textsc{gen}  &  genitive \\

\textsc{if}  &  indefinite future \\

\textsc{loc}  &  locative \\

\textsc{mf} &   middle future \\

\textsc{neg}  &  negative \\

\textsc{nf, nf2} & near future \\

\textsc{poss}  &  possessive \\

\textsc{prog}  &  progressive \\

\textsc{recp}  &  recent past \\

\textsc{rp}  &  remote past \\

\textsc{rf}  &  remote future \\

\textsc{sp}  &  subject prefix \\

\textsc{subj}   & subjunctive \\
& \\
\end{tabularx}
\begin{longtable}[l]{ll}



\end{longtable}

\section*{Acknowledgments}

The analysis presented here stems from a two-quarter Field Methods course at UCLA, with Mwabeni Indire as Logoori consultant. Many thanks to him for his patience and generosity. The analyses here may differ from his own. Thanks to Sandra Nichols for making clips from her 1976 film available for analysis, and to Michael Diercks for allowing access to his narratives and songs corpus. The Diercks corpus was created with support from the National Science Foundation Collaborative Research Grant, Structure and Tone in Luyia: BCS-1355749. Doris Payne and two anonymous reviewers provided very detailed and helpful comments. General thanks to Michael Diercks, Mary Paster, and Michael Marlo for welcoming input from Luyia newbies.

\printbibliography[heading=subbibliography,notkeyword=this]

\end{document}