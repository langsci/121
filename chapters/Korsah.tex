\documentclass[output=paper]{LSP/langsci} 
\author{Sampson Korsah \affiliation{Universit\"at Leipzig}}

\title{Beyond resumptives and expletives in {A}kan} 

\abstract{In Akan, resumption is obligatory for extraction from a subject position. Accordingly, there is usually agreement between the resumptive pronoun (RP) and its referent constituent. However, data from the Asante-Twi dialect suggest that it is possible to have a non-agreeing pronoun in the extraction site. But this is only possible for the   highest subject position. In  this paper I show  that what occupies the subject position in the non-agreeing situation is an instance of the insertion of a default pronoun. Gaps in subject positions are not allowed whatsoever in Akan. In order to avoid violating this constraint in the case of the highest subject, the language has two options; it either uses a resumptive pronoun, or a default pronoun.}



\maketitle

\begin{document}

\section{Introduction} 
It is a standardly held view in Generative syntax traditions that follow the Chomskyan approach e.g. \cite{Chomsky81}, that various constituents in a given syntactic structure may be extracted from argument positions (\textsc{A-positions}) to non-argument positions (\textsc{\=A-positions})  for various information structure purposes. For instance, the internal argument of a verb may be fronted to head a  relative clause, as in   \REF{ex:korsah:1}. I will refer to such extractions as \textsc{\=A-operations}.

\ea\label{ex:korsah:1}
[The woman]$_i$ who \sout{the woman}$_i$ bought the car is rich.
\z

\=A-operations in languages may exhibit various kinds of reflexes \citep{Georgi14}. In some languages, e.g.\ English, the extraction site shows no phonetic signs of such operations. In such cases, there are \textsc{gaps}, see e.g.\ \cite{Salzmann11}.
But there are languages which do not permit gaps; rather they require \textsc{resumptive pronouns} (RP) in the extraction site. For languages that allow RPs, some allow gaps in only certain positions \citep[see][]{Klein14}.
Unlike languages like English however,  Akan (Kwa, Niger-Congo) sometimes does not permit gaps in any extraction site. As \REF{ex:korsah:2} shows, the RP is always obligatory.\footnote{The data presented here is based on the intuitions of three native speakers. Akan has both high and low tones. But I will mark only high tones.} 

Note that an RP agrees with its referent NP in terms of number, person, animacy, and case features. For instance, \textit{Amma} in \REF{ex:korsah:2a} is a third person singular animate NP which has been extracted from a subject position, and thus has nominative case. \textit{Kofi} \REF{ex:korsah:2b} has similar properties except that, because it is extracted from an object position, it has accusative case. 
   
\ea\label{ex:korsah:2}
\ea\label{ex:korsah:2a}
\gll Amma$_i$ na  *\O $_i$/\oor$_i$-h\h u-u Kofi. \\
A. \textsc{foc} \textsc{3sg.nom.anim-}see\textsc{-pst} K. \\
\glt  `\textsc{ama}  (and not say, John) saw Kofi.'

\ex \label{ex:korsah:2b}
\gll Kofi$_i$  na Amma h\h u-u *\O $_i$/n\h o$_i$  n\h o. \\
K. \textsc{foc} A. see\textsc{-pst} \textsc{3sg.acc.anim} \textsc{cd} \\
\glt  `Ama saw \textsc{kofi} (and not say, John).'
\z
\z   
   
This paper focuses on resumption in Akan, particularly resumption that affects contituents in subject position in the Asante-Twi dialect, as in \REF{ex:korsah:2}. (Thus the use of the term \textit{Akan} here is restricted to only this dialect.) Where necessary, the facts of the other dialects will be pointed out. The interesting thing about  subject resumption  in Akan is that sometimes the expected agreement between an extracted NP and its RP does not seem to obtain. The pronominal form that is used in this non-agreeing configuration is \textit{\eer-}.  In \REF{ex:korsah:3}, although the subject NP is animate, its (otherwise) RP does not necessarily  agree with it in terms of number and animacy.  Yet, \REF{ex:korsah:3} is perfectly grammatical \citep[but see][]{Marfo05}.

 
\ea\label{ex:korsah:3} 
\gll Amma$_i$ na  \eer$_i$-h\h u-u Kofi.\\
A.  \textsc{foc} \textsc{3.nom.-}see\textsc{-pst} K. \\
\glt `\textsc{ama}  (and not say, John) saw Kofi.'
\z

This paper has two main aims regarding this phenomenon. First, it examines the status of the non-agreeing \textit{\eer-} in relation to other homophonous pro-forms in Asante-Twi. I present evidence to show whether it is a regular RP, an expletive, or a default pronoun. Second, the paper investigates the constraints on the distribution of the non-agreeing \textit{\eer-} vis-\lw a-vis the other competing forms. 

Regarding the former aim, I will argue that the non-agreeing \textit{\eer-} is a special kind of pronoun; though it is referential, it has certain underspecified pronominal features. This makes it the default pronoun, and it is inserted in contexts where the (agreeing) RP is not possible due to certain constraints.
Regarding the latter, I will assume, following \cite{KorsahnMurphy15}, and contrary to  \cite{Saah94, Saah10}, that  \=A-operations in Akan are movement-based, and that in the particular instances of \=A-operations involving subjects, two factors affect the nature of what can occupy the extraction site. If the extraction is done from an embedded position,  then the extracted constituent can only reach its final   \=A position via the intermediate subject position of the clause. In such a situation, only an (agreeing) RP is permitted at the extraction site. But when the extraction is done from a non-embedded, i.e. the highest, subject position then Asante-Twi has the  option of either skipping the subject position, or going through it. 
When the former option is adopted, then the default pronoun  \textit{\eer-} is inserted. This insertion is necessary in order to repair a violation of a constraint that places a ban on gaps in subject position in Akan syntax.

% \citep{RnS07}.
 
 The remainder of this paper is structured as follows: \sectref{sec:korsah:2} gives a general overview of resumption in Akan, and how it can be analyzed. \sectref{sec:korsah:3} shows instances  of agreement mismatches, and discusses the kinds of features that the relevant pronouns have. \sectref{sec:korsah:4} deals with the constraints on the distribution of the non-agreeing pronoun. \sectref{sec:korsah:5} is the conclusion.
 

\section{Resumption in Akan}\label{sec:korsah:2}
 

In Akan, it is possible to extract NPs from various argument and non-argument positions \citep{MnB05, Saah10} for various \=A-operations. In the present discussion, I will focus on extraction from only subject and object positions. As in many languages \citep{KnC77, Klein14}, the resumptive pronouns correspond to the personal pronoun paradigm (see \tabref{tab:korsah:1}).\footnote{Where there are options, the forms to the right of the slash are used in the Fante dialect.}

\begin{table}[h]
  	\caption{Pronouns in Akan}
  \label{tab:korsah:1}
  	\centering
  	\begin{tabular}{lll p{2cm} c}
  \lsptoprule	
  		\textsc{numb}& \textsc{pers} & \textsc{nom} & \textsc{acc} \\
  		\midrule
  		%\hline
  		%\textsc{sg} & & & \\
  		%\hline
  		\textsc{sg} & 1 &  me- & me/-m\\
  		%    \hline
  		& 2 & wo-/{i} & wo/-w \\
  		%   \hline
  		&  3(\textsc{anim}) & {\oor-}/{no} & no/-n\\
  		
  		&  3(\textsc{inanim}) & {\eer-}{no}/ {\oor-}{no} & (no/-n) \\
  		%\hline
  		%\textsc{pl} &   & & \\
  		%\hline
  		\textsc{pl} & 1 &  {y\eer-} & {y\eer n}/{h\eer n}\\
  		%    \hline
  		& 2 & {mo-/hom-} & {mo/hom} \\
  		%   \hline
  		&  3(\textsc{anim}) & {w\oor-}& {w\oor n/h\oor n}\\
  		
  		&  3(\textsc{inanim}) & {\eer-}(no)/{\oor-}(no)& {w\oor n/h\oor n} \\
  	 \lsptoprule
  	\end{tabular}
  	
  \end{table}

  
    
  In this section, I discuss three issues: the nature of subject and object resumption, how they may be explained, and how to deal with a constraint that I will refer to as the \textsc{no-subject-gap} (NSG) constraint.  
  
\subsection{Subject and object resumption}\label{sec:korsah:2.1}

 
As indicated earlier, Akan allows (and sometimes requires) both subject and object  resumption. Resumption is obligatory for all extracted subjects , whether the extracted constituent is animate \REF{ex:korsah:4a}, or inanimate \REF{ex:korsah:4b}.  Similar restrictions have been reported  for languages like Hebrew \citep{RnS07}. For instance, while it is fine to extract \textit{ha-sulxan} as the object of  \textit{raca} \REF{ex:korsah:5a}, extracting  \textit{ya'ale} which is in subject position renders the construction ungrammatical \REF{ex:korsah:5b}.
 
\ea\label{ex:korsah:4} 
\ea\label{ex:korsah:4a}
\gll\scalebox{1.5}{\oor}b{\'{a}}{\'{a}}$_i$  {\'{a}}a   *\O $_i$/\oor$_i$-w{\'{a}}r\h e-e  Kofi n{\'{o}} \ fi Aburi.\\
woman \textsc{rel} \textsc{3sg.nom}-marry-\textsc{pst} K. \textsc{cd} be.from A. \\
\glt  `The woman who married Kofi is from Aburi.'\hfill   \citep[92]{Saah10}

\ex \label{ex:korsah:4b}
\gll [ K{\'{r}}ata{\'{a}} n{\'{o}} ]$_i$  {\'{a}}a  *\O $_i$/\eer$_i$-d{\'{a}} p{\'{o}}n{\'{o}} n{\'{o}} s{\'{o}}  n{\'{o}} y\h \eer \, f\eer.\\
 ~ paper  \textsc{def} ~  \textsc{rel} \textsc{3sg.nom.inanim}-lie table \textsc{def} top \textsc{cd} be nice \\
\glt  `The paper that is on the table is nice.'
\z
\z
 

\ea\label{ex:korsah:5} \textit{Hebrew}   \citep[120-121]{RnS07}\\
\ea\label{ex:korsah:5a}
\gll kaniti et ha-\v sulxan$_i$ \v se oto xana amra \v se dalya ma'amina \v se Kobi raca \O$_i$.\\
(I).bought \textsc{acc} the-table that him Hannah said that Dalya believes that Kobi wanted \\
\glt  `I bought the table that Hannah said that Dalya believes that Kobi wanted.'

\ex\label{ex:korsah:5b}
\gll *kaniti et ha-sulxan$_i$ \v se hu xana amra \v se dalya ta'ana \v se \O$_i$ ya'ale harbe kesef.\\
(I).bought \textsc{acc} the-table that he Hannah said that Dalya claimed that {} will.cost a.lot money \\
\glt  `I bought the table that Hannah said that Dalya claimed that will cost a lot of money.'
\z
\z

The agreement requirements between the RP and the extracted NP in Akan include person, number, animacy, and case specifications. (I defer a discussion of how Akan handles the ban on gaps in the extraction sites of subjects to \sectref{sec:korsah:2.3}.) 
 
For object extraction, resumption is  obligatory for only animate NPs.  For instance,  the RP \textit{n\h o} in \REF{ex:korsah:7} is obligatory. For inanimate object NPs, however, the overt realization of an RP appears to be optional; gaps are sometimes permitted, as in when \REF{ex:korsah:8a} and \REF{ex:korsah:8b} are compared. This distribution of gaps and RPs with regards to extracted object inanimate NPs in Akan, may be attributable to independent properties of either the pronoun system, or the verbs involved (see e.g.\ \citealt{Chinebuah76} and \citealt{Larson05}  properties in other Kwa languages).


\ea\label{ex:korsah:7} 
\gll  Me-huu \oor b{\'{a}}{\'{a}}$_i$  {\'{a}}a Kofi w{\'{a}}re-e n{\'{o}}$_i$/*\O $_i$ n{\'{o}}.\\
 \textsc{1sg}-see woman \textsc{rel} K. marry-\textsc{pst}  \textsc{3sg.acc} \textsc{cd} \\
\glt  `I saw the woman whom Kofi married.' \hfill  \citep[92]{Saah10}
\z


\ea\label{8} %\textit{Extracted inanimate object}\\
\ea\label{ex:korsah:8a}
\gll [ K{\'{r}}ata{\'{a}} n{\'{o}} ]$_i$ {\'{a}}a Kofi h{\'{u}}-ui *n{\'{o}}$_i$/\O $_i$ n{\'{o}} da p{\'{o}}n{\'{o}} \ n{\'{o}} s{\'{o}}.\\
 ~ paper  \textsc{def} ~  \textsc{rel} K. see-\textsc{pst} \textsc{3sg.acc} \textsc{cd} lie table \textsc{def} top \\
\glt  `The paper that Kofi saw is on the table.'

\ex\label{ex:korsah:8b}
\gll [ K{\'{r}}ata{\'{a}} n{\'{o}} ] {\'{a}}a Kofi t\h e-e n{\'{o}}$_i$/*\O $_i$ n{\'{o}} da p{\'{o}}n{\'{o}} \ n{\'{o}} s{\'{o}}.\\
~ paper \textsc{def} ~ \textsc{rel} K. tear-\textsc{pst}  \textsc{3sg.acc} \textsc{cd} lie table \textsc{def} top \\
\glt  `The paper that Kofi tore is on the table.'

\end{xlist}
\z
 
 The data that we have seen so far show that resumption may result from extraction within a single clause. But unlike languages like Tsez, as reported by \cite{PolinskynPotsdam01}, resumption in Akan is not clause-bound; it is possible to extract NPs from deeply-embedded contexts, across several clauses. In \REF{ex:korsah:9}, for instance the extracted NP \textit{\oor b{\'{a}}{\'{a}}} spans three clauses i.e.\ three CPs.
 
 %EXX
 \ea\label{ex:korsah:9} 
\gll Me-hu-u  [\textsc{np} \oor b{\'{a}}{\'{a}}$_i$ {\'{a}}a Ama p{\'{\eer}} [\textsc{cp} s\eer {} {\'{o}}b{\'{i}}{\'{a}}r{\'{a}} t{\'{e}} [\textsc{cp} s\eer {} Kofi {\'{a}}-k{\'{a}} [\textsc{cp} s\eer{} \oor-b{\'{\eer}}-war{\'{e}} n{\'{o}}$_i$  n{\'{o}}]]]].\\
\textsc{1sg}-see-\textsc{pst} ~ woman  \textsc{rel} A. like ~ \textsc{comp} everybody hear ~ \textsc{comp} K. \textsc{perf}-say ~  \textsc{comp} \textsc{3sg.nom}-\textsc{fut}-marry \textsc{3sg.acc} \textsc{cd}\\
\glt  `I saw the woman whom Ama wants everybody to hear that Kofi has said that he will marry her.'

\z

 
\subsection{Analysis of Akan resumption}\label{sec:korsah:2.2}
 
Resumption in Akan has been traditionally  analyzed as involving base-generation \citep[see][]{Saah92, Saah94, Saah10}. This view suggests that resumption is not due to movement but rather the result of binding. However, following \cite{KorsahnMurphy15}, I will assume that resumption in Akan involves movement.  
 
The main argument for the base-generation approach to resumption is that resumption is  possible in contexts which have been argued to be syntactic islands in languages like English. For instance, movement of $who$  \REF{ex:korsah:10b}, out of the relative clause for the purposes of question formation is illicit. This is because relative clauses are syntactic islands \citep{Ross67}. Furthermore, movement out of such syntactic configurations results in ungrammaticality cross-linguistically.
 
 
%EXX
\ea\label{ex:korsah:10} 
\ea[]{\label{ex:korsah:10a}
John met the lady that owns the publishing company.}

\ex[*]{\label{ex:korsah:10b}
Who$_i$ did John meet  \O$_i$ that owns the publishing company?}
\end{xlist}
\z
 
 
%(I will talk about the verbs in boldface in (\ref{11}) later.)
Given the above, displacing \textit{Sikan\h i} in \REF{ex:korsah:11} is predicted to be illicit if it involves movement. But, it is obviously acceptable in Akan \REF{ex:korsah:11b}.   It follows then that whatever process that results in such NPs ending up in \=A positions in Akan could not have involved movement. Under this assumption, the NP in the \=A position is assumed to be directly merged at its surface position. Its surface relationship with the RP is established via semantic binding, as sketched in \REF{ex:korsah:12}. A similar analysis has been proposed by \cite{McCloskey11} for Irish.

%EXX
\ea\label{ex:korsah:11} 
\gll  Me-\textbf{nim} baabi aa Sikan\h i n\h o f\h i.\\ 
  \textsc{1sg}-know where \textsc{rel} rich.man \textsc{def} come.from\\
 \glt `I know where the rich man comes from.'
\z



\ea\label{ex:korsah:11b}  
 \gll  Sikan\h i$_i$ \h aa me-\textbf{n\h im} baabi aa \oor$_i$-f\h i n\h o \h a-da.\\ 
 rich.man \textsc{rel} \textsc{1sg}-know where \textsc{rel} \textsc{3sg.nom}-come.from  \textsc{cd} \textsc{perf}-sleep \\
 \glt `\#The rich man who I know where he comes from is asleep.'
\z

%TREE1       TREE2

\begin{multicols}{2}
 \ea\label{ex:korsah:12} \textit{Base-generation}	~\\	
 \begin{tikzpicture}[scale=0.65][sibling distance=15pt]
  	\tikzset{every tree node/.style={align=center,anchor=north}}
  	\Tree [.CP \node(wh){Sikan\h i}; [.C\1 \h aa 
  	[.TP \node(x){\textbf{\oor-}}; 
  	[.{} 
  	] ] ] ]
  %	\draw[semithick, <-] (x.south)..controls +(south:2) and +(south:2)..node {}(y.south);
  	\draw[->] (wh.south) to [bend right=1200] node [midway,fill=white]
  	{\textsc{bind}}
  	(x.south);
  	\end{tikzpicture}
 \z 
 
  %	\ex. \textit{Movement}~\\
  \ea\label{ex:korsah:13} \textit{Movement}	~\\
  \begin{tikzpicture}[scale=0.65][sibling distance=15pt]
  	\tikzset{every tree node/.style={align=center,anchor=north}}
  	\Tree [.CP \node(wh){Sikan\h i}; [.C\1 \h aa 
  	[.TP \node(x){\textbf{\oor-}}; 
  	[.{}  
  	] ] ] ]
  	(x.south);
  	%\draw[semithick, <-] (x.south)..controls +(south:2) and +(south:2)..node {}(y.south);
  	\draw[semithick,dashed,<-] (wh.south) to [bend right=80] node [midway,fill=white]{\textsc{move}}
  	(x.south);
  	(x.south);
  	\end{tikzpicture}  	
 \z 
\end{multicols}


Contrary to the base-generation approach, the movement approach would proceed as sketched in \REF{ex:korsah:13}. Here, the target NP is literally extracted from one position to the final landing site (in the direction of the dashed arrow). \REF{ex:korsah:13} is a simplified illustration, but when the extraction is done from a much lower position in the structure, the movement path includes all available intermediate landing sites. This idea is succinctly expressed by the notion of \textsc{Successive Cyclic Movement} proposed by \cite{Chomsky77}. Going by the movement approach, the explanation for the agreement relationship between an RP and its antecedent NP in \=A contexts is that a resumptive pronoun is the most economical way of realizing the features of the copy of an extracted NP, see e.g.\ \cite{Nunes04}.
    
One core issue to be addressed  if one is to  account for resumption in Akan in terms of movement is the empirical justification for movement in the language. Empirical evidence in  Akan comes from tones. \cite{KorsahnMurphy15} show that there are movement reflexes on the stem of every verb across which movement has taken place. This is registered in the tonal changes. For instance, when we compare the tone of the verb  stem \textit{nim} in \REF{ex:korsah:11}, in which the target NP is in-situ, with the same verb in \REF{ex:korsah:11b}, where the NP has moved, we observe this tonal difference. The claim here is that the tonal change from low to high on the verb \textit{nim} indicates that \textit{Sikan\h i} has undergone movement. Thus, for Akan, we need not rely solely on island effects when talking about \=A-operations; evidence may be found in phonological reflexes of such operations. The interested reader is referred to \cite{KorsahnMurphy15} for the details. But as far as the present discussion is concerned, this will be the theoretical assumption for all \=A-operations.
  
     
\subsection{Dealing with the \textit{no-subject-gap} constraint}\label{sec:korsah:2.3}

We showed in \sectref{sec:korsah:2.1} that Akan permits no gaps in subject positions. This property seems to fit into a larger picture which suggests that some languages tend to disprefer extraction from subject positions. However, sometimes it is imperative for languages to deploy strategies that force extraction from such positions, and thus a violation of this constraint becomes inevitable. When this is the case, then languages tend to adopt one of the following strategies as a repair mechanism.
  
  
%EXX 
\ea\label{ex:korsah:14} \textit{Repair strategies for NSG} \citep[see][]{RnS07} \\
\ea\label{ex:korsah:14a}
Base-generation or movement, resulting in a resumptive pronoun

\ex\label{ex:korsah:14b} 
Skipping subject position, and filling it with an expletive pronoun
\z
\z

Regarding \REF{ex:korsah:14a}, I follow \cite{Shlonsky92} and propose that Akan uses RPs as a repair strategy against violating the NSG. I will demonstrate later that the language also has the option of using what looks like a version of the \textsc{expletive insertion} strategy in \REF{ex:korsah:14b}.

  
   
\subsection{Summary}\label{sec:korsah:2.4}

So far we have made the following observations and arguments about \=A-opera\-tions, and resumption in Akan. First, When the extraction is from a subject position,  it is obligatory to  overtly realize an RP. But from object positions, this may be optional. Second, the most crucial property about the resumption process for the present discussion is the requirement that the extracted NP and its RP must agree in terms of number, person, animacy, and case features. Third, \=A-operations in Akan involve movement. Fourth, Akan uses an RP  as a repair strategy for what would otherwise be a violation of the NSG constraint.
   
\section{Subject agreement mismatches}\label{sec:korsah:3}
  
The agreement requirement between a moved NP and its RP is particularly crucial for subjects, given the NSG constraint. However, I show in this section that a non-agreeing pronominal element \textit{\eer-} may fill the extraction site of the highest subject position. I will claim that this is a default pronoun; it is less specified than the inanimate \textit{\eer-} form in \tabref{tab:korsah:1}, and it is also not an expletive pronoun.
  
\subsection{Non-agreeing subject pronoun}\label{ex:korsah:3.1}

In many contexts (of particularly spoken Asante-Twi), sometimes the expected agreement between an extracted NP and its RP does not obtain. Consider  \REF{ex:korsah:16}-\REF{ex:korsah:17}, where I have glossed the otherwise RP as \textsc{dflt}. I will argue later on that it is a \textsc{default} pronoun; see also \cite{McCracken13}. The expected RP for \textit{\oor b\h a\h a} in \REF{ex:korsah:16} is \textit{-\oor}, i.e.\ {\sc{3sg.subj.anim}},  and that of \textit{Nn{\'{i}}p{\'{a}} d{\'{u}} n{\'{o}} } in \REF{ex:korsah:16b} is \textit{w\oor-}, i.e.\ {\sc{3pl.subj.inanim}}. Yet speakers of Asante-Twi have absolutely no problem with parsing \REF{ex:korsah:16}-\REF{ex:korsah:16b}.  
  
  

%EXX
\ea\label{ex:korsah:16} 
\gll \scalebox{1.5}{\oor}b{\'{a}}{\'{a}}$_i$ {\'{a}}a  \eer$_i$-w{\'{a}}r\h e-e Kofi n{\'{o}} fi A.\\
 woman \textsc{rel} \textsc{dflt}-marry-\textsc{pst} K. \textsc{cd} from A.  \\
\glt  `The woman who married Kofi is from Aburi.'

\z


\ea\label{ex:korsah:16b} 
\gll [ N-n{\'{i}}p{\'{a}} d{\'{u}} n{\'{o}} ]$_i$ {\'{a}}a  \eer$_i$-hy\h i\h a-{\'{\eer}} n{\'{o}} ma-a h{\'{\oor}} s\oor-\oor e.\\
~ \textsc{pl}-person ten \textsc{def} ~ \textsc{rel} \textsc{dflt}-meet-\textsc{pst} \textsc{cd} cause-\textsc{pst} there light-\textsc{pst}.\\
\glt  `The ten people who met made the place exciting.'

\z



With \REF{ex:korsah:17}, it is not obvious whether the \textit{\eer-} is the non-agreeing form since it is homophonous with the expected RP; see \tabref{tab:korsah:1}. Thus syncretism cannot be ruled out in these contexts. I do not intend to address this issue in  this paper. 


%EXX
\ea\label{ex:korsah:17} 
\ea\label{ex:korsah:17a}
\gll [ K{\'{r}}ata{\'{a}} n{\'{o}} ]$_i$ {\'{a}}a  \eer$_i$-y\h er\h a-a\eer{} n\h o n\h i\h e.\\
 ~ paper  \textsc{def} ~  \textsc{rel} \textsc{dflt/3.sg.inanim}-lost-\textsc{pst} \textsc{cd} this \\
\glt  `This is the paper which got missing.'

\ex\label{ex:korsah:17b}
\gll [ N-k{\'{r}}ata{\'{a}} \textbf{n{\'{o}}} ]$_i$ {\'{a}}a  \eer$_i$-y\h er\h a-a\eer{} n\h o n\h i\h e.\\
 ~ \textsc{pl}-paper \textsc{def} ~ \textsc{rel} \textsc{dflt/3.pl.inanim}-lost-\textsc{pst} \textsc{cd} this\\
\glt  `These are the papers which got missing.'

\z
\z



The agreement mismatch observed above is not restricted to only relative clause constructions. As \REF{ex:korsah:18} and \REF{ex:korsah:19} show, it also obtains in focus constructions and  ex-situ content questions. Note that for constructions like \REF{ex:korsah:19b}, some speakers prefer the non-agreeing pronoun over the agreeing form. 
 

        

%EXX
\ea\label{ex:korsah:18} 
\ea\label{ex:korsah:18a}
\gll [ \scalebox{1.5}{\oor}b{\'{a}}{\'{a}} y{\'{i}} ]$_i$ na  \oor$_i$-/\eer$_i$-b\h \eer-w{\'{a}}r\h e Kofi.\\
 ~ woman \textsc{dem} ~ \textsc{foc} \textsc{3sg.anim.nom}/\textsc{dflt}-\textsc{fut}-marry K. \\
\glt  `This woman (as opposed to some other woman) will marry Kofi.'

\ex\label{ex:korsah:18b}
\gll [ Nn{\'{i}}p{\'{a}} \textbf{d{\'{u}}} p{\'{\eer}} ]$_i$ na  w\oor$_i$-/\eer$_i$-hy\h i\h a-\h \eer.\\
 ~ people ten only ~ \textsc{foc} \textsc{3pl.anim.nom}/\textsc{dflt}-meet-\textsc{pst} \\
\glt  `Only ten people (as opposed to more people) met.'

\z
\z


%EXX
\ea\label{ex:korsah:19} 
\ea\label{ex:korsah:19a}
\gll Hw{\'{a}}n$_i$ na  \oor$_i$-/\eer$_i$-b\h \eer-w{\'{a}}r\h e Kofi?\\
Who \textsc{foc} \textsc{3sg.anim.nom}/\textsc{expl}-marry-\textsc{pst} K. \\
\glt  `Who married Kofi?'

\ex\label{ex:korsah:19b}
\gll Hw{\'{a}}n-m{\'{o}}$_i$ na ?w\oor$_i$-/\eer$_i$-hy\h i\h a-\eer.\\
Who-\textsc{pl} \textsc{foc} \textsc{3pl.anim.nom}/\textsc{dflt}-meet-\textsc{pst} \\
\glt  `Who met?'

\z
\z

The agreement mismatch reported here obtains only in the Asante-Twi dialect.
For instance, \REF{ex:korsah:20b} shows that the equivalent of the non-agreeing \textit{\eer-} in similar syntactic contexts  is illicit in the Fante dialect of Akan.\footnote{Note that Fante uses the same  form, i.e.\ \textit{\textbf{\oor-}}, for both \textsc{3sg.nom.+/-anim}, and \textsc{expl}(=non-referential pronoun).} 
%EXX
\ea\label{ex:korsah:20}  \textit{Fante dialect} \\
\ea\label{ex:korsah:20a}
\gll [ Ny{\'{i}}p{\'{a}} k{\'{o}}r p{\'{\eer}} ]$_i$ na  \h \oor$_i$-b\h a-{\'{\i}}.\\
 ~ person one only ~ \textsc{foc} \textsc{3sg}-come-\textsc{pst}\\
\glt  `Only one person (as opposed to more people) came.'

\ex\label{ex:korsah:20b}
\gll [ N-ny{\'{i}}p{\'{a}} d{\'{u}} p{\'{\eer}} ]$_i$ na  w\h \oor$_i$/*\h \oor$_i$-hy\h ia-{\'{\i}}.\\
~  \textsc{pl-}person ten only ~ \textsc{foc} \textsc{3pl}/\textsc{dflt}-meet-\textsc{pst}\\
\glt  `Only ten people (as opposed to more people) met.'

\z
\z


\subsection{Distribution of non-agreeing pronoun \textit{\eer-}}\label{sec:korsah:3.2}
    
Interestingly, the non-agreeing \textit{\eer-} is restricted to only matrix subject positions; subjects of embedded clauses do not seem to allow this non-agreeing \textit{\eer-} \REF{ex:korsah:21a}. Also, this \textit{\eer-} cannot refer to extracted  objects, irrespective of their agreement features and level of embedding. Thus \textit{\eer-} cannot be co-indexed with \textit{\oor b\h a\h a} in \REF{ex:korsah:21b}. 
  

%EXX

\ea\label{ex:korsah:21}  
\ea\label{ex:korsah:21a}
\gll [ N-n{\'{i}}p{\'{a}} d{\'{u}} n{\'{o}} ]$_i$ \h aa  me-n\h im {s\eer} w\oor$_i$-/*\eer$_i$-hy\h i\h a-{\'{\eer}}  n{\'{o}} n\h i\h e.\\
  ~ \textsc{pl}-person ten \textsc{def} ~ \textsc{rel}  \textsc{1sg}-know that \textsc{3pl}/\textsc{dflt}-meet-\textsc{pst} \textsc{cd} \textsc{dem}\\
\glt  `These are the ten people who I know met.'

\ex\label{ex:korsah:21b}
\gll \scalebox{1.5}{\oor}b{\'{a}}{\'{a}}$_i$  {\'{a}}a Kofi w{\'{a}}r\h e-e  n{\'{o}}$_i$/*\eer$_i$ n{\'{o}}  fi Aburi.\\
 woman \textsc{rel} K. marry-\textsc{pst} \textsc{3sg.acc}/\textsc{dflt} \textsc{cd} be.from A.\\
\glt  `The woman whom Kofi married is from Aburi.'

\z
\z



\subsection{Summary}\label{sec:korsah:3.3}

I have shown that in the Asante-Twi dialect of Akan, there is optional agreement between an extracted matrix subject NP and what replaces it at the extraction site. This is possible in all \=A-operations. The form \textit{\eer-} is always used in the contexts where this optionality is allowed. These observations raise at least two fundamental questions: First, what is the  nature the pro-form \textit{\eer-} that is allowed in these non-agreeing configurations, and how different is it from other homophonous forms in the language? Second, why is it only the highest subject position that allows this agreement optionality? I address these issues in the next section.
    
\section{Accounting for \textit{\eer}- pro-forms} \label{sec:korsah:4}

In this section I propose an account for the non-agreeing pronoun and its distribution. As far as I can see, there are at least two ways to deal with this. One way is to assume that \textit{\eer-} is a default pronoun that is inserted whenever movement skips a subject position. The alternative approach is to assume that the regular RP \textit{\eer-}  loses some of its pronominal features in the context of the matrix subject. In the present discussion, I will argue for the plausibility of the former. I will treat the non-agreeing \textit{\eer-} as a \textsc{default} (\textsc{dflt}) pronoun that is inserted whenever movement skips the (matrix) subject position. I show that this \textit{\eer-}  has certain features that make it less specified as a personal pronoun, and which also makes it different from similar pro-forms in the language. I argue further that this special feature makes it the default choice in instances where a repair is needed for what otherwise would be a violation of the NSG constraint.
    
\subsection{Three types of \textit{\eer-} in Akan} \label{sec:korsah:4.1}	

Based on the anaphoric properties of the \textit{\eer-} pro-forms that we have seen so far, I distinguish between three kinds of \textit{\eer-} pro-forms  in Asante-Twi: the \textsc{agreeing} form(s), the \textsc{default} form, and the \textsc{expletive} form. I summarize the properties of these homophonous forms in \tabref{tab:korsah:2}. In order to account for their distinctions, I assume that every personal pronoun in Akan needs to specify a value for at least one of the following properties \textbf{in order to be anaphoric or referential}: 

i. \textsc{person}: Participant (i.e.\ 1st/2nd person) or Non-participant (i.e.\ 3rd person).

ii. \textsc{number}: Singular (Sg) or Plural (Pl)
 
iii. \textsc{animacy}: Animate or Inanimate (Inan) 
 
\begin{table}[h]
 	\caption{Features of \textit{\eer}- pro-forms}
 	\label{tab:korsah:2}
 	\centering
 	\begin{tabular}{lccc p{2cm} c}
  \lsptoprule
 		
 		&\textsc{pro-form}& \textsc{person} & \textsc{number}  & \textsc{animacy}   \\
 		\hline
 		\textsc{agreeing} &\eer & NPart & Sg & Inan  \\
 		%\hline
 		%   \hline
 		\textsc{agreeing}&	\eer & NPart & Pl  & Inan \\
 		
 		\textbf{\textsc{default}}&	\textbf{\eer}&	\textbf{NPart} 	&	\textbf{-} & 	\textbf{-}  \\
 		
 		\textsc{expletive}& 	\eer& - 	&- & -   \\
 	 \lsptoprule
 		
 	\end{tabular}
 	
 \end{table}
 
 


  
Given \tabref{ex:korsah:2}, all agreeing \textit{\eer-} forms have specified values for all the pronominal properties. I treat these as one type of \textit{\eer-}, i.e.\ they all fully agree with their antecedent NPs, as in \REF{ex:korsah:23}, and \REF{ex:korsah:4b}.
  
 %EXX
\ea\label{ex:korsah:23} 
\gll [ N-k{\'{r}}ata{\'{a}} n{\'{o}} ]$_i$  {\'{a}}a  \eer$_i$-d{\'{a}} p{\'{o}}n{\'{o}} n{\'{o}} s{\'{o}}  n{\'{o}} y\h \eer \, f\eer.\\
~ \textsc{pl-}paper  \textsc{def} ~ \textsc{rel} \textsc{3pl.nom.inan}-lie table \textsc{def} top \textsc{cd} be nice \\
\glt  `The papers that are on the table are nice.'

\z

 
 % \subsubsection{The default \eer-}
The default \textit{\eer-} differs from the agreeing pro-forms in the sense that it has no specified values for number and animacy. But since it is specified for at least third person, it is still referential, as in \REF{ex:korsah:24a}. We know this because when a native speaker is presented with \REF{ex:korsah:24b}, independent of \REF{ex:korsah:24a}, the \textit{\eer-} in \REF{ex:korsah:24b} helps to select the pronoun that matches the coindexed gap in terms of person features, as \REF{ex:korsah:24c} shows. Thus, \textit{\eer-} here simply refers to third person.  In this regard, the use of \textit{\eer-} in \REF{ex:korsah:24a} is comparable to the use of \textit{their} in \REF{ex:korsah:25a}. At least some native speakers of English use \textit{their} in such variable binding contexts (although some speakers would use \REF{ex:korsah:25b} instead).  Here, although \textit{their} and \textit{your driver} do not totally agree, there still exists a binding relationship between them. I emphasize the striking similarity between the person features of both  \textit{their} and the default \textit{\eer-} in Akan; both are third person.


%EXX
\ea\label{ex:korsah:24} 
\ea\label{ex:korsah:24a}
\gll [Amma$_i$ (ne Yaa)]$_i$ na  \eer$_i$-w{\'{a}}r\h e-e Kofi.\\
 A. \textsc{conj} Y. \textsc{foc} \textsc{dflt}-marry-\textsc{pst} K.\\
\glt  `\textsc{ama} (and \textsc{yaa}) married Kofi.'

\ex\label{ex:korsah:24b}
\gll ---$_i$ na  \eer$_i$-w{\'{a}}r\h e-e Kofi.\\
{} \textsc{foc} \textsc{dflt}-marry-\textsc{pst} K. \\
\glt  `\textbf{---} married Kofi.'

\ex\label{ex:korsah:24c}
\gll \scalebox{1.5}{\oor}no$_i$/W\h\oor n$_i$/*w\h o$_i$/*m\h e$_i$ na  \eer$_i$-w{\'{a}}r\h e-e Kofi.\\
\textsc{3sg/pl/2sg/1sg} \textsc{foc} \textsc{dflt}-marry-\textsc{pst} K.\\
\glt  `\textsc{s/he/they} married Kofi.'

\z
\z



\ea\label{ex:korsah:25} 
\ea\label{ex:korsah:25a}  
If there are any changes to the service, [your driver]$_i$ will do \textbf{their}$_i$ best to inform you.\footnote{This is an extract from an audio clip played on some National Express coaches in Britain. }

\ex\label{ex:korsah:25b} 
If there are any changes to the service, [your driver]$_i$ will do \textbf{his}$_i$ or \textbf{her}$_i$ best to inform you.

\z
\z

 

 
The expletive \textit{\eer-} in \tabref{tab:korsah:2} has no specified feature for any of the three properties. It is only similar to the other two in terms of form. This may be due to the fact that all three \textit{\eer-} forms get the same case feature, i.e.\ nominative, by virtue of where in the syntax they are permitted, i.e.\ subject position. This is the only feature that it shares with the agreeing \textit{\eer-}, and the default \textit{\eer}. But certainly this property is not an inherent property of this pronoun. 
 
 
The expletive \textit{\eer-} is like expletive pronouns in other languages, e.g.\ \REF{ex:korsah:26} where the subject pronoun is absolutely non-referential. As \REF{ex:korsah:27} shows, similar constructions exist in Akan. Here, \textit{\eer-} has a purely formal function, and neither agrees nor refers to any NP.
 
 
 %EXX
 \ea\label{ex:korsah:26} 
 It is raining.

\z

 
 %EXX
\ea\label{ex:korsah:27} \textit{Expletive \eer-}\\
 
\gll \scalebox{1.5}{\eer}-w\oor \, s{\'{\eer}}  ob{\'{i}}{\'{a}}{\'{a}}$_i$ t{\'{u}}m{\'{i}} ky{\'{e}}r{\'{\eer}} n$_i$-{\'{a}}dw{\'{e}}n. \\
\textsc{expl}-be that everyone  able show \textsc{3sg.poss}-mind \hfill \\
\glt  `It ought to be the case that everybody is able to express their opinion.'

\z

 
Based on the above discussions, I conclude  that Asante-Twi has at least three kinds of \textit{\eer-} pro-forms: the agreeing one, the default one, and the expletive one. This distinction is based on their pronominal properties. Contexts that require a less specific but referential \textit{\eer-} favor the default form over the expletive form. 

 
\subsection{On the distribution of agreeing and default  \textit{\eer-} } \label{sec:korsah:4.2}
  
 I showed early on  that Akan uses resumptive pronouns as a repair strategy for the NSG. Given that RPs are the result of movement (see \sectref{sec:korsah:2.3}), I deduce that all instances of RPs in subject positions in Akan are derived via movement, as represented in \REF{ex:korsah:28}.  I propose further in this section that apart from this RP strategy, Akan also sometimes inserts the default \textit{\eer-} as a repair mechanism for a potential violation of the NSG constraint. This latter strategy is, however, relativized to only the matrix subject position in the syntactic structure. I propose that the matrix subject position is a privileged position in Akan. This is supported by the fact that there is a back-up strategy to repair any instance of an NSG violation in case resumption fails. 
  
  
 We recall from \sectref{sec:korsah:2.2} that a language may also skip a subject position altogether in order to avoid an extraction that would result in a violation of the NSG constraint. Given the assumption that the use of a RP in the extraction site results from movement only, I deduce that any instance of the non-agreeing pronoun in the extraction site cannot be attributed to movement, because movement yields only (agreeing) resumptive pronouns. I claim therefore that such non-agreeing cases involve the skipping of the subject position. In order to avoid a violation of the NSG constraint, the  default \textit{\eer-} pronoun is inserted. This strategy is available only for the matrix subjects. Crucially, this process is ordered, as shown in \REF{ex:korsah:29}.
  
  
\begin{multicols}{2}
    	\ea\label{ex:korsah:28} \textit{Movement}~\\
    	\begin{tikzpicture}[scale=0.65][sibling distance=15pt]
    	\tikzset{every tree node/.style={align=center,anchor=north}}
    	\Tree 
    	[.CP \node(wh){NP}; [.C\1 C 
    	[.TP \node(x){\textbf{AGR}}; 
    	[.T\1 T
    	[.$v$P \node(y){\sout{\textbf{NP}}}; 
    	[.\node(t){$v$\1}; $v$ \edge[dashed];{}  ]]] ] ] ]    
    	\draw[semithick, <-] (x.south)..controls +(south:2) and +(south:2)..node {}(y.south);
    	\draw[semithick,<-] (wh.south) to [bend right=80] node {}
    	(x.south);
    	\end{tikzpicture}
    \z	
     	
     	\ea\label{ex:korsah:29} \textit{Deriving default \eer-} ~\\  
     	\begin{tikzpicture}[scale=0.6][sibling distance=15pt]
     	\tikzset{every tree node/.style={align=center,anchor=north}}
     	\Tree 
     	[.CP \node(wh){\textbf{NP} }; 
     	[.C\1 C\\{} 
     	[.TP \node(x){\textsc{dflt}\\{\textbf{\eer-}}}; 
     	[.T\1 T
     	[.$v$P \node(y){\sout{\textbf{NP} }}; 
     	[.\node(t){$v$\1}; $v$ \edge[dashed];
     	{}  ]]] ] ] ]
     	\node(z) at (0,-5) {};
     	%\draw[semithick, <-] (x.south) to [bend right=45] (y.west);
     	\draw[semithick, <-] (wh.south)..controls +(south:5) and +(south:3)..node {\ding{203}}(y.south);
     	\draw[semithick, <-] (x) to node [right=1mm] {\ding{202}} (z);
     	\end{tikzpicture}
    	\z
    \end{multicols}
  
  
  
 
\section{Summary and conclusion} \label{sec:korsah:5}


The distribution of (agreeing) resumptive pronouns and the (non-agreeing) default pronoun in Akan can be summarized  as presented in \tabref{tab:korsah:3}.

 \begin{table}[h]
 	\caption{Distribution of RPs in Akan}
 	\label{tab:korsah:3}
 	\centering
 	\begin{tabular}{lcccc}
    \lsptoprule
 		Strategy& Matrix Subj &  Embedded Subj  &   Obj \\
 		\hline
 		Gap &   \ding{55} & \ding{55}  &  (\ding{51})\\    
 		
 		RP& (\ding{51}) &  \ding{51}  & \ding{51} \\
 		\textsc{dflt} insertion &   (\ding{51}) & \ding{55}  &  \ding{55}\\  
 	 \lsptoprule
 	\end{tabular}
 \end{table} 

With regard to the two main issues that this paper set out to address, I have argued that based on the relevant pronominal properties, three types of \textit{\eer-} pro-forms can be distinguished in Asante-Twi: the agreeing, the default, and the expletive forms. I have claimed that the non-agreeing pronoun is analyzable as a default pronoun that is inserted in the matrix subject position whenever resumption fails. Thus Akan, like Yoruba \citep{Adesola10}, has two strategies for repairing a potential violation of the \textsc{no-subject-gap}-like constraint; it uses either a RP, or a default pronoun. In \sectref{sec:korsah:4}, I suggested a potential alternative explanation for the agreement mismatches with regard to the distribution of these agreeing and non-agreeing subject pronouns in Akan. Further discussion of this will be insightful for how \textsc{spell-out} works in Generative syntax. But this is an issue that  future research on this phenomenon will attempt to address.


\section*{Abbreviations}

\begin{tabularx}{.45\textwidth}{lX}
{\sc{1,2,3}} &   first, second, third \\ 
& person      \\    
 		
{\sc{acc}} & accusative       \\

 {\sc{cd}} &   clausal determiner      \\    
 		
{\sc{comp}} & complementizer      \\   

{\sc{conj}} & conjunction     \\ 

{\sc{def}} &   definite      \\

{\sc{dem}} & demonstrative      \\

{\sc{foc}} &   focus      \\    
 		
{\sc{fut}}  & future       \\
 
{\sc{hab}}  &   habitual     \\
\end{tabularx}
\begin{tabularx}{.45\textwidth}{lX}
{\sc{(in)anim}} &   (in)animate     \\

{\sc{nom}}  &   nominative     \\  

NPart &  non-participant     \\ 

Obj &  object     \\ 
 
{\sc{perf}} &   perfective     \\

 {\sc{pl}}  &   plural      \\    
 		
{\sc{poss}} & possessive       \\

{\sc{pst}}& past        \\
 
{\sc{rel}} &   relativizer     \\

{\sc{sg}} &  singular     \\ 

Subj &  subject     \\ 
\end{tabularx}




\section*{\textbf{Acknowledgments}}

This research was funded by the DFG-funded Interaktion Grammatischer Bausteine `Interaction of Grammatical Building Blocks' (IGRA) program at the University of Leipzig. I am grateful to Andrew Murphy, Chris Collins, Gereon M\"uller, Martin Salzmann, Martina Martinovi\v c,  two anonymous reviewers, and an editor of this volume, for their comments and suggestions. I also thank Augustina P. Owusu for corroborating the Asante-Twi data. All inadequacies are however solely mine.



\printbibliography[heading=subbibliography,notkeyword=this]

\end{document}

