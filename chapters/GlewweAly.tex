\documentclass[output=paper]{langsci/langscibook} 
\title{Coronal palatalization in Logoori} 
\author{% 
 Eleanor Glewwe\affiliation{University of California, Los Angeles}\lastand 
 Ann Aly Bailey \affiliation{University of California, Los Angeles}
}
% \sectionDOI{} %will be filled in at production


\abstract{
Logoori (Bantu, Kenya, JE41) exhibits several palatalization processes affecting both coronal and dorsal consonants. These processes give rise to derived instances of [t͡ʃ], [d͡ʒ], and [ʃ]. While the post-alveolar affricates [t͡ʃ] and [d͡ʒ]  also correspond to independent phonemes of Logoori, Leung’s (1991) rules-based phonological analysis of the language considered [ʃ] to be only an allophone of /h/. We provide new instances of surface [ʃ] that, if also derived from /h/, would lead to a rule ordering paradox involving the palatalization process Consonant-Glide Reduction and the rule /h/-Plosivization. The ordering paradox can be resolved by claiming either that [ʃ] is in fact a phoneme or that these new instances of [ʃ] are derived from something other than /h/. We opt for the latter analysis and argue that Logoori palatalizes /s/ to [ʃ] before a palatal glide (sj} \textit{ ʃ), a coronal palatalization process not previously described for the language. We also discuss evidence that the phonemicization of [ʃ] is in progress in Logoori.}   

\maketitle
\begin{document}

 



\section{Introduction}

This paper examines palatalization processes in Logoori, a Bantu language of western Kenya (JE41, Luyia) \citep{Mould1981,Bastin2003}. Logoori has a series of postalveolar consonants, [t͡ʃ], [d͡ʒ], and [ʃ], that occur as palatalized allophones of other sounds. The postalveolar affricates [t͡ʃ] and [d͡ʒ] also correspond to independent phonemes, but in the previous literature [ʃ] was considered to be only an allophone of /h/. We review the palatalization processes that have been described for Logoori and identify a rule ordering paradox in an earlier phonological analysis of the language. We then argue, still in a rules-based framework \citep{ChomskyHalle1968}, that [ʃ] can be derived from /s/ as well as from /h/. The proposed palatalization of /s/ to [ʃ] resolves the rule ordering paradox. Finally, we discuss evidence that [ʃ] is in the process of becoming an independent phoneme of Logoori.

\section{Background}
\subsection{Palatalization processes in Logoori} %2.1 /

\citet{Leung1991}, the only phonological description of a Logoori dialect, includes a number of palatalization processes that apply to both coronal consonants and back consonants. While Leung described a different dialect than the one we consider, most of the rules she formalized are active in the dialect under study, with some differences.\footnote{The data in this paper were gathered in a graduate field methods class at UCLA during the 2014/2015 academic year. Our consultant, Mwabeni Indire, is a male native speaker of Logoori in his early thirties. He was raised in Nairobi, Kenya and also spent time with extended family in Vihiga County. In addition to Logoori, he speaks English and Kiswahili. If according to our consultant an utterance can stand as a complete sentence, we captitalize and punctuate the free translation accordingly; otherwise no punctuation is used.} 

Leung found two coronal palatalization processes, which she grouped together as the rule Palatalization of Dental Consonants:


\ea{}

\textbf{Palatalization of Dental Consonants} \citep[117]{Leung1991}




j̪ $\rightarrow$ j / \_\_ \{i, u\}

n̪ $\rightarrow$ ɲ / \_\_ \{i, u\}
\z


The unusual dental glide /j̪/ that appears in the first part of \REF{ex:glewwe:1} is never realized in the dialect considered here. Where Leung observed [j̪] on the surface, we observe only the palatal glide [j]. We retain the phoneme /j̪/, however, because nasals assimilate to it in place:
\todo{I am unsure how the examples in this paper should be rendered. Please check. SN}

\ea{}
 /Nj̪anzaa/  $\rightarrow$  [n̪anzaa]  \\{}
\textsc{1sg}{}-like-\textsc{prs}\\{}
\glt ‘I like’
\z

Nevertheless, in the dialect under study, the first part of Leung’s Palatalization of Dental Consonants must be subsumed under a more general rule that renders palatal all dental glides that do not undergo other rules.

The second part of Leung’s Palatalization of Dental Consonants, whereby the dental nasal becomes palatal before a high vowel, does apply in the dialect we consider, as shown in \REF{ex:glewwe:3}:


\ea\label{ex:glewwe:3}{}
 /Nman̪i/  $\rightarrow$  [maɲi]   \\{} 
\textsc{1sg}-know-\textsc{fv}\\{}
\glt {‘I} know’
\z

In Leung’s phonological description of Logoori, the dental consonants /j̪/ and /n̪/ are the only coronal consonants that palatalize.    

For back consonants, Leung posited the following rule, according to which the velar stops and /h/ palatalize before a high front vowel:


\ea{}
\textbf{Palatalization of Back Consonants} \citep[116]{Leung1991}\\{}
k $\rightarrow$ t͡ʃ / \_\_ i
\\{}
g $\rightarrow$ d͡ʒ / \_\_ i
\\{}
h $\rightarrow$ ʃ / \_\_ i      
\\{}
(except before the causative /iz/)
\z


In the dialect we consider, the velar stops do not undergo Palatalization of Back Consonants. Examples in \REF{ex:glewwe:5} shows that /k/ and /g/ do not palatalize before [i]:


\ea\label{ex:glewwe:5}{}
   \ea{} 
 /Nhandeki/  $\rightarrow$  [mbandeki]  (*[mbandet͡ʃi])  \\{}
\textsc{1sg}-write-\textsc{hod.pfv}\\{}
\glt ‘I wrote.’ 

\ex{}
 /akaragi/  $\rightarrow$  \textsc{[}akaɾagi]  (*[akaɾad͡ʒi])\\{}
\textsc{3sg}cut\textsc{hod.pfv}\\{}
\glt ‘He cut.’\\{}
\z
\z 

While the dialect Leung described palatalizes the final /k/ of the verb root ‘write’ and the final /g/ of the verb root ‘cut’ before the hodiernal perfective suffix /i/, the dialect we consider does not. However, it does palatalize /h/ before [i], as seen in \REF{ex:glewwe:6}:\footnote{The real picture is actually more complex. Palatalization of /h/ does not seem to apply across morpheme boundaries in this dialect: /Nvehi/ ‘1\textsc{sg}lie\textsc{hod.pfv}’ $\rightarrow$ [mbehi] (*[mbeʃi]). Additionally, even morphemeinternally there is both type and token variation in the application of /h/ palatalization. [ehiiɾi] ‘\textsc{cl}9ethnic group’ does not exhibit palatalization of /h/ while /mahiga/ ‘\textsc{cl}6cooking stones’ is realized as both [mahiga] and [maʃiga].} 

\ea\label{ex:glewwe:6}{}
 /ahiraa/  $\rightarrow$  [aʃiɾaa]\\{}
\textsc{3sg}-ride-\textsc{prs}\\{}
\glt ‘he is riding’
\z

A second palatalization process Leung described is Consonant-Glide Reduction:

\ea\label{ex:glewwe:7}{} 
 \textbf{Consonant}-\textbf{Glide Reduction} \citep[116]{Leung1991}\\{}
kj $\rightarrow$ t͡ʃ \\{}
gj $\rightarrow$ d͡ʒ \\{}
hj $\rightarrow$ ʃ \\{}
\z

Due to the shape of Logoori morphemes, the palatal glides in the targets of ConsonantGlide Reduction are always derived from vowels. That is, Consonant-Glide Reduction is always fed by a gliding process. Logoori has seven vowels, which we transcribe as /i e ɛ a ɔ o u/ (cf. \citealt{Leung1991}). Vowel length is underlyingly contrastive, making for a total of fourteen vowel phonemes. A high vowel (/i/, /e/, /o/, or /u/) becomes [syllabic] when it occurs before another vowel, and the formation of the glide induces compensatory lengthening of the second vowel in the sequence. This process is formalized for the front vowels in \REF{ex:glewwe:8}:

\ea\label{ex:glewwe:8}{}
 \textbf{Gliding}\\{}
 \begin{tabular}{llll}
{i/e} & {V} & $\rightarrow$ & {j 2 2}\\{}
 {1} & {2} & & \\{}
 \end{tabular}
\z

Consonant-Glide Reduction is active for both velar stops and /h/ in the dialect considered here. The following derivation demonstrates how Gliding and ConsonantGlide Reduction apply in an example involving /k/:

\ea\label{ex:glewwe:9}{}
\begin{tabular}{ll}
/kej̪isiŋgaa/ & \textsc{cl7-refl}-wash-\textsc{prs}\\{} 
keisiŋgaa  &  /j̪/Deletion{\footnotemark}
{kjiisiŋgaa}  &  {Gliding and Compensatory Lengthening}\\{}
t͡ʃiisiŋgaa  &  Consonant-Glide Reduction\\{}
[t͡ʃiisiŋgaa] & ‘It washes itself.’\\{}
\end{tabular}
\z
\footnotetext{/j̪/Deletion \citep[116]{Leung1991}: /j̪/ $\rightarrow$ Ø / V + \_\_} 

Note that since Consonant-Glide Reduction is always fed by Gliding, which entails compensatory lengthening of the following vowel, the vowel following a postalveolar affricate derived by Consonant-Glide Reduction should always be long.

The example in \REF{ex:glewwe:10} shows that /h/ also palatalizes and fuses with a following derived [j] (the Underlying Representation (UR) for the root ‘new’ will be justified below):


\ea\label{ex:glewwe:10}{}
/kehia/{\rmfnm} $\rightarrow$  [keʃa]\footnote{The reason the surface form of ‘\textsc{cl7}new’ is [keʃa] and not [keʃaa] is because Gliding-induced compensatory lengthening is blocked wordfinally \citep{Leung1991}.} \\{}
\textsc{cl7}-new\\{}
\glt ‘new’
\z
\footnotetext{It is not possible to determine whether the underlying vowel after /h/ in the root ‘new’ is /i/ or /e/ because both glide to [j] before another vowel, and the underlying vowel never appears on the surface. We have arbitrarily written this vowel as /i/.}  



To recapitulate, in this dialect /h/ palatalizes to [ʃ] through both Palatalization of Back Consonants and Consonant-Glide Reduction, but /k/ and /g/ only palatalize to the postalveolar affricates [t͡ʃ] and [d͡ʒ] through Consonant-Glide Reduction. Compared with other Bantu languages, this dialect of Logoori exhibits relatively limited velar palatalization. \citet{HymanMoxley1996} proposed a typology of Bantu velar palatalization classifying languages according to the extent of the environments in which they palatalize /k/ and /g/. With velar palatalization occurring only as the fusion of [kj] and [gj], the dialect we consider falls into \citegen{HymanMoxley1996} most restrictive category. 

From a broader crosslinguistic perspective, the palatalization processes Logoori exhibits are typical. Across languages, the most common types of consonants to be palatalized are coronal and dorsal consonants \citep{Bateman2011}. Logoori shows cases of both coronal and dorsal palatalization, and we will show that there are more cases of coronal palatalization than previously thought. Additionally, palatalization is most commonly triggered by high front vowels or the palatal glide \citep{Bateman2011}, and these are the triggers seen in Logoori’s palatalization processes.

\subsection{The phonemic status of [t͡ʃ] and [d͡ʒ]} %2.2 /
[t͡ʃ] and [d͡ʒ{]} correspond to independent phonemes of Logoori. Surface postalveolar affricates can be identified as underlying if they are followed by a short vowel. Derived [t͡ʃ{]} and [d͡ʒ{]} in this dialect always arise as a result of ConsonantGlide Reduction. As mentioned above, the vowels following these derived affricates are always long because the palatalization rule is fed by Gliding, which triggers compensatory lengthening on the vowel. Therefore, if a postalveolar affricate is not followed by a long vowel, we infer that it is nonderived, i.e. underlying.]{\textmd{In addition to being allophones of the velar stops, the postalveolar affricates [t͡ʃ] and [d͡ʒ] correspond to independent phonemes of Logoori. Surface postalveolar affricates can be identified as underlying if they are followed by a short vowel. Derived [t͡ʃ] and [d͡ʒ] in this dialect always arise as a result of ConsonantGlide Reduction. As mentioned above, the vowels following these derived affricates are always long because the palatalization rule is fed by Gliding, which triggers compensatory lengthening on the vowel. Therefore, if a postalveolar affricate is not followed by a long vowel, we infer that it is nonderived, i.e. underlying.}}

Many words containing a postalveolar affricate, particularly [t͡ʃ], exhibit variation in their realization. Consider the following representative set of words, for which we observed variable Surface Representations (SRs): 


\ea\label{ex:glewwe:11}{}
  \ea\label{ex:glewwe:11a}
 /keaŋge/  $\rightarrow$ [kjaaŋge]/[t͡ʃaaŋge]\\{}
\textsc{cl7}-my\\{}
\glt ‘my’

\ex\label{ex:glewwe:11b}{}
 /kokia/    $\rightarrow$  [kokja]/[kot͡ʃa]\\{}
\textsc{cl15}-dawn\\{}
\glt ‘dawn’

\ex\label{ex:glewwe:11c}{}
 /ekiova/  $\rightarrow$  [ekjoova]/[et͡ʃoova]\\{}
\textsc{cl9}-outside\\{}
\glt ‘outside’
\z
\z 
In (11a–c), a surface [t͡ʃ] varies with a surface [kj] sequence. In \REF{ex:glewwe:11a}, we know the [t͡ʃ] in the palatalized variant derives from /k/ because the morpheme is the \textsc{cl}7 concord, which appears as [ke] in other words. In \REF{ex:glewwe:11b} and \REF{ex:glewwe:11c}, there is no evidence from alternations that the roots ‘dawn’ and ‘outside’ begin with /k/ underlyingly, but the fact that the [kj] variant is possible suggests that [t͡ʃ] is not underlying in these roots. We therefore propose the underlying forms /kokia/ and /ekiova/ for these words. Cases like \REF{ex:glewwe:11b} and \REF{ex:glewwe:11c} are very common in Logoori. In such cases, we take the underlying phoneme to be the velar stop because the postalveolar affricate can vary with the velar stop on the surface. 

A few morphemes containing underlying postalveolar affricates are given in \REF{ex:glewwe:12}:

\ea\label{ex:glewwe:12}{}
  \ea\label{ex:glewwe:12a}
 /d͡ʒi-/  [ad͡ʒiɾoɾaa]\\{}
\textsc{cl}4  3\textsc{sg}-\textsc{cl}4.\textsc{obj}-see-\textsc{prs}\\{}
\glt ‘He sees it.’

\ex\label{ex:glewwe:12b}{}
 /d͡ʒib/  [ked͡ʒibaa]\\{}
answer  \textsc{cl7}-answer-\textsc{prs}
\glt ‘It is answering.’

\ex\label{ex:glewwe:12c}{}
 /d͡ʒirit͡ʃi/  [ed͡ʒiɾit͡ʒi]\\{}
bull  \textsc{cl9}-bull
    ‘bull’

\ea\label{ex:glewwe:12d}{}
 /t͡ʃɛɛre/  [m̩t͡ʃɛɛɾe]\\{}
rice  \textsc{cl3}-rice\\{}
\glt    ‘rice’
\z
\z 

In (12a–c), we know the affricates are underlying because they are followed by short vowels. In \REF{ex:glewwe:12d}, the affricate is followed by a long vowel, but we consider this vowel to be underlyingly long. We analyze the [t͡ʃ] in the root ‘rice’ as underlying because it does not alternate with [k] and because, unlike \REF{ex:glewwe:11b} and \REF{ex:glewwe:11c}, it does not exhibit any variation in its realization ([m̩t͡ʃɛɛɾe] cannot be produced as [m̩kjɛɛɾe]).

\subsection{The phonemic status of [ʃ]} %2.3 /

While the affricates [t͡ʃ] and [d͡ʒ] can be either underlying or derived from /k/ and /g/ by ConsonantGlide Reduction, Leung argues that [ʃ] is \textit{only} ever derived from /h/. That is, according to Leung, [ʃ] is only an allophone of /h/ and does not (also) correspond to a contrastive phoneme like /t͡ʃ/ and /d͡ʒ/. It can be derived from /h/ by Palatalization of Back Consonants, as in [aʃiɾaa] (see ex. 6 above), or it can be derived from an underlying /hV/ sequence by ConsonantGlide Reduction, as in [keʃa] (see ex. 10 above). Leung’s claim that [ʃ] is always derived from /h/ rests on two types of evidence. The first type is evidence from alternations. Many instances of surface [ʃ] can be shown to alternate with [h]. The following is a straightforward example:


\ea\label{ex:glewwe:13}{}
   \ea\label{ex:glewwe:13a}
 /koroha/  $\rightarrow$  [koroha]\\{}
\textsc{cl15}-get.tired-\textsc{fv}\\{}
\glt ‘to get tired’ \citep[38]{Leung1991}

\ex\label{ex:glewwe:13b}{}
 /korohi/  $\rightarrow$  [koroʃi]\\{}
\textsc{1pl}-get.tired-\textsc{hod.pfv}\\{}
\glt ‘We got tired.’ \citep[38]{Leung1991}
\z
\z 

The [ʃ] in \REF{ex:glewwe:13b} can be identified as being derived from /h/ by Palatalization of Back Consonants since the same segment surfaces as [h] when it does not precede [i], as in \REF{ex:glewwe:13a}.. 

Other instances of [ʃ] can be shown to be derived from /h/ because they alternate with [b]. Logoori has a rule whereby /h/ becomes [b] after a nasal. This is exemplified by the following partial paradigm:

\ea\label{ex:glewwe:14}{}
   \ea\label{ex:glewwe:14a}
/N-handek-aa/  $\rightarrow$  [mbandekaa] \\{}
\textsc{1sg}write\textsc{prs}
\glt ‘I am writing.’

\ex\label{ex:glewwe:14b}{}
/o-handek-aa/  $\rightarrow$  [ohandekaa]\\{}
\textsc{2sg}write\textsc{prs}\\{}
\glt ‘You are writing.’

\ex\label{ex:glewwe:14c}{}
/a-handek-aa/  $\rightarrow$  [ahandekaa]\\{}
\textsc{3sg}write\textsc{prs}\\{}
\glt ‘He is writing.’
\z
\z 

We formalize this rule as /h/-Plosivization (cf. Leung’s Stop \citet[117]{Formation1991}):



\ea\label{ex:glewwe:15}{}
\textbf{/h/-Plosivization}: h $\rightarrow$ b / [+nas] \_\_ {\rmfnm}
\z
\footnotetext{The alternation of [h] and [b] has a historical explanation. Logoori /h/ came from *\textit{p}. The bilabial stop lenited to the glottal fricative everywhere except after nasals, where it was preserved \citep{Hyman2003}. This is why it surfaces in 1\textsc{sg} forms like \REF{ex:glewwe:14a}, which have a nasal prefix. Although the verb root alternations in (14a–c) have a historical account, presumably the root ‘write’ has been restructured to /handek/ in modern speakers’ grammars, so we posit the synchronic rule /h/Plosivization to explain the alternations.}




Also relevant is Nasal Place Assimilation:


\ea\label{ex:glewwe:16}{}
\textbf{Nasal Place Assimilation} \citep[116]{Leung1991}\\{}
[+nas] $\rightarrow$ [\alpha{place}] / \_\_ {[} -son, \alpha{place} {]}
\z



Consider again the surface form of ‘\textsc{cl}7new,’ given in \REF{ex:glewwe:10} as [keʃa]. This surface [ʃ] does not precede [i], so it cannot be the result of Palatalization of Back Consonants. We claimed it derived from an underlying /hi/ sequence by Consonant-Glide Reduction but provided no evidence for the UR /hia/ ‘new’. In the absence of such evidence, we would be forced to conclude that [ʃ] is underlying and therefore a phoneme of Logoori. However, the evidence exists in the form of the SR of ‘\textsc{cl}9-new,’ [embja]. The derivations of the \textsc{cl}7 and \textsc{cl}9 forms of ‘new’ together establish the correct UR of the root ‘new’ and demonstrate a crucial rule ordering:


\ea\label{ex:glewwe:17}{}
\begin{tabular}{lll}
/eN-hia/
 \textsc{cl9}-new & /ke-hia/  \textsc{cl7}-new & \\
eN-hja   &   ke-hja    &    Gliding\\
eN-bja   &   {}-   &     /h/-Plosivization\\
{}-    &    ke-ʃa     &   Consonant-Glide Reduction\\
em-bja  &    {}-     &   Nasal Place Assimilation\\
[embja]  ‘new’  &  [keʃa]     ‘new’
\end{tabular}
\z


The surface alternation between [bj] and [ʃ] in the two forms of ‘new’ in \REF{ex:glewwe:17} can only be accounted for by an underlying /hi/. The reason [a] is not long in [embja] and [keʃa], despite Gliding having applied, is because Gliding-induced compensatory lengthening is blocked wordfinally \citep{Leung1991}. The derivations in \REF{ex:glewwe:17} also show that /h/-Plosivization must bleed Consonant-Glide Reduction. If Consonant-Glide Reduction were ordered before /h/-Plosivization, the SR of ‘\textsc{cl}9new’ would contain a [ʃ], which is not the case. The derivation also shows that Nasal Place Assimilation must follow /h/-Plosivization, since the nasal of the prefix gets its place feature from [b].

The [ʃ]/[b] alternation can also be seen in the paradigm of the verb ‘ride’:


\ea\label{ex:glewwe:18}{}
   \ea\\label{ex:glewwe:18a}
 /Nhiraa/  $\rightarrow$  [mbiɾaa]\\{}
\textsc{1sg}-ride-\textsc{prs}\\{}
\glt ‘I am riding’

\ex\label{ex:glewwe:18b}{}
 /ohiraa/  $\rightarrow$  \textsc{[}oʃiɾaa]\\{}
\textsc{2sg}-ride-\textsc{prs}
\glt ‘you are riding’  

\ex\label{ex:glewwe:18c}{}
 /ahiraa/  $\rightarrow$  [aʃiɾaa]  \\{}
\textsc{3sg}-ride-\textsc{prs}
\glt ‘he is riding’
\z
\z 
The underlying /h/ in the verb root ‘ride’ surfaces as [b] in the 1\textsc{sg} due to /h/-Plosivization and as [ʃ] in the other persons due to Palatalization of Back Consonants. The fact that the form [mbiɾaa] contains [b] and not [ʃ] illustrates that /h/-Plosivization must also bleed Palatalization of Back Consonants. 

The preceding discussion has demonstrated that many surface [ʃ]s in Logoori can be shown to be derived from /h/ because of alternations with [h] and [b]. 

Leung’s second type of evidence for the nonphonemicity of [ʃ] is distributional. In her data, those surface [ʃ]s that do not alternate with known allophones of /h/ always precede [i], the vowel that triggers palatalization of /h/ to [ʃ]. She therefore posits that these [ʃ]s are also derived from /h/. This analysis neatly accounts for the quite limited distribution of these nonalternating [ʃ]s. Thus Leung concludes that all Logoori [ʃ]s are derived from /h/ and that /ʃ/ is not a phoneme.

\section{A rule-ordering paradox}

In the dialect of Logoori considered here, there are surface [ʃ]s that do not seem to be derived from /h/. This creates a problem for Leung’s phonological analysis and at least requires us to say something different about this dialect. Consider the following partial paradigms that feature [ʃ]:


\ea\label{ex:glewwe:19}
{verbs with [ʃ]}
  \ea\\label{ex:glewwe:19a}
  [Ø-ʃɔɔm-aa] \\
\textsc{1sg}-wail-\textsc{prs}  
\glt ‘I am wailing.’


\ex\label{ex:glewwe:19b}{}
 [o-ʃɔɔm-aa] \\
\textsc{2sg}-wail-\textsc{prs}  \\
\glt ‘You are wailing.’


\ex\label{ex:glewwe:19c}{} 
[a-ʃɔɔm-aa] \\
\textsc{3sg}-wail-\textsc{prs}\\
\glt ‘He is wailing.’


\ex\label{ex:glewwe:19d}{} 
[Ø-ʃoov-aa]  \\
\textsc{1sg}-throw.out-\textsc{prs} \\
\glt ‘I am throwing out’


\ex\label{ex:glewwe:19e}{}
 [o-ʃoov-aa]  \\
\textsc{2sg}-throw.out-\textsc{prs} \\
\glt ‘you are throwing out’

\ex\label{ex:glewwe:19f}{}
 [a-ʃoov-aa]  \\
\textsc{3sg}throw.out\textsc{prs} \\
\glt ‘he is throwing out’
\z
\z 

The examples in \REF{ex:glewwe:19} do not appear in Leung’s description, so the forms they would have in the dialect she studied are uncertain. Under Leung’s analysis, however, the instances of surface [ʃ] in \REF{ex:glewwe:19} cannot be derived through Palatalization of Back Consonants because they do not precede [i]. They must therefore be derived from underlying /hV/ sequences through Consonant-Glide Reduction. This would lead us to posit URs for the verb roots ‘wail’ and ‘throw out’ that begin with /hi/,\footnote{See Footnote 4.} like the root of the adjective ‘new.’ Since the 1\textsc{sg} subject prefix is /N/ and the present tense suffix is /aa/, the URs of the 1\textsc{sg} forms of ‘wail’ and ‘throw out’ in \REF{ex:glewwe:19a} and \REF{ex:glewwe:19d} would then be /Nhiɔmaa/ and /Nhiovaa/, respectively. The beginnings of these URs are like that of ‘\textsc{cl}9new’ in \REF{ex:glewwe:17}, but while the /Nhi/ sequence in \REF{ex:glewwe:17} surfaces as [mbj], in \REF{ex:glewwe:19a} and \REF{ex:glewwe:19d} the hypothesized /Nhi/ sequences unexpectedly surface as [ʃ]. 

The absence of a nasal prefix may have an independent explanation. Logoori has a rule that deletes a nasal before [s]:



\ea\label{ex:glewwe:20}{}
 \textbf{Nasal Deletion} \citep[116]{Leung1991}\\{}
 [+nas] $\rightarrow$ Ø / \_\_ s\\
\z

This rule is active in our speaker’s dialect:

\ea\label{ex:glewwe:21}{}
 /Nsɔmaa/  $\rightarrow$  [sɔmaa]\\
\textsc{1sg}read\textsc{prs}\\
\glt ‘I am reading.’
\z

While Leung does not mention it, it might be argued that nasals delete before all sibilants, that is, [s] and [ʃ]. Thus the reason for the absence of a nasal prefix in the 1\textsc{sg} forms in \REF{ex:glewwe:19a} and \REF{ex:glewwe:19d} is the same as the reason for the absence of a nasal prefix in the 1\textsc{sg} form in \REF{ex:glewwe:21}. 

Returning now to the forms in \REF{ex:glewwe:19a} and \REF{ex:glewwe:19d} in their entirety, if /h/Plosivization bleeds ConsonantGlide Reduction, as was demonstrated in \REF{ex:glewwe:17} with the derivation of [embja], /Nhiɔmaa/ and /Nhiovaa/ should be realized as [mbjɔɔmaa] and [mbjoovaa]. Instead, they have the [ʃ]initial forms given in \REF{ex:glewwe:19}. The alternation between [bj] and [ʃ] that we see for the paradigm of the adjective ‘new’ does not arise in the paradigms for the verbs ‘wail’ and ‘throw out.’ If we assume that all three roots are underlyingly /h/initial, as demanded by Leung’s analysis, we arrive at a rule ordering paradox. This paradox is demonstrated in \REF{ex:glewwe:22} and \REF{ex:glewwe:23} with the words ‘\textsc{cl}9-new’ and ‘I am throwing out\textsc{’}:

\ea\label{ex:glewwe:22}{}
\begin{tabular}{lll}
 /eN-hia/  \textsc{cl9}-new & /N-hiov-aa/\\
  \textsc{1sg}-throw.out-\textsc{prs} & \\
eN-hja   &   N-hjoov-aa    &    Gliding\\
eN-bja   &   N-bjoov-aa    &    /h/-Plosivization\\
{}-     &   {}-       &   Consonant-Glide Deletion\\
{}-     &   {}-      &    Nasal Deletion\\
em-bja  &    m-bjoov-aa    &    Nasal Place Assimilation\\{}
[embja]  ‘new’  &  *[mbjoovaa]  ‘I am throwing out’\\
\end{tabular}
\z

The ordering of /h/-Plosivization before Consonant-Glide Reduction, established in \REF{ex:glewwe:17} to derive the correct form of ‘\textsc{cl}9-new,’ yields the incorrect surface form for ‘\textsc{I} am throwing out.’ Switching the order of ConsonantGlide Reduction and /h/Plosivization does not solve the problem:


\ea\label{ex:glewwe:23}{}
\begin{tabular}{111}
 /eN-hia/  \textsc{cl9}-new & /N-hiov-aa/ 
 \textsc{1sg}-throw.out-\textsc{prs} & \\
eN-hja  &    N-hjoov-aa   &     Gliding\\
eN-ʃa   &    N-ʃoov-aa    &    Consonant-Glide Reduction\\
{}-     &   {}-       &   /h/-Plosivization\\
e-ʃa    &   ʃoov-aa   &    Nasal Deletion{\rmfnm}\\
{}-     &   {}-      &    Nasal Place Assimilation\\
\*[eʃa]    ‘new’  &  [ʃoovaa] 
\glt ‘I am throwing out’ \\
\z

\footnotetext{Here the environment of Nasal Deletion is provisionally extended to \_\_ ʃ. See \sectref{sec:glewwe:5} for further discussion.}
Now the correct SR of ‘I am throwing out’ can be derived, but the SR yielded for ‘\textsc{cl}9-new’ is incorrect. [embja] requires /h/-Plosivization to be ordered before Consonant-Glide Reduction while [ʃoovaa] requires Consonant-Glide Reduction to be ordered before /h/-Plosivization, generating the rule ordering paradox. Note that if Consonant-Glide Reduction were indeed to bleed /h/-Plosivization, there would never be any surface alternations within a paradigm between [bj] and [ʃ]. One might expect speakers to stop maintaining a UR beginning with /hi/ in their grammars if there was no surface evidence pointing to the existence of this UR. However, the paradigm of the adjective ‘new’ proves that such alternations do exist, necessitating the ordering of /h/-Plosivization before Consonant-Glide Reduction. 

If this ordering is correct, a new account of the derivation of verb forms like [ʃoovaa] and [ʃɔɔmaa] must be found. Two explanations suggest themselves. The first is that [ʃ] is a phoneme of Logoori after all, and the [ʃ] in 1\textsc{sg} forms like [ʃoovaa] is underlying. The second is that the [ʃ] in [ʃoovaa] and similar forms is derived from something other than [hj]. We argue for the latter option, proposing that [ʃ] can also be derived from [sj].

\section{A new coronal palatalization rule}

We call the palatalization and fusion of [sj] to [ʃ] Coronal ConsonantGlide Reduction and formalize it in \REF{ex:glewwe:24}:


\ea\label{ex:glewwe:24}{}
 \textbf{Coronal Consonant-Glide Reduction}\\
 sj $\rightarrow$ ʃ\\{}
\z

The evidence for \REF{ex:glewwe:24} comes from several quarters. First, there is intraspeaker variation that suggests that [ʃ] in [ʃɔɔmaa] and [ʃoovaa] is derived and not underlying. Consider the variants below:
 
\ea\label{ex:glewwe:25}{}
{palatalization and fusion of [sj] to [ʃ]} \\
   \ea\label{ex:glewwe:25a}
 [Ø-ʃoov-aa]\\{}
\textsc{1sg}-throw.out-\textsc{prs}\\{}
\glt ‘I am throwing out’

\ex\label{ex:glewwe:25b}{}
 [Ø-ʃjoov-aa]\\{}
\textsc{1sg}-throw.out-\textsc{prs}\\{}
\glt ‘I am throwing out’

\ex\label{ex:glewwe:25c}{}
 [Ø-ʃɔɔm-aa]\\{}
\textsc{1sg}-wail-\textsc{prs}\\{}
\textsc{‘I} am wailing.’

\ex\label{ex:glewwe:25d}{}
 [Ø-sjɔɔm-aa]\\{}
\textsc{1sg}-wail-\textsc{prs}\\{}
\glt ‘I am wailing.’
\z 
\z 

The variation between [ʃ] and [ʃj] and between [ʃ] and [sj] in \REF{ex:glewwe:25} is reminiscent of the variation we see in words with derived postalveolar affricates, such as the 1\textsc{sg} possessive pronoun for \textsc{cl}7 possessums (two of the three variants in \REF{ex:glewwe:26} were given in \REF{ex:glewwe:11} above):

\ea\label{ex:glewwe:26}{}
   \ea\label{ex:glewwe:26a}
 [t͡ʃ-aaŋge]\\{}
\textsc{cl7}-my\\{}
\glt ‘my’

\ex\label{ex:glewwe:26b}{}
 [t͡ʃj-aaŋge]\\{}
\textsc{cl7}-my\\{}
\glt ‘my’

\ex\label{ex:glewwe:26c}{}
 [kj-aaŋge]\\{}
\textsc{cl7}-my\\{}
\glt ‘my’
\z
\z 

The fact that a palatal glide sometimes surfaces in ‘I am wailing’ and ‘I am throwing out’ suggests that underlyingly there is a high front vowel between the sibilant and the vowel that is always present in the surface form. The variants in \REF{ex:glewwe:26} show that, in cases of velar palatalization, the surface form sometimes exhibits a palatalized consonant fused with the glide (as in 26a), a palatalized consonant with the glide still present (as in 26b), and the underlying consonant and the palatal glide, with no application of ConsonantGlide Reduction (as in 26c). These variants all have analogues in \REF{ex:glewwe:25}: \REF{ex:glewwe:25a} and \REF{ex:glewwe:25c} show full palatalization and fusion, like \REF{ex:glewwe:26a}; \REF{ex:glewwe:25b} shows palatalization with incomplete fusion, like \REF{ex:glewwe:26b}; and \REF{ex:glewwe:25d} shows a form to which neither palatalization nor fusion have applied, like \REF{ex:glewwe:26c}. Crucially, the underlying consonant that is revealed in \REF{ex:glewwe:25d} is [s], not [h]. If the root of ‘wail’ was underlying /h/-initial, we would expect a variant form like [hjɔɔmaa], since the [k] in [kjaaŋge] corresponds to the underlying /k/ of the \textsc{cl}7 prefix. In \REF{ex:glewwe:25d}, though, we see [s] instead of [h]. This is an indication that the underlying initial consonant of ‘wail’ is actually /s/. 

Another piece of evidence comes from the forms of the verb ‘grind’. \citet{Leung1991} gives the UR of the infinitive form of this verb as /ko-siɛ-a/\footnote{Again, it is impossible to determine whether the first vowel in the root is /i/ or /e/. \citet{Leung1991} uses a special notation for a vowel that is unspecified as to whether it is /i/ or /e/, but here we simply write /i/.}  ‘\textsc{cl}15-grind-\textsc{fv}’ and its SR as [kosja]. The /i/ in the root glides before /ɛ/, and /ɛ/ in turn deletes before /a/. Leung’s SR exhibits no palatalization of /s/ to [ʃ], let alone fusion with [j]. In our data, however, the infinitive form of ‘grind’ is [koʃja]. Assuming the underlying form of the root ‘grind’ is still /siɛ/ in this dialect, the form [koʃja] constitutes evidence for the palatalization of /s/ in the language. 

There are further, dialectinternal reasons to think the [ʃ] in [koʃja] comes from an underlying /s/. Consider the following surface forms:

\ea\label{ex:glewwe:27}{}
  \ea\label{ex:glewwe:27a}
 [ko-ʃ{}-a]\footnote{The underlying form of the root ‘burn’ is /he/.} \\{}
\textsc{cl15}-burn-\textsc{fv}\\{}
\glt ‘to burn’

\ex\label{ex:glewwe:27b}{}
 [m-bez-aa]\\{}
\textsc{1sg}-burn-\textsc{prs}\\{}
\glt ‘I am burning.’


\ex\label{ex:glewwe:27c}{}
 [a-hez-aa]\\{}
\textsc{3sg}-burn-\textsc{prs}\\{}
\glt ‘He is burning.’


\ex\label{ex:glewwe:27d}{}
 [ko-ʃj-a]\\{}
\textsc{cl15}-grind-\textsc{fv}\\{}
\glt ‘to grind’


\ex\label{ex:glewwe:27e}{}
 [-ʃjɛɛz-aa]\\{}
\textsc{1sg}grind\textsc{prs}\\{}
\glt ‘I am grinding’


\ex\label{ex:glewwe:27f}{}
 [a-ʃjɛɛz-aa]\\{}
\textsc{3sg}grind\textsc{prs}\\{}
\glt ‘he is grinding’
\z
\z
The verbs ‘burn’ and ‘grind’ have very similar infinitive forms, but in inflected forms they diverge. The 1\textsc{sg} present form of ‘burn’ begins with [mb], pointing to an underlying /h/, while the 1\textsc{sg} present form of ‘grind’ begins with [ʃj], the same consonants seen in the infinitive. That [koʃa] is underlyingly /kohea/ ‘\textsc{cl}15burn\textsc{fv’} is confirmed by the 3\textsc{sg} present form [ahezaa] in \REF{ex:glewwe:27c}, which exhibits on the surface the /he/ sequence that undergoes Gliding and ConsonantGlide Reduction to become [ʃ] in the infinitive. The fact that the paradigm of ‘grind’ in (27d–f) is not parallel to that of ‘burn’ in (27a–c) strongly suggests that the [ʃ] in [koʃja] does not derive from /h/. It must instead derive from /s/, which is consistent with the UR Leung gives for ‘grind.’ We note that the forms of ‘grind’ we observed always seem to show incomplete ConsonantGlide Reduction in that /s/ palatalizes to [ʃ] but the glide does not fuse with the sibilant. This appears to be part of the variation that Logoori shows in the application of its phonological processes.

\section{Nasal Deletion revisited}

The absence of the nasal prefix in 1\textsc{sg} forms like [ʃoovaa] and [ʃɔɔmaa], together with the necessary ordering of /h/Plosivization before ConsonantGlide Reduction, constitutes evidence that the initial consonant of the verb roots in these forms is not /h/. We have argued that it is in fact /s/ and drew a parallel between the absence of the nasal 1\textsc{sg} prefix in [ʃoovaa] and [ʃɔɔmaa] and its absence in [sɔmaa] ‘I am reading.’ Leung’s rule of Nasal Deletion, given in \REF{ex:glewwe:20}, is repeated in \REF{ex:glewwe:28}:

\ea{}
 \textbf{Nasal Deletion}: [+nas] $\rightarrow$  / \_\_ s\\{}
\z

This formulation only deletes nasals before [s]. If we retain this version of Nasal Deletion, it must be ordered before Coronal ConsonantGlide Reduction to derive [ʃoovaa] and [ʃɔɔmaa]. [s] must still be present in the derivation to trigger the deletion of the 1\textsc{sg} nasal prefix. This is illustrated in \REF{ex:glewwe:29} with ‘I am throwing out,’ whose UR we now give with the root as /siov/:

\ea{}
 /Nsiovaa/  \textsc{1sg}throw.out\textsc{prs}\\{}
Nsjoovaa  Gliding\\{}
sjoovaa   Nasal Deletion\\{}
ʃoovaa  Coronal ConsonantGlide Reduction\\{}
[ʃoovaa] \\{}
\glt  ‘I am throwing out’
\z

If Coronal ConsonantGlide Reduction were ordered before Nasal Deletion in its current form, it would bleed it, and the 1\textsc{sg} nasal prefix would not delete as it must. Ordering Nasal Deletion before Coronal ConsonantGlide Reduction allows us to maintain Nasal Deletion as given in \REF{ex:glewwe:28} and not expand its environment to include [ʃ] as well as [s]. 

In fact, though, expanding the environment of Nasal Deletion to include both [s] and [ʃ] actually results in a featurally simpler formulation. Compare \REF{ex:glewwe:30}, the featural equivalent of \REF{ex:glewwe:28}, and \REF{ex:glewwe:31}, whose environment covers both sibilants:

\ea{}
 \textbf{Nasal Deletion}: [+nas] $\rightarrow$  / \_\_ \\{}
\z

\ea{}
 \textbf{Nasal Deletion}: [+nas] $\rightarrow$  / \_\_ \\{}
\z



The environment that comprises [s] and [ʃ] (shown in 31) can be described with one less feature than the environment that comprises only [s] (shown in 30). Specifically, the additional feature [+anterior] is required to isolate [s] in \REF{ex:glewwe:30}. If simpler rules make an analysis more desirable, then the formulation of Nasal Deletion in \REF{ex:glewwe:31} is to be preferred. If Nasal Deletion is as in \REF{ex:glewwe:31}, then it can be ordered either before or after Coronal ConsonantGlide Reduction and still yield [ʃoovaa]. 

Notice that if Nasal Deletion deletes nasals before [ʃ] as well as [s], the deletion of the 1\textsc{sg} nasal prefix need not be limited to verb roots whose initial consonants are underlyingly /s/. Rather, the rule could delete the 1\textsc{sg} nasal prefix before verb roots whose initial consonants are underlyingly /ʃ/. Thus far, we have continued to assume that [ʃ] is always derived, either from /h/, as Leung showed, or from /s/, as we have shown. If Logoori had verb roots that began with /ʃ/ underlyingly, though, the new formulation of Nasal Deletion would ensure that they lacked the 1\textsc{sg} nasal prefix in the present tense, which would seem to be the right effect. For instance, if we proposed that the underlying form of the verb root ‘throw out’ were /ʃoov/ instead of /siov/, Nasal Deletion would correctly delete the 1\textsc{sg} prefix, yielding [ʃoovaa] ‘I am throwing out.’ In the next section, we reconsider the phonemic status of [ʃ] and present evidence that its phonemicization may be in progress.

\section{The phonemic status of [ʃ] revisited}

In the absence of surface alternations between [s] and [ʃ] in paradigms like that of ‘grind’ (27d–f), claiming that [ʃ] derives from /sV/ requires accepting that speakers store an abstract UR for which they have little to no evidence. One reason to propose that surface [ʃ] is always derived from /h/ or /s/ and never reflects an underlying phoneme /ʃ/ is simply analytical economy. We do not want to increase the size of the phoneme inventory if we are not forced to.

We have made the case for the palatalization of /s/ to [ʃ], showing that the verb root ‘grind’ whose UR Leung gave as /siɛ/ is realized as [ʃjɛɛ] in this dialect. The variation between [ʃɔɔmaa] and [sjɔɔmaa] for ‘I am wailing’ suggests that the UR of the root ‘wail’ is also /s/initial. However, it is conceivable that, without consistent alternations like those seen between [bj] and [ʃ] for /hV/ sequences, underlying /s/s that always surface as [ʃ] may restructure to /ʃ/. 

In fact, there is already evidence of restructuring in the aforementioned cases of [bj]/[ʃ] alternations. Consider the following partial paradigms for the verbs ‘haunt’ and ‘hurry’:



\ea{}
 [ko-ʃookeɾ-a]\\{}
\textsc{cl15}haunt\textsc{fv}\\{}
\glt ‘to haunt’
\z


\ea{}
 [m-bjookeɾ-aa]/[-ʃookeɾ-aa]\\{}
\textsc{1sg}haunt\textsc{prs}\\{}
\glt ‘I am haunting’
\z


\ea{}
 [o-ʃookeɾ-aa]\\{}
\textsc{2sg}haunt\textsc{prs}\\{}
\glt ‘you are haunting’
\z


\ea{}
 [a-ʃookeɾ-aa]\\{}
\textsc{3sg}haunt\textsc{prs}\\{}
\glt ‘he is haunting’
\z


\ea{}
 [ko-ʃoog-a]\\{}
\textsc{cl15}hurry\textsc{fv}\\{}
\glt ‘to hurry’
\z


\ea{}
 [m-bjoog-aa]/[-ʃoog-aa]\\{}
\textsc{1sg}hurry\textsc{prs}\\{}
\glt ‘I am hurrying’
\z


\ea{}
 [o-ʃoog-aa]\\{}
\textsc{2sg}hurry\textsc{prs}\\{}
\glt ‘you are hurrying’
\z


\ea{}
 [a-ʃoog-aa]\\{}
\textsc{3sg}hurry\textsc{prs}\\{}
\glt ‘he is hurrying’
\z

The 1\textsc{sg} present forms of ‘haunt’ and ‘hurry’ (\ref{ex:glewwe:32b} and \ref{ex:glewwe:32f}) exhibit variation: they are sometimes produced with an initial [mbj] sequence and sometimes produced with an initial [ʃ]. Like in the paradigm for the adjective ‘new,’ the alternations between [bj] and [ʃ] in \REF{ex:glewwe:32} can only be accounted for by an underlying /h/. The 1\textsc{sg} forms with [mbj] show that the verb roots ‘haunt’ and ‘hurry’ must have the URs /hiok/ and /hiog/, respectively. ‘Haunt’ and ‘hurry’ show evidence of paradigm leveling, however. The 1\textsc{sg} forms have variants with [ʃ], the consonant that appears rootinitially in the other present tense forms of the verbs (see \ref{ex:glewwe:32c}, d, g and h). Moreover, these [ʃ]initial variants seem to be preferred; we observed more tokens of them than of the [mbj]initial variants. When [ʃookeɾaa] and [ʃoogaa] are the 1\textsc{sg} forms, the paradigms are free of alternations. There is also no more evidence for an underlying /h/, so the verb roots have presumably been restructured. 

One possibility is that their new underlying forms are respectively /siok/ and /siog/. The Coronal ConsonantGlide Reduction rule we put forth would apply to these forms, yielding surface paradigms in which the verb roots always began with [ʃ]. Some evidence in favor of an underlying /sV/ sequence, at least for ‘haunt,’ comes from the fact that we also observed the infinitive [koʃookeɾa] ‘to haunt’ produced as [koʃjookeɾa] and [kosjookeɾa]. The presence of the glide suggests an additional underlying vowel in the root, and the [s] in [kosjookeɾa] suggests the underlying consonant is /s/, not /ʃ/. We saw the same types of variation in [ʃoovaa]/[ʃjoovaa] ‘I am throwing out’ and [ʃɔɔmaa]/[sjɔɔmaa] ‘I am wailing’. 

A second possibility is that the underlying forms of the verb roots ‘haunt’ and ‘hurry’ are restructuring to /ʃook/ and /ʃoog/, respectively. If [ʃ] does not alternate with [bj] in the paradigm and if [ʃ] does not vary with [s], there is no reason for speakers to store a UR that begins with anything other than /ʃ/. If this type of restructuring appears to be occurring for /h/initial roots like /hiok/ ‘haunt’ and /hiog/ ‘hurry,’ it seems likely that it could also be occurring for /s/initial roots like /siɛ/ ‘grind,’ /siɔm/ ‘wail,’ and /siov/ ‘throw out’. Speakers for whom this restructuring is complete must have /ʃ/ as a phoneme. 

According to the analysis of Logoori we presented in the previous sections, [ʃ] is always derived by Palatalization of Back Consonants, ConsonantGlide Reduction, or Coronal ConsonantGlide Reduction. Palatalization of Back Consonants only palatalizes /h/ to [ʃ] before the vowel [i], so in all other cases where [ʃ] appears, it must be derived through a process that is fed by Gliding. As a result, all instances of [ʃ] that appear before a vowel other than [i] should be followed by a long vowel. If a systematic study measuring the duration of vowels following [ʃ] were to uncover that [ʃ] can be followed by short vowels other than [i], we would have to conclude that [ʃ] has phonemicized. Just as the existence of short vowels after the postalveolar affricates [t͡ʃ] and [d͡ʒ] shows that they are independent phonemes of the language, the existence of short vowels besides [i] after [ʃ] would show that it is also an independent phoneme. Such a vowel duration study is a task for future research.

\section{Conclusion}

Logoori exhibits a range of palatalization processes, including Palatalization of Dental Consonants (j̪, n̪), Palatalization of Back Consonants (/h/ $\rightarrow$ [ʃ]), and Consonant-Glide Reduction (\{[kj], [gj], [hj]\} $\rightarrow$ \{[t͡ʃ], [d͡ʒ], [ʃ]\}). While [t͡ʃ] and [d͡ʒ] are independent phonemes as well as allophones of /k/and /g/, Leung’s (1991) phonological analysis considered [ʃ] to be only an allophone of /h/. Another allophone of /h/, [b], is derived by postnasal /h/Plosivization. New data revealed a rule ordering paradox involving /h/Plosivization and ConsonantGlide Reduction. Forms like /eNhia/ required /h/Plosivization to precede ConsonantGlide Reduction to yield [embja] while forms like /N-hiov-aa/ required ConsonantGlide Reduction to precede /h/Plosivization to yield [ʃoovaa]. To resolve this paradox, we argued that certain presumed underlying /h/s were actually /s/s and proposed a new rule, Coronal ConsonantGlide Reduction ([sj] $\rightarrow$ [ʃ]), thereby demonstrating that coronal palatalization is more widespread in Logoori than previously recognized. Additionally, the phonemic status of [ʃ] in this dialect of Logoori appears to be in flux. In certain cases, underlying /h/ and /s/ may be restructuring to /ʃ/. While [ʃ] was not considered a phoneme of Logoori in the past, it seems on its way to becoming one.

\section{Acknowledgements} We gratefully acknowledge our consultant, Mwabeni Indire. Many thanks to Hannah Sarvasy, Kie Zuraw, Jesse Zymet, and audiences at the UCLA Phonology Seminar and ACAL 46 for their contributions and helpful feedback. We also thank the editors and two anonymous reviewers for their comments.

\section*{Abbreviations}

\textsc{1}    first person

\textsc{2}    second person

\textsc{3}    third person

\textsc{cl}    noun class

\textsc{fv}    final vowel

\textsc{obj}    object

\textsc{pl}    plural

\textsc{prs}    present

\textsc{hod.pfv}  hodiernal perfective

\textsc{refl}    reflexive

\textsc{sg}    singular
  

\begin{verbatim}%%move bib entries to  localbibliography.bib


\end{verbatim}
 

\printbibliography[heading=subbibliography,notkeyword=this]

\end{document}