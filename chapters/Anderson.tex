\documentclass[output=paper]{langsci/langscibook} 
\title{STAMP morphs in the Macro-Sudan Belt} 
\author{%
Gregory D. S. Anderson
\affiliation{Living Tongues Institute for Endangered Languages \& University of South Africa (UNISA)} 
}
% \chapterDOI{} %will be filled in at production


\abstract{
STAMP morphs $-$portmanteau subject-tense-aspect-mood-polarity morphs that exhibit functional and formal properties of both pronominals and auxiliary verbs– are characteristic of many of the genetic units across the Macro-Sudan Belt. STAMP morphs typically occur within a constructional frame, which may include a verb in a fully unmarked form or various construction-specific finite or non-finite forms. STAMP morphs often originate in, and synchronically may appear in, auxiliary verb constructions. Further, univerbation of STAMP morphs with following verbs has yielded various series of inflectional prefixes in a range of families and individual languages. After introducing the functional and formal/constructional properties of STAMP morphs, the paper discusses the origin of STAMP morph constructions and their subsequent developments into bound inflectional or conjugational prefix series. The paper closes by presenting a typology of STAMP morphs in three important Macro-Sudan Belt macro-groupings, viz. Chadic, Central Sudanic, and Niger-Congo languages; and for the last mentioned, particularly the Benue-Congo taxon. 
}

\maketitle
\begin{document}


\section{Introduction}\label{sec:Anderson:1}

In this article I introduce and exemplify a curious and characteristic feature of a number of languages found across different genetic units spanning the large areal complex of equatorial Africa called the Macro-Sudan Belt \citep{Güldemann2008}. This feature has challenged analysts pursuing descriptions of various western and central African languages, exhibiting functional properties typically associated with both subject pronouns and auxiliary verbs in many other languages; it has recently been called a STAMP morph \citep{Anderson2012, Anderson2015}. This is mnemonic for what these elements largely are, portmanteau morphs that encode the referent properties of semantic arguments that typically play the syntactic role of `S[ubject]'–that is, the person, number and gender properties of such an actant–in combination with categories of T[ense], A[spect], M[ood] and P[olarity]. Such elements have also been previously called the \textit{tense-person complex} \citep{Creissels2005}, and \textit{pronominal predicative markers} or \textit{pronominal auxiliaries} \citep{Vydrin2011,Vydrin:2006,Erman2002} in the Africanist literature. 

  Portmanteau STAMP morphs are found throughout languages representing different genetic units of the Macro-Sudan Belt. In their most basic form, they are the sole means of encoding referent properties of the syntactic subject, in addition to most of the types of categories encoded by verbal Tense, Aspect and \textit{Aktisonsart}, Mood and Polarity morphology. An example of the simplest STAMP morph construction can be seen in \REF{ex:anderson:1} and \REF{ex:anderson:2} from Tarok, a language of Nigeria belonging to the Tarokoid Plateau genetic unit (Benue-Congo stock, Niger-Congo phylum).
  
\ea\label{ex:anderson:1}
Tarok \citep[238]{Sibomana1981}        [\textsc{Tarokoid/Plateau}]\\
\gll \textbf{n}        wá     ù-dɨŋ    \\
\textsc{1.pfv}  drink \textsc{clsfr}-water\\
\glt 'I have drunk the water.'      
\z

\ea\label{ex:anderson:2}
Tarok   \\
\gll \textbf{mi}    wá  a-tí      ipín  \\
1\textsc{.irr}  drink  \textsc{clsfr}-tea tomorrow \\
\glt 'I will drink tea tomorrow.'
\z

In this study, I discuss some of the characteristics of STAMP morphs and STAMP morph constructions in a set of languages and genetic units of the Macro-Sudan Belt. \sectref{sec:Anderson:2} discusses the genetic units that constitute the typological sampling of African languages used in this study. \sectref{sec:Anderson:3} discusses the range of functions associated with STAMP morph constructions, and \sectref{sec:Anderson:4} explores their formal constructional features. \sectref{sec:Anderson:5} offers some thoughts on the origins of STAMP morphs.  \sectref{sec:Anderson:6} discusses the subsequent role such formations have played in the development of prefixed conjugation series that are likewise found across the different genetic units of this linguistic area. Sections 7-9 briefly examine the role of STAMP morph constructions and prefix conjugations that developed out of erstwhile STAMP morphs in three macro-level groupings of genetic units (stocks) in the Macro-Sudan Belt. This includes the Chadic stock (\sectref{sec:Anderson:7}), the Central Sudanic stock of Nilo-Saharan (\sectref{sec:Anderson:8}), and the various genetic units found within the Benue-Congo stock of Niger-Congo (\sectref{sec:Anderson:9}).

\section{Typological Sampling and the Macro-Sudan Belt}\label{sec:Anderson:2}

In his seminal work on the Macro-Sudan Belt [MSB], \cite{Güldemann2008} remained agnostic about the usefulness of higher order units in Africa, at least with respect to typological sampling. Instead he suggested the fourteen genera in \tabref{tab:Anderson:1}, with Ijoid, Dogon, Chadic and Songhay being peripheral to the proposed area insofar as they show fewer of the characteristic core-features identified as definitional of the MSB, than do the majority of other identified genera.

\begin{table}
\caption{Genera in the Macro-Sudan Belt according to \cite{Güldemann2008}} 
\label{tab:Anderson:1} 
\begin{tabularx}{\textwidth}{XllX}
\lsptoprule
 Adamawa &  non-Bantu Benue Congo &  Bongo-Bagirmi &  Dogon\\
\tablevspace
\tablevspace
 Kwa &  Kru &  Gur &  Songhay\\
\tablevspace
 Mande &  Moru-Mangbetu &  Ubangian &  Ijoid\\
\tablevspace
 Chadic &  Atlantic &  & \\
\lspbottomrule
\end{tabularx}
\end{table} 

Since 2008, a number of revisions to standard classifications of African languages have been offered, including statements by \cite{Dimmendaal2008, Dimmendaal2011} and \citet{Sands2009}. In previous work on African linguistic typology, I have suggested a larger range of sampling units as appropriate for cross-linguistic work, which have internal diversification on the level of Germanic or Romance; I call these `genetic units for typological sampling' \citep{Anderson2011}, and write them in small-caps to distinguish them from names of both languages and larger taxa. Many genera in \tabref{tab:Anderson:1} would thus be split into multiple different genetic units; in this system, the number of such genetic units in the Macro-Sudan Belt to use in typological sampling totals over fifty. Not all these genetic units have attested STAMP morph constructions, but the vast majority does. In my corpus the genetic units that include STAMP morph constructions (or prefix conjugation series that historically derive from such constructions, see section 5) include at least those in \tabref{tab:Anderson:2}.

\begin{table} 
\caption{Genetic units w/STAMP morphs in MSB \citep{Anderson2011, Dimmendaal2008, Dimmendaal2011, Sands2009}}
\label{tab:Anderson:2}
\centering
\textit{Atlantic-Congo families, isolates and stocks}\\
\begin{tabular}{llll}
\hline
\textsc{Bantoid} & \textsc{Gbe} & \textsc{Kulango} & \textsc{Potou-Tano}\\
\textsc{Bendi} & \textsc{Gur} & \textsc{Leko-Nimbari} & \textsc{Senufic}\\
\textsc{Cangin} & \textsc{Igboid} & \textsc{Mbum-Day} & \textsc{Tarokoid}\\
\textsc{Cross River} & \textsc{Jen Bambukic} & \textsc{Na-Togo} & \textsc{Tenda}\\
\textsc{Edoid} & \textsc{Ka-Togo} & \textsc{Nupoid} & \textsc{Ukaan}\\
\textsc{Ega} & \textsc{Kainji} & \textsc{Okoid} & \textsc{Waja-Kam}\\
\textsc{Fali} & \textsc{Kru} & \textsc{Plateau} & \textsc{Wolof} \\
\textsc{Ga-Dangme} &  & & \\
\end{tabular}

\textit{Chadic sub-families}\\
\begin{tabular}{llll}
\textsc{Biu-Mandara (Central)} & \textsc{East Chadic} & \textsc{Masa} & \textsc{West Chadic}\\
\end{tabular}

\bigskip

\textit{Central Sudanic families}\\
\begin{tabular}{lll}
\textsc{Kresh} & \textsc{Lendu} & \textsc{Mangbetu}\\
\textsc{Mangbutu-Lese/Efe} & \textsc{Moru-Ma'di} & \textsc{Sara-Bongo-Bagirmi}\\
\end{tabular}

\bigskip


\textit{Mande families}\\
\begin{tabular}{llll}
\textsc{East Mande} &  \textsc{Northwest Mande} & \textsc{Southeast Mande} & \textsc{Southwest Mande}
\end{tabular}

\bigskip


\textit{Ubangian families}\\
\begin{tabular}{llll}
\textsc{Gbaya} & \textsc{Ngbandi} & \textsc{Mba} & \textsc{Zande}(+traces in \textsc{Ngbaka, Banda)}\\
\end{tabular}
\end{table}

\section{Functions of STAMP morphs}\label{sec:Anderson:3}

STAMP morphs usually combine with a following verb to encode the TAM categories of the event, and the person, number and, where relevant, gender categories of the subject. Such forms can be found in as diverse array of languages of the Macro-Sudan Belt as the \textsc{Biu-Mandara (Central) Chadic} language Merey \REF{ex:anderson:3} of Cameroon, the \textsc{Kru} language Wobé \REF{ex:anderson:4} of Côte d'Ivoire, and Bongo of the \textsc{Sara-Bongo-Bagirmi} family of Central Sudanic \REF{ex:anderson:5} spoken in South Sudan.

\ea\label{ex:anderson:3}
Merey  \citep[8]{Gravina2007}                  \textsc{[Biu-Mandara Chadic]}\\
\gll na  zal    \\
\textsc{1.pst}  call   \\
\glt `I called.'  
\z


\ea\label{ex:anderson:4}
\ea\label{ex:anderson:4a}
Wobé \citep[Wobé 3]{HoferLink:1973}\footnote{Hofer \& Link's volume does not number the chapters consecutively, but rather each chapter begins anew at page 1. Thus this form comes from page 3 of the Wobé chapter. Their page-numbering convention is followed here.}    \textsc{[Kru]}\\
\gll ẽ\textsuperscript{ 2}    gyi\textsuperscript{32}            \\
1.\textsc{pst}  come  \\
\glt `I have come.'            
\ex \label{ex:anderson:4b}
Wobé\\
\gll  ma\textsuperscript{2}  gyi\textsuperscript{32}\\
1.\textsc{npst}  come\\
\glt `I am coming.'  
\z
\z

\ea\label{ex:anderson:5}
Bongo   \citep[75]{Tucker&Bryan1966}       [\textsc{Sara-Bongo-Bagirmi}]\\
\gll ma    bi\\
1.indef  give  \\
\glt `I am giving.'
\z

One of the curious and characteristic features of STAMP morph constructions is the possible encoding of negative polarity without any distinct negative polarity scope operator, the negative polarity being encoded together with TAM categories and referent properties of the subject in the STAMP morph itself. Formations of this type can be found in languages of the Macro-Sudan Belt such as Duka (\textsc{Kainji}) \REF{ex:anderson:6} or Mano (\textsc{Southeast Mande}) \REF{ex:anderson:7}. 

\ea\label{ex:anderson:6}
Duka  \citep[13]{BendorSamuel1973}      [\textsc{Kainji}]\\
\gll mân  hé  ò-kɔt  \\
\textsc{I.fut.neg}  go  to-bush\\
\glt `I won't go to the bush.'      
\z

\ea\label{ex:anderson:7}
Mano \citep[226]{Vydrine2009}           [\textsc{Southeast Mande}]\\
\gll lɛɛ      máá    wè    gè\={e}\\
\textsc{3sg.neg[hab]}  Mano    language  speak\\
\glt `S/he doesn't speak Mano.'
\z

As noted earlier, the referent categories expressed in STAMP morphs are typically restricted to subjects, but in a small number of instances one also finds portmanteau subject > object forms within a STAMP morph, as in the following sentence from Kohumono (\textsc{Cross River}).

\ea\label{ex:anderson:8}
 Kohumono \citep[355]{Cook1972}         [\textsc{Cross River}]\\
\gll $\beta ɔ$    fà\\
1>2.\textsc{npst}  bite\\
\glt `I bite you.'
\z

In Guro (\textsc{Southeast Mande}), one finds portmanteau subject > object-encoding STAMP morphs that also reference polarity, as in \textit{ɓ}\textit{e} vs. \textit{yaa} in the following sentences (note the tone differences on the verbs in these constructions as well as the presence of the postverbal negator \textit{ɗ}\textit{o} in \REF{ex:anderson:9b}). 

\ea\label{ex:anderson:9}
\ea\label{ex:anderson:9a}
Guro \citep[239]{Vydrine2009}       [\textsc{Southeast} \textsc{Mande}]\\
\gll  ɓe      zuru-o      \\
  \textsc{2sg>3sg.ipfv}    wash-\textsc{ipfv} \\
\glt `[You] wash him/her/it.'       

\ex \label{ex:anderson:9b}
\gll  yaa      zùrù-ò    ɗo\\
  \textsc{2sg>3sg.ipfv.neg}  wash-\textsc{ipfv}  \textsc{neg}\\
\glt `[You] don't wash him/her/it.'
\z
\z

\sectref{sec:Anderson:4} now turns to the range of constructional types in which STAMP morphs participate.


\section{Formal subtypes of STAMP morph constructions}\label{sec:Anderson:4}


STAMP morph constructions exhibit a wide-range of formal sub-types. As mentioned previously, the simplest construction consists of a STAMP morph and an unmarked or bare stem form of the verb \REF{ex:anderson:10}.

\ea\label{ex:anderson:10}
Bare Stem Construction:  STAMP  Verb
\z

Some examples of this constructional type of STAMP morph may be found in Fyem \REF{ex:anderson:11}. Wolof also has an extensive system of such STAMP morph constructions, as do a number of Mande languages. Upwards of ten separate paradigms may be found.

\ea\label{ex:anderson:11}
\ea\label{ex:anderson:11a}
Fyem \citep[32, 35]{Nettle1998}          \textsc{[Plateau]}\\
\gll   náá  soo  Gindiríŋ    \\
\textsc{I.prf}  go  Gindiri  \\
\glt `I went to Gindiri.'   
\ex \label{ex:anderson:11b}
\gll ín  soo  dirámméka\\
\textsc{I.ipfv}  go  farm.your.\textsc{oblq}\\
\glt `I will go to your farm.'
\z
\z

In a second STAMP morph construction the verb appears in a construction-specific tonal form, suggesting that there is an associated floating tone projecting from the STAMP morph onto the verb stem.

\ea\label{ex:anderson:12}
 STAMP + Floating Tone Construction:   STAMP Verb\textsubscript{<construction-specific tonal form>}  
 \z

West Chadic Guus (also known as Sigidi) also has a highly elaborated system of STAMP morphs (\tabref{tab:Anderson:3}), some of which have associated floating tones that they project onto the verb. The Recent Past (\textsc{rec)} form projects a high tone onto the following verb. The Future (\textsc{fut)} does the same, and differs from the otherwise identical-looking subjunctive [\textsc{sbjn]} in whether it assigns this floating high tone (\textsc{fut}), or does not (\textsc{sbjn}); the two are thus \textit{constructionally} distinct.

%%please move \begin{table} just above \begin{tabular
\begin{table}
\caption{STAMP morphs in Guus [Sigidi], West Chadic \citep[8-9]{Caron2001}}
\label{tab:Anderson:3}

\begin{tabularx}{\textwidth}{XXXXXXXXX} & \scshape aor & \scshape imm & \scshape sbjn & \scshape fut & \scshape hab & \scshape pfv & \scshape rec & \scshape irr\\
\lsptoprule
1 & \itshape ma & \itshape maa & \textit{m}ə & \textit{m}ə\textit{\textsuperscript{+H}} & \itshape mak & \itshape map & \textit{mam}\textit{\textsuperscript{+}}\textsuperscript{H} & \textit{m}ə\textit{s}\\
2 & \itshape ka & \itshape kaa & \textit{k}ə & \textit{k}ə\textit{\textsuperscript{+H}} & \itshape kak & \itshape kap & \textit{kam}\textit{\textsuperscript{+}}\textsuperscript{H} & \textit{k}ə\textit{s}\\
\lspbottomrule
\end{tabularx}
\end{table}

\ea\label{ex:anderson:13}
Guus [Sigidi]  \citep[11]{Caron2001}          \textsc{[West Chadic]}\\
\gll ən  ka    ɗuu  karáŋ    tʃí    məʃi\\
if  you\textsc{.irr}  beat  dog    s/he\textsc{.fut}  die\\
\glt `If you beat the dog, it will die.'   
\z

The divergent Atlantic-Congo (or \textsc{Gur)} isolate language Kulango is another language of the Macro-Sudan Belt that reflects floating tones associated with STAMP morphs. In Kulango \REF{ex:anderson:14} the first singular habitual STAMP morph projects a low tone onto the first stem syllable of the verb, while the corresponding subjunctive form projects a high tone. Both STAMP morphs bear high tones themselves, but one projects a floating low tone and the other a floating high tone.

\ea\label{ex:anderson:14}
Kulango \citep[193]{Elders2007}          [K\textsc{ulango}]\\
\ea\label{ex:anderson:14a}
\gll má     dɔlɪ  \\
  \textsc{I.hab}\textsubscript{<+L>}  sell\\        
\glt `I sell.'            

\ex \label{ex:anderson:14b}
\gll mɪ    dɔlɪ\\
  1.\textsc{sbjn}\textsubscript{<+H>}  sell\\
\glt 'May I sell.'
\z
\z

The \textsc{Waja-Kam} language Dadiya shows similar constructions.

\ea\label{ex:anderson:15}
Dadiya   \citep[196]{Jungraithmayr1968}       [\textsc{Waja-Kam]}  \\
\gll \'{n}    já    \\
you\textsc{.prf}  eat:\textsc{prf}    \\
\glt `You have eaten.'  
\z

In a third sub-type of STAMP morph construction, the verb appears in a construction-specific co-grammaticalized aspectual form \REF{ex:anderson:16}. 

\ea\label{ex:anderson:16}
Aspect Construction:    STAMP  Verb\textsubscript{<\textsc{asp}>}    
\z

Such STAMP morph constructions are found in Tiv (\textsc{Bantoid}) \REF{ex:anderson:17}, and Ndut-Falor (\textsc{Cangin}) \REF{ex:anderson:18}.

\ea\label{ex:anderson:17}
\ea\label{ex:anderson:17a}
Tiv \citep[114]{Arnott1958}          \textsc{[Bantoid/Tivoid]}\\
\gll vé     pìnè        \\
  they\textsc{.pst}   ask          \textsc{}\\
\glt `They asked.'          

\ex \label{ex:anderson:17b}
\gll mbá       pi\textsuperscript{{\textbar}}ne-n  \\
  they.\textsc{prs}    ask-\textsc{asp}\\
\glt `They are asking.'      
\z
\z

\ea\label{ex:anderson:18}
Ndut-Falor \citep[Ndut-Falor 4]{Pichl1973/1980}      [\textsc{Cangin}]\\
\gll mi    acɛ          \\
I.\textsc{rls}  come:\textsc{prf}\\
\glt `I have come.'  
\z

A sub-type of this construction is found with the verb stem having a as morphological aspect marker as well being reduplicated \REF{ex:anderson:19}. 

\ea\label{ex:anderson:19}
   STAMP  \textsc{redpl}-Verb-\textsc{asp} 
\z

This is found in the following formation from Dadiya of the \textsc{Waja-Kam} genetic unit \REF{ex:anderson:20}:

\ea\label{ex:anderson:20}
Dadiya   \citep[197]{Jungraithmayr1968}        [\textsc{Waja-Kam}]\\
\ea\label{ex:anderson:20a}
\gll mən  nò-lɛ\\
  I.\textsc{npst}  drink-\textsc{prog}        \\
\glt `I am drinking.'         

\ex \label{ex:anderson:20b}
\gll  mon    jà-jà-l \\
  you.\textsc{npst}  \textsc{rdpl}-eat-\textsc{prog}\\
\glt `You are eating.'
\z
\z 

In some African languages this construction-determined aspectual form is itself realized tonally \REF{ex:anderson:21}.

\ea\label{ex:anderson:21}
  Tonal Morphology  STAMP   Verb\textsc{\textsubscript{<Asp=Tone>}}
\z

An example of such comes from `Bozom, a \textsc{Gbaya Ubangi} language \REF{ex:anderson:22}, in which are the \textsc{ipfv} and \textsc{pfv} verb forms are distinguished tonally.

\ea\label{ex:anderson:22}
`Bozom   \citep[159]{Monino1995}        [\textsc{Gbaya Ubangi}]\\
\ea\label{ex:anderson:22a}
\gll ʔà̰    ré  \\
  he\textsc{.rls}     enter.\textsc{ipfv}\\
\glt `He enters.' 

\ex \label{ex:anderson:22b}
\gll  má  rè \\
  he\textsc{.irr}  enter\textsc{.pfv}\\
\glt `He will enter.'
\z
\z

Verbs in STAMP morph constructions can also appear in specific construction-dependent modal forms \REF{ex:anderson:23}. 


\ea\label{ex:anderson:23}
Modal Construction  STAMP  Verb\textsubscript{<}\textsc{\textsubscript{modal}}\textsubscript{>}\\
\z

Examples of this type can be found in Ndut-Falor \REF{ex:anderson:24} and Duka \REF{ex:anderson:25}.

\ea\label{ex:anderson:24}
Ndut-Falor \citep[Ndut-Falor 4]{Pichl1973/1980}      [\textsc{Cangin}]\\
\gll ma[y]  ayɛ\\
I.\textsc{fut}  come:\textsc{mod}  \\
\glt `I will come.' 
\z

\ea\label{ex:anderson:25}
 Duka  \citep[96-98; 105]{Bendor-Samuel1973}    \textsc{[Kainji]}\\
 \gll mɛ    heɛ   \\
I.\textsc{irr}  go:\textsc{cond}  \\
\glt `If I go...'  
\z

Split negative marking is relatively common in AVCs (Anderson 2006; 2011) where the negative is encoded on a lexical verb but other obligatory categories are on the auxiliary. In Africa, this pattern is a feature of auxiliary verb constructions in the languages of the \textsc{Rashad Kordofanian} genetic unit, e.g. in Rashad proper \REF{ex:anderson:26}.

\ea\label{ex:anderson:26}
Rashad \citep[297]{Tucker&Bryan1966}      [\textsc{Rashad Kordofanian}]\\
\gll ŋi   fas    k-eyɛ    y-ɛn\\
I  meat    \textsc{neg}-eat  1-\textsc{aux}\\
\glt `I am not eating meat.'
\z

Thus, it is not overly surprising that negative can also be encoded in STAMP morph constructions as well \REF{ex:anderson:27}, albeit reflecting different syntactic configurations in the languages of the Macro-Sudan Belt (the lexical verb follows the functional element) than in the Nuba Hills languages (where the reverse tends to be true).

\ea\label{ex:anderson:27}
Split Negative Construction:  \textsc{STAMP  V}erb\textsc{\textsubscript{<neg>}}
\z

Such a formation in the negative future in Gã of the \textsc{Ga-Dangme Kwa} genetic unit \REF{ex:anderson:28}.\footnote{Negative can be encoded in the STAMP morph itself as well. When so, the verb is thus in a `co-negative' form.}

\ea\label{ex:anderson:28}
Ga \citep[105]{Kropp-Dakubu1988}        [\textsc{Ga-Dangme Kwa}]\\
\gll e\textsuperscript{{\textbar}}    bá\textsuperscript{!}-\'{ŋ}\\
he\textsc{.fut}  come-\textsc{neg}\\
\glt `He will not come.'
\z

The other common split inflectional pattern found cross-linguistically, but one that is not overly common in the languages of Africa and particularly not in those of the Macro-Sudan Belt \citep{Anderson2011}, is a subject-object split pattern \citep{Anderson2006}. In the split subject-object pattern, the object is encoded on the lexical verb that subcategorizes for it, and the other obligatory inflectional categories including subject information appear on the auxiliary as in the auxiliary verb construction in \REF{ex:anderson:29} from Gidar.

\ea\label{ex:anderson:29}
Gidar \citep[263]{Frajzyngier2008}        \textsc{[Biu-Mandara Chadic]}\\
\gll wá-nə    mpər-kó\\
\textsc{fut}-1    chew-2\\
\glt `I will eat you.'
\z

This pattern also appears to be reflected in STAMP morph constructions as well \REF{ex:anderson:30}. 

\ea\label{ex:anderson:30}
Split Object Construction:   STAMP  Verb\textsc{\textsubscript{<obj>}}
\z

Such a formation may be found in a small number of Chadic languages, for example, \textsc{West Chadic} Polci \REF{ex:anderson:31} or \textsc{Biu-Mandara (Central) Chadic} Mofu-Gudur \REF{ex:anderson:31}. 

\ea\label{ex:anderson:31}
Polci \citep[153]{Caron2008}          \textsc{[West Chadic]}\\
\gll Gǎrbà   kən  ndʒaŋ  sloː  wú  ɗe  kə   fǔː-m \\
Garba  \textsc{cop}  cut  meat  \textsc{acc}  \textsc{inj}  \textsc{2:aor}  say-1  \\
\glt `If Garba slaughters a beast, tell me'
\z

\ea\label{ex:anderson:32}
Mofu-Gudur \citep[4]{Pohlig1992}          \textsc{[Biu-Mandara Chadic]}\\
\gll fá    tá-ka    ɗáf      \\
\textsc{prog.3}  prepare-2\textsc{.io}  food    \\
\glt `She is preparing you food.'
\z

\section{Origin of STAMP morphs from auxiliary constructions}\label{sec:Anderson:5}

Where might STAMP morphs have come from; in other words, how did they develop? In known instances, they result from the fusing of a pronominal marker or pronoun with a following auxiliary verb, fused into a single portmanteau complex \citep{Anderson2006, Anderson2011}.\footnote{There are also instances where STAMP morphs have no obvious etymology, as in many Mande languages.} An example showing synchronic variation between a STAMP morph and its source auxiliary verb construction comes from Limbum \REF{ex:anderson:33}.

\ea\label{ex:anderson:33}
Limbum \citep[Limbum 3-4]{Fiore&Peck1973/1980}      \textsc{[Bantoid]}\\
\gll mɛ:\textsuperscript{32}  wɨ\textsuperscript{3}   {\textasciitilde}   mɛ\textsuperscript{3}  ʃe\textsuperscript{2}  wɨ\textsuperscript{3 }   mɛ:\textsuperscript{32} < mɛ\textsuperscript{3}=ʃe\textsuperscript{2}\\
\textsc{1.prog}  come  ~  1  \textsc{prog}  come\\
\glt `I am coming.'
\z

Researchers on individual languages and families have recognized this connection. Thus \citet[101]{Shimizu1983}, when describing STAMP morphs in the Leko-Nimbari language Zing Mumuye states that ``the surface differences in subject pronouns are in fact due to the TAM markers, which are realized on them or contracted with them.'' Further, \citet[35]{Babaev2010}, in discussing issues in the reconstruction of Benue-Congo cautions the reader that ``… various phonological processes of merging person markers with predicative markers of tense, aspect, modality and polarity have made the situation in many languages obscure.''

  Language-specific internal evidence sheds light on this most likely source for the development of STAMP morphs. In Sara of the \textsc{Sara-Bongo-Bagirmi} family of the Central Sudanic stock, the so-called indefinite series subject pronouns are not used with finite, inflected forms of a following vowel-initial verb, but rather with an infinitive form.
  
  \ea\label{ex:anderson:34}
  Sara \citep[75]{Tucker&Bryan1966}      [\textsc{Sara-Bongo-Bagirmi}]\\
\gll ma    k-usa\\
1:\textsc{def}  \textsc{inf}-eat\\
\glt `I am eating.', `I eat (\textsc{hab}).' 
\z

Though a finite construction that seemingly consists of a combination of a subject pronoun  plus an infinitive verb form is odd typologically, such a formation is entirely consistent with typological norms if the subject pronoun actually has an auxiliary verb `hidden' within it, or in other words is actually a STAMP morph. Sara and its sister languages (see \sectref{sec:Anderson:8} below) are far from the only languages that show constructions with infinitive verb forms in combination with specific STAMP morphs. The entirely unrelated \textsc{Bantoid} language Nomaande shows similar structures. 

\ea\label{ex:anderson:35}
Nomaande \citep[22]{Wilkendorff2001}          \textsc{[Bantoid]}\\
\gll yɔ    ɔcɔba  ná-áyá\\
3.\textsc{prs}  \textsc{inf}:go  to-\textsc{3sg}\\
\glt `He is going to him (i.e., his home).' 
\z

Similarly, in Dott (also known as Dass or Zoɗi) of the \textsc{West Chadic} family \REF{ex:anderson:36}, the continuous or progressive STAMP morph series requires a verbal noun form of the lexical verb in the predicate. 

\ea\label{ex:anderson:36}
Dott/Zoɗi \citep[165]{Caron2002}            [W\textsc{est} \textsc{Chadic}]\\
\gll taa    tʃét-ti\\
\textsc{3pl.cont}  come-\textsc{vn}\\
\glt `They are coming.'
\z

Yet other languages of the Macro-Sudan Belt require dependent-marked forms of verbs to appear in combination with STAMP morphs in specific constructional subtypes. Such languages include the \textsc{Kainji} language Duka \REF{ex:anderson:37} and 'Bozom \REF{ex:anderson:38} of the \textsc{Gbaya Ubangi} family. 

\ea\label{ex:anderson:37}
Duka  \citep[96-98; 105]{Bendor-Samuel1973}     \textsc{[Kainji]}\\
\gll mɛ     əm-hà     á\\
I.\textsc{irr}  \textsc{dep}-go   \textsc{neg}  \\
\glt `I am not going.' 
\z

\ea\label{ex:anderson:38}
`Bozom \citep[159]{Monino:1995}            \textsc{[Gbaya Ubangi]}\\
\gll ʔà̰    rè-á      \\
he.\textsc{rls}  enter.\textsc{prf}-\textsc{dep} \\
\glt `He has entered.'
\z

\section{STAMP morphs developing into prefixal conjugations}\label{sec:Anderson:6}

%%start here

Having established that STAMP morph constructions often derive from AVCs, the question arises as to where these formations ``go'' in their subsequent historical developments. In ``normal'' AVCs, an auxiliary is often drawn into the verbal complex, and loses its phonological integrity as a freestanding element. Similar developments are seen with STAMP morph constructions. In the Macro-Sudan Belt one typically finds that old STAMP morphs have gotten drawn into the verb to become prefixal conjugation markers. This can be seen in cognate formations in Nchumuru varieties. Bejamso-Grubi Nchumuru \REF{ex:anderson:39} has freestanding STAMP morphs, while in closely related Banda Nchumuru \REF{ex:anderson:40}, these have become a prefixed conjugation set.

\ea\label{ex:anderson:39}
Bejamso-Grubi Nchumuru  \citep[Nchumuru 5]{Price1975/1980}   [\textsc{Potou-Tano} K\textsc{wa}]\\
\gll màá  bà\\
I.\textsc{prf}  come\\
\glt `I have come.'
\z

\ea\label{ex:anderson:40}
Banda Nchumuru \citep[Nuchumuru 4]{Cleal1973/1980}      [\textsc{Potou-Tano} K\textsc{wa}]\\
\gll mà-ba\\
\textsc{1pst}:\textsc{prf}-come\\
\glt `I have come.'
\z

Such developments have been known in African linguistics for more than a century.\footnote{\citet[211]{Seidel1898} pointed out that in Kwa languages of Togo ``die verbalen Präfixe, wahrscheinlich Reste ehemahliger Hilfsverben, verschmelzen nicht selten mit den Pronominalpräfixen zu einer Silbe''. What the status of STAMP morphs might be prosodically in a given language in this region bears close examination, as they may turn out to be clitics or prefixes. To be sure, \citet[93]{CreisselsEtAl2008} caution that ``many descriptions of African languages do not identify pronominal markers appropriately, treating them as independent words''. Such is definitely the case in the East Mande Boko/Busa cluster \citep{Jones1998}, where the orthography treats STAMP morphs as freestanding elements but phonologically they are prefixes. As a whole, one can find certain variation between Anglophone vs. Francophone meta-analytic traditions in the Africanist literature, with Anglophone analysts preferring more prefixes in their descriptions while Francophone scholars rather often favor analyses with pre-verbal free-standing elements. This is not an absolute.} 

I now turn to brief examination of STAMP morph constructions and historically-related prefixal conjugations in Chadic, Central Sudanic, and certain Niger-Congo families.

\section{STAMP morphs and prefix conjugations in Chadic languages}\label{sec:Anderson:7}

This section briefly presents STAMP morphs in three of the representative stocks found in the Macro-Sudan Belt (each consisting of several genetic units for typological sampling). STAMP morphs are a characteristic of the entire Chadic macro-family.\footnote{It is important to not confuse STAMP morphs with a series of intransitive copy pronouns that are also characteristic of Chadic \citep{Frajzyngier1977, Burquest1986}. They are unrelated phenomena.} Simplex formations of the most basic type \REF{ex:anderson:10} are characteristic of \textsc{West Chadic} languages like Angas \REF{ex:anderson:41} and Gerka \REF{ex:anderson:42}.

\ea\label{ex:anderson:41}
Angas  \citep[38]{Burquest1973}         [\textsc{West Chadic]}\\
\gll ŋâː    jì\\
1.\textsc{compl}  come\\
\glt  `I have come.'
\z

\ea\label{ex:anderson:42}
Gerka  \citep[173]{Jungraithmayr1968}\\
\gll kà  tà  ƒàm\\
\textsc{2.prf}  drink  water\\
\glt `You have drunk water.'
\z

On the other end of the spectrum, in some \textsc{Biu-Mandara Chadic} languages, e.g., Vamé \REF{ex:anderson:43}, one now finds only bound prefix conjugations which have derived from STAMP morph constructions. 

\ea\label{ex:anderson:43}
Vamé  \citep[11]{Kinnaird2006}          [\textsc{Biu-Mandara Chadic}]\\
\gll əŋ-lɛ\\
fut.1-go\\
\glt `I will go.'
\z

A common pattern seen throughout the Macro-Sudan Belt is for a perfective series to be bound prefixal conjugations, but an imperfective series to remain in freestanding STAMP morph constructions. An example of such a system is found in the \textsc{Biu-Mandara Chadic} language Mbuko \REF{ex:anderson:44}.

\ea\label{ex:anderson:44}
Mbuko  \citep[7]{Gravina2001}           [\textsc{Biu-Mandara Chadic}]\\
\ea\label{ex:anderson:44a}
\gll nə-zlàmbál    \\
  1\textsc{pfv/ant}-throw:\textsc{ant} \\
\glt `I threw.'    

\ex \label{ex:anderson:44b}
\gll n\={i}  zl\={a}mb\={a}l\\
\textsc{1.ipfv}  throw\\
\glt `I am throwing.'  
\z
\z

Merey (also \textsc{Biu-Mandara Chadic}) shows an unusual intra-pardigmatic split between free-standing forms in the first singular past but bound prefix elements in the corresponding third  person \REF{ex:anderson:45} (masculine singular) past form. 

\ea\label{ex:anderson:45}
Merey  \citep[8]{Gravina2007}          [\textsc{Biu-Mandara Chadic}]\\
\ea\label{ex:anderson:45a}
\gll na    zal      \\
  \textsc{1.pst}    call\\
\glt `I called.'

\ex \label{ex:anderson:45b}
\gll a-zal\\
  3\textsc{.pst}-call\\
\glt `He called.'
\z
\z

Once bound, such prefix conjugational elements can appear with new auxiliary verbs, as in the following Buduma formation \REF{ex:anderson:46}.

\ea\label{ex:anderson:46}
Buduma \citep[376]{Pawlak2001}, \citep[55]{Lukas1939}    [\textsc{Biu-Mandara Chadic}]\\
\gll a-kol  a  jai-ni\\
\textsc{3.prs}-be  at  sit\textsc{-vn}\\
\glt `He is/was sitting.' 
\z

While prefix conjugations are archaic features of Afroasiatic languages \citep{Hodge1971, Schuh1976, Mukarovksy1983, Voigt1987}, the ones attested in Chadic languages do not reflect inherited structures \citep{Schuh1976, Voigt1989, Jungraithmayr2005, Jungraithmayr2006}, but rather reflect secondary developments typically derived from the the univerbation of STAMP morphs with following verbs \citep{Caron2006, Shay2008, Anderson2011, Anderson2012}.

\section{STAMP morphs and conjugational prefixes in Central Sudanic languages}\label{sec:Anderson:8}

STAMP morphs and prefixed conjugation series derived from STAMP morphs are also found in the various families of the Central Sudanic stock \citep{Anderson2015}. Within the analytic tradition of Ma'di \REF{ex:anderson:47}, a member of the \textsc{Moru-Ma'di} genetic unit Blackings \& \citet{Fabb2003} analyze the freestanding STAMP morph \textit{ka}/\textit{kɔ} as a pronominal, but Tucker \& \citet{Bryan1966} analyze it as an auxiliary. Such disagreement over what traditional part-of-speech to assign these complexes to underscores the unusual but characteristic features of STAMP morphs. 

\ea\label{ex:anderson:47}
Ma'di \citep[13]{Blackings&Fabb2003}      [\textsc{Moru-Ma'di}]\\
\ea\label{ex:anderson:47a}
\gll ká    gbándà  {\textasciigrave}ɲa  \\
  \textsc{3sg.indef}  cassava  \textsc{npst}:eat\\
\glt `He eats/is eating cassava.'      

\ex\label{ex:anderson:47b}
\gll kɔ    {\textasciigrave}ɲa-ʔa\\
  3\textsc{sg.indef}  \textsc{npst}:eat-\textsc{obj}\\
\glt `He eats/is eating it.'
\z
\z

Various \textsc{Sara-Bongo-Bagirmi} languages require infinitive forms of vowel-initial verbs when used with freestanding STAMP morphs, for example, Ngambay \REF{ex:anderson:48}, Kabba \REF{ex:anderson:49} and Kenga \REF{ex:anderson:50}.

% \newcommand{\andersonA}{\textcyrillic{а}}
\todo{check intended meaning of cyrillic here}
\newcommand{\andersonA}{a}
\ea\label{ex:anderson:48}
Ngambay \citep[118]{Vandame1963}      [\textsc{Sara-Bongo-Bagirmi}]\\
\ea\label{ex:anderson:48a}
m{\andersonA}\={}    k-ào   àl   ngà\\
   1.\textsc{fut}   \textsc{inf}-go  \textsc{neg}   \textsc{adv}\\
\glt `I will not go again.'

\ex \label{ex:anderson:48b}
\gll á   k-ùs{\andersonA}\={}    né   ngà   uà\\
 2.\textsc{fut}   \textsc{inf}-eat thing   \textsc{adv}   \textsc{q}\\
\glt `You are already going to eat?'
\z
\z

\ea\label{ex:anderson:49}
Kabba  \citep[220]{Moser2004}         [\textsc{Sara-Bongo-Bagirmi}]\\
\gll má  k-àw  lò  tə  àáng    \\
1:\textsc{fut}  \textsc{inf}-go  place  \textsc{loc}  \textsc{neg}    \\
\glt `I shall go nowhere.'
\z

\ea\label{ex:anderson:50}
Kenga  \citep[15]{Neukom2010}        [\textsc{Sara-Bongo-Bagirmi}]\\
\gll \={m}    k-ɔsɔ\\
1.\textsc{fut}  \textsc{inf}-eat\\
\glt `I will eat.'
\z

As with Chadic, many languages in different sub-families of Central Sudanic have a bound `definite' perfective STAMP series that contrasts with an unbound `indefinite' imperfective series. The pattern is found in Bongo \REF{ex:anderson:51}, Lugbara \REF{ex:anderson:52}, Lendu \REF{ex:anderson:53} and Oke'bu \REF{ex:anderson:54}, from four different genetic units of Central Sudanic. 

\ea\label{ex:anderson:51}
Bongo   \textit{*mi- < m-i} \citep[75]{Tucker&Bryan1966}    [\textsc{Sara-Bongo-Bagirmi}]\\
\ea\label{ex:anderson:51a}
\gll mi-bi\\
1.\textsc{def}-give\\
\glt `I give, I gave.'

\ex \label{ex:anderson:51b}
\gll ma    bi\\
 1.\textsc{indef}  give\\
\glt  `I am giving.'    
\z
\z

\ea\label{ex:anderson:52}
Lugbara \citep[47]{Tucker&Bryan1966}         \textsc{[Moru-Ma'di]} \\
\ea\label{ex:anderson:52a}
\gll á-tsɔ    mvá\\
  1.\textsc{def}-beat  child\\
  \glt `I beat the child.'
  
  \ex \label{ex:anderson:52b}
\gll ma    mvá  tsɔ\\
  1.\textsc{indef}  child  beat\\
\glt `I am beating the child.'
\z
\z

\ea\label{ex:anderson:53}
Lendu  \citep[46]{Tucker&Bryan1966}         [\textsc{Lendu}]\\
\ea\label{ex:anderson:53a}
\gll má-drr    mbí\\
  1.\textsc{def}-pull  rope\\
  \glt `I pull the rope.' 

\ex\label{ex:anderson:53b}
\gll má    mbi  dr\'{r}\\
  \textsc{1.indef}  rope  pull\\
\glt `I am pulling the rope.'
\z
\z

\ea\label{ex:anderson:54}
Oke'bu \citep[48]{Tucker&Bryan1966}        \textsc{[Mangbutu]}\\
\ea\label{ex:anderson:54a}
\gll l-ómá     ùnzu\\
  \textsc{2.def}-beat  child\\
\glt `You beat the child.'


\ex \label{ex:anderson:54b}
\gll láà    unzú  òma\\
  \textsc{2.indef}  child  beat\\
\glt `You are beating the child.'
\z
\z

As the reader may have noticed, there is also a difference in the basic clausal syntax of the two series in these Central Sudanic languages, with VO order in the definite/perfective series and OV in the indefinite/imperfective one. The VO order in the definite series vs. OV order in the indefinite is, however, not universal in Central Sudanic and is not found, for example, in Baka \REF{ex:anderson:55} of the \textsc{Sara-Bongo-Bagirmi} family or Kresh \REF{ex:anderson:56}.

\ea\label{ex:anderson:55}
Baka \citep[75]{Tucker&Bryan1966}        [\textsc{Sara-Bongo-Bagirmi}]\\
\ea\label{ex:anderson:55a}
\gll m-áne    yí\\
  \textsc{1.def}-eat  thing\\
\glt `I ate (something).'

\ex \label{ex:anderson:55b}
\gll má    y-ane    yi\\
 1.\textsc{indef}  \textsc{inf}-eat    thing\\
\glt `I am eating (something).'
\z
\z

\ea\label{ex:anderson:56}
Kresh \citep[75]{Tucker&Bryan1966}     [\textsc{Kresh}]\\
\ea\label{ex:anderson:56a}
\gll m-omò    nòmò\\
  1-drink   drink    \\
\glt I drink a drink.'     


\ex \label{ex:anderson:56b}
\gll ma     y-òmò    nòmò\\
  \textsc{1.indef}  \textsc{inf}-drink   drink\\
\glt `I am drinking.'
\z
\z

Note that these two languages however also have infinitive forms of the verb with vowel initial stems with the indefinite series STAMP morph, like those in \REF{ex:anderson:48}-\REF{ex:anderson:50}. This feature further underscores the likely origin of STAMP morph in an auxiliary verb construction in Central Sudanic languages.

Of course, once a bound STAMP morph prefix exists in a given Central Sudanic language, it is free to attach to auxiliary verbs in new AVCs, as seen in the following Lugbara sentence \REF{ex:anderson:57}.

\ea\label{ex:anderson:57}
Lugbara \citep[46, 47]{Tucker&Bryan1966}      \textsc{[Moru-Ma'di]}\\
\gll ma-ŋga  mvá  tsɔ\\
\textsc{1.def-aux}  child  beat\\
\glt `I shall beat the child.'    
\z

Variation in closely related varieties between bound and unbound formations can be seen in \textsc{Sara-Bongo-Bagirmi} languages. Compare in this regard Furu \REF{ex:anderson:58}, where a freestanding STAMP morph is found, with the corresponding sentence in its close sister language Bagiro \REF{ex:anderson:59}, where it is a prefix.

\ea\label{ex:anderson:58}
Furu \citep[91]{Boyeldieu1990}           [\textsc{Sara-Bongo-Bagirmi}]\\
\gll mí  gáli  gɔ          \\
I   know  \textsc{neg}      \\
\glt `I don't know.'
\z

\ea\label{ex:anderson:59}
Bagiro  \citep[91]{Boyeldieu1990}           [\textsc{Sara-Bongo-Bagirmi}]\\
\gll mú-gáꜜli  gɔ\\
1-know  \textsc{neg}\\
\glt `I don't know.'  
\z

\textsc{Mangbetu} languages only have bound prefixes, but nevertheless reflect a possible trace of two originally distinct STAMP sets, albeit with the opposition having become phonologized in such languages as Mangbetu and Meje. Thus vowel initial stems take the second singular prefix \textit{ni-} \REF{ex:anderson:60}-\REF{ex:anderson:61} while consonant-initial stems take \textit{(m)ú-} \REF{ex:anderson:62}-\REF{ex:anderson:63}:

% \newcommand{\andersonE}{$\text{\textgreek{'e}}$} 

\todo{check intended meaning of greek letter here}
\newcommand{\andersonE}{\'{ɛ}}
\ea\label{ex:anderson:60}
Mangbetu \citep[106]{Larochette1958}        \textsc{[Mangbetu]}\\
\gll ni-{\andersonE}si-(a)      \\
2.\textsc{prs/fut}-do-\textsc{tam}        \\
\glt `You (will) do.' 
\z

\ea\label{ex:anderson:61}
Meje \citep[106]{Larochette1958}           \textsc{[Mangbetu]}\\
\gll ni-{\andersonE}sí-a\\
\textsc{prs/fut}-do-\textsc{tam}\\
\glt `You (will) do.' 
\z

\ea\label{ex:anderson:62}
Mangbetu \citep[106-7]{Larochette1958}        \textsc{[Mangbetu]}\\
\gll mú-ta\\
2.\textsc{prs/fut}-carry    \\
\glt `You (will) carry.'     
\z

\ea\label{ex:anderson:63}
Meje \citep[106-7]{Larochette1958}        \textsc{[Mangbetu]}\\
\gll ú-ta\\
\textsc{prs/fut}-carry\\
\glt `You (will) carry.' 
\z

The \textsc{Mangbutu} language Lese also has only bound prefix conjugations. Again these appear to reflect two originally distinct series, such as the present progressive in the first singular in \textit{má}- \REF{ex:anderson:64} and the future progressive first singular in \textit{mʊ-} \REF{ex:anderson:65}.

\ea\label{ex:anderson:64}
Lese \citep[51]{Tucker&Bryan1966}        \textsc{[Mangbutu]}\\
\gll má-dzɔ    kɔɗí  à-nʊ   \\
1:\textsc{tam-aux}  meat  \textsc{dep}-eat   \\
\glt `I am eating meat.'   
\z

\ea\label{ex:anderson:65}
Lese \citep[51]{Tucker&Bryan1966}        \textsc{[Mangbutu]}\\
\gll mʊ-dzá    kɔɗí  à-nʊ\\
1:\textsc{tam-aux}  meat  \textsc{dep}-eat\\
\glt `I shall be eating meat.'  
\z

While widespread and hence tempting to reconstruct the bound perfective series and unbound imperfective series pattern all the way back to Proto-Central Sudanic, the overall frequency of such patterning in the Macro-Sudan Belt suggests that we should do so only with caution. Nevertheless, I tentatively reconstructed this to Proto-Central Sudanic \citep{Anderson2015}. This is because the sub-families that lack this pattern have two bound series, and never the opposite patterning, such that bound imperfective series appear to be attested only if bound perfective series exist; this distribution is also found across the Benue-Congo and Chadic languages of the Macro-Sudan Belt.

\section{STAMP morphs and conjugational prefixes in Niger-Congo languages}\label{sec:Anderson:9}

STAMP morphs and conjugational prefixes derived from STAMP morphs are found in a wide range of Niger-Congo languages. These include Klao of the \textsc{Kru} family \REF{ex:anderson:66}, Dadiya of the \textsc{Waja-Kam} family \REF{ex:anderson:67}, Gã of the \textsc{Ga-Dangme Kwa} genetic unit \REF{ex:anderson:68}, and Kulango \REF{ex:anderson:69}.

\ea\label{ex:anderson:66}
Klao \citep[3, 18]{Marchese1982}            [\textsc{Kru}]\\
\gll ɔɔ    bl\={e}  \\
he\textsc{:ipfv}   sing\\
\glt `He is singing.'
\z

\ea\label{ex:anderson:67}
Dadiya   \citep[196]{Jungraithmayr1968/1969}        [\textsc{Waja-Kam]}\\
\gll \'{n}    já   \\
you\textsc{.pfv}  eat.\textsc{pfv} \\
\glt `You have eaten.'
\z

\ea\label{ex:anderson:68}
Ga \citep[105]{Kropp-Dakubu1988}          \textsc{[Ga-Dangme Kwa]}  \\
\gll èe     n\`{\~u}  n\`{\~u}  \\
3\textsc{:prs}  water  drink\\
\glt `He is drinking water.'
\z

\ea\label{ex:anderson:69}
Kulango \citep[193]{Elders2007}            \textsc{[Kulango]}\\
\ea\label{ex:anderson:69a}
\gll mɪ   dɔlɪ\\
  \textsc{1.sbjn}  sell\\
\glt `May I sell.'    

\ex \label{ex:anderson:69b}
\gll   mɪɪ   dɔlɪ\\
  1.\textsc{prog}  sell\\
  \glt `I am selling.'  
\z
\z

Such formations are also widespread in Mande languages, such as the following example from Jo(wulu) of the \textsc{Northwest Mande} genetic unit \REF{ex:anderson:70}.

\ea\label{ex:anderson:70}
Jo (Jowulu) \citep[11]{Kim2002}          [\textsc{Northwest Mande}]  \\
\gll kaa    ku-ki  kulu  n\~u  \\
\textsc{3pl.fut}  \textit{to}-\textsc{def}  hot  eat    \\
\glt `They will eat hot \textit{to.}'        
\z

Within Benue-Congo there is evidence for two (possibly three) sets of contrastive STAMP morphs series.\footnote{The reconstructions for Proto-Benue-Congo are preliminary and impressionistic. Space limitations prevent further demonstration of the details of the complex Benue-Congo situation.} One is realized as a bound prefix series, while the other(s) remain(s) freestanding STAMP morphs in most but not all relevant languages. These series consist of a bound realis/perfect(ive) series marked by \textit{*}\textit{ma-} (possibly with an associated low tone *\textit{mà}-) in the first singular, which contrasts with one (or two) other \textit{m}-initial series with front vowels (or no vowel) marking irrealis/imperfective action. Thus, within Benue-Congo itself, there is some evidence for the two non-perfect(ive) series. However, the broader comparative data from Atlantic-Congo (or Volta-Congo) languages suggest that there were originally two sets of forms, and one may well have split into two contrasting series during the development of Benue-Congo, or, alternatively, in the history of individual sub-groups within Benue-Congo. However, they likely originated in a single imperfective/irrealis series from an earlier state of the language. One of these sets generally has a high-tone in the first singular and marks irrealis or future semantics. The first singular marker in the other set appears optionally without a vowel and with a non-high (mid or low) tone, and encodes present/non-past/imperfective semantics \REF{ex:anderson:71}.

\ea\label{ex:anderson:71}
 Proto-Benue-Congo\\
Pattern 1  \textit{*ma-} I.\textsc{prf/pst/rls  :}  \textit{*mé}  I.\textsc{irr/fut}\\  
Pattern 2  \textit{*ma-} I.\textsc{prf/pst/rls}  :  \textit{*mé} I.\textsc{irr/fut}  :   \textit{*mi, *m(V)} I.\textsc{prs/npst}\\
\z

Evidence for the *\textit{ma-} ({\textasciitilde}\textbf{*}\textit{mà}-) perfective/realis/past series in Benue-Congo \REF{ex:anderson:71} comes from a range of different sub-groups of the stock. Thus one finds formally and functionally cognate elements in such Benue-Congo languages as \textsc{Bantoid} Ndemli \REF{ex:anderson:72}, Eleme of the \textsc{Cross River} family \REF{ex:anderson:73}, and Berom of the \textsc{Plateau} sub-stock \REF{ex:anderson:74}.

\ea\label{ex:anderson:72}
Ndemli   \citep[72]{Ngoran1999}            [\textsc{Bantoid}]\\
\gll mà-tóm \\
\textsc{I.pst-}send\\
\glt  `I sent.'   
\z

\ea\label{ex:anderson:73}
Eleme  \citep[1482]{Bond2008}, \citep{Bond2010}        [\textsc{Cross River]}\\
\gll ma-ʔà\\
1.\textsc{ant.prf}-leave\\
\glt `I left.'
\z

\ea\label{ex:anderson:74}
Berom  \citep[299, 301]{Bouquiaux1970}          [\textsc{Plateau}]\\
\gll mà-ciŋ\\
1.\textsc{pst/prf}-dig\\
\glt `I (have) dug.'
\z

Evidence for the high-toned irrealis or future series marked in the first singular by \textit{*mé} can be found in a range of Benue-Congo languages, e.g., the \textsc{Bantoid} language Tiv \REF{ex:anderson:75}, the \textsc{Kainji} language Duka \REF{ex:anderson:76} or Berom of the \textsc{Plateau} stock \REF{ex:anderson:77}.

\ea\label{ex:anderson:75}
 Tiv \citep[Tiv 4]{Arnott1967/1980}        \textsc{[Bantoid]}\\
 \gll mé    !va\\
\textsc{1fut}  come\\
\glt `I will come.'
\z

\ea\label{ex:anderson:76}
Duka \citep[17]{Bendor-Samuel1973}      \textsc{[Kainji]}\\
\gll  mɛ    róà    sə  á\\
I.\textsc{irr}  \textsc{rem.fut}  drink  \textsc{neg}\\
\glt `I would not drink it.'
\z

\ea\label{ex:anderson:77}
Berom  \citep[300]{Bouquiaux1970}        \textsc{[Plateau]}\\
\gll mé    ciŋ\\
1.\textsc{sbjn}  dig\\
\glt `I must dig.'
\z

Evidence for a third series of STAMP morphs in Proto-Benue-Congo is more tenuous and must remain beyond of the scope of the present study.

\section{Summary}\label{sec:Anderson:10}

This study has been a preliminary survey of STAMP morph constructions in a range of genetic linguistic units spoken across the Macro-Sudan Belt. A comparative analysis of STAMP morphs across the languages of the Macro-Sudan Belt, and the prefixal conjugations that historically derived from them, yields significant insights into various layers in the history of verbal conjugation in many genetic units. Perhaps the most noteworthy insight is that there is an absolute pattern whereby, if present, a bound series encodes perfective/realis categories and an unbound series rather encodes imperfective/irrealis categories. This situation is found in Chadic, Central Sudanic and Benue-Congo languages alike. This pattern has been reconstructed to Proto-Central Sudanic \citep{Anderson2015}. One can also reconstruct a probably bound perfective/realis conjugation series (e.g., with first singular \textit{*ma-/=)} for Proto-Benue-Congo, but its immediate ancestral formation may still have been an unbound STAMP morph. The corresponding imperfective/irrealis series (e.g., with first singular \textit{*mé {\textasciitilde} *mI {\textasciitilde} *mÍ}) appears to have certainly been an unbound STAMP morph in the proto-language. As in Central Sudanic, Benue-Congo languages only have bound imperfective series STAMP morphs if perfective series elements are also bound. The precise modeling of the Chadic developments is the object of current ongoing research, but the same distributional trend appears to be manifested across the Chadic languages as well. 

While insights into the inflectional history of the genetic units in this macro-region are gained by such an analysis (e.g. how secondary prefix conjugations arose Central Sudanic languages), others remain unanswered to date. The most salient of such questions is the following: Why is it that if bound/prefixal conjugations are found which appear to derive from STAMP morph constructions, in Chadic, various Niger-Congo families and Central Sudanic languages across the Macro-Sudan Belt the pattern is always one where the bound series encodes perfective/realis and the unbound series imperfective/irrealis categories, but never the reverse situation? Resolving this important and intriguing question remains a primary goal of future research. 

\section*{Abbreviations}

 

\begin{tabularx}{.3\textwidth}{ll}
1 & first person\\
2 & second person\\
3 & third person\\
>  & acting on  \\
\textsc{acc}  & accusative\\  
\textsc{adv}   & adverb\\
\textsc{ant}   & anterior \\
\textsc{aor}   & aorist\\  
\textsc{asp}   & aspect\\
\textsc{aux}   & auxiliary  \\
\textsc{clsfr}   & classifier  \\
\textsc{compl}   & completive\\
\textsc{cond}   & conditional  \\
\textsc{cont}   & continuative  \\
\textsc{cop}   & copula\\
\textsc{def}   & definite\\  
\textsc{dep}   & dependent \\ 
\end{tabularx} 
\begin{tabularx}{.3\textwidth}{ll}
\textsc{fut}   & future\\
\textsc{+H}   & high tone\\  
\textsc{io}  & indirect object  \\
\textsc{hab}   & habitual  \\
\textsc{imm}  & immediate\\
\textsc{indef}  & indefinite\\  
\textsc{inf}  & infinitive  \\
\textsc{inj}  & injunctive  \\
\textsc{ipfv}  & imperfective  \\
\textsc{irr}   & irrealis  \\
\textsc{+L}  & low tone  \\
\textsc{loc}  & locative  \\
\textsc{mod}  & modal  \\
\textsc{neg}  & negative  \\
\textsc{npst}  & non-past \\
\textsc{obj}  & object  \\
\textsc{oblq}  & oblique  \\
\end{tabularx} 
\begin{tabularx}{.33\textwidth}{ll}
\textsc{pfv}  & perfective  \\
\textsc{pl}  & plural\\
\textsc{prf}  & perfect  \\
\textsc{prog}  & progressive \\ 
\textsc{prs}  & present\\
\textsc{pst}  & past  \\
\textsc{q}  & interrogative\\  
\textsc{rec}  & recent\\
\textsc{rdpl}  &reduplication  \\
\textsc{rem}  & remote  \\
\textsc{rls}  & realis\\
\textsc{sbjn}  & subjunctive\\  
\textsc{sg}  & singular  \\
\textsc{tam}  & tense-aspect-mood \textsc{}\\
\textsc{vn    }   & verbal noun\\
\\
\\
\end{tabularx}


%
%\bfseries
%\begin{verbatim}%%move bib entries to  localbibliography.bib
%@book{Anderson2006,
%	address = {Oxford},
%	author = {Anderson, Gregory D. S.},
%	publisher = {Oxford University Press},
%	title = {\textit{Auxiliary verb constructions}},
%	year = {2006}
%}
%
%@article{Anderson2011,
%	author = {Anderson, Gregory D. S},
%	journal = {\textit{Studies in African Linguistics}} ,
%	number = {1-2},
%	pages = {1-409},
%	title = {Auxiliary verb constructions in the languages of Africa},
%	volume = {40},
%	year = {2011}
%}
%
%@misc{Anderson2012,
%	author = {Anderson, Gregory D. S},
%	title = {S/TAM/P morphs in the history of Benue-Congo and Niger-Congo conjugation. Presented at Niger-Congress, Paris, September 2012.},
%	year = {2012}
%}
%
%@incollection{Anderson2015,
%	address = {Köln},
%	author = {Anderson, Gregory D. S},
%	booktitle = {\textit{Nilo-Saharan: Models and descriptions},},
%	editor = {Mietzner, Angelika \& Storch, Anne},
%	pages = {151-167},
%	publisher = {Rüdiger Köppe Verlag},
%	title = {STAMP morphs in Central Sudanic languages},
%	year = {2015}
%}
%
%@article{Arnott1958,
%	author = {Arnott, D. W},
%	journal = {\textit{Bulletin of the School of Oriental and African Studies}},
%	number = {1},
%	pages = {111-133},
%	title = {The classification of verbs in Tiv},
%	volume = {21},
%	year = {1958}
%}
%
%Arnott, D. W. 1967/1980. Tiv. In Kropp-Dakubu, M. E. (ed.), \textit{West African language data sheets} \textit{volume 2}. 8 pp. Accra: West African Linguistic Society.
%
%@article{Babaev2010,
%	author = {Babaev, Kirill},
%	journal = {\textit{Journal of Language Relationship/Voprosy Jazykovogo Rodstva}},
%	pages = {1–45},
%	title = {Reconstructing Benue-Congo person marking II},
%	volume = {4},
%	year = {2010}
%}
%
%\begin{styleHeader}
%@book{Bendor-Samuel1973,
%	address = {Zaria},
%	author = {Bendor-Samuel, John, Skitch, Donna  and  Cressman, Esther.},
%	number = {3},
%	publisher = {Institute of Linguistics and Centre for the Study of Nigerian Languages, Abdullahi Bayero College, Ahmadu Bello University, Kano},
%	series = {Studies in Nigerian Languages},
%	title = {\textit{Duka sentence, clause and phrase}.},
%	year = {1973}
%}
%\end{styleHeader}
%
%@book{Blackings2003,
%	address = {Berlin},
%	author = {Blackings, Mairi  and  Fabb, Nigel.},
%	publisher = {Mouton de Gruyter},
%	title = {\textit{A grammar of Ma'di.} (Mouton Grammar Library 32)},
%	year = {2003}
%}
%
%Bond, Oliver. 2008. Multiple analytical perspectives of the Eleme anterior-perfective. In \textit{Current issues in unity and diversity of languages: Collection of papers selected from the CIL 18}, 1480-1496. Seoul: Linguistic Society of Korea.
%
%@article{Bond2010,
%	author = {Bond, Oliver},
%	journal = {\textit{Studies in Language}},
%	pages = {1-35},
%	title = {Intra-paradigmatic variation in Eleme verbal agreement},
%	volume = {34},
%	year = {2010}
%}
%
%@book{Bouquiaux1970,
%	address = {}Paris},
%	author = {Bouquiaux, Luc},
%	publisher = {Les Belles Lettres},
%	title = {\textit{La langue Birom (Nigeria septentrional) –phonologie, morphologie, syntaxe},
%	year = {1970}
%}
%
%@incollection{Burquest1973,
%	address = {Accra},
%	author = {Burquest, Donald},
%	booktitle = {\textit{West African language data sheets} \textit{volume 1},},
%	editor = {Kropp-Dakubu, M. E.},
%	pages = {35-43},
%	publisher = {West African Linguistic Society},
%	title = {/1980. Angas},
%	year = {1973}
%}
%
%@incollection{Burquest1986,
%	address = {Tübingen},
%	author = {Burquest, Donald},
%	booktitle = {\textit{Pronominal systems},},
%	editor = {Wiesemann, Ursula},
%	pages = {71–101},
%	publisher = {Gunter Narr Verlag},
%	title = {The pronoun system of some Chadic languages},
%	year = {1986}
%}
%
%@article{Caron2001,
%	author = {Caron, Bernard},
%	journal = {\textit{Afrika und Übersee}},
%	pages = {1-60},
%	title = {Dott, aka Zoɗi, (Chadic, West-B, South Bauchi): Grammatical notes and vocabulary},
%	volume = {84},
%	year = {2001}
%}
%
%@article{Caron2002,
%	author = {Caron, Bernard},
%	journal = {\textit{Afrika und Übersee}},
%	pages = {161-248},
%	title = {Guus, aka Sigidi (Chadic, West-B, South Bauchi): Grammatical notes, vocabulary and text},
%	volume = {85},
%	year = {2002}
%}
%
%Caron, B. 2006. South Bauchi West pronominal and TAM systems. In Caron, B. \& Zima, P.~(eds.), \textit{Sprachbund in the West African Sahel,} 93-112\textit{.} Louvain/Paris: Peeters.
%
%@incollection{Caron2008,
%	address = {Louvain/Paris},
%	author = {Caron, Bernard},
%	booktitle = {\textit{Subordination, dependence et parataxe dans les langues africaines},},
%	editor = {Caron, Bernard},
%	pages = {145-158},
%	publisher = {Peeters},
%	title = {La structure énonciative des subordonnées conditonnelles},
%	year = {2008}
%}
%
%\begin{styleHeader}
%Cleal, Alizon M. 1973/1980. Nchumuru. In Kropp-Dakubu, M. E. (ed.), \textit{West African language data sheets} \textit{volume} \textit{2}, 11 pp. Accra: West African Linguistic Society.
%\end{styleHeader}
%
%\begin{styleHeader}
%@incollection{Cook1972,
%	address = {Accra},
%	author = {Cook, Thomas L},
%	booktitle = {\textit{West African language data sheets} \textit{volume} \textit{1},},
%	editor = {Kropp-Dakubu, M. E.},
%	pages = {350-356},
%	publisher = {West African Linguistic Society},
%	title = {/1980 (1977) Kohumono},
%	year = {1972}
%}
%\end{styleHeader}
%
%Creissels, Denis. 2005. A typology of subject and object markers in African languages\textit{.} In Voeltz, F. K. Erhard (ed.), \textit{Studies in African Linguistic Typology}, 43-70. (Typological Studies in Language 64). Amsterdam: John Benjamins.
%
%\begin{styleHeader}
%@incollection{Creissels2008,
%	address = {Cambridge},
%	author = {Creissels, Denis, Dimmendaal, Gerrit J., Frajzyngier, Zygmunt  and  König, Christa},
%	booktitle = {\textit{A linguistic geography of} \textit{Africa},},
%	editor = {Heine, Bernd \& Nurse, Derek},
%	pages = {86-150},
%	publisher = {Cambridge University Press},
%	title = {Africa as a morphosyntactic area},
%	year = {2008}
%}
%\end{styleHeader}
%
%@article{Dimmendaal2008,
%	author = {Dimmendaal, Gerrit},
%	journal = {\textit{Language and} \textit{Linguistics Compass}} ,
%	number = {5},
%	pages = {840-858},
%	title = {Language ecology and linguistic diversity in Africa},
%	volume = {2},
%	year = {2008}
%}
%
%@book{Dimmendaal2011,
%	address = {Amsterdam},
%	author = {Dimmendaal, Gerrit.},
%	publisher = {John Benjamins},
%	title = {\textit{Historical linguistics and the comparative study of African languages}},
%	year = {2011}
%}
%
%@book{Dimmendaal2009,
%	address = {Amsterdam},
%	editor = {Dimmendaal, Gerrit},
%	publisher = {Benjamins},
%	title = {\textit{Coding participant marking. Construction types in twelve African languages}},
%	year = {2009}
%}
%
%@incollection{Elders2007,
%	address = {(SOAS Working Papers in Linguistics 15). London},
%	author = {Elders, Stephan},
%	booktitle = {\textit{Bantu in Bloomsbury: Special issue on Bantu linguistics},},
%	editor = {Kula, Nancy C. \& Marten, Lutz},
%	pages = {187-200},
%	publisher = {University of London},
%	title = {Complex verb morphology in Kulango (Gur): Similarities and dissimilarities with Bantu},
%	year = {2007}
%}
%
%\begin{styleHeader}
%@incollection{Èrman2002,
%	address = {Sankt-Petersburg},
%	author = {Èrman, Anna V},
%	booktitle = {\textit{Juzhnye mande: Lingvistika afrikanskikh ritmakh. Materialy peterburgskoj èkspeditsii v Kot d'Ivuar (K 50-letiju Konstantina Pozdnjakov)},},
%	editor = {Vydrin, Valentin F. \& Zheltov, Aleksandr Ju.},
%	pages = {154-82},
%	publisher = {Evropejskij Dom},
%	title = {Sub''ektnye mestoimenija v dan-blovo I modal'no-aspektno-temporal'nye znachenija [Subject pronouns in Dan-Blowo and (their) modal-aspectual-temporal meanings]},
%	year = {2002}
%}
%\end{styleHeader}
%
%Frajzyngier, Zygmunt. 1977. On the intransitive copy pronouns in Chadic. \textit{Studies in African Linguistics, Supplement} VII. 73-84.
%
%@book{Frajzyngier2008,
%	address = {Frankfurt},
%	author = {Frajzyngier, Zygmunt.},
%	number = {13},
%	publisher = {Peter Lang},
%	series = {Research in African studies},
%	title = {\textit{A Grammar of Gidar}.},
%	year = {2008}
%}
%
%@book{Gravina2001,
%	address = {Yaoundé},
%	author = {Gravina, Richard.},
%	publisher = {SIL},
%	title = {\textit{The verb phrase in Mbuko}},
%	year = {2001}
%}
%
%@book{Gravina2007,
%	address = {Yaoundé},
%	author = {Gravina, Richard.},
%	publisher = {SIL},
%	title = {\textit{The verb phrase in Merey}. (Ministère de la recherche scientifique et de l'innovation)},
%	year = {2007}
%}
%
%@incollection{Güldemann2008,
%	address = {}Cambridge},
%	author = {Güldemann, Tom},
%	booktitle = {\textit{A linguistic geography of} \textit{Africa,},
%	editor = {Heine, Bernd \& Nurse, Derek},
%	pages = {151-185},
%	publisher = {Cambridge University Press},
%	title = {The Macro-Sudan belt: Towards identifying a linguistic area in northern sub-Saharan Africa},
%	year = {2008}
%}
%
%@book{Hodge1971,
%	address = {The Hague},
%	editor = {Hodge, Carleton T.},
%	number = {163},
%	publisher = {Mouton},
%	series = {Janua Linguarum Series Practica,},
%	title = {\textit{Afroasiatic: A survey}.},
%	year = {1971}
%}
%
%Hofer, Verena \& Link, Christa. 1973/1980. Wobé. In Kropp-Dakubu, M. E. (ed.), \textit{West African language data sheets} \textit{volume} \textit{2}. 8 pp. Accra: West African Linguistic Society.
%
%@book{Jones1998,
%	address = {Munich},
%	author = {Jones, Ross McCallum.},
%	number = {30},
%	publisher = {Lincom Europa},
%	series = {Lincom Studies in African Linguistics},
%	title = {\textit{The Boko/Busa language cluster}.},
%	year = {1998}
%}
%
%@article{Jungraithmayr1968,
%	author = {Jungraithmayr, Herrmann},
%	journal = {\textit{Afrika und Übersee}},
%	pages = {161-204},
%	title = {/1969. The class languages of the Tangale-Waja district},
%	volume = {52},
%	year = {1968}
%}
%
%Jungraithmayr, Herrmann. 2005. Prefix and suffix conjugation in Chadic. \textit{Proceedings of the 10th Meeting of Hamito-Semitic (Afroasiatic) Linguistics}, 411-419.
%
%Jungraithmayr, Herrmann. 2006. The verb in Chadic–state of the art.~In Caron, B. \& Zima, P.~(eds.), \textit{Sprachbund in the West African Sahel,} 167-183. Louvain/Paris: Peeters.
%
%@book{Kim2002,
%	address = {Bamako},
%	author = {Kim, Hae-Kyung,},
%	publisher = {SIL Mali},
%	title = {\textit{Aspect, temps et modes en jowulu}},
%	year = {2002}
%}
%
%@book{Kinnaird2006,
%	address = {Yaoundé},
%	author = {Kinnaird, William J.},
%	publisher = {SIL},
%	title = {\textit{The Vamé verbal system}},
%	year = {2006}
%}
%
%Kropp Dakubu, Mary Esther (ed.). 1975-1980. \textit{West African language data sheets. Volumes 1-2.} Accra: West African Linguistic Society.
%
%@book{Kropp-Dakubu1988,
%	address = {London},
%	editor = {Kropp-Dakubu, Mary Esther},
%	publisher = {Kegan Paul International for the International African Institute},
%	title = {\textit{The languages of Ghana}},
%	year = {1988}
%}
%
%@book{Larochette1958,
%	address = {Tervuren},
%	author = {Larochette, Joe.},
%	number = {18},
%	publisher = {Annales du Musée Royal du Congo Belge},
%	series = {Sciences de L'homme,},
%	title = {\textit{Grammaire des dialects mangbetu et medje, suivie d'un manuel de conversation et d'un lexique}.},
%	year = {1958}
%}
%
%@book{Lukas1939,
%	address = {Leipzig},
%	author = {Lukas, Johannes.},
%	publisher = {F. A. Brockhaus},
%	title = {\textit{Die Sprache der Buduma im Zentral Sudan}. \textit{Auf Grund eigener Studien und des Nachlasses von Gustav Nachtigal}},
%	year = {1939}
%}
%
%@article{Marchese1982,
%	author = {Marchese, Lynell},
%	journal = {\textit{Journal of West African Languages}},
%	number = {1},
%	pages = {3–23},
%	title = {Basic aspectual categories in Proto-Kru},
%	volume = {12},
%	year = {1982}
%}
%
%@book{Moñino1995,
%	address = {Paris},
%	author = {Moñino, Yves.},
%	publisher = {Peeters},
%	title = {\textit{Le Proto-Gbaya.} \textit{Essai de linguistique comparative sur vingt-et-une langues d'Afrique centrale}},
%	year = {1995}
%}
%
%@book{Moser2004,
%	address = {München},
%	author = {Moser, Rosemarie.},
%	number = {63},
%	publisher = {LINCOM},
%	series = {LINCOM studies in African linguistics},
%	title = {\textit{Kabba: A Nilo-Saharan language of the Central African Republic}.},
%	year = {2004}
%}
%
%Mukarovsky, Hans G.~1983. \textit{Pronouns and prefix conjugation in Chadic and Hamito-Semitic}. Hamburg: Helmut Buske Verlag.
%
%@article{Nettle1998,
%	author = {Nettle, Daniel},
%	journal = {\textit{Afrika und Übersee}},
%	pages = {253-79},
%	title = {Materials from the southeastern Plateau languages of Nigeria},
%	volume = {81},
%	year = {1998}
%}
%
%@book{Neukom2010,
%	address = {\textit{Description grammaticale du kenga (langue nilo-saharienne du Tchad).} Köln},
%	author = {Neukom, Lukas.},
%	publisher = {Rüdiger Köppe},
%	year = {2010}
%}
%
%@book{Ngoran1999,
%	address = {Yaoundé},
%	author = {Ngoran Loveline Lenaka.},
%	publisher = {University of Yaoundé. (MA Thesis)},
%	title = {\textit{A sketch outline of the phonology of Ndemli}},
%	year = {1999}
%}
%
%@article{Pawlak2001,
%	author = {Pawlak, Nina},
%	journal = {\textit{Sprache und Geschichte in Afrika}},
%	pages = {355-386},
%	title = {Diachronic typology of locative phrases in Chadic},
%	volume = {16/17},
%	year = {2001}
%}
%
%Pichl, Walter J. 1973/1980. Ndut-Falor. In Kropp-Dakubu, M. E. (ed.), \textit{West African language data sheets} \textit{volume 2}, 8 pp. Accra: West African Linguistic Society.
%
%@book{Pohlig1992,
%	address = {Yaoundé},
%	author = {Pohlig, James N.},
%	publisher = {SIL},
%	title = {\textit{An account of Mofu-Gudur verb infixes and suffixes}. (Ministry of Scientific and Technical Research)},
%	year = {1992}
%}
%
%Price, Norman. 1975/1980. Nchumuru. In Kropp-Dakubu, M. E. (ed.), \textit{West African language data sheets} \textit{volume} \textit{2}, 11 pp. Accra: West African Linguistic Society.
%
%@incollection{Pyne1972,
%	address = {Accra},
%	author = {Pyne, P. C},
%	booktitle = {\textit{West African language data sheets} \textit{volume} \textit{1},},
%	editor = {Kropp-Dakubu, Mary Esther},
%	pages = {45-55},
%	publisher = {West African Linguistic Society},
%	title = {/1980. Anyi},
%	year = {1972}
%}
%
%@article{Sands2009,
%	author = {Sands, Bonny},
%	journal = {\textit{Language and linguistics compass}} ,
%	number = {2},
%	pages = {559-580},
%	title = {Africa's linguistic diversity},
%	volume = {3},
%	year = {2009}
%}
%
%Schuh, Russell G.~1976. \textit{The Chadic verbal system and its Afroasiatic nature}. Malibu, California: Undena Publication.
%
%@article{Seidel1898,
%	author = {Seidel, August},
%	journal = {\textit{Zeitschrift für afrikanische und oceanische Sprachen}},
%	pages = {201-286},
%	title = {Beiträge zur Kenntnis der Sprachen im Togo},
%	volume = {4},
%	year = {1898}
%}
%
%@incollection{Shay2008,
%	address = {Amsterdam},
%	author = {Shay, Erin},
%	booktitle = {\textit{Interaction of morphology and syntax. Case studies in Afroasiatic},},
%	editor = {Frajzyngier, Zygmunt},
%	pages = {85-105},
%	publisher = {Benjamins},
%	title = {Coding the unexpected. Subject pronouns in East Dangla},
%	year = {2008}
%}
%
%@book{Shimizu1983,
%	address = {Hamburg},
%	author = {Shimizu, Kiyoshi.},
%	publisher = {Helmut Buske Verlag},
%	title = {\textit{The Zing dialect of Mumuye. A descriptive grammar. With a Mumuye-English dictionary and an English-Mumuye index}},
%	year = {1983}
%}
%
%@article{Sibomana1981,
%	author = {Sibomana, Leo},
%	journal = {Tarok III: Das Verbalsystem und der Satz\textit{.} \textit{Afrika und Übersee}} ,
%	number = {2},
%	pages = {237–247},
%	title = {/82},
%	volume = {64},
%	year = {1981}
%}
%
%@book{Tucker1966,
%	address = {London},
%	author = {Tucker, Archibald N.  and  Bryan, Margaret A.},
%	publisher = {Oxford University Press},
%	title = {\textit{Linguistic analyses: The non-Bantu languages of north-eastern Africa}},
%	year = {1966}
%}
%
%@book{Vandame1963,
%	address = {Dakar},
%	author = {Vandame, Charles.},
%	publisher = {IFAN},
%	title = {\textit{Le Ngambay-Moundou. Phonologie, grammaire et textes}},
%	year = {1963}
%}
%
%Voigt, Rainer M. 1989. Verbal conjugation in Proto-Chadic.~In Frajzyngier, Zygmunt (ed.), \textit{Current progress in Chadic linguistics: Proceedings of the international symposium on Chadic linguistics, Boulder, Colorado, 1-2 May, 1987}, 267-284. (Vol. 62). Amsterdam: John Benjamins.
%
%@article{Voigt1987,
%	author = {Voigt, Rainer M},
%	journal = {The two prefix-conjugations in East Cushitic, East Semitic, and Chadic.~\textit{Bulletin of the School of Oriental and African Studies}~},
%	number = {2},
%	pages = {330-345},
%	volume = {50},
%	year = {1987}
%}
%
%\begin{styleHeader}
%@incollection{Vydrine2009,
%	address = {(Typological Studies in Language 87). Amsterdam and Philadelphia},
%	author = {Vydrine, Valentin F},
%	booktitle = {\textit{Negation Patterns in West African} \textit{Languages},},
%	editor = {Cyffer, Norbert \& Ebermann, Erwin \& Ziegelmeyer, Georg},
%	pages = {223-260},
%	publisher = {John Benjamins},
%	title = {Negation in southern Mande},
%	year = {2009}
%}
%\end{styleHeader}
%
%@article{Vydrine2011,
%	author = {Vydrine, Valentin F},
%	journal = {\textit{Studies in Language}} ,
%	number = {2},
%	pages = {409-443},
%	title = {Ergative/absolutive and active/stative alignment in West Africa},
%	volume = {35},
%	year = {2011}
%}
%
%Wilkendorf, Patricia 2001. \textit{Sketch grammar of Nɔmaándɛ: Sections 1-4.} Yaoundé: SIL.
%
%\end{verbatim} 


\printbibliography[heading=subbibliography,notkeyword=this]
\end{document}