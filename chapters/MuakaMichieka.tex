\documentclass[output=paper]{langsci/langscibook} 
\title{Humor in Kenyan comedy} 
\author{%
 Martha Michieka   \affiliation{ East Tennessee State University}\lastand 
 Leonard Muaka\affiliation{Winston Salem State University, North Carolina}
}
% \chapterDOI{} %will be filled in at production


\abstract{
In multiethnic and multilingual communities of Africa, speakers claim certain ethnic affiliations through their speech. As an identity marker, language further compartmentalizes speakers into certain groups and leads to attitudes, labeling, stereotyping and perceptions that go beyond language itself. Based on data drawn from Kenyan media discourse, this paper focuses on Kenyan ethnic humor to analyze absurdities, incongruity, and stylized speech by Kenyan humorists and their role in identity marking, boundary construction, performativity and personal experiences. The paper combines several approaches to examine how contemporary Kenyan humorists such as Eric Omondi and Byron Otieno style their Swahili and English discourses to index Kenyanness and Kenya’s diverse ethnolinguistic identities. The paper examines how this ethnic humor is perceived and what comedians do to ensure that their humor is not offensive to implied ethnic groups. The use of ethnic flavored discourse styles that strategically include and exclude different participants brings humor to the audience and helps downplay the stigma associated with certain styles of speech, ethnic practices, and marginalized ethnic groups. Given how ethnicity is intertwined with language, this paper argues that the use of incongruity, absurdities, and stigmatized speech forms by comedians in their jokes helps create humor and neutralizes the ethnic tensions that exist amongst different ethnic groups. 
}

\maketitle
\begin{document}

 

\section{Introduction: An overview of Kenyan humor}

Humor is understudied not only in Kenya, but in Africa in general. However, with a new generation of humorists driven by liberalized media, technology, and globalization, African humor is beginning to draw attention from researchers. In Kenya, comedy has been made possible due to several factors, but more importantly due to the political and socioeconomic changes in the last two decades. This paper seeks to highlight how the aforementioned changes have impacted Kenyan comedy and uniquely positioned it as Kenyan humor. 

In the following subsections, a short overview of humor in the Kenyan context is offered.
\section{Kenyanness and Kenyan humor}

What is considered “Kenyan” and consequently what is Kenyan humor? The Kenyan community consists of over~42 million~people belonging to more than 42 different ethnic groups. Generally, to be Kenyan is to belong to, or to identify with at least one of these ethnic groups. Most Kenyans are multilingual and can speak at least one ethnic language and either one or both of the Kenyan official languages (English and Swahili). When Kenyans of the same ethnic group meet, they are immediately connected through the shared language and cultural information. Most Kenyans can identify members of their ethnic groups through their easily identifiable accents as well as their shared names. Names such as Kamau, Wanjiku, Wamalwa, Kiprono, and Mogaka typically belong to particular Kenyan~ethnic groups, and Kenyans, whether~at home or in the diaspora, identify with those names because they indicate Kenyanness. There are also other unique ethnic group characteristics such as the choice of foods, lifestyles, and common economic, religious, and traditional activities that specific groups generally engage in. The shared knowledge~Kenyans have about various ethnic groups and each group’s unique characteristics make the idea of “Kenyanness” stand out.



Although ethnicity is clearly a mark of identity in Kenya, at the larger Kenyan community level, Kenyans have shared practices that extend across ethnic subgroups as well, that distinguish them from other African nationalities. This is the uniqueness that \citet[5]{Tumusiime2013} says Kenyan humorists tap into in order to indigenize their humor and make a career. Most Kenyans, for instance, are proficient in the Kenyan varieties of English and Swahili – varieties which are distinct from those spoken in neighboring countries. These language varieties unify Kenyans, as well as distinguish them from speakers of English and Swahili in neighboring countries. Unless one has been to Kenya and learned the Kenyan ways, he/she may not, for example, understand the typical Kenyan \textit{kitu kidogo }‘something small’, an expression that every Kenyan understands to mean a bribe. Apart from the unifying lingua franca(s), Kenyans also share traditions, practices and foods such as chapati, ugali, and nyama choma. These cross-Kenyan shared characteristics are also what Kenyan humor draws from and which, when performed, resonate with Kenyan audiences. As \citet{Chiaro1992} points out, the appreciation of jokes requires, among other things, shared cultural backgrounds.


\section{Locating Kenyan humor }

Humor is a universal phenomenon and it may even address universal themes and functions \citep[xiii]{NorrickChiaro2009}. However, certain humor trends unique to each society are influenced by sociopolitical and historical factors. Before being shared out in public, humorous acts can be said to have occurred within in-group members the same way riddles and stories are told. In the Kenyan case, humor has come of age and gone public largely due to a liberalized media sector that has broadened the platform of tapping into the local talent market. During the one-party political system period in Kenya, media waves were largely state controlled and therefore only a few comedians could display their potential and talent. However, following the introduction of multiparty politics that reduced media censorship and the emergence of entrepreneurship among young people, many media houses emerged and ended up creating opportunities for upcoming and talented comedians. Without a doubt, globalization had exposed many Kenyans to foreign humor that was no longer “magical” and many wanted authentic Kenyan humor. This mindset has inspired local artists, who are adored by Kenyans happy to embrace Kenyan pride and culture both locally and abroad. Consequently this trend has given local artists a ready market that allows them to reflect on their own society and tell their own stories. 

In essence, Kenyan humor appeals to many Kenyans because of the shared values between the~humorists~and their~audience and because the humor mirrors the lives of the audience. It finds its true meaning and identity in the multifaceted linguistic and cultural landscape of the Kenyan people. Kenyan accents, styles, cultural practices, and other practices, lead to identity marking that not only in-group members acknowledge, but also allows out group members to formulate stereotypes about them. Kenya’s in-group identity is dynamic and available at different levels and among different groups. Therefore, although humor in Africa shares certain similarities, Kenyan humor is unique to its people’s ways of life.

As a genre, the continued growth of Kenyan humor owes its present state to some of the early artists of the 1980s such as Amka Twende (Benjamin Otieno), Othorong’ong’o (Joseph Anyona), Mutiso (Kimunyo Mbuthia), Mzee Ojwang Hatari (Benson Wanjau), Mama Kayai (Mary Khavere), Baba Zero, Masanduku arap Simiti (Sammy Muya), Wariahe (Said Mohammed Said) and Otoyo Obambla (Samuel Mwangi) who featured prominently on \textit{Vitimbi}, a local Swahili comedy show (\textit{Passion For Life Magazine} July 2011). Although most of these comedians performed in Swahili, contemporary comedians tend to code switch between English and Swahili or utilize the urban variety of Sheng. Aside from television humorists, the Kenyan audiences were also entertained through print humor by artists such as Wahome Mutahi (Whispers), James Tumusiime (Bogi Benda), Cheka na Baraza,\footnote{ See \citet{Rhoades1977}.} and Juha Kalulu~(Edward Gicheri Gitau), the last two being in Swahili. These artists were particularly humorous because many Kenyans could identify with them. Mzee Ojwang, for example, though not Luo by ethnicity, performed on stage by stylizing his Swahili to project a Swahili speaker of Luo ethnicity entertaining all Kenyans by his parody of the Luo community.

\section{The state of Kenyan humor}

In the 1990s a trio of Kenyatta University students emerged on the entertainment scene impersonating President Moi and his cabinet to entertain Kenyans at a time when ridiculing the president was almost unheard of. The group known as Redykyulass was composed of Walter Mongare, Tony Njuguna, and John Kiarie. Political changes on the continent at the time favored multi-party politics and freedom of speech, consequently allowing political satire into the Kenyan media. Kenyans embraced this new way of entertainment, ushering in a new generation of comedians such as Churchill (Daniel Ndambuki), Eric Omondi and Byron Otieno that are now known regionally and internationally. 

Ndambuki’s entry onto the Kenyan comedy scene in the mid-1990s brought a kind of comedy that artists and the society took seriously, and it has since given rise to young talent that draws heavily on Kenya’s diverse ethnic groups. Ndambuki has nurtured many aspiring artists seeking to follow in the footsteps of the aforementioned pioneering comedians. Beyond the Kenyan scene and owing to globalization and regional collaboration, Kenyan humorists partner with their contemporaries in other African countries to entertain and build what can be referred to as African standup comedy. Kenyan comedians thrive on the diversity that exists in the country and as stated in \sectref{sec:muaka:1.3}, the themes that they focus on revolve around ethnolinguistic, cultural, socioeconomic, and political aspects of the society. While they may not come from the ethnic community they project, these comedians are able to capitalize on shared knowledge in their performative acts. As Ndambuki notes in an interview, Kenya’s middle class has converged toward Kenyan material as Kenyan artists seek to tell their own story and celebrate their different cultures, devoid of foreign material \citep{Kimani2014}. This is what has propelled Kenya’s showbiz, unlike in the past when Kenyans bypassed their local culture and entertainment for something new and foreign. 

\section{Conceptual framework}

This study focuses on Kenyan humor by analyzing absurdities, performativity, incongruity, and stylized speech in Kenyan humor and the role these elements play in identity marking and boundary construction. However, before we analyze Kenyan ethnic humor and how it is perceived, it is important to define two key terms, namely, ethnicity and ethnic humor as used in this paper.

\section{Ethnicity and ethnic humor}

Humor is a part of who we are as human beings, and although not everyone may be considered to be humorous, we all tend to understand what humor is in our culture. How do comedians ensure that their humor is socially acceptable, especially when it is ethnic centered? \citet[1]{Davies1990} notes that “the term ethnic tends to be used in a broad way about a group that sees itself and is seen by others as a ‘people’ with a common cultural tradition, a real or imagined common descent and a distinctive identity.” A group has to recognize in its members several factors that distinguish it from the rest of the communities around it. These factors should also be clear enough that outsiders notice the difference(s) and classify those who subscribe to this group as different. \citet[1]{Davies1990} further states that “an ethnic identity may be chosen or changed either by individuals or by the collective decision of a group, whether suddenly and deliberately or as a gradual result of other decisions … and the members of an ethnic group may collectively choose to assert or neglect the factors that set them apart from others.”. Ethnic humor, on the other hand, has been defined as a “type of humor in which fun is made of the perceived behavior, customs, personality, or any other traits of a group or its members by virtue of their specific sociocultural identity” \citep[108]{Apte1985}. The humorist can be either an outsider expressing superiority or an insider poking fun at self. 

This paper seeks to analyze Kenyan humor and establish what makes ethnic humor unique and funny. The term \textit{ethnic} will be used in reference to the slightly over 42 different ethnic groups mentioned above. Prior to addressing these questions, a cursory overview of the frameworks used in this paper is provided. 

\section{Theories of humor}

According to \citet{Raskin1985}, there are three major theories useful in the analysis and understanding of humor: superiority, incongruity, and release. In our analysis we apply these three theories as well as focus on how Kenyan comedians achieve humor without being offensive in a context that could otherwise be ethnically sensitive. Along with these concepts, we also take note of concepts such as performativity, stylization, and identity. 

\subsection{Superiority theory}

Proponents of the superiority theory view humor as an avenue through which one group pokes fun at a marginalized group and by so doing, laughs at the group’s folly, perhaps because it is not their own. \citet[8]{Carroll2014} notes that for superiority theorists such as Hobbes, people’s laughter results from perceiving infirmities in others which [then] reinforce our own sense of superiority. Such kind of humor tends to portray the targeted groups as “stupid, ignorant or unclean” \citep[115]{Apte1985}. However, superiority does not necessarily entail enmity. We laugh because we experience the joy of getting at the enemy and defeating them in some way. \citet[7]{Davies1990} maintains that “a further factor that underlies much humor and is brought out particularly well in ethnic jokes is the sense of sudden vicarious superiority felt by those who devise, tell or share a joke. Ethnic jokes “export” a particular unwanted trait to some other group and we laugh at their folly, perhaps glad or relieved that it is not our own”. Whether listeners laugh at such humor or find it offensive will depend on their level of group identity. As \citet[34]{Apte1987} claims, “if ethnic humor is initiated in a small group interaction where a listener’s ethnic group is the target, the listener may choose to ignore it or to affirm his or her identity and to protest, thereby either embarrassing the teller of the joke or putting that person on the defensive.” Labrador’s (2004) work shows that when humor is directed at others, it can be offensive. He explores humor in Hawaii where local jokes poke fun at “the Manong” (recently arrived Filippino immigrants), who are ridiculed not just for their language and perceived heavy Filipino accent but also for the foods they eat and how they socialize in their homes. As \citet[35]{Apte1987} notes, fury against negative humor is not unusual. For instance, “Politicians and other public figures who either intentionally or inadvertently engage in ridicule at the expense of some ethnic group have suffered serious consequences for their action.”

However, humor, even the superiority type, can serve various positive functions such as that of correcting social behavior. \citet[x]{Ziv1988} points out that we can achieve this social function by laughing “at forms of behavior or thought that are contrary to what is socially expected and accepted. Therefore laughter can have a punitive effect aimed at correcting behavior.” In this article, the superiority approach allows for an analysis of Kenyan humor by seeking to show how some ethnic groups may laugh at the folly of a disempowered group. The power relations involved are not just political but they may be linguistic, social, or cultural. 

\subsection{Incongruity theory}

Unlike superiority theory, incongruity theory attributes the source of amusement to the ambiguity or the unexpected ways of looking at a situation. Proponents of this approach believe there cannot be humor without incongruity. \citet{Berk1999} and \citet{Caroll2014}, for example, view incongruity as the most basic structure of humor consisting of at least two elements: The “expected content” and the “unexpected content” (see \citealt[7]{Berk1999}). When the unexpected is what the audience gets, humor becomes funnier. Therefore humor arises from the perception of an incongruity between a set of expectations and what is actually perceived.  \citet[308]{AttardoRaskin1999} argue that there is usually a punchline that “triggers the switch from one script to the other by making the hearer backtrack and realize that a different interpretation of the joke was possible from the very beginning.” In Kenyan humor, the unexpected twists in the selected acts are what make the jokes hilarious. 

\subsection{Release theory}

Release theory posits that there is a psychic release as a result of humor. As a type of self-deprecating humor, artists may present jokes that target their own follies or defects, as part of a defense or tension-relieving mechanism that is not necessarily intended to be hostile or aggressive \citep{Davies1990,Gulas2006}. According to \citet[30]{Apte1987}, “engaging in mild self-deprecatory humor is generally perceived as clear indication of having a sense of humor.” People who are able to laugh at themselves are perceived as having a good sense of humor and being easygoing. \citet[x]{Ziv1988} defines self-disparaging humor as those “instances in which we ourselves are ‘victims’ of the joke”. Similarly \citet[12]{Berk1999} observes that “self-effacing or self-deprecating humor in the form of \textit{self-downs} is not only an acceptable form but a highly desirable one to break barriers....” Some minority groups may even find strength in self-disparaging humor as that humor helps them rethink the values that bind them together as a community. \citet[172]{GonzalesWiseman2005} observe that ethnic humor directed at self is useful among minority groups since these small and marginalized groups will use this humor as “a means of trying to preserve their ethnic identity and tradition.” In a similar manner, \citet[298]{Labrador2004} argues that {local comedy and ethnic jokes are important sites for the practice and performance of~local identity and culture.” }This type of humor helps to construct identity among the affected people \citep{Labrador2004,Rappoport2005}; and as \citet[36]{Apte1987} observes, comedians who make fun of their ethnic groups are quite popular. 

In this article, these approaches are applied in the discussion of Kenyan humor. Each act is analyzed using an approach or a combination of approaches that best captures the intended humor. We also examine how comedians avoid being offensive. This article demonstrates that jokes occur in context and their effectiveness depends largely on shared knowledge and stereotypes. Therefore, the appreciation of ethnic humor depends on how much we know about and identify with the joke and its target and the shared background knowledge. 

\section{Data and data analysis}

Data used in this paper were gathered from Kenyan standup comedy clips from YouTube. These episodes were recorded in Kenya, Rwanda and Zambia. The main show, however, is recorded and produced in Kenya on the Churchill Show at Carnivore Restaurant. The Churchill Show has several videos that have been collected over time and for our analysis we chose over 50 online video clips from some of the most recent episodes. Since we cannot include audio files in this paper, the clips that are relevant to the current analysis are transcribed to help the reader understand the context as well as the analysis. 

\section{Superiority at the local and international level}

Ethnicity as defined above can be viewed within the larger Kenyan society as well as at the international level. In the Kenyan context, comedians entertain their audiences using jokes that poke fun at other Kenyan ethnic groups. At the international level, local ethnic differences are overlooked and the focus shifts to the idea of Kenyanness. Comedians’ identities continue to evolve depending on the audience as well as the context.

In the joke below, Majimoto, a humorist from the coastal region of Kenya, privileges his coastal people over the Meru people due the coastal people’s higher proficiency in Swahili. He imitates the way the Meru people speak Swahili to make his joke successful. What is also interesting is that he delivers this joke to an audience dominated by Meru speakers.
 
\ea
 Wajua mi natatizika sana nyie watu wa bara mwatatizika sana na Kiswahili. Halafu si ndio tumebarikiwa twajua Kiswahili sana. Ndio maana wale mapresenta wa ripoti za saa moja wengi ni wa Mombasa. Sasa wajua nini, kuna ripota mmoja ni rafiki wangu sana. Sasa tulikuja naye hapa town moja sijui inaitwa nini – akaripoti hapo. Anaanza tu kuripoti: “Tuko katika mji huu wa Chuka, kaunti ya Meru ambapo asubuhi ya leo katika bucha ya Bwana Karimi kumetokea wizi wa mabavu.” Wameru wakamkatisha: “Mbro tangaza ukweli. Hao watu hawajaiba mbavu, wameiba steak.” \textup{[Laughter]} \\
\glt ‘You know I am always troubled by the way you people from upcountry struggle with Swahili. And then for us we are blessed to know Swahili very well. That is why most 7:00 pm news reporters are from Mombasa. Now you know, there is one reporter who is a great friend of mine. Now I accompanied him to a town nearby whose name I can’t recall– and he was reporting. He begins to report: “We are in Chuka town, of Meru County where this morning at Mr. Karimi’s Butchery, there has been robbery with violence.” Meru people cut him short: “Brother, report the truth. Those people did not steal ribs, they stole steak.”’ (Churchill Show 2015)
\z

Here, Majimoto pokes fun at the pronunciation and limited Swahili proficiency of Meru people. The phrase \textit{wezi wa mabavu} which means ‘violent robbery’ is misunderstood by the Meru people to mean \textit{mbavu} ‘ribs’ and hence the rest of the joke. To the Meru people, who are known for their nasalized speech, \textit{Mabavu} and \textit{mbavu} are homophonous. It may be argued that the humorist is bragging about the coastal people’s high proficiency level in Swahili. Most of the coastal people speak Swahili natively while other Kenyans acquire Swahili as a second language. In this context, the superiority humor is due to the failure by the Meru speakers to decode a Swahili idiom, as native speakers would do. Majimoto’s joke therefore expresses the idea of one ethnicity being superior to another linguistically. Although Kenyans are multilingual, their levels of proficiency vary and they do not necessary seek to acquire the coastal Swahili variety.

Another humorist, Eric Omondi of Luo ethnicity capitalizes on shared local and international knowledge and stereotypes. In some cases his jokes concern competing ethnic communities such as the Luo and the Kikuyu, the most dominant ethnic groups in Kenya politically and, to some extent, economically. In the following joke, the humorist portrays Kikuyus as entrepreneurs whereas the Luos are portrayed as proud, snobbish, and arrogant people. Viewed through the lens of the superiority framework, this humorist privileges one group over another. 
 
\ea
  The Luos and Kikuyus … (are) very different, very unique, almost the same but very different. Like the Kikuyus, Kikuyu is the national language, and Luo is the international.\textup{ [laughter]} For Luos, money is not a problem and for Kikuyus, money is not everything; it is the only thing… \textup{[laughter]. }\\
  \citep{Quarshe2015} 
\z

As \citet{Apte1985} points out, there is nothing new that the comedians bring to the audience but rather it is how they repackage what is already known in form of jokes to entertain their audience that triggers excitement and laugher. Omondi portrays the Luo as a superior ethnic community. He introduces the two ethnic groups as being “almost the same but very different,” signaling the groups’ political dominance. But he also succeeds at privileging his Luo group over Kikuyus. If Kikuyus dominate at the national level, then the Luos are superior by dominating at the international level. Omondi further jokes that money is not a problem to the Luos; however to Kikuyus it is the only thing, thus satirizing the Kikuyus’ obsession with money. The wordplay and the unexpected twists further make the joke more hilarious. 

In yet another joke Omondi connects with his audience by portraying Luos  as superior and arrogantly so, living lavishly and driving the latest models of cars; while the Kikuyus are misers, saving every penny and driving old outdated cars. The humor comes not only from shared stereotypes about the lifestyles and boastful nature of the Luo people, but also from the humorist’s stylization of this conversation.

\ea
{Announcer speaks in English; Caller speaks in English with an unmistakably Luo accent}\\
You know I used to hear people say Luo is a lifestyle I never understood what it meant until two weeks ago a certain Luo man made a phone call to a certain radio station. \\
\textup{Caller}: Hello, we are on our way to Thika. We are going to support Gor Mahia. \\
\textup{Announcer}: OK sir, how many are you?\\
\textup{Caller}: Do not ask me how many we are; ask how many cars have we. \textup{[laughter]}\\
\textup{Announcer}: OK sir, I am sorry. How many cars do you have?\\
\textup{Caller}: Do not ask me how many cars have we, ask me which models? \textup{[laughter]}\\
\textup{Announcer}: OK I am really sorry, which model sir?\\
\textup{Caller}: We are having 30 Mercedes and 4 NZs. That means we have 30 Luos and 4 Kikuyus. \\
\citep{Quarshe2015}
\z

At the international level, the emerging African film industry, such as Nollywood in Nigeria and Riverwood in Kenya, gives Omondi a subject of entertainment. Nollywood films saturate the Kenyan market and although there is rivalry, artists from both West and East Africa have been collaborating in different entertainment ventures. When Omondi compares Kenyans and Nigerians he rises above the ethnic boundaries and views himself as a Kenyan, privileging Kenyanness over Nigerianess. At this level the projection is elevated to that of a Kenyan irrespective of the local differences that exist among certain Kenyan ethnic communities. He pokes fun at Nigerian etiquette and mannerisms. As \citet{Labraddor2004} notes, we poke fun at other people’s follies, thankful that we are not in their position. Omondi’s jokes privilege the groups that he belongs to and therefore it becomes “we” versus “them.” In example \REF{ex:muaka:4} below, Omondi constructs complex ethnic boundaries that portray a positive Kenyan identity compared to the Nigerian one that is vain.

\ea
{English }\\
  You know Nigerians, they talk big but when it comes to actions, there is a problem. Like I saw a Nigerian who went to a hotel and he was asking for everything, “waiter do you have fried chicken? Ok do you have pizza, pizza, big large pizza? Ok wait a minute, do you have milkshake? OK, do you have the big hamburger \textup{[hombaga]}? Ok if you have all those, in that case, give me a cup of tea. \\
  \citep{Omogi2012a}
\z

Additionally, in example \REF{ex:muaka:5} Omondi praises Kenyan women for their politeness and good manners as contrasted to Nigerian women. This performance can be viewed from both superiority and incongruity frameworks.

\ea
{English, with Nigerian portrayed as speaking in a heavy Nigerian-English accent}\\
Kenyan women are so kind and polite and patient. …Even Safaricom is one of our sponsors. They chose a Kenyan lady to be talking to us on the phone. When you call Safaricom, there is a lady who is very polite and very patient. She tells you the mobile subscriber cannot be reached. And you can call again and she tells you the mobile subscriber cannot be.… \textup{[The audience fills in).]} And you can call a million times and she will be so patient, she will keep telling you, the mobile subscriber cannot be reached. She will even go into Swahili and say, Samahani, mteja wa nambari uliyopiga hapatikani.” \textup{[Audience echoes.]} Sorry, the mobile subscriber cannot be reached.\\
Go to Nigeria; the correspondence in Nigeria, the lady from Nigeria, she tells you 
once! “The mobile subscriber cannot be found!” \textup{ [In a heavy Nigerian English accent]. [Laughter].}\\
Try calling again. “Hello, didn’t I just tell you the mobile subscriber cannot be found? \\
\textup{[Laughter]} What is your problem, stupid man?” \textup{[In a heavy Nigerian English accent].}\\
 \citep{Omogi2012b}
\z

Another humorist who compares ethnicities across national borders is Pablo from Uganda. He entertains his audience by portraying Kenyans’ inability to give directions. However, because of his desire not to antagonize the Kenyan audience, he resorts to inclusivity as seen in clip \REF{ex:muaka:7}, and softens the joke showing that Ugandans are not any different.

\ea
We all don’t know how to give directions. I met a young boy. He was around 9 years old and I asked him, “Where is the museum?” And the boy looked at me and told me “I don’t know.” And so as I was walking back the boy called me in the most local way \textup{(whistles)} and I knew that was meant for me because I am a VIP (very Improved Peasant). So I come back thinking that maybe the boy has remembered where the museum is, instead he showed me his friend and told me “This my friend, also he don’t know. 
\citep{Omogi2012c}
\z

\ea
 But it is not only in Kenya. It also happens in Uganda. You ask for directions at your own risk. 
 \citep{Omogi2012d}
\z

So while the first part of the joke was humorous due to its incongruity, there is a possibility of it being interpreted as offensive for portraying a sense of superiority by an outsider. However, the artist quickly returns to self-deprecation and places Kenyans and Ugandans on the same level. Unlike Omondi who does not make such apologies in his talk about Nigeria, maybe because Kenya and Nigeria are not neighbors and therefore offending Nigerians will not have any serious consequences, Kenya and Uganda are neighbors and offending one might have serious repercussions. The context favors Omondi, unlike Pablo whose act is in front of Kenyan audience.

\section{Incongruity}

Incongruity theory attributes humor to an unexpected twist. In this section we show examples of how comedians employ incongruity to create humor and entertain their audience. In example \REF{ex:muaka:8}, MCJessie ridicules televangelists who lie about receiving special messages from God.

\ea
Swahili spoken in a Kikuyu accent\\ 
Ningependa uangalie kwenye television, Mimi nimekuwa mchungaji kwa miaka twenty…. na Mungu amenionyesha pande hii ya reft kuna Mzee mmoja ako na miaka ninety five. Amekuwa akililia Mwenyezi Mungu kwamba haoni. Macho yake anafungua lakini anaona giza. Mungu ameniambia nikuulize. Mzee miaka ninety five, umeishi ninety five years, ni nini ingine unataka kuona.\\
\glt ‘I would like you to look at your TV screen [ addressing viewers], I have been a pastor for twenty years …. the Lord has shown me in this left side, there is a ninety five year old man who has been crying to the Lord that he can’t see. He opens his eyes but all he sees is darkness. The Lord has told me to ask you [now directing his message to the old man], “Old man, you have lived for ninety five years, what else do you need to see?”’  (NTV 2014a)
\z

Usually the audience will expect the preacher to bring some good news to the worshippers and so this joke takes a humorous twist when the preacher tells the old man that he has lived long enough and has seen everything there is to see and therefore he should not be bothering God anymore by asking for better eyesight. If the comedian didn’t add this unexpected turn or if he told the old man that God had promised sight, nothing could have been funny. 

In the joke below, Smart Joker and his friend parody robbers; and as robbers, they have a “constitution” to regulate their robbery. Their first rule states

\ea
{Swahili in a Luhya accent}\\
 Rule namba one kwa constitution ilisemanga hivi: Hakuna kuipia mama mja mzito… tumuachie nafasi azae mtoto apate laptop\\
\glt ‘Rule number one on the constitution states that: There should be no robbing of a pregnant woman… we should give her time to deliver so that the child can get a laptop.’ (Churchill Show 2014)
\z

Smart Joker entertains his audience by his strong western Kenya Luhya regional accent (Luhya ) that clearly portrays where he comes from. However, what makes the above joke funny is the twist in especially the second part. Smart Joker sounds very human when he states that they should not steal from a pregnant woman and the audience might just conclude that even robbers are understanding enough not to bother pregnant women. In the video, the audience does not laugh at the joke until Smart Joker adds the twist or the unexpected ending. It is then that the audience realizes they were wrong in thinking the robbers were being kind as most people generally are to pregnant women. Apparently the most valued item for the robbers is a laptop and they are so desperate for laptops that they are willing to wait until a child is born and grows to be of an age that he/she can be given a laptop at school which the robbers can then steal. It is shared knowledge among Kenyans that when the current Kenyan government was campaigning, they promised to give school-going children laptops. (Unfortunately the project did not take off.) 

\section{Release or self-deprecation}

As mentioned earlier, release or self-deprecating humor involves jokes that target one’s own follies. The examples in this section involve artists who expose some of the follies of their own ethnic groups and, because they belong to those groups, the jokes are considered self-deprecating. 

In the following example, another comedian Owago-Onyiro, laughs at his own “Luoness.”

\ea
{Swahili and English, with some Kikuyu}\\
   Ujaluo utakuua-\\
\glt ‘“Luoness” will kill you.’
\z

\ea
 Lazima upende mahali unatoka.\\
\glt ‘You have to appreciate your origins.’ 
\z

\ea
 People from Nyanza like prefixes. A prefix is a title that comes before the name: Dr Geofrey Otieno, Engineer Obado, Lecturer Omondi. People from Central also like something called suffixes. Suffix is a title that comes after the name: Mwangi wa Equity, Wanjiku wa makara.\\
\glt ‘Mwangi of Equity’ [equity bank], ‘Wanjiku of Makara’ [‘charcoal’ in Kikuyu, or ‘Wanjiku who runs the charcoal business’] ({Churchill Show 2013})
\z

This comedian addresses several issues and, since he is a Luo, he can poke fun at the Luo lifestyle without worrying about the consequences. He is clearly making fun of the Luos for their arrogance and their love of lengthy titles. However, since he is Luo, he has the right to say whatever he wants about himself and his people. This can be likened to the situation in the USA where it is ok for an African American to use the N-word but it would be considered inappropriate if someone from a different race used the same term. It is also clear that through Onyiro’s jokes, the Kikuyu ethnic group is associated more with business ventures, unlike the Luo’s lifestyle. 

In this next joke, Omwami is supposedly Luhya as indicated by his name and he jokes freely about the Luhya people.

\ea
{Swahili with Luhya features}\\
 Unajua Mluhya ni mtu anakutusi lakini vile amepackage hiyo matusi uwezi jua amekutusi. Mluhya anaweza sema, ah, “Bosi yaani umefika, asante sana kwa kufika show... \textup{[stretching out his hand to give a handshake]} kamatako hii.\\
‘You know a Luhya can curse you but the way the curse is “packaged” you won’t realize you have been cursed. A Luhya can say to you, “Boss, so you made it here, thank you for coming to the show… [high five]’ (NTV 2014b)
\z

The wordplay here is the twist that delivers the humor to an audience that shares in the language stereotypes. There is a tendency for Luhya dialects to add a particle or suffix like \textit{ko} and \textit{nga} to a conjugated verb. When \textit{ko} is added to the verb \textit{kamata}, the new word created could be a dimunitive for ‘butt’. Consequently in Omwami’s joke, the high five could also mean ‘you little butt…’

Kenyan audiences laugh at the Luhya speech, but at the same time they realize that such jokes empower the Luhya speakers. If the speaker chooses to, s/he can use the language to attack his boss; but since the listeners know that Luhyas often add that suffix to their words, nobody can punish the speaker. The artist being Luhya himself has the right to make fun of his own people without offending them. 

In the next example, Majimoto this time pokes fun at his community that is often considered less educated and limited in its knowledge of English. As a coastal resident, he pokes fun at his own Mombasa people because he is one of them. This will not be an easy joke for an outsider to make considering that the people from this area are limited in their English proficiency due to marginalization and unfair allocation of public resources. Moreover, this is supposedly a joke about his own mother. What the audience finds funny is the misinterpretation of the phrase \textit{high class} by the mother. \textit{High class} is used here to mean ‘sophisticated /classy’ but the mother understands it to mean an upper grade in school. 

\ea
 Ikabidi nimchukue bwana mtoto nikampeleke nyumbani. Sasa mtoto wa Kikamba ikabidi nikamintroduce kwa mother, unaona. “Mama waona bwana nakuletea mkaza mwana – mkaza mwana sasa ni bibi ya mtoto wako. Sasa nakuletea mkaza mwana mtoto wa kikamba, mtoto high class. Mama nakuletea mtoto wa high class.” Mama aliposikia mtoto wa “high class,” mama akaanza kunisomea: “utafungwa wewe, utafungwa wewe yaani mtoto bado yuasoma yuko high class wewe wamchukua wamleta nyumbani wataka kumfanya bibi”.\\
\glt ‘It was necessary I take her home. Now, the Kamba girl, I had to introduce her to my mother, you see. “Mother you see this is your daughter in-law – a daughter in law is a wife of your son/child. Now I am bringing to you a daughter in law, a Kamba girl, a high-class girl. Mother, I bring you a high class girl.” When mom heard “a high class girl” she started lecturing me, “you will be jailed, you will be jailed; the girl is still in school in “high class” and you are taking her and bringing her home to be your wife”.’ (NTV 2015).
\z

\section{How do humorists avoid being offensive? }

In our analysis we notice a number of tactics the comedians employ to avoid being offensive especially to other ethnic groups. Some of the common tactics they use include self-effacing, inclusivity, outright apology, and careful explanation of their joke.

\subsection{Self effacing}

As we have seen in the examples above, most artists choose to poke fun at themselves or at their own ethnic groups instead of laughing at other groups. Luo comedians such as Omondi and Owago make fun of Luo practices, while Luhya comedians like Omwami freely joke about their Luhya follies. Instead of outsiders poking fun at the Luhya or Luo accents in spoken English, the comedians from those groups take it upon themselves to bring this out in the jokes. Owago, for example, discusses the issue of “shrubbing” among Luos. \textit{Shrubbing} (or \textit{srub}) is a Sheng word which refers to the unintentional [${\int}$]/ [s] substitution characteristic of many Luo speakers of English or Swahili as second languages.

\ea
{English with Sheng code-mixing (in bold)}\\
 You say haa you Luos you \textbf{srub}. \textbf{Mimi ninasrub}? Be careful if you have a friend from Nyanza. Luos only \textbf{srub} what they don’t like. \textbf{Kama sida} –\textup{[laughter]}…\textbf{mandasi} … \textbf{shitikashel} \textup{[more laughter]}... \textbf{aishishi} . \textbf{We maisani mwako ushawai ona mjaluo akisema Mashidishi ama Range Rover Shport}?\\
‘You say that haa you Luos you shrub. Do I shrub? Be careful if you have a friend from Nyanza. Luos only shrub what they don’t like. Things like problems (standard pronunciation [ʃida])…difficulties… City Council (standard [sitikasel])… ICC (standard [aisisi]). In your entire life, have you ever heard a Luo say “Mashdishi” (< Mercedes) or “Range Rover Shport”? (< Sport)?”’ \citep{Churchill2013c}
\z

Here Owago, a Luo comedian, pokes fun at himself and his fellow Luos by performing and stylizing an embarrassing language aspect that many Luos struggle with. However, he makes it light and even empowers his people by arguing that their pronunciation mistakes are deliberate and intentional. To Owago, it is not necessarily a pronunciation problem that they struggle with, but rather something they have power over and can choose to avoid whenever they want.

\subsection{Inclusivity }

Some comedians avoid being offensive by finding a way to include themselves in the group they are poking fun at. They do this either by showing that they are good friends and are therefore in agreement with what that particular group does, or by showing that that particular folly is widespread and not just limited to one specific group. Pablo from Uganda in example \REF{ex:muaka:7} above employs inclusivity. In example \REF{ex:muaka:14} below, Omondi begins his joke by introducing Kikuyus as friends. 

\ea
{English with Swahili code mixing (in bold)}\\
I love Kikuyus, I love Kikuyus because they are very, they are very entrepreneurial, they are very business minded yeh. \textbf{Kama} last Sunday I was with a Kikuyu friend in church; \textbf{nimekaa hapa, beste yangu amekaa hapa}. My tight friend. \textbf{Pastor akakuja akasema,} “And Jesus went to heaven to make each one of us a mansion. Everyone will have his own mansion.” \textbf{Huyo Mkikuyu akaniangalia akaniambia, “Eriko si tukifika heaven tuishi kwa moja turent hiyo ingine?}” That is a business mind. \\
\glt ‘I love Kikuyus, I love Kikuyus because they are very, they are very entrepreneurial, they are very business minded, yeh. \textbf{Like} last Sunday I was with a Kikuyu friend in church; \textbf{I am seated right next to my best friend}. My tight friend. \textbf{The pastor said}, “And Jesus went to heaven to make each one of us a mansion. Everyone will have his own mansion.” \textbf{That Kikuyu turned to me and said, “Eric, when we get to heaven, can’t we just stay in one of the mansions and rent out the other one?}” That is a business mind.’ \citep{Quarshe2015}
\z

In this case Eric Omondi, who is from a different ethnic group, pokes fun at the extreme love of money expressed by Kikuyus; but he makes it less offensive by introducing his joke with the fact that he loves Kikuyus, and the person he is talking about is a very close friend of his – “a tight friend”$-$ so close that his joke cannot be intended to hurt or ridicule him.

\subsection{Outright apology and careful explanation of their joke }

Sometimes the jokes are too satirical and the artists resort to outright apology or justification for their choice of words. 

In this next joke Mammito talks about human beings having been created from various types of soil: loam, clay, and sand. She then goes on to say:

\ea
{Swahili with English code-mixing}\\
Mafans wa Gor Mahia hao hawakutengenezwa na mchanga, hao walikaviwa kutoka kwa mawe. \textup{[Loud laughter.]} Hao ndio unasikianga… Siyo kwa ubaya. Mawe ni kitu ya... tunasikianga Yesu ni mwamba. Mawe ni kitu ya muhimu sana. Wafans wa Gor Mahia ndio mnasikianga stone age people, stone age people.\\
\glt ‘Gor Mahia fans were not created out of soil but they were carved out of rocks. Those are the people… No ill intended. Rocks are very important… we talk of Jesus being a Rock. Rocks are very important. Gor Mahia fans are what you often hear referred to as Stone Age people, Stone Age people.’ (NTV 2014c)
\z

In this clip the artist wants to poke fun at the Gor Mahia fans who are mainly of the Luo ethnic group. Gor Mahia is one of the major Kenyan soccer leagues. Many Kenyans know that whenever this team loses a game, there is usually a huge fracas and its fans end up throwing stones at the opposing fans. Mammito, however, tries to defend her use of the term \textit{mawe }‘rock’ here by even calling attention to the fact that Jesus is referred to as the rock of salvation. She is, however, very intentional about discussing the disgusting practice of throwing rocks at opponents. It is a backward practice and people who do that belong to the Stone Age.

While Mammito tries to explain this joke and sounds apologetic to show her sensitivity to local ethnicity, she doesn’t do the same for groups that will not pose danger. 
 
\ea
 Wazungu, hao walitengenezwa kutoka kwa sand, sand ni ngumu kumold, ndio hawananga shape \\
\glt ‘Europeans, they were created from sand, sand it is difficult to mold, and that is why they don’t have a shape.’ (NTV 2014c)
\z

Apparently since there are hardly any Europeans in her audience, Mammito does not feel obligated to offer any apologies. However, the analogies she creates are very humorous and depend on shared knowledge. 

\section{Conclusion}

While the Kenyan humor industry is still in its developmental stages, there is every evidence that it is a growing field and Kenyans are rapidly learning to appreciate their own humor and to reward their artists. The humor captures the ethnolinguistic diversity inherent in the Kenyan society and shows that the humorists are fully aware of this diversity and are very cautions not to intentionally offend specific ethnicities. Kenyans are also learning to laugh at their follies, not necessarily because they are proud of them but because these follies define who they are as Kenyans. This humor seems to be a way forward for Kenyans to reflect on various issues and maybe serves as a channel for correcting socially unacceptable behavior without causing too much pain and repercussions.

The jokes show a clear interface between language ideologies that shape people’s language attitudes. It is evident that speakers use language, dialects, and accents as identity markers. \\

% \section*{Abbreviations}
% \section*{Acknowledgements}

\printbibliography[heading=subbibliography,notkeyword=this]

\end{document}