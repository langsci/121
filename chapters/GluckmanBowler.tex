\documentclass[output=paper]{langsci/langscibook} 
\title{The anticausative alternation in Luragooli} 
\author{% 
 John Gluckman\affiliation{UCLA}\lastand 
 Margit Bowler \affiliation{UCLA}
}
% \sectionDOI{} %will be filled in at production
\abstract{We discuss the distribution of the verbal suffix \textup{-Vk} in Luragooli (Luyia, Bantu) based on original fieldwork with a native speaker. We show that \textup{-Vk} patterns like an anticausative marker with respect to a number of different diagnostics, including licensing of theta-roles and interaction with lexical aspect. We compare Luragooli to other languages with anticausative morphology and identify different classes of verbs based on their behavior with the \textup{-Vk} suffix.} 

\maketitle
\begin{document}


\section{Introduction}

This paper addresses the distribution and meaning of the Luragooli (Luyia, Bantu) verbal suffix \textit{-Vk} (variously realized as -\textit{ek}, -\textit{ik}, -\textit{ok}, and -\textit{uk}).\footnote{Luragooli (also called Maragoli, Logoori, Lulogoori, and Logooli) is a Bantu language in the Luyia subfamily, spoken by approximately 618,000 people in Kenya and Tanzania \citep{LewisEtAl2015}.} This suffix occurs in a number of Bantu languages, including Chichewa \citep{Mchombo1993,Simango2009} and Swahili \citep{SeidlDimitriadis2003}. \textit{-Vk} constructions in these languages have variously been called statives, middles, neuter-passives, quasi-passives, anticausatives, and intransitivized constructions, among others \citep{Mchombo1993,DubinskySimango1996,SeidlDimitriadis2003,Fernando2013}. For now, we will refer to and gloss the suffix with the neutral term \textit{-Vk}.

The primary goal of this paper is to attain a descriptively adequate account of \textit{-Vk} in Luragooli by addressing the following research questions through original fieldwork with a native Luragooli speaker:

\begin{itemize}
\item What is the distribution of \textit{-Vk}?
\item What meaning(s) is/are associated with the use of \textit{-Vk}?
\end{itemize}

Based on the answers to these two questions, we suggest that Luragooli \textit{-Vk} should be analyzed as an anticausative suffix. That is, \textit{-Vk} can be treated as a marker of intransitivity analogous to the English anticausative in \REF{ex:gluckman:1b} below. While in English there is no morphological difference between the causative and anticausative forms of the verb \textit{break} (cf. \ref{ex:gluckman:1a} and \ref{ex:gluckman:1b}), in Luragooli the difference between the two is marked by presence versus absence of the –V\textit{k} suffix, as in (\ref{ex:gluckman:2}a-b):


\ea\label{exx:}
%%1st subexample: change \ea\label{...} to \ea\label{...}\ea; remove \z  
%%further subexamples: change \ea to \ex; remove \z  
%%last subexample: change \z to \z\z 
\ea
\langinfo{}{}{  English}\\
 a.  John broke the vase.~ ~ ~ ~ ~       (lexical) causative \\
 b.  The vase broke. ~ ~ ~ ~ ~ ~       anticausative \\
 \z
\z

\ea
\ea
%%1st subexample: change \ea\label{...} to \ea\label{...}\ea; remove \z  
%%further subexamples: change \ea to \ex; remove \z  
%%last subexample: change \z to \z\z 
\langinfo{}{}{  Luragooli}\\
\gll a.  Sira a-han-i       muriaŋgo.~ ~ ~ ~     (lexical) causative\\
       Sira 1-close-\textsc{fv} 3door\\
\glt ‘Sira closed the door.’
\ex
\gll b.  muriaŋgo gu-han-ik-i.~ ~ ~ ~ ~     anticausative\\
       3door ~3-close\textit{-}\textsc{Vk}-\textsc{fv}\\
\glt ‘The door closed.’
\z
\z

We show that the distribution and use of \textit{-Vk} pattern similarly to cross-linguistic diagnostics for anticausative markers. However, we also present a number of further uses that fall outside of the characteristic anticausative domain. It is therefore a matter of ongoing theoretical research as to whether these functions can be subsumed under the anticausative use.

This paper is organized as follows. We present a brief overview of the core anticausative alternation in \sectref{sec:gluckman:2}. The remainder of the paper focuses on how the \textit{-Vk} form differs from “plain” intransitives and from valency decreasing processes like passivization. In \sectref{sec:gluckman:3} we look at what sorts of oblique theta-roles are permitted with each of these three types of intransitive verb stem. In \sectref{sec:gluckman:4} we look at how the –V\textit{k} suffix interacts with Aktionsart, or lexical aspect, and show that \textit{-Vk} anticausatives correlate with a telic reading of the event. In \sectref{sec:gluckman:5} we detail two sub-classes of \textit{-Vk} intransitives which challenge our typology. \sectref{sec:gluckman:6} concludes the paper. 

\section{Background on anticausatives}

The examples in \REF{ex:gluckman:1} above show the anticausative alternation in English. The verb \textit{break} can appear in a transitive construction where the \textsc{patient} is a direct object \REF{ex:gluckman:1a} or in an intransitive construction where the \textsc{patient} is the subject \REF{ex:gluckman:1b}. Cross-linguistically, it is generally true that a verb like \textit{break} can have both a causative and an anticausative form.\footnote{Throughout the literature on transitivity alternations, there is a great deal of variation in terminology. Here we follow the terminology of \citet{Schäfer2008,Schäfer2009} and \citet{AlexiadouEtAl2015}. We use the term \textit{causative (verb)} to refer to any transitive verb which is semantically reducible to \textsc{cause}-verb. We will not go into detail on the various possible causative forms in Luragooli. See the appendix in \citet{GluckmanBowler2015} for an overview of these forms. We use the term \textit{anticausative (verb)} to refer to a non-passive intransitive use of a causative verb. If an anticausative form uses special morphology, we refer to this as a \textit{marked anticausative}. Since we aim to be as neutral as possible in our terminology classification, where possible we will try to use the terms \textit{transitive} instead of \textit{causative}, and \textit{marked} and \textit{unmarked intransitive} instead of \textit{marked} and \textit{unmarked anticausative.} We suggest the reader consult \citet{Schäfer2008,Schäfer2009} and \citet{AlexiadouEtAl2015} for a more substantive debate about terminological issues.} Likewise, it is generally true that a verb such as \textit{bloom} \REF{ex:gluckman:3a} tends not to have a (simple) causative counterpart \REF{ex:gluckman:3b}.

\ea\label{exx:}
%%1st subexample: change \ea\label{...} to \ea\label{...}\ea; remove \z  
%%further subexamples: change \ea to \ex; remove \z  
%%last subexample: change \z to \z\z 
\langinfo{}{}{English}\\
\ea
 The flower bloomed.        \jambox{anticausative}
\ex
  *The sun bloomed the flower.      \jambox{causative}
 \z
 \z
 
What allows a given verb to have an (anti-)causative counterpart is a matter of ongoing research (\citealt{Smith1970,Haspelmath1993,LevinRappaportHovav1995,Reinhart1996,Folli2002,Folli2005,AlexiadouAnagnostopoulou2006,Schäfer2008}; among others). One of the core debates concerns the number of event subcomponents associated with each form in \REF{ex:gluckman:1}. One influential proposal is that anticausatives lack a \textsc{cause} semantic sub-event \citep{Haspelmath1993}, and hence, a \textsc{causer/agent} argument which brings about the result state. Under this approach, the anticausative verb \textit{break} in \REF{ex:gluckman:1a} means, essentially, ‘the vase \textsc{became} broken’, while the causative verb \textit{break} in \REF{ex:gluckman:1b} contains a \textsc{cause} event: ‘John \textsc{caused} the vase to \textsc{become}-broken.’\footnote{The second core debate concerns the derivational relationship between the forms in \REF{ex:gluckman:1}. We will not be concerned with this issue here.}

Other proposals argue that causatives and anticausatives are identical in terms of the number of sub-events, and differ in, essentially, thematic structure determined by factors other than types of sub-events. According to these proposals, both verbs in \REF{ex:gluckman:1} have the meaning ‘\textsc{cause}-break’ (or ‘\textsc{cause-become}-broken’).\footnote{Among other things, this allows for the possibility of a spontaneous internal \textsc{cause.}} However, the verbs differ in whether or not they encode an external force which acts to bring about this event, i.e., an \textsc{agent} (or \textsc{instrument}) of the \textsc{cause} sub-event \citep{LevinRappaportHovav1995,Schäfer2008}. That is, while \REF{ex:gluckman:1b} encodes that \textit{John} is the\textsc{ agent} that brings about the \textsc{cause} event of the vase’s breaking, \REF{ex:gluckman:1a} does not encode reference (explicitly or implicitly) to such an argument. 

Our study of Luragooli is consistent with this second hypothesis. We contend that Luragooli intransitive verbs with \textit{-Vk} contain a \textsc{cause} event, but lack an external argument which brings this event about. That said, this paper aims for descriptive coverage; as such, we do not take a strong theoretical stance. Our study starts from the well-known typological observation that languages can differentiate between three classes of verbs that display anticausative alternations, i.e., intransitive “causer-less” forms \citep{Haspelmath1993,Schäfer2008}:\footnote{This three-way classification is reported to reflect a scale of “spontaneity,” or how likely it is that the event is perceived as needing an external force to bring it about \citep{Haspelmath1993}. Verbs without the marker are expected to be less likely to require an external effort (that is, they are “internally caused”), while verbs with the marker are perceived as requiring some external force to make the event occur.} 
\begin{itemize}
     \item[\textbf{  Class I}]: Intransitive forms that need a special anticausative marker\
     \item[\textbf{  Class II}]: Intransitive forms that cannot have an anticausative marker\\
     \item[\textbf{  Class III}]: Intransitive forms that can optionally have an anticausative marker\\ 
\end{itemize}

     
Such a partitioning is also present in Luragooli. Three classes of verbs can be distinguished based on how the intransitive version of an otherwise syntactically transitive verb is morphologically expressed:

\begin{itemize}
     \item[\textbf{  Class I}]: Intransitive forms that must occur with \textit{-Vk}\\
     \item[\textbf{  Class II}]: Intransitive forms that cannot occur with \textit{-Vk}\\
     \item[\textbf{  Class III}]: Intransitive forms that optionally occur with \textit{-Vk}\\
\end{itemize}

     
We give examples of each of these verb classes in \tabref{tab:1}.\footnote{The forms in \tabref{tab:1} bear the infinitival prefix \textit{ku}-/\textit{kw}- and the \textit{–a} final vowel. For a more complete list of all the verbs of the types discussed in this study, see the appendix in \citet{BowlerGluckman2015}.}

\begin{table}
\caption{Intransitive verb classes in Luragooli based on the distribution of -Vk}
\label{tab:1}

\begin{tabularx}{\textwidth}{XXX}
\lsptoprule

\textbf{Class I} (\textit{intransitive with -Vk}) & \textbf{Class II} (\textit{intransitive without -Vk}) & \textbf{Class III} (\textit{intransitive with or without -Vk})\\
\textit{kwoneka} ‘to be destroyed’ & \textit{kwigora} ‘to open’ & \textit{kuhana/kuhaneka} ‘to close’\\
\textit{kubameka} ‘to be flattened’ & \textit{kumeeda} ‘to increase’ & \textit{kwiina/kwiineka} ‘to sink’\\
\textit{kuzuganyika} ‘to be mixed’ & \textit{kugomagoma} ‘to roll’ & \textit{kwoma/kwomeka} ‘to dry’\\
\textit{kuharagateka} ‘to be scraped’ & \textit{kumera} ‘to grow’ & \textit{kuzurula/kuzuruleka} ‘to wilt’\\
\lspbottomrule
\end{tabularx}
\end{table}


Thus, at first glance, \textit{-Vk} seems to pattern as we might expect for an anticausative marker. In \sectref{sec:gluckman:3}-4 we review further parallels in Luragooli to anticausative alternations that have been observed cross-linguistically. We begin \sectref{sec:gluckman:3} by introducing the basic distribution of \textit{-Vk} in Luragooli.

\section{The distribution of \textit{-Vk} in Luragooli}

The suffix \textit{-Vk} attaches to certain transitive verbs (roots or stems) to form non-passive intransitives, i.e., anticausatives. For instance, the Luragooli transitive verb \textit{kuhana} ‘to close’ can appear as a (non-passive) intransitive in two different ways:

\ea\label{exx:}
%%1st subexample: change \ea\label{...} to \ea\label{...}\ea; remove \z  
%%further subexamples: change \ea to \ex; remove \z  
%%last subexample: change \z to \z\z 
\ea
\langinfo{}{}{Causative}\\
\gll Sira    a-han-i      muriaŋgo.      \jambox{causative}\\
     1Sira 1-close-\textsc{fv}  3door\\
\glt ‘Sira closed the door.’
\footnote{Luragooli has an extremely complex tense/aspect system, which we largely ignore in this paper. We also do not mark any tones. Luragooli is analyzed as having 2 tones (high and non-high). See \citet{SamuelsPaster2015} for a description of the Luragooli (verbal) tonal system.}
\z
\z

\ea\label{exx:}
%%1st subexample: change \ea\label{...} to \ea\label{...}\ea; remove \z  
%%further subexamples: change \ea to \ex; remove \z  
%%last subexample: change \z to \z\z 
\langinfo{}{}{Anticausative}\\
\ea
\gll muriaŋgo gu-han-i        \jambox{plain intransitive}\\
       3door       3-close-\textsc{fv}\\
\glt ‘The door closed.’
\ex
\gll muriaŋgo gu-han-ik-i.      -\textup{Vk} \jambox{intransitive}\\
       3door         3-close-\textsc{Vk}-\textsc{fv}\\
\glt ‘The door closed.’/‘The door was closed.’
\z
\z

Example \REF{ex:gluckman:5a} is consistently translated as ‘The door closed.’ We refer to this form as the \textit{plain intransitive}.However, \REF{ex:gluckman:5b} is translated more frequently as ‘The door was closed.’ We refer to this form as the \textit{-Vk intransitive} or \textit{-Vk form}. Curiously, the English passive translation in \REF{ex:gluckman:5b} is available despite the fact that the verbal passive suffix \textit{-w} is not present. We take this as initial evidence that \textit{-Vk} makes a semantic contribution in addition to its syntactic contribution of detransitivization. The question that this data raises is exactly how to define this semantic contribution. In the remainder of \sectref{sec:gluckman:3}, we investigate this question by looking at how oblique theta-roles interact with intransitives in Luragooli. We conclude that the Luragooli \textit{-Vk} form patterns similarly to what is reported for anticausatives cross-linguistically.

\subsection{Oblique theta-roles} %3.1 /

Our first set of diagnostics concerns the interaction of the three intransitive classes (\tabref{tab:1}) with oblique theta-roles. Anticausatives interact with theta-roles cross-linguistically in consistent ways. Anticausatives generally do not permit \textsc{agents} or \textsc{instruments} in oblique phrases \citep{LevinRappaportHovav1995}. However, they do tend to license \textsc{causers} in oblique phrases (Schäfer 2008).\footnote{For descriptions of the relevant theta-roles, see Levin \& Rappaport-\citet{Hovav1995}. Also, the reader should be aware that the properties reviewed here are robust crosslinguistic trends, but are not universally true, even within a single language. In this section, we present these diagnostics merely to establish that both intransitive \textit{-Vk} forms are distinct from a true passive in the relevant respect.} This differentiates them from passives, which generally permit \textsc{agents} in oblique phrases, but do not permit \textsc{instruments} or \textsc{causers}.

For example, German passives generally permit \textsc{agents (6}a) but not \textsc{causers (}i.e. \textsc{forces}) in oblique phrases\textsc{} \REF{ex:gluckman:6b}, while German anticausatives permit \textsc{causers} (\textsc{forces}) but not \textsc{agents} (6b, 6c).\footnote{{} German examples are adapted from \citet{Schäfer2008}. For reasons of space, we present a slight oversimplification of the data in that there is variability with respect to which verbs permit which oblique theta-roles. We find the same complexities in Luragooli as well.}\textsuperscript{}  Note that German anticausatives have an unmarked and marked form; the latter is accomplished with the reflexive \textit{sich}, as in \REF{ex:gluckman:6c}.

\ea\label{exx:}
%%1st subexample: change \ea\label{...} to \ea\label{...}\ea; remove \z  
%%further subexamples: change \ea to \ex; remove \z  
%%last subexample: change \z to \z\z 
\langinfo{German}{}{\citep{Schäfer2008}}\\
\ea\label{ex:}
\langinfo{}{}{\textbf{passive}}\\
\gll   Die Tür wurde von Peter/??vom      Windstoß  geöffnet\\
       the door was    by   Peter/by.the  wind.gust   opened\\
\glt ‘The door was opened by Peter /??by the gust of wind.’

\ex\label{ex:}
\langinfo{}{}{\textbf{unmarked anticausative}}\\
\gll   Das Segel zerriss (*von Peter/durch den Sturm)\\
       the  sail    tore        by    Peter/through the storm\\
\glt (*‘The sail tore by Peter.) / ‘The sail tore from the storm.’

\ex\label{ex:}
\langinfo{}{}{\textbf{marked anticausative}}\\
\gll   Die Tür   öffnete sich (*von Peter/durch einen Windstoß)\\
       the door opened \textsc{refl}  by   Peter/through a wind-gust\\
\glt (‘The door opened by Peter.’)/‘The door opened from a gust of wind.’
\z
\z

Note that the availability of a \textsc{causer} (or \textsc{force}) semantic role is not predicted under the proposal that anticausatives lack a\textsc{ cause} event (\citealt{Haspelmath1993}, among others). This type of data is therefore used by \citet{Schäfer2008}, among others, to argue that anticausatives do contain a \textsc{cause} event, but do not encode (in their terms, “license”) an \textsc{agent} which brings this event about. 

The following three sub-sections show how oblique theta-roles combine with the three types of intransitives in Luragooli: passives, plain intransitives, and \textit{-Vk} intransitives. The plain and \textit{-Vk} intransitives pattern similarly with respect to the theta-roles they license in oblique phrases, as we expect from anticausatives.
\todo{there is only 3.1, but no 3.2.}

\subsubsection{Oblique \textsc{agents}} 

In this section, we determine whether an oblique \textsc{agent} (that is, an agentive ‘by’-phrase) is permitted with each type of Luragooli intransitive construction. Oblique \textsc{agents} are permitted only with the Luragooli passive \REF{ex:gluckman:7a}. Oblique agents are not permitted with either the \textit{-Vk} intransitive \REF{ex:gluckman:7b} or the plain intransitive \REF{ex:gluckman:7c}:

\ea\label{exx:}
%%1st subexample: change \ea\label{...} to \ea\label{...}\ea; remove \z  
%%further subexamples: change \ea to \ex; remove \z  
%%last subexample: change \z to \z\z
  \ea
  \langinfo{}{}{\textbf{passive}}\\
  \gll muriaŋgo gu-han-w-i            (na Sira).\\
      3door       3-close-\textsc{pass}-\textsc{fv}    by  Sira\\
  \glt ‘The door was closed (by Sira).’
  \ex
  \langinfo{}{}{\textbf{ plain intransitive}}\\
  \gll muriaŋgo gu-han-i   (*na Sira).\\
      3door      3-close-\textsc{fv}   by   Sira\\
  \glt ‘The door closed (*by Sira).’
  \ex
  \langinfo{}{}{\textbf{\textit{-Vk}}\textbf{ intransitive}}\\
  \gll muriaŋgo gu-han-ik-i     (*na Sira).\\
      3door       3-close\textit{-}\textsc{Vk}-\textsc{fv}~ by  Sira\\
  \glt     ‘The door closed (*by Sira).’\\
  \z
\z
     
\subsubsection{Oblique \textsc{causers}}

Oblique \textsc{causers} are permitted with both the plain intransitive \REF{ex:gluckman:8b} and the \textit{-Vk} intransitive \REF{ex:gluckman:8c}, but not with the passive \REF{ex:gluckman:8a}:

\ea\label{exx:}
%%1st subexample: change \ea\label{...} to \ea\label{...}\ea; remove \z  
%%further subexamples: change \ea to \ex; remove \z  
%%last subexample: change \z to \z\z 
  \ea
  \langinfo{}{}{\textbf{passive}}\\
  \gll muriaŋgo gu-araminy-w-i  (*kutorona na imboza).\\
      3door 3-open-\textsc{pass}-\textsc{fv}      from   by 9wind\\
  \glt ‘The door was opened (*because of/from the wind).’
  \ex
  \langinfo{}{}{ \textbf{plain intransitive}}\\
  \gll muriaŋgo gu-aram-i (kutorona na imboza).\\
      3door 3-open-\textsc{fv}    from by 9wind\\
  \glt ‘The door opened (because of/from the wind).’
  \ex
  \langinfo{}{}{ \textbf{\textit{-Vk}} \textbf{intransitive}}\\
  \gll muriaŋgo gu-aram-ik-i    (kuturona na imboza).\\
      3door       3-open\textit{-}\textsc{Vk}-\textsc{fv}   from       by 9wind\\
  \glt ‘The door opened (because of/from the wind).’
  \z
\z

\subsubsection{Oblique \textsc{instruments}}

Oblique \textsc{instrument} theta-roles are licensed only by the passive in Luragooli \REF{ex:gluckman:9a}. \textsc{Instruments} are not permitted with plain intransitives \REF{ex:gluckman:9b} or \textit{-Vk} intransitives \REF{ex:gluckman:9c}.\footnote{In general, instrumental subjects are not permitted in Luragooli. The active transitive version of \REF{ex:gluckman:9a} with \textit{itahoro} ‘the towel’ as the subject would be ungrammatical.}

\ea\label{exx:}
  \ea
  %%1st subexample: change \ea\label{...} to \ea\label{...}\ea; remove \z  
  %%further subexamples: change \ea to \ex; remove \z  
  %%last subexample: change \z to \z\z 
  \langinfo{}{}{\textbf{passive}}\\
  \gll imbwa y-um-iny-w-i            (na    itahoro).\\
      9dog 9-dry-\textsc{caus}-\textsc{pass}-\textsc{fv}   \textsc{prt}  9towel\\
  \glt ‘The dog was dried (with a towel).’
  \ex
  \langinfo{}{}{\textbf{plain intransitive}}\\
  \gll imbwa y-um-i          (*na   itahoro).\\
      9dog   9-dry-\textsc{fv}         \textsc{prt} 9towel\\
  \glt ‘The dog dried (*with a towel).’
  \ex
  \langinfo{}{}{\textbf{\textit{-Vk}}\textbf{ intransitive}}\\
  \gll imbwa  y-um-ik-i      (*na  itahoro).\\
      9dog    9-dry\textit{-}\textsc{Vk}-\textsc{fv} \textsc{prt} 9towel\\
  \glt ‘The dog dried (*with a towel).’
  \z
\z

Thus, in terms of oblique theta-roles, the plain intransitive and the \textit{-Vk} intransitive pattern together, separately from the Luragooli passive. This is summarized in \tabref{tab:2}.

\begin{table}
\caption{Theta-role properties of the intransitive constructions}
\label{tab:2}

\begin{tabularx}{\textwidth}{llXX} & \textbf{Passive} & \textbf{Plain intransitive} & \textbf{\textit{-Vk}} \textbf{intransitive}\\
\lsptoprule
\textbf{Oblique agents} & yes & no & no\\
\textbf{Oblique causers} & no & yes & yes\\
\textbf{Oblique instruments} & yes & no & no\\
\lspbottomrule
\end{tabularx}
\end{table}

The Luragooli patterns in \tabref{tab:2} largely parallel properties of anticausative versus passive constructions in other languages. The anticausative forms do not permit oblique \textsc{agents} or \textsc{instruments}, but are compatible with oblique \textsc{causers.}

\section{Lexical aspect}

In \sectref{sec:gluckman:3} we demonstrated how both plain and \textit{-Vk} intransitives are distinct from the passive. In this section, we will show how \textit{-Vk} intransitives are distinct from the plain intransitive and the passive. The data in \sectref{sec:gluckman:4} concerns lexical aspect, or Aktionsart. We use four pieces of evidence to show that \textit{-Vk} intransitives differ from the other two intransitive forms with respect to lexical aspect. Our evidence involves interaction with negation, complementation under ‘want’, progressive aspect, and continuations. Our data suggest the provisional generalization in \REF{ex:gluckman:10}, which we will revise in \sectref{sec:gluckman:4.5}.

\ea\label{exx:}
%%1st subexample: change \ea\label{...} to \ea\label{...}\ea; remove \z  
%%further subexamples: change \ea to \ex; remove \z  
%%last subexample: change \z to \z\z 
\langinfo{}{}{\textbf{Telicity Restriction}}\\
 \textit{-Vk} only attaches to telic predicates. (\textit{to be revised})
\z

This generalization is consistent with cross-linguistic findings on anticausatives; an interaction with telicity is also reported for anticausatives in other languages \citep{Labelle1992,Folli2002,FolliHarley2005}. Marked anticausatives tend to entail a telic reading of the event denoted by the predicate in Greek \citep{AlexiadouAnagnostopoulou2004}, Italian \citep{Folli2002}, and French \citep{ZribiHertz1987}. For example, in Italian, the marked anticausative (with the reflexive \textit{si}) cannot occur with a ‘for’-temporal phrase \REF{ex:gluckman:11b}, while the unmarked intransitive form can \REF{ex:gluckman:11a}.\footnote{We refer the reader to the large body of work on Italian anticausatives, in particular \citet{Folli2002}, for a full explanation of the data.}

\ea\label{exx:}
%%1st subexample: change \ea\label{...} to \ea\label{...}\ea; remove \z  
%%further subexamples: change \ea to \ex; remove \z  
%%last subexample: change \z to \z\z 
  \ea
  \langinfo{}{}{Italian}\\
  \ex
  \gll  Il   cioccolato è  fuso     per pochi secondi/in pochi secondi\\
	the chocolate  is melted for few    seconds/in few    seconds\\
  \ex       
  \gll   Il    cioccolato si     è  fuso     *per pochi secondi/in pochi secondi\\
	the chocolate  \textsc{refl} is melted  for   few   seconds/in few seconds. \\
  \glt ‘The chocolate melted for a few seconds/in a few seconds.’     (Schäfer 2009)
  \z
\z

We are not aware of a convincing explanation for why such a correlation between anticausatives and telicity should exist. It is not our aim to explain this correlation. Instead, we will merely show that such a pattern is consistent with what we find in Luragooli as well. 

Finally, we note that the next two pieces of evidence involving complementation under ‘want’ and negation are, as far as we know, specific to Luragooli (or perhaps Bantu languages more generally). The negation test is inspired by Dubinsky \& Simango’s (1996) work on Chichewa. It remains to be seen what the results of such tests are in other languages. 

\subsection{Complements of ‘want’}

We first observe a contrast in interpretation when embedding the three intransitives under a verb like \textit{kwenya} ‘to want.’\footnote{The form of the embedded verb in this context is subjunctive, indicated by the final vowel /ɛ/.} We find that with the passive \REF{ex:gluckman:12a} and plain intransitive \REF{ex:gluckman:12b} the object of wanting can only be the beginning of the event, not the result state.\footnote{The term “beginning of the event” is possibly not quite accurate. For passive and plain intransitive complements of ‘want’, the object of wanting is perhaps best described as “anything that is not the result state,” which includes the beginning, but may also include the middle of the event as well.}  Conversely, with the \textit{-Vk} intransitive in \REF{ex:gluckman:12c}, the thing that is wanted can only be the result state of the embedded verb. Thus, in a context where the door is already closed, it is infelicitous to use either \REF{ex:gluckman:12a} or \REF{ex:gluckman:12b}. We take this as evidence that \textit{-Vk} imposes a telicity restriction, i.e. requires a telic predicate; only with the \textit{-Vk} form is the result state entailed.

\ea\label{exx:}
%%1st subexample: change \ea\label{...} to \ea\label{...}\ea; remove \z  
%%further subexamples: change \ea to \ex; remove \z  
%%last subexample: change \z to \z\z 
\langinfo{}{}{Context: The door is closed.}\\
  \ea\label{ex:}
  \langinfo{}{}{\textbf{passive}}\\
  \gll   \# n-eny-a       murianggo gu-han-w-ɛ.\\
      {}  1sg-want-\textsc{fv} 3door         3-close-\textsc{pass}-\textsc{fv}\\
  \glt ‘I want the door to be closed.’
  \ex
  \langinfo{}{}{b. \textbf{plain intransitive}}\\
  \gll   \# n-eny-a       murianggo gu-han-ɛ\\
      {}  1sg-want-\textsc{fv} 3door        3-close-\textsc{fv}\\
  \glt ‘I want the door to close.’
  \ex
  \langinfo{}{}{\textbf{\textit{-Vk}} \textbf{intransitive}}\\
  \gll   n-eny-a         murianggo gu-han-ek-ɛ.\\
	1sg-want-\textsc{fv} 3door        3-close\textit{-}\textsc{Vk}-\textsc{fv}\\
  \glt ‘I want the door closed.’\footnote{A reviewer asks whether \REF{ex:gluckman:12c} can be translated as ‘I want the door to be closed.’ We think this translation is misleading for two reasons: a) it either suggests a passive reading of this sentence, or, b) it suggests a stative reading.}
  \z
\z

The plain intransitive and the passive again pattern similarly in that the object of wanting is the movement of the door: ‘I want the event of door-closing.’ These forms cannot target the result state. In contrast, in \REF{ex:gluckman:12c}, the object of wanting can be either the event of door-closing or the result state: ‘I want the state of the door to be closed.’ This second reading is not available in \REF{ex:gluckman:12a} and \REF{ex:gluckman:13b}.

Lastly, we note that at first glance, the data in \REF{ex:gluckman:12} might be taken to indicate that the \textit{-Vk} form is a stative, as argued in \citet{DubinskySimango1996}. However, we observe that \REF{ex:gluckman:12c} can have the same reading as \REF{ex:gluckman:12b}. That is, the object of wanting can be the event of closing. In other words, \textit{-Vk} intransitives can still be interpreted as eventive. Furthermore, recall that both the plain and \textit{–Vk} intransitive forms permit \textsc{causer} theta-roles, which should be impossible with stative verbs.\footnote{It is unlikely that the \textit{-Vk} form can be treated as an adjective. In Luragooli, a deverbal adjectival form would trigger a different set of agreement (concord) markers than verbal agreement.}

\subsection{Negation} 

Our second piece of evidence that \textit{-Vk} intransitives differ from the other two intransitive constructions comes from which parts of the event can be targeted by negation. We find that \textit{-Vk} intransitives only permit the end of the event, i.e., the result state, to be negated, while both passives and plain intransitives permit either the beginning or end of the event to be negated. While less obvious, we think this can also be taken as evidence for a telicity restriction in the \textit{-Vk} form. If the end of the event is entailed by the assertion, then it can be targeted by negation. 

Given the context below in \REF{ex:gluckman:13} in which the door has not moved at all, both the passive \REF{ex:gluckman:13a} and plain intransitive \REF{ex:gluckman:13b} are felicitous. Conversely, the \textit{-Vk} intransitive \REF{ex:gluckman:13c} is infelicitous. Example \REF{ex:gluckman:13c} is only felicitous if the door moved, but didn’t finish closing.

\ea\label{exx:}
%%1st subexample: change \ea\label{...} to \ea\label{...}\ea; remove \z  
%%further subexamples: change \ea to \ex; remove \z  
%%last subexample: change \z to \z\z 
\langinfo{}{}{Context: The door hasn’t moved at all.}\\
  \ea\label{ex:}
  \langinfo{}{}{\textbf{passive}}\\
  \gll   murianggo gu-han-w-i           daave.\\
	3door        3-close-\textsc{pass}-\textsc{fv}    \textsc{neg}\\
  \glt ‘The door wasn’t closed.’
  \ex
  \langinfo{}{}{\textbf{plain intransitive}}\\
  \gll   murianggo gu-han-i     daave.\\
	3door         3-close-\textsc{fv}  \textsc{neg}\\
  \glt ‘The door didn’t close.’
  \ex
  \langinfo{}{}{c. \textbf{\textit{-Vk}}\textbf{ intransitive}}\\
  \gll   \# murianggo gu-han-ek-i      daave.\\
	  {}  3door         3-close\textit{-}\textsc{Vk}-\textsc{fv}   \textsc{neg}\\
  \glt ‘The door didn’t close.’
  \z
\z

Out of context, both \REF{ex:gluckman:13a} and \REF{ex:gluckman:13b} are ambiguous. They mean that either ‘the door didn’t start to close’ or ‘the door didn’t finish closing.’ That is, \REF{ex:gluckman:13a} and \REF{ex:gluckman:13b} can have a reading that the event of the door starting to close – the beginning of the event – didn’t occur. However, the \textit{-Vk} intransitive in \REF{ex:gluckman:13c} is only compatible with a scenario in which the door moved, but didn’t get all the way closed. Example \REF{ex:gluckman:13c} only has the reading that the state of the door being closed didn’t occur.\footnote{This reading is also compatible with the passive and plain forms in Luragooli. This differs from what \citet{DubinskySimango1996} report for Chichewa. We find our Luragooli data curious. It is unclear to us why the start of the event is not a possible target for negation with the \textit{-Vk} form.}  We suggest that this follows if the result state is entailed in the \textit{-Vk} form, and so can be targeted by negation. Since there is no such entailment with either the plain or passive form, the result state is not a possible target for negation.

\subsection{Progressive aspect}

Telic predicates require the culmination of the event that they denote. As a result, we should expect to see an interaction with progressive (grammatical) aspect, since the progressive aspect asserts that the event is on-going, i.e., incomplete, with respect to a reference time. In Luragooli, both the passive \REF{ex:gluckman:14a} and plain intransitive \REF{ex:gluckman:14b} forms are compatible with the progressive.\footnote{A similar set of facts is apparently reported for Greek in \citet{Mavromanolaki2002}, as cited in \citet{AlexiadouEtAl2015}, although we have not been able to locate this source.} In contrast, \textit{-Vk} verb forms are ungrammatical in combination with the progressive aspect, as shown in \REF{ex:gluckman:14c}.

\ea\label{exx:}
%%1st subexample: change \ea\label{...} to \ea\label{...}\ea; remove \z  
%%further subexamples: change \ea to \ex; remove \z  
%%last subexample: change \z to \z\z 
  \ea
  \langinfo{}{}{\textbf{passive}}\\
  \gll mpira gu-toony-w-ang-a.\\
      3ball  3-drop-\textsc{pass}-\textsc{prog}-\textsc{fv}\\
  \glt ‘The ball was being dropped.’
  \ex
  \langinfo{}{}{\textbf{ plain intransitive}}\\
  \gll mpira gu-toony-ang -a.\\
      3ball  3-drop-\textsc{prog}-\textsc{fv}\\
  \glt ‘The ball is dropping.’
  \ex
  \langinfo{}{}{\textbf{ -Vk intransitive}}\\
  \gll *mpira gu-toony-ik-ang-a.\\
	3ball    3-drop\textit{-}\textsc{Vk}-\textsc{prog}-\textsc{fv}\\
  \glt  (‘The ball is being dropped.’/‘The ball is dropping.’)
  \z
\z

The ungrammaticality of \REF{ex:gluckman:14c} follows if \textit{-Vk} requires that the event culminate—that is, if \textit{-Vk} must combine with a telic predicate, as proposed in \REF{ex:gluckman:10}.

\subsection{Continuations}

Our last piece of evidence on the interaction of \textit{-Vk} and telicity concerns overt continuations. Related to the negation diagnostic above, we examine the felicity of continuations that deny the result state of an intransitive verb form. We find that continuations of both the passive \REF{ex:gluckman:15a} and plain intransitive \REF{ex:gluckman:15b} forms are felicitous if the result state is denied. However, the result state of a \textit{-Vk} intransitive cannot be felicitously denied \REF{ex:gluckman:15c}. This supports the generalization in \REF{ex:gluckman:10} in that only \textit{-Vk} forms entail that the event culminate. As a result, it is infelicitous to later assert that the event did not culminate.

\ea\label{exx:}
%%1st subexample: change \ea\label{...} to \ea\label{...}\ea; remove \z  
%%further subexamples: change \ea to \ex; remove \z  
%%last subexample: change \z to \z\z 
  \ea
  \langinfo{}{}{\textbf{passive}}\\
  \gll maguta  ga-diny-iz-w-i                     (netare ga-ker-e       ma-doto).\\
      6butter  6-harden-\textsc{caus}-\textsc{pass}-\textsc{fv}       but      6-be.still-\textsc{fv} 6-soft\\
  \glt ‘The butter was hardened (but it’s still soft).
  \ex
  \langinfo{}{}{\textbf{plain intransitive}}\\
  \gll maguta ga-diny-i      (netare ga-ker-e       ma-doto).\\
      6butter 6-harden-\textsc{fv}   but     6-be.still-\textsc{fv}  6-soft\\
  \glt ‘The butter hardened (but it’s still soft).’
  \ex
  \langinfo{}{}{\textbf{\textit{-Vk}}\textbf{ intransitive}}\\
  \gll maguta ga-diny-ik-i        (\#netare ga-ker-e       ma-doto).\\
      6butter 6-harden\textit{-}\textsc{Vk}-\textsc{fv}     but      6-be.still-\textsc{fv} 6-soft\\
  \glt ‘The butter hardened (\#but it’s still soft).’
  \z
\z

In \REF{ex:gluckman:15a} and \REF{ex:gluckman:15b} we get a reading in which the butter has hardened somewhat, but still remains soft. However, \REF{ex:gluckman:15c} is infelicitous if it is later asserted that the butter hasn’t completed the hardening process. This follows if \textit{-Vk} is required to attach only to telic predicates that denote a culminated event (i.e., telic predicates).

Thus, for contexts targeting lexical aspect, the \textit{-Vk} form patterns distinctly from the passive and plain forms. These data suggest that \textit{-Vk} requires that the event of the verb culminate, supporting the telicity generalization in \REF{ex:gluckman:10}. We summarize the aspectual properties of the Luragooli passive, plain intransitive, and -\textit{Vk} intransitive in \tabref{tab:3}. 

\begin{table}
\caption{Lexical aspect properties of passive, plain intransitive and \textit{-Vk} intransitive}
\label{tab:3}

\begin{tabularx}{\textwidth}{lp{2.5cm}XX} & \textbf{Passive} & \textbf{Plain intransitive} & \textbf{\textit{-Vk}}\textbf{ intransitive}\\
\lsptoprule
\textbf{Negation} & entire event & entire event & result state\\
\textbf{‘want’} & entire event & entire event & result state\\
\textbf{Progressive} & grammatical & grammatical & ungrammatical\\
\textbf{Continuations} & can deny result state & can deny result state & cannot deny result state\\
\lspbottomrule
\end{tabularx}
\end{table}

However, there are a number of counterexamples in Luragooli to the telicity generalization in \REF{ex:gluckman:10}. Not all verbs pattern similarly with respect to the four tests above. For instance, the \textit{-Vk} form of \textit{kwoma} ‘to dry’ fails the four diagnostics in \tabref{tab:3}. Given a \textit{-Vk} form of \textit{kwoma} ‘to dry,’ the object of wanting cannot be the result state \REF{ex:gluckman:16a} of the event described by the verb. Negation can target the beginning of the event as well as the result state \REF{ex:gluckman:16b}. The \textit{-Vk} form is compatible with progressive aspect \REF{ex:gluckman:16c}. Finally, a continuation that denies that result state is felicitous \REF{ex:gluckman:16d}.

\ea\label{exx:}
%%1st subexample: change \ea\label{...} to \ea\label{...}\ea; remove \z  
%%further subexamples: change \ea to \ex; remove \z  
%%last subexample: change \z to \z\z 
  \ea
  \langinfo{}{}{\textbf{complement of} \textbf{\textit{want}}}\\
  \gll n-eny-a          imbwa y-um-ik-e.\\
      1sg-want-\textsc{fv} 9dog    9-dry-\textsc{Vk}-\textsc{fv}\\
  \glt ‘I want the dog to be dry.’\\
  \glt Consultant’s comment: Strange if the dog is already dry.
  \ex
  \langinfo{}{}{\textbf{negation}}\\
  \gll imbwa y-um-ek-i     daave.\\
      9dog    9-dry-\textsc{Vk}-\textsc{fv} \textsc{neg}\\
  \glt ‘The dog didn’t dry.’
  \glt Consultant’s comment: OK if the dog doesn’t dry at all, or only dries halfway.
  \ex
  \langinfo{}{}{\textbf{progressive}}\\
  \gll imbwa y-um-ek-a.\\
      9dog   9-dry\textit{-}\textsc{Vk}-\textsc{fv}\\
  \glt ‘The dog is drying.’
  \ex
  \langinfo{}{}{\textbf{continuation}}\\
  \gll imbwa y-um-ik-i      (netare i-ker-e         i-nzilu hadi).\\
      9dog   9-dry\textit{-}\textsc{Vk}-\textsc{fv} but       9-be.still-\textsc{fv} 9-wet some.of\\
  \glt ‘The dog dried (but it’s still a little wet).’
  \z
\z

That said, these exceptions do not necessarily argue against a treatment of \textit{-Vk} as an anticausative marker, since such variation is consistent with what is observed cross-linguistically. \citet{Schäfer2008} argues convincingly that the telicity restrictions for Greek, Italian, and French fail for a number of lexical items, and are not consistent across languages. For instance, in Italian, some marked anticausatives permit modification by ‘for’-temporal adverbial phrases, and so are not necessarily telic.

\ea\label{exx:}
%%1st subexample: change \ea\label{...} to \ea\label{...}\ea; remove \z  
%%further subexamples: change \ea to \ex; remove \z  
%%last subexample: change \z to \z\z 
\langinfo{Italian}{}{\citep{Schäfer2008}}\\
\gll La temperatura  si     è   alterata per  due ore.\\
     the temperature \textsc{refl} is  altered  for   two hours\\
\glt ‘The temperature altered for two hours.’        
\z

While there is an overall tendency for anticausatives to co-occur with a telic reading of the predicate, telicity is not an absolute requirement for morphologically marked anticausatives. It is still an open question as to why some telicity diagnostics fail with certain Luragooli \textit{-Vk} verbs. Our hypothesis, adopted from \citet{Schäfer2008}, is that there is something inherent about the semantics of the verb root itself that lead to the failure of a particular diagnostic. Further in-depth examination of lexical classes in Luragooli (along the lines of \citet{LevinRappaportHovav1995} and \citet{Haspelmath2005}) are needed to tease apart these differences. 

\section{Subclasses of Class I verbs}

In \sectref{sec:gluckman:3} and \ref{sec:gluckman:4} we provided evidence that Luragooli \textit{-Vk} intransitives generally pattern distinctly from the passive and the plain intransitive forms in terms of theta-roles and lexical aspect. The accumulated evidence led us to conclude that \textit{-Vk} is the anticausative marker in Luragooli. Our conclusion was based largely on a comparison with cross-linguistic observations. In this section, we detail some “anomalous” uses of \textit{-Vk} that fall outside of what is typically associated with an anticausative alternation cross-linguistically.

We begin by designating two additional subclasses of Class I, that is, verbs which require \textit{-Vk} to form non-passive intransitives: Class Ia and Class Ib.\footnote{We thank a reviewer for helping us with the overall classification of the verbs.} These classes are differentiated based on semantic criteria.\footnote{As far as we know, there is one exceptional verb, \textit{kunwa} ‘to drink.’ The (true) passive of this verb is expressed with the \textit{-Vk} form \textit{kunwahuka} ‘to be drunk.’ This verb must be listed as an idiosyncratic exception.} 

\textbf{Class Ia:} Verbs that (loosely) denote an epistemic state, i.e., that license a mental experiencer argument.

\textbf{Class Ib:} Verbs that have an affected argument. (We will return shortly to what we mean by “affected.”)

Examples of verbs in these classes are shown in \tabref{tab:4}.

\begin{table}
\caption{Non-canonical anticausative verb classes}
\label{tab:4}

\begin{tabularx}{\textwidth}{lp{2.5cm}XX} & \textbf{Transitive} & \textbf{Intransitive with} \textbf{\textit{-Vk}} & \textbf{Passive}\\
\lsptoprule
\textbf{Class Ia} & \textit{kuhola} \newline ‘to hear’ & \textit{kuholeka} \newline ‘to be heard’ & \textit{kuholwa} \newline ‘to be heard’\\
& \textit{kurora} \newline ‘to see’ & \textit{kuroreka} \newline ‘to be seen’ & \textit{kurorwa} \newline ‘to be seen’\\
\hhline{~---} & \textit{kudiira} \newline ‘to touch’ & \textit{kudiirika} \newline ‘to be touched’ & \textit{kudiirwa} \newline ‘to be touched’\\
\hhline{~---} & \textit{kumena} \newline ‘to taste/lick’ & \textit{kumeneka} \newline ‘to be tasted/licked’ & \textit{kumenwa} \newline ‘to be tasted/licked’\\
\textbf{Class Ib} & \textit{kuhola} \newline ‘to punch’\textsuperscript{a} & \textit{kuholeka} \newline ‘to be punched’ & \textit{kuholwa} \newline ‘to be punched’\\
& \textit{kurasa} \newline ‘to throw’ & \textit{kurasika} \newline ‘to be thrown’ & \textit{kuraswa} \newline ‘to be thrown’\\
\hhline{~---} & \textit{kuroomba} \newline ‘to make’ & \textit{kuroombika} \newline ‘to be made’ & \textit{kuroombwa} \newline ‘to be made’\\
\hhline{~---} & \textit{kulia} \newline ‘to eat’ & \textit{kuliika} \newline ‘to be eaten’ & \textit{kuliwa} \newline ‘to be eaten’\\
\hhline{~---} & \textit{kunyanya} \newline ‘to chew’ & \textit{kunyanyeka} \newline ‘to be chewed’ & \textit{kunyanywa}~\newline ‘to be chewed’\\
\hhline{~---}
\lspbottomrule
\end{tabularx}
\textsuperscript{a} In Luragooli, ‘to hear’ and ‘to punch’ are entirely homophonous (\textit{kuhola}), with no tonal differences.

\end{table}

The Class Ia \textit{-Vk} intransitives are productively formed with any verb that takes an experiencer subject. They pattern separately from the passive in not being able to occur with an oblique “demoted” subject \REF{ex:gluckman:18b}.\footnote{However, an oblique argument is sometimes licensed in the presence of \textit{-Vk} with the addition of the reciprocal -\textit{an}. Such facts have also been reported for Chichewa and Swahili \citep{DubinskySimango1996,SeidlDimitriadis2003}.} Passives, however, are acceptable with an \textsc{experiencer} subject that is expressed obliquely \REF{ex:gluckman:18a}.

\ea\label{exx:}
  \ea
  %%1st subexample: change \ea\label{...} to \ea\label{...}\ea; remove \z  
  %%further subexamples: change \ea to \ex; remove \z  
  %%last subexample: change \z to \z\z 
  \langinfo{}{}{\textbf{passive}}\\
  \gll iroli      i-ror-w-e         na Sira\\
      9truck  9-see-\textsc{pass-fv} by 1Sira\\
  \glt ‘The truck was seen by Sira.’ 
  \ex
  \langinfo{}{}{\textbf{\textit{-Vk}}\textbf{ intransitive}}\\
  \gll iroli    i-ror-ek-e       (*na Sira)\\
      9truck 9-see-\textsc{Vk}\textsc{-}\textsc{fv}   by 1Sira\\
  \glt ‘The truck was seen (by Sira).’
  \z
\z

The only commonality that we can identify among Class Ib verbs is a notion of “affectedness.” Class Ib transitive verbs all involve an affected object argument.\footnote{\citet{DubinskySimango1996} make a similar claim for \textit{-Vk} in Chichewa.} Things that are ‘punched,’ ‘thrown,’ ‘made,’ ‘eaten,’ and ‘chewed’ are affected in a broad sense. However, a verb like \textit{kwomba} ‘to sing’ does not have a form with \textit{-Vk} (*\textit{kwombeka}), presumably because songs are not affected by the action of singing.\footnote{The Class Ib verbs might all be classified as change of state verbs, although it requires us to loosen the definition of change of state considerably. See \citet{DubinskySimango1996} for discussion of change of state and \textit{-Vk} in Chichewa.}

The Class Ia and Class Ib verbs are a prima facie problem for our analysis of \textit{-Vk} as an anticausative marker; these classes of verbs are not generally reported to have anticausative forms in other languages. Moreover, it is unclear how the diagnostics concerning thematic roles and lexical aspect are applicable to the Class Ia verbs, some of which seem to be inherently stative and non-agentive/non-causative. A potential way to incorporate these verbs into the more general analysis of anticausativization in Luragooli would be to appeal to Beavers’s (2011) criteria for affected objects.\footnote{We thank an anonymous reviewer for pointing out the relevance of this work to us.}Objects\textsc{} of the Class Ib verbs can be thought of as being “physically impinged on” to some extent. We could possibly extend this to the experiencer verbs in Class Ia by assuming that experiencer subjects are also (mentally) impinged on. Thus, the descriptive generalization is that  \textit{-Vk} attaches to any verb that takes an affected (“impinged”) argument, in the sense of \citet{Beavers2011}. This generalization subsumes canonical anticausative verbs (e.g. \textit{break} and \textit{melt}) as well, since these verbs also involve affected arguments: the \textsc{patient}. We find this a promising avenue for further research, but we must leave it open for now.

\section{Conclusion}

In this paper, we have shown that the Luragooli morpheme \textit{-Vk} has a wide distribution. While coinciding nicely with what we expect from an anticausative morpheme, as documented in \sectref{sec:gluckman:3} concerning theta-roles, and \sectref{sec:gluckman:4} concerning lexical aspect, \sectref{sec:gluckman:5} has shown that \textit{-Vk}’s range extends beyond what are canonically seen as anticausative environments. Further investigation of the semantics of Classes Ia and Ib should provide a clearer picture as to what governs the distribution of \textit{-Vk}.\footnote{\textit{-Vk} forms have been reported to mean ‘V-able’ in Chichewa \citep{Simango2009} and Kikongo \citep{Fernando2013}. This reading does not seem to be present with \textit{-Vk} for our consultant, although further investigation is required to settle the matter. We further note that treating \textit{-Vk} as a marker of a middle voice (e.g., ‘This cheese cuts easily’) is not straightforwardly possible.} Nonetheless, we do not view this exceptional data as an insurmountable obstacle to our proposal. Even in Romance and Germanic languages, the ‘anticausative’ morpheme does not solely mark anticausatives: it is also the reflexive morpheme. Having an anticausative marker that does double-duty with other functions is therefore not cross-linguistically unusual. Still, the Luragooli data suggest that more in-depth cross-linguistic research would be beneficial to our understanding of anticausatives in general, since the majority of in depth work on anticausatives has been done for western European languages.

\section*{Acknowledgments}

We would like to thank our wonderful Luragooli consultant, Mwabeni Indire, for sharing his language with us and making each elicitation session a joy. All of the Luragooli data in this paper comes from our own fieldwork with Mwabeni. We would also like to thank Michael Diercks, Mary Paster, Meredith Landman, participants in the Spring 2014 undergraduate field methods class at Pomona College, and audience members at ACAL 46. We are grateful to Hilda Koopman for extensive comments and feedback. Thanks also to Doris Payne and Sara Pacchiarotti for very helpful suggestions for improvements and clarifications.

\section*{Abbreviations}

Luragooli has 20 noun classes. Following Bantuist convention, we mark noun classes through numerals at the beginning of nouns and verbs. The following abbreviations are used in this paper: 

\textsc{caus}     causative 

\textsc{dem}     demonstrative 

\textsc{fv}     final vowel

\textsc{neg}     negative   

\textsc{prog}     progressive 

\textsc{pass}     passive 

\textsc{prt}     particle   

\textsc{plact}   plural act 

\textsc{refl}     reflexive


\begin{verbatim}%%move bib entries to  localbibliography.bib


Samuels, Alex & Paster, Mary. 2015. Verbal tone in Logoori. (Paper presented at the Workshop on Luyia Bantu Languages at ACAL 46, Eugene, Oregon, 26-28 March, 2015.)


Seidl, Amanda & Dimitriadis, Alexis. 2003. Statives and reciprocal morphology in Swahili. In Sauzet, Patrick & Zribi-Hertz, Anne (eds.), Typologie des langues d’Afrique et universaux de la grammaire, 239-284. Paris: L’Harmattan



\end{verbatim}

\printbibliography[heading=subbibliography,notkeyword=this]

\end{document}