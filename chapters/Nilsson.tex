\documentclass[output=paper]{langsci/langscibook} 
\title{Somali gender polarity revisited} 
\author{%
 Morgan Nilsson \affiliation{University of Gothenburg }
}
% \sectionDOI{} %will be filled in at production


\abstract{
The notion of \textsc{gender polarity }is a well established part of the description of Somali nouns. It refers to the phenomenon that in most Somali nouns a change in number is accompanied by a simultaneous change in gender, and it implies that number is actually expressed by means of the change in gender. However, from a synchronic point of view there seems to be little evidence for such an interpretation, as plural gender is realised solely through the shape of the definite article suffix on the noun itself. In this article the main arguments for the standard description are investigated and an alternative solution based on typological considerations of the data is proposed, claiming that gender is not relevant in the plural. Instead, the form of the plural definite article is predictable from the gender of the noun in the singular, together with some morpho-phonological characteristics of the stem. Additionally, many nouns traditionally claimed to be plural are argued here to be formally singular and mostly collective.
}

\maketitle
\begin{document}

 

 
% Keywords: Somali, polarity, gender, number, agreement.

\section{Gender polarity}

According to the traditional point of view, Somali has two genders and two morphemes expressing definiteness, one for each gender: \{k\}\footnote{Curly brackets indicate underlying morphemes that are realised differently in different phonological contexts.} for masculine gender and \{t\} for feminine gender. The same morphemes are used both in the singular and in the plural, and the majority of Somali nouns change their gender when they change their number. All nouns that are feminine and take the article \{t\} in the singular become masculine and take the article \{k\} in the plural; whereas most nouns that are masculine and take the article \{k\} in the singular become feminine and take the article \{t\} in the plural. A smaller group of nouns which are masculine in the singular remain masculine also in the plural, and hence take the article \{k\} irrespective of their number. This standard point of view, illustrated in \tabref{tab:1},\footnote{The horizontal line in the middle of \tabref{tab:1} is discontinuous as some nouns are masculine in both numbers.
} is presented in practically all modern works on Somali, among others the reference grammars by \citet{Moreno1955}, Saeed (1993; 1999), Puglielli \& \citet{Mansuur1999} and \citet{Berchem2012}.
 
\begin{table}
\caption{Gender polarity – the traditional view on gender in Somali nouns}
\label{tab:1}

\begin{tabularx}{\textwidth}{XXXX} & \textsc{Singular}& \textsc{Plural}& \\
\lsptoprule
 \textsc{Feminine}& \{t\}& \{k\}& \textsc{Masculine}\par\\
\hhline{--~~} &  &  & \\
\hhline{~~--}
 \textsc{Masculine}& \{k\}& \{t\}& \textsc{Feminine}\par\\
\lspbottomrule
\end{tabularx}

\end{table} 


Some examples of actual occurrences of these articles are given in \tabref{tab:2}. The ordering of suffixes added to noun stems is 1) plural morpheme, 2) definite article (marked by boldface), and 3) “case” ending. Somali also exhibits several morpho-phonological alternations. Depending on the stem-final phoneme, the masculine definite article suffix \{k\} is realised as one of the three posterior consonants /k/, /g/, /h/ or as zero; whereas the feminine definite article suffix \{t\} is realised as one of the three coronal consonants /t/, /d/ or /ʃ/. The most common plural suffix is realised as /o/ when word final, but as /a/ when non-final. For example, in the form \textit{sababaha} ‘the reasons’ at the end of the first line in \tabref{tab:2}, /sabab/ is the stem, the following /a/ marks the plural, the /h/ is a realisation of the definite article \{k\}, and the final /a/ is a final vowel occurring in the unmarked base form of a definite noun, traditionally referred to as the absolutive case.
 
\begin{table}
\caption{Nouns exemplifying the distribution of gender and number in Somali}
\label{tab:2}

\begin{tabularx}{\textwidth}{XXXX}
\lsptoprule
{\textsc{f.sg.indef}} & {\textsc{f.sg.def}} & {\textsc{m.pl.indef}} & {\textsc{m.pl.def}}\\
{\textit{sabab} ‘reason’}

{\textit{mindi} ‘knife’} & {\textit{sabab-}\textbf{\textit{t}}\textit{{}-a}}

{\textit{mindi-}\textbf{\textit{d}}\textit{{}-a}} & {\textit{sabab-o}}

{\textit{mindi-yo}} & {\textit{sabab-a-}\textbf{\textit{h}}\textit{{}-a}}

{\textit{mindi-ya-}\textbf{\textit{h}}\textit{{}-a}}\\
{\textsc{m.sg.indef}} & {\textsc{m.sg.def}} & {\textsc{f.pl.indef}} & {\textsc{f.pl.def}}\\
{\textit{gambar} ‘stool’}

{\textit{derbi} ‘wall’} & {\textit{gambar-}\textbf{\textit{k}}\textit{{}-a}}

{\textit{derbi-}\textbf{\textit{g}}\textit{{}-a}} & {\textit{gambar-ro}}

{\textit{derbi-yo}} & {\textit{gambar-ra-}\textbf{\textit{d}}\textit{{}-a}}

{\textit{derbi-ya-}\textbf{\textit{d}}\textit{{}-a}}\\
{\textsc{m.sg.indef}} & {\textsc{m.sg.def}} & {\textsc{m.pl.indef}} & {\textsc{m.pl.def}}\\
{\textit{miis} ‘table’} & {\textit{miis-}\textbf{\textit{k}}\textit{{}-a}} & {\textit{miis-as}} & {\textit{miis-as-}\textbf{\textit{k}}\textit{{}-a}}\\
\lspbottomrule
\end{tabularx}

\end{table} 

The first two nouns in \tabref{tab:2}, \textit{sabab }‘reason’ and \textit{mindi} ‘knife’, have a feminine definite article, i.e. a coronal consonant, in the singular; whereas they have a masculine definite article, i.e. a back consonant, in the plural. They are therefore traditionally considered feminine in the singular and masculine in the plural. The next two examples, \textit{gambar} ‘stool’ and \textit{derbi} ‘wall’, have a masculine (back) definite article in the singular and a feminine (coronal) definite article in the plural. They are therefore traditionally considered masculine in the singular and feminine in the plural. The final example consists of the noun \textit{miis} ‘table’, which is masculine both in the singular and in the plural.

It is traditionally argued that a distinct gender value is associated with the plural form of nouns\footnote{\citet{Lecarme2002}, however, suggests that the gender value is associated with the individual plural morphemes, and that it is the plural morpheme that bears the gender value in the plural forms of nouns, not the nouns themselves.} and that the definite article is then assigned according to the plural gender of each specific noun (El-Solami-Mewis 1988; Saeed 1999: 54{}--55; Berchem 2012: 48{}--49; among others).\footnote{From a pedagogical perspective, it is also highly relevant that the prevailing description is perceived by most learners of Somali grammar (both Somali speakers and foreign learners) as a very complicated way of describing the rather simple facts of present-day Somali.}  In this article, however, I argue that the form of the plural definite article is morphologically predictable without reference to plural gender. The principles for the distribution of the definite article suffixes in the plural will be discussed in detail in §2 below.

Another important exponent of gender is also regularly pointed out in the literature, namely the position of tonal accent, realised as high pitch, which falls on the second to last mora of the stem in most masculine singular nouns and on the last mora of the stem in most feminine singular nouns, as shown in \tabref{tab:3}.\footnote{This is not an exceptionless principle, but it applies to the vast majority of nouns ending in a consonant or a vowel other than \textit{{}-e} or \textit{–o}; nouns ending in these two vowels are subject to specific tonal accent assignment rules of their own.}
 
\begin{table}
\caption{The interdependence between gender and tonal accent}
\label{tab:3}

\begin{tabularx}{\textwidth}{XXXXX} & {\textsc{indef}} &  & {\textsc{def}} & \\
\lsptoprule
{\textsc{masculine}}

{\textsc{feminine}} & {\textit{ínan}}

{\textit{gámbar}}

{\textit{inán}}

{\textit{sabáb}} & {‘a boy’}

{‘a stool’}

{‘a girl’}

{‘a reason’} & {\textit{ínan-}\textbf{\textit{k}}\textit{{}-a}}

{\textit{gámbar-}\textbf{\textit{k}}\textit{{}-a}}

{\textit{inán-}\textbf{\textit{t}}\textit{{}-a}}

{\textit{sabáb-}\textbf{\textit{t}}\textit{{}-a}} & {‘the boy’}

{‘the stool’}

{‘the girl’}

{‘the reason’}\\
\lspbottomrule
\end{tabularx}

\end{table} 


This gender-based distribution of tonal accent plays an important role in the traditional argumentation because of a group of nouns referred to as the 5th declension by Saeed (1993: 134; 1999: 61) and \citet[48]{Orwin1995}. In this group of nouns we find, for instance, \textit{dibi} ‘ox’. The nouns in the 5th declension lack an overt plural morpheme, and therefore they may be considered to exhibit “pure” gender polarity as they form their plural simply by changing their gender, which in the indefinite form is expressed merely by means of tonal accent shift.
 
\begin{table}
\caption{A noun in the 5th declension}
\label{tab:4}

\begin{tabularx}{\textwidth}{XXXXX} & {\textsc{indef}} &  & {\textsc{def}} & \\
\lsptoprule
{\textsc{singular}}

{\textsc{plural}} & {\textit{díbi}}

{\textit{dibí}} & {‘ox’} & {\textit{díbi-}\textbf{\textit{g}}\textit{{}-a}}

{\textit{dibí-}\textbf{\textit{d}}\textit{{}-a}} & {\textsc{masculine}}

{\textsc{feminine}}\\
\lspbottomrule
\end{tabularx}

\end{table} 

The plural \textit{dibí}, with the feminine definite form \textit{dibída}, does not contain any overt plural morpheme. The plural is simply expressed by the change in gender, which is realised in the form of the definite article and the typical feminine tonal accent on the last mora of the stem. 

Furthermore, with forms of the 5th declension such as \textit{dibída} ‘oxen’, traditionally considered to be feminine plural, agreeing verbs and pronouns may be either singular or plural. Consider (1), where \textit{dibída} occurs in the subject case form.

\ea
\ea
\gll \textup{a. } Dibí-d-u             waa  ay{\rmfnm}   daaq-ay-aan.   \\
        ox{\textbackslash}\textsc{pl-f.def-sbj  decl   pro.3}   graze-\textsc{prog-prs.3pl}\\
\ex
\langinfo{}{}{     ‘The oxen are grazing.’}\\
\gll \textup{b.  }Dibí-d-u        waa  ay   daaq-ay-saa.    \\
       ox{\textbackslash}\textsc{pl-f.def-sbj  decl   pro.3}  graze-\textsc{prog-prs.3fsg}\\
\glt   ‘The oxen are grazing.’
\z
\z

\footnotetext{The subject pronoun ay ‘she, it, they’, which is used in positive statements even if there is a subject noun, is homonymous in the feminine singular and in the plural.} 

In (1a) \textit{dibídu} ‘the oxen’ is followed by the predicate verb \textit{daaqayaan} ‘are grazing’ in the plural, whereas it may equally well be followed by the verb form \textit{daaqaysaa} ‘is grazing’ in the feminine singular (1b). This variation in agreement between the plural and the singular has been taken as evidence that plural forms like \textit{dibí} ‘oxen’ must be feminine, as the singular agreement marker is feminine (Hetzron 1972: 259{}--260, Zwicky \& Pullum 1983: 391{}--393; Lecarme 2002: 134{}--137). It has also been pointed out that the same type of variation occurs with plurals of Arabic origin which lack any Somali plural morpheme, e.g. the feminine plural \textit{kutub-t-a} ‘the books’ of the masculine singular noun \textit{kitaab-k-a} ‘the book’; or the feminine plural \textit{macallim-iin-t-a} ‘the teachers’ of the masculine singular \textit{macallin-k-a} ‘the teacher’.

The reason, however, for having singular verb agreement with a plural noun as in (1b) is not made very clear. In §3 of this article, I will argue that the reason for the variation in agreement is that these nouns should not be interpreted as plurals, but as collective nouns that are formally feminine singular. A strong general connection in Afroasiatic languages between feminine suffixation and the derivation of collective nouns was pointed out already by \citet{Speiser1938}, who at the same time was strongly opposed to the notion of gender polarity in general, but not specifically in Somali.

\section{No gender distinction in the plural}

Before going into the alternative analysis of the Somali data, it is crucial to take some typological considerations into account. Elaborating the claim by \citet[231]{Hockett1958} that gender is “reflected in the behavior of associated words”, Corbett (2013: 89{}--90) states that the “relevant ‘reflection’ in the associated words is agreement […]. No amount of marking on a noun can prove that the language has a gender system; the evidence that nouns have gender values in a given language lies in the agreement targets which show gender.” Building on Corbett’s definition, I hence assume that gender is not present in a noun unless it is reflected as agreement on other associated words. If morphological differences in the noun itself would be enough to distinguish between genders, there are as many as seven different realisations of the definite article suffix in the singular of Somali nouns. However, nobody has proposed seven genders, precisely because the agreement patterns in pronouns and verbs only give evidence for two genders. 

In order to establish whether Somali has a gender distinction in the plural, we will therefore have a closer look at the morphological exponents of agreement in Somali in other parts of speech than the noun itself. A practically exhaustive list of such exponents is found in \tabref{tab:5}.\footnote{No gender distinction is made in the singular of adjectives, and plurality is optionally expressed by means of reduplication of the initial CV(C).}


\begin{table}
\caption{Exponents of agreement in Somali}
\label{tab:5}

\begin{tabularx}{\textwidth}{XXXXX} &  &  & {\textsc{sg}} & {\textsc{pl}}\\
\lsptoprule
{\textsc{Verbs}} & {\textsc{Present indicative}} & {\textsc{m}} & {\textit{{}-aa}} & {\textit{{}-aan}}\\
\hhline{~~~~-} &  & {\textsc{f}} & {\textit{{}-taa}} & \\
\hhline{~----} & {\textsc{Past indicative}} & {\textsc{m}} & {\textit{{}-ay}} & {\textit{{}-een}}\\
\hhline{~~~~-} &  & {\textsc{f}} & {\textit{{}-tay}} & \\
\hhline{~----} & {\textsc{Present subjunctive}} & {\textsc{m}} & {\textit{{}-o}} & {\textit{{}-aan}}\\
\hhline{~~~~-} &  & {\textsc{f}} & {\textit{{}-to}} & \\
{\textsc{Pronouns}} & {\textsc{Reduced personal}} & {\textsc{m}} & {\textit{uu}} & {\textit{ay}}\\
\hhline{~~~~-} &  & {\textsc{f}} & {\textit{ay}} & \\
\hhline{~----} & {\textsc{Full personal}} & {\textsc{m}} & {\textit{isaga}} & {\textit{iyaga}}\\
\hhline{~~~~-} &  & {\textsc{f}} & {\textit{iyada}} & \\
\hhline{~----} & {\textsc{Demonstrative}} & {\textsc{m}} & {\textit{kan}} & {\textit{kuwan}}\\
\hhline{~~~~-} &  & {\textsc{f}} & {\textit{tan}} & \\
\hhline{~----} & {\textsc{Possessive}} & {\textsc{m}} & {\textit{{}-iis}} & {\textit{{}-ood}}\\
\hhline{~~~~-} &  & {\textsc{f}} & {{}-\textit{eed}} & \\
\hhline{----~}
\lspbottomrule
\end{tabularx}

\end{table} 

There exist a few other verb categories and types of pronouns than those listed in \tabref{tab:5}, but the pattern regarding the type of forms remains just the same. The important point is that there is not a single category in which we find a gender distinction in the plural. This is typologically quite a common pattern, and is also found in languages such as Russian, Swedish and German.

As it is clear that Somali does not exhibit any agreement distinctions in gender in the plural in any associated words, there is no reason to define the gender of a noun in the plural. The variation found in the morphology of plural nouns themselves, that is, in the form of the plural definite article, is instead easily predictable on morphophonological grounds. The singular gender and the syllabic structure of the noun are enough to make the right choice of the plural definite article. The principles are that 1) feminine nouns take the definite article \{k\} in the plural; 2) masculine nouns in which the plural morpheme is preceded by a monosyllabic stem take the definite article \{k\} in the plural;\footnote{The only systematic violations of this principle are exhibited by nouns forming their plural by reduplication of a final /l/. In this group of nouns, those with a long root vowel have a definite form with \{t\} (which becomes /ʃ/ under influence of the preceding /l/), probably due to analogy with the very many nouns ending in /e/ that all take the plural suffix \textit{{}-yaal}, e.g. \textit{wiilal} ‘boys’, def. \textit{wiilasha}, in line with \textit{aabbayaal} ‘fathers’, def. \textit{aabbayaasha}. Apart from these, there is also a very small number of high frequency words that behave in an irregular manner, e.g. \textit{walaal} m. ‘brother’ and \textit{walaal} f. ‘sister’, which both correspond to the plural \textit{walaalo} ‘siblings’, def. \textit{walaalaha}.} and 3) masculine nouns with bisyllabic or longer stems take the definite article \{t\} in the plural.

As there is no need for a gender distinction in the plural for agreement purposes, there is consequently no need to posit a gender shift for the majority of nouns as in the traditional analysis. Instead, just as in many other languages such as Russian, Swedish or German, a noun is most conveniently interpreted as having the same gender at all times, as proposed in \tabref{tab:6}. 
 
\begin{table}
\caption{Nouns exemplifying a simplified view of gender in Somali}
\label{tab:6}


\begin{tabularx}{\textwidth}{XXXX}
\lsptoprule
\multicolumn{2}{X}{{\textsc{feminine nouns}}

} &  & \\
{\textsc{sg.indef}} & {\textsc{sg. def}} & {\textsc{pl.indef}} & {\textsc{pl.def}}\\
{\textit{sabab} ‘reason’}

{\textit{mindi} ‘knife’}

{\textit{hooyo} ‘mother’} & {\textit{sabab-}\textbf{\textit{t}}\textit{{}-a}}

{\textit{mindi-}\textbf{\textit{d}}\textit{{}-a}}

{\textit{hooya-}\textbf{\textit{d}}\textit{{}-a}} & {\textit{sabab-o}}

{\textit{mindi-yo}}

{\textit{hooyo-oyin}} & {\textit{sabab-a-}\textbf{\textit{h}}\textit{{}-a}}

{\textit{mindi-ya-}\textbf{\textit{h}}\textit{{}-a}}

{\textit{hooyo-oyin-}\textbf{\textit{k}}\textit{{}-a}}\\
\multicolumn{2}{X}{{\textsc{masculine nouns}}

} &  & \\
{\textsc{sg.indef}} & {\textsc{sg.def}} & {\textsc{pl.indef}} & {\textsc{pl.def}}\\
&  & \multicolumn{2}{X}{{\textsc{monosyllabic stem}}

}\\
\hhline{~~--}
{\textit{miis} ‘table’}

{\textit{geed} ‘tree’}

{\textit{jilib} ‘knee’} & {\textit{miis-}\textbf{\textit{k}}\textit{{}-a}}

{\textit{geed-}\textbf{\textit{k}}\textit{{}-a}}

{\textit{jilib-}\textbf{\textit{k}}\textit{{}-a}} & {\textit{miis-as}}

{\textit{geed-o}}

{\textit{jilb-o}} & {\textit{miis-as-}\textbf{\textit{k}}\textit{{}-a}}

{\textit{geed-a-}\textbf{\textit{h}}\textit{{}-a}}

{\textit{jilb-a-}\textbf{\textit{h}}\textit{{}-a}}\\
&  & \multicolumn{2}{X}{{\textsc{polysyllabic stem}}

}\\
\hhline{~~--}
{\textit{gambar} ‘stool’}

{\textit{derbi} ‘wall’}

{\textit{aabbe} ‘father’} & {\textit{gambar-}\textbf{\textit{k}}\textit{{}-a}}

{\textit{derbi-}\textbf{\textit{g}}\textit{{}-a}}

{\textit{aabba-}\textbf{\textit{h}}\textit{{}-a}} & {\textit{gambar-ro}}

{\textit{derbi-yo}}

{\textit{aabba-yaal}} & {\textit{gambar-ra-}\textbf{\textit{d}}\textit{{}-a}}

{\textit{derbi-ya-}\textbf{\textit{d}}\textit{{}-a}}

{\textit{aabba-yaa-}\textbf{\textit{sh}}\textit{{}-a}}\\
\lspbottomrule
\end{tabularx}
\end{table} 


As becomes evident from \tabref{tab:6}, the polar syncretism is not total. Many masculine nouns have the same definite article both in the singular and in the plural.\footnote{Actually, according to my native speaker consultants, some Somali dialects in Ogaden only use the definite article \{k\} for all nouns in the plural, meaning that nouns have come to behave just like all the other parts of speech by exhibiting only one common form in the plural.} Actually, such polar syncretisms are not unique. To a varying extent, they can also be observed in other languages, for instance in the Swedish definite article (\tabref{tab:7}), or in the nominative dual and plural suffixes of Slovene nouns (\tabref{tab:8}).
 
\begin{table}
\caption{Definite forms of Swedish nouns}
\label{tab:7}

\begin{tabularx}{\textwidth}{XXXX}
\lsptoprule
{\textsc{defininte forms}} & {\textsc{singular}} &  & {\textsc{plural}}\\
{\textsc{common gender}}

{\textsc{neuter}} & {\textit{sten-}\textbf{\textit{en}}}

{\textit{hus-et}} & {‘the stone’}

{‘the house’} & {\textit{sten-arna}}

{\textit{hus-}\textbf{\textit{en}}}\\
\lspbottomrule
\end{tabularx}

\end{table} 

 
\begin{table}
\caption{Nominative forms of Slovene nouns}
\label{tab:8}
\begin{tabularx}{\textwidth}{XXXXX}
\lsptoprule
{\textsc{nominative forms}} & {\textsc{singular}} &  & {\textsc{dual}} & {\textsc{plural}}\\
{\textsc{masculine}}

{\textsc{neuter}} & {\textit{vlak}}

{\textit{okn-o}} & {‘train’}

{‘window’} & {\textit{vlak-}\textbf{\textit{a}}}

{\textit{okn-}\textbf{\textit{i}}} & {\textit{vlak-}\textbf{\textit{i}}}

{\textit{okn-}\textbf{\textit{a}}}\\
\lspbottomrule
\end{tabularx}

\end{table} 

\section{Agreement variation}

In this section I discuss agreement as evidence for gender polarity in Somali. It is crucial to recall, as was pointed out in \tabref{tab:5}, that there is no gender distinction in the exponents of agreement in the plural. The argument used by \citet{Hetzron1972}, Zwicky \& \citet{Pullum1983} and \citet{Lecarme2002} for forms like \textit{dibída} ‘the oxen’ to be feminine plural is not that they trigger agreement in the feminine plural, as no such form exists, but that they exhibit a variation in agreement between the plural form and the feminine singular form of a predicate verb. The discussion in the articles mentioned is limited to nouns in the 5th declension, such as \textit{tuugta} ‘the thieves’ in (2a) and (2b), and nouns with Arabic plural forms, such as \textit{kuraasta} ‘the chairs’ in (2c) and (2d), both groups showing a variation between plural and feminine singular agreement in the predicate verb.

\ea
\ea
\gll a. Si-d-ee             tuug-t-u             u  feker-aan?\\
       manner-\textsc{def-q} thieves-\textsc{def-sbj} in think-\textsc{prs.3pl}\\
\glt ‘How do thieves think?’
\ex
\gll b. Si-d-ee             tuug-t-u             u  feker-taa?\\
       manner-\textsc{def-q} thieves-\textsc{def-sbj} in think-\textsc{prs.3fsg}\\
\glt ‘How do thieves think?’
\ex
\gll c. Kuraas-t-u      waxa  ay     ku wareeg-san       yihiin         miis weyn.\\
       chairs-\textsc{def-sbj ffoc pro.3} in go.around\textsc{{}-ptcp }be\textsc{.prs.3pl} table big\\
\glt ‘The chairs are placed around a big table.’ 
\ex
\gll d. Kuraas-t-u      waxa  ay     ku wareeg-san       tahay           miis weyn.\\
       chairs-\textsc{def-sbj ffoc pro.3} in go.around\textsc{{}-ptcp }be\textsc{.prs.3fsg} table big\\
\glt ‘The chairs are placed around a big table.’ 
\z
\z

The literature does not discuss whether there are other nouns, apart from the 5th declension and Arabic-borrowing types, which exhibit similar variation in their agreement patterns. It is therefore important to point out that many other frequently-occurring nouns do exhibit the same type of agreement patterns in predicates and pronouns with regard to gender and number. In (3a) the feminine noun \textit{carrurtu} ‘the children’ triggers plural agreement in the finite verb and the possessive suffix \textit{{}-ood} ‘their’, whereas in (3b) the same noun is followed by the feminine singular finite verb \textit{karto} ‘can’. This noun, however, is traditionally not claimed to be plural, but instead to be a singular collective noun.

\ea
\ea
\gll a. Carruur-t-u          waxa ay      jecel  yihiin           waalid-k-ood.\\
       children-\textsc{def.f-sbj ffoc pro.3 }fond  be\textsc{{}-prs-3pl} parents-\textsc{def.m}{}-\textsc{poss.3pl}\\
\glt ‘The children love their parents.’
\ex
\gll b. Kor u qaad buug-ga  si          ay      carruur-t-u          u   arki kar-to.\\
     top to take book-\textsc{def} manner \textsc{pro.3} children-\textsc{def-sbj} for see can-\textsc{sbjv.3fsg}\\
\glt ‘Hold up the book so that the children can see it.’
\ex
\gll c. Maxaa     ay          samay-nay-aan     dad-k-u?\\
       what{\textbackslash}\textsc{foc pro.3}  do-\textsc{prog-prs.3pl}  people\textsc{{}-def-sbj}\\
\glt ‘What are people doing?’
\ex
\gll d. Ma sheegi kar-taa         xayawaan-no  uu        dad-k-u             dhaqd-o?\\
       \textsc{q}     tell     can-\textsc{prs.2sg} animal\textsc{{}-pl  pro.3msg}  people\textsc{{}-def-sbj} breed-\textsc{sbjv.3msg}\\
\glt ‘Can you say some animals that people breed?’
\z
\z

Even more interestingly, a substantial proportion of such nouns are masculine, and these nouns therefore trigger variation between the plural and the masculine singular. In (3c) the masculine noun \textit{dadku} ‘the people’ triggers plural agreement in the subject pronoun\footnote{Subject pronouns are used in positive statements even if there is a subject noun.} \textit{ay} and the finite verb \textit{samaynayaan} ‘do’, whereas in (3d) the very same noun is accompanied by the masculine singular subject pronoun \textit{uu} and the masculine singular finite verb \textit{dhaqdo} ‘breeds’. 

Having a larger set of data, it also becomes clear that this type of variation in predicate and pronoun agreement patterns occurs only with nouns which lack an overt synchronic Somali plural morpheme, but which have a meaning that may be perceived as plural, as in (4a) and (4b). However, nouns which have an indisputable plural morpheme, i.e. an overt synchronic Somali plural ending, never trigger any such variation. Such nouns are always accompanied by verbs and pronouns in the plural, as in (4c), whereas (4d) is incorrect.

\ea
\ea
\gll a. Dibi-d-u     w-ay     daaq-ay-aan.\\
       oxen-\textsc{def-sbj   decl-pro.3}  graze-\textsc{prog-prs.3pl}\\
\glt ‘The oxen are grazing.’
\ex
\gll a. Dibi-d-u    w-ay    daaq-ay-saa.\\
       oxen\textsc{{}-def-sbj  decl-pro.3}  graze-\textsc{prog-prs.3fsg}\\
\glt ‘The oxen are grazing.’
\ex
\gll c. Dibi-ya-d-u    w-ay    daaq-ay-aan.\\
       ox\textsc{{}-pl-def-sbj  decl-pro.3}  graze-\textsc{prog-prs.3pl}\\
\glt ‘The oxen are grazing.’
\ex
\gll d. *Dibi-ya-d-u    w-ay    daaq-ay-saa.\\
       ox\textsc{{}-pl-def-sbj  decl-pro.3}  graze\textsc{{}-prog-prs.3fsg}\\
\glt ‘The oxen are grazing.’
\z
\z

From a typological point of view, variation between agreement in the singular and the plural is quite common. It is reported from a diversity of languages, and it occurs, for instance, in both Swedish and English with words such as ‘the team’, as in (5) and (6).

\ea
\ea
\gll a.  Lag-et           kom          in på plan-en        inställ-\textbf{t}            på att vinn-a.\\
       team-\textsc{sg.def} come-\textsc{prt} in on field-\textsc{sg.def} determined-\textsc{sg} on to  win-\textsc{inf}\\

\ex
\gll b.  Lag-et           kom          in på plan-en        inställ-\textbf{da}        på att vinn-a.\\
       team-\textsc{sg.def}  come-\textsc{prt} in on field-\textsc{sg.def} determined-\textsc{pl} on to  win-\textsc{inf} \\
\glt ‘The team entered the playground determined to win.’
\z
\z

\ea a.  The team \textbf{is} friendly.
\z

\ea
 b.  The team \textbf{are} friendly.
\z

As pointed out by \citet[187]{Corbett2000}, the typologically interesting distinction is between two types of agreement, namely syntactic agreement, determined by the form of the noun; and semantic agreement, determined by the meaning of the noun. Typologically, there are certain types of nouns that typically trigger this kind of variation, and Somali agreement variation fits neatly into this typological pattern, as the Somali nouns triggering such variation belong to typologically expected categories.

First, collective nouns commonly cause this type of variation in a number of languages. Corbett (2000: 118–119) uses the term \textit{collective} to indicate that a noun is “referring to a group of items considered together rather than a number of items considered individually. […] The primary function of collectives is to specify the cohesion of a group”. Most of the Somali nouns exhibiting variation in their agreement patterns can be included in this category. Actually, Puglielli \& \citet[82]{Siyaad1984} state that the plurals of the Somali 5th declension have been recategorised as collective forms, but despite this claim they still choose to treat forms like \textit{dibída} ‘oxen’ as formally plural.

It should, however, be pointed out that the collective nouns of the 5th declension in most instances also have a regular plural counterpart based on the masculine singular form of the same word; hence there are two forms expressing plural meaning, but only one form with a plural morpheme. Because the feminine collective form exhibits agreement variation in the predicate verb, allowing both the plural and the feminine singular verb forms, I claim that the feminine collective form should be treated as formally singular, as shown in \tabref{tab:9}.
 
\begin{table}
\caption{Typical forms of nouns with both a plural and a collective form}
\label{tab:9}

\begin{tabularx}{\textwidth}{XXXXX} & {\textsc{sg.indef}} & {\textsc{sg.def}} & {\textsc{pl.indef}} & {\textsc{pl.def}}\\
\lsptoprule
{\textsc{m}}

{\textsc{f.coll}} & {\textit{díbi  }‘ox’}

{\textit{dibí  }‘oxen’} & {\textit{díbi-}\textbf{\textit{g}}\textit{{}-a  }‘the ox’ }

{\textit{dibí-}\textbf{\textit{d}}\textit{{}-a  }‘the oxen’} & {\textit{dibi-yó  }‘oxen’} & {\textit{dibi-yá-}\textbf{\textit{d}}\textit{{}-a  }‘the oxen’}\\
\lspbottomrule
\end{tabularx}

\end{table} 


The type of nouns found in the 5th declension are highly interesting, as these nouns exhibit a singular, a regular plural and a collective form based on the very same root. The number of such noun stems is just a couple of dozen, but interestingly enough some newer words, often not mentioned in the literature, have also found their way into this group, e.g. the Arabic\footnote{Which in its turn is most probably a loan from the French \textit{vapeur} ‘steam’ (Jan Retsö, personal communication).} loanword \textit{baabuur} ‘car’, illustrated by (4), as well as the English loanword \textit{buug} ‘book’.

\ea
\gll Baabuur-t-u     waa    ay    nooc-yo  badan   yihiin.\\
     car{\textbackslash}\textsc{f.coll-def-sbj  decl    pro.3  }type-\textsc{pl}  many   be.\textsc{prs.3pl}\\
\glt ‘There are cars of many types.’
\z

\tabref{tab:10} presents some further nouns of the 5th declension type with both their regular plural form and their collective singular form.
 
\begin{table}
\caption{Further nouns with both a plural and a collective form}
\label{tab:10}

\begin{tabularx}{\textwidth}{XXXXX} & {\textsc{sg.def}} &  & {\textsc{pl.def}} & \\
\lsptoprule
{\textsc{m}}

{\textsc{f.coll}} & {\textit{baabúur-}\textbf{\textit{k}}\textit{{}-a}}

{\textit{baabuúr-}\textbf{\textit{t}}\textit{{}-a}} & {‘the car’}

{‘the cars’} & {\textit{baabuur-rá-}\textbf{\textit{d}}\textit{{}-a}} & {‘the cars’}\\
{\textsc{m}}

{\textsc{f.coll}} & {\textit{búug-}\textbf{\textit{g}}\textit{{}-a}}

{\textit{buúg-}\textbf{\textit{t}}\textit{{}-a}} & {‘the book’}

{‘the books’} & {\textit{buug-ág-}\textbf{\textit{g}}\textit{{}-a}\textsuperscript{a}} & {‘the books’}\\
{\textsc{m}}

{\textsc{f.coll}} & {\textit{túug-}\textbf{\textit{g}}\textit{{}-a}}

{\textit{tuúg-}\textbf{\textit{t}}\textit{{}-a}} & {‘the thief’}

{‘the thieves’} & {\textit{tuug-ág-}\textbf{\textit{g}}\textit{{}-a}} & {‘the thieves’}\\
\lspbottomrule
\end{tabularx}

\end{table} 

\textsuperscript{a} \textit{Buugagga} is the regular form. There is, however, a general preference for an irregular plural with a long epenthetic /a:/ and the article \{t\}, i.e. \textit{buugaagta} ‘the books’.

Interestingly enough, a very similar pattern can be found in Swedish for the noun \textit{mygga} ‘mosquito’ (\tabref{tab:11}).\textit{ }


\begin{table}
\caption{Forms of the Swedish noun \textit{mygga} ‘mosquito’}
\label{tab:11}
\begin{tabularx}{\textwidth}{XXXXX} & {\textsc{sg.indef}} & {\textsc{sg.def}} & {\textsc{pl.indef}} & {\textsc{pl.def}}\\
\lsptoprule
{\textsc{indiv}}

{\textsc{coll}} & {\textit{mygg-a}}

{\textit{mygg}} & {\textit{mygg-a-n}}

{\textit{mygg-en}} & {\textit{mygg-or}} & {\textit{mygg-or-na}}\\
\lspbottomrule
\end{tabularx}

\end{table} 



Judging from further corpus data, it also becomes clear that borrowed Arabic plural forms behave in the same manner as indigenous Somali collective nouns. The varying agreement is also confirmed by Puglielli \& \citet[86]{Siyaad1984}, but they refrain from calling these forms collectives. I will, however, argue that such feminine forms, containing Arabic plural morphemes, are not plurals in Somali, but collective forms which are grammatically singular. The argument is based on the same type of variation in agreement as that encountered with nouns of the 5th declension, and on the fact that these nouns also exhibit a regular plural with an overt Somali plural morpheme (\tabref{tab:12}).
 
\begin{table}
\caption{Typical forms of nouns exhibiting a borrowed Arabic plural}
\label{tab:12}
\begin{tabularx}{\textwidth}{XXXXX} & {\textsc{sg.indef}} & {\textsc{sg.def}} & {\textsc{pl.indef}} & {\textsc{pl.def}}\\
\lsptoprule
{\textsc{m}}

{\textsc{f.coll}} & {\textit{kursi} ‘chair’}

{\textit{kuraas}} & {\textit{kursi-}\textbf{\textit{g}}\textit{{}-a}}

{\textit{kuraas-}\textbf{\textit{t}}\textit{{}-a}} & {\textit{kursi-yo}} & {\textit{kursi-ya-}\textbf{\textit{d}}\textit{{}-a}}\\
\lspbottomrule
\end{tabularx}

\end{table} 

Some speakers prefer the plural forms over collective forms when referring to a smaller number of individualised objects, whereas the collective nouns refer to many objects as a coherent group. 

Other examples of nouns that behave as collective nouns, not included within the two types already discussed, are the feminine nouns \textit{carruur-t-a} ‘the children’ and \textit{lo’-d-a} ‘the cattle’, as well as the masculine nouns \textit{dumar-k-a} ‘the women’, \textit{dad-k-a} ‘the people’, and \textit{rag-g-a} ‘the men’. None of these has a corresponding individualising singular form based on the same root. Instead, they have corresponding individualising forms based on another root, as shown in \tabref{tab:13}.


\begin{table}
\caption{Different lexemes for collective and individualising meanings}
\label{tab:13}
\begin{tabularx}{\textwidth}{XXXXX} & {\textsc{sg.def}} &  & {\textsc{pl.def}} & \\
\lsptoprule
{\textsc{f}}

{\textsc{m.coll}} & {\textit{naag-}\textbf{\textit{t}}\textit{{}-a}}

{\textit{dumar-}\textbf{\textit{k}}\textit{{}-a}} & {‘the woman’}

{‘the women’} & {\textit{naag-a-}\textbf{\textit{h}}\textit{{}-a}} & {‘the women’}\\
{\textsc{m}}

{\textsc{m.coll}} & {\textit{nin-}\textbf{\textit{k}}\textit{{}-a}}

{\textit{rag-}\textbf{\textit{g}}\textit{{}-a}} & {‘the man’}

{‘the men’} & {\textit{nim-an-}\textbf{\textit{k}}\textit{{}-a}} & {‘the men’}\\
\lspbottomrule
\end{tabularx}

\end{table} 



Another interesting group of Somali nouns, which are indisputably singular, form a totally regular plural; and yet, in the singular, they may trigger both syntactic agreement in the singular and semantic agreement in the plural. These words are, according to \citet[188]{Corbett2000}, so-called \textit{corporate nouns}, i.e. “nouns which are singular morphologically and (typically) have a normal plural and yet, when singular, may take plural agreement”. This pattern can frequently be observed in Somali and seems to be equally common in both genders. Some examples are in \tabref{tab:14}.
 
\begin{table}
\caption{Somali corporate nouns}
\label{tab:14}
\begin{tabularx}{\textwidth}{XXXXX}
\lsptoprule
{\textsc{sg.indef}} &  & {\textsc{sg.def}} & {\textsc{pl.indef}} & {\textsc{pl.def}}\\
{\textit{qoys}} & {‘family’} & {\textit{qoys-}\textbf{\textit{k}}\textit{{}-a}} & {\textit{qoys-as}} & {\textit{qoys-as-}\textbf{\textit{k}}\textit{{}-a}}\\
{\textit{koox}} & {‘group, team’} & {\textit{koox-}\textbf{\textit{d}}\textit{{}-a}} & {\textit{koox-o}} & {\textit{koox-a-}\textbf{\textit{h}}\textit{{}-a}}\\
{\textit{geel}} & {‘herd of camels’} & {\textit{geel-a}\textsuperscript{a}} & {\textit{geel-al}} & {\textit{geel-a-}\textbf{\textit{sh}}\textit{{}-a}}\\
\lspbottomrule
\end{tabularx}

\end{table} 


\textsuperscript{a} This noun is a unique exception, realising the definite article \{k\} as zero in the singular.

In (8), we find the masculine singular noun \textit{qoyskiisu} ‘his family’ with the predicate verb \textit{dhaqdaan} ‘breed’ and the subject pronoun \textit{ay}, here\footnote{The homonymous subject pronoun \textit{ay} ‘they; she, it’ represents the 3\textsuperscript{rd} person plural as well as the feminine 3\textsuperscript{rd} person singular. When it occurs together with a masculine subject noun, it may therefore only be interpreted as plural.} representing the plural meaning ‘they’. The verb and the pronoun could, however, just as well have been in the masculine singular, like the predicate verb \textit{yahay} ‘is’ and the subject pronoun \textit{uu} ‘he, it’ in (9).

\ea
\gll Faarax  qoys-\textbf{k}{}-iis-u        waxa  \textbf{ay}    dhaqd-\textbf{aan}  geel.\\
     Faarax  family{\textbackslash}\textsc{msg-def-poss.3msg-sbj    ffoc}  \textsc{pro.3}  breed-\textsc{prs.3pl}  camel{\textbackslash}\textsc{m.coll}\\
\glt ‘Faarax’s family breeds camels.’
\z

\ea
\gll Qoys-\textbf{k}{}-iis-u      waxa  \textbf{uu}    ka  kooban    \textbf{yahay}    shan  ruux.\\
     family{\textbackslash}\textsc{msg-def-poss.3msg-sbj  ffoc  pro.3msg}  of  consisting  be.\textsc{prs.3msg}  five  person\\
\glt ‘His family consists of five persons.’
\z

Typologically, many languages also exhibit variation in number when nouns are used with a generic meaning. English, as well as Swedish, exhibits such variation between the definite singular and the indefinite plural form of the generic noun itself, as shown in (10).

\ea
 a.  The tiger is in danger of becoming extinct.\\
 b.  Tigers are in danger of becoming extinct.\\
\z

In Somali, generic meaning is very often expressed by the singular definite form of the noun, as with \textit{diinku} ‘the turtle’ in (11). The agreement of accompanying constituents is, however, often in the plural, like the verb \textit{sameeyaan} ‘make’ in the same example.

\ea
\gll Diin-k-u    inta  badan  ma  sameey-\textbf{aan}  wax  dhaqdhaqaaq  ah.\\
     turtle{\textbackslash}\textsc{msg-def-sbj  }amount  much  not  make-\textsc{prs.3pl  }thing  movement  being.\\
\glt ‘For a long period of time turtles don’t make any movements.’
\z

Based on the quite diverse categories of nouns which have been shown to trigger variation between singular and plural agreement, I claim that such variation is a typical general trait of Somali syntax, reaching far beyond the examples frequently discussed as results of the traditionally posited gender polarity.  

\section{Summary}

Building on the data presented in this article, I would like to make two important claims. 


\begin{itemize}
\item In Somali, gender is not a relevant category in the plural as there is no need to refer to the gender of a noun in the plural for the sake of agreement. Therefore, nouns should only be ascribed one gender value based on their behaviour in the singular. 

\item Variation in number in agreement patterns has nothing to do with plural gender. Instead, nouns that trigger variation in number agreement are grammatically singular. They are collective nouns, corporate nouns or common nouns used in a generic sense. Agreement in the singular is syntactically conditioned, whereas agreement in the plural is semantically conditioned.

\end{itemize}

Therefore, instead of the traditional view referred to as \textit{gender polarity}, presented above in \tabref{tab:1}, I propose a far more simple interpretation of the definite articles and the gender system, as shown in \tabref{tab:15}. This analysis of the Somali gender system and its morphological exponents of definiteness is typologically uncontroversial. The notion of polarity may, of course, still be applied; but if so, only in order to refer to the morphological exponents (i.e. forms) of definiteness.


\begin{table}
\caption{Polarity of the exponents of definiteness in Somali nouns}
\label{tab:15}

\begin{tabularx}{\textwidth}{XXX} & \textsc{Singular}& \textsc{Plural}\par\\
\lsptoprule
 \textsc{Feminine}& \{t\}& \{k\}\par\\
\hhline{--~} &  & \\
\hhline{~~-}
 \textsc{Masculine}& \{k\}& \{t\}\par\\
\lspbottomrule
\end{tabularx}

\end{table} 

\section*{Acknowledgements}

I wish to express my sincere gratitude to the more than 30 students, all L1 speakers of Somali, in the Somali mother tongue teachers’ programme at the University of Gothenburg for the many enlightening discussions and remarks during our grammar classes in 2014/2015. I also want to thank Doris Payne, Moubarak Ahmed and an anonymous reviewer for many useful comments and suggestions during the work on this text.

The Somali examples have primarily been retrieved from the Somali corpus at the Swedish Language Bank {\textless}http://spraakbanken.gu.se/korp/?mode=somali{\textgreater}, but some examples have also been constructed in cooperation with the already mentioned native speakers.

\section*{Abbreviations}

2  2nd person

3  3rd person

\textsc{coll}  collective

\textsc{decl}  declarative particle

\textsc{def}  definite

\textsc{f}  feminine

\textsc{ffoc}  final focus, i.e. the clause final NP is focused

\textsc{foc}  focus particle

\textsc{fsg}  feminine singular

\textsc{indef}  indefinite

\textsc{indiv}  individualising

\textsc{inf  }infinitive

\textsc{m}  masculine

\textsc{msg}  masculine singular

\textsc{pl}  plural

\textsc{poss}  possessive

\textsc{pro}  pronoun

\textsc{prog}  progressive

\textsc{prs}  present

\textsc{prt}  preterite

\textsc{ptcp  }participle

\textsc{q  }question marker

\textsc{sbj}  subject

\textsc{sg}  singular
 
 

\printbibliography[heading=subbibliography,notkeyword=this]

\end{document}