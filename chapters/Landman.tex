\documentclass[output=paper]{langsci/langscibook} 
\title{Quantification in Logoori} 
\author{%
 Meredith Landman \affiliation{Pomona College} 
}
% \chapterDOI{} %will be filled in at production


\abstract{
In this paper I examine how quantification is expressed in Logoori, a Luyia (Bantu) language spoken in western Kenya. I focus on the two universal quantifiers in Logoori, viz., \textup{vuri} ‘every' and -\textup{oosi} ‘all'. I show that these two quantifiers display a number of syntactic and semantic differences and present a compositional analysis to account for those differences. Throughout, I discuss how the Logoori patterns relate to previous cross-linguistic work on quantification, both on Bantu \citep{ZerbianKrifka2008} as well as across languages more generally \citep{Matthewson2013}.

}

\maketitle
\begin{document}
 
% Keywords: Bantu, Luyia, semantics, syntax, quantification
 

\section{Introduction}

In this paper I examine how quantification is expressed in Logoori [ISO 639-3
% \href{http://www-01.sil.org/iso639-3/documentation.asp?id=rag}
{rag}], a Luyia (Bantu) language spoken in western Kenya.\footnote{All data are from field notes collected via elicitation interviews with Isaac K. Thomas, a native Logoori speaker in his late 30s. In the orthographic conventions I use here, \textit{ng'} represents a velar nasal [ŋ], \textit{ny} a palatal nasal [ɲ], \textit{y} a palatal glide [j], and \textit{ch} a voiceless palatal affricate [tʃ]; tone is not transcribed.
} I focus mainly on the two universal quantifiers in Logoori, namely, \textit{vuri} ‘every’ \REF{ex:landman:1a} and -\textit{oosi} ‘all’ \REF{ex:landman:1b}.

\ea\label{ex:landman:1}
\ea\label{ex:landman:1a}
\gll \textbf{vuri}  muundu  a-syeev-i\\
     every  1person  1-dance-\textsc{pst}\\
\glt ‘Every person danced.’

\ex\label{ex:landman:1b}
\gll   vaandu  v-\textbf{oosi}    va-syeev-i\\
     2person  2-all    2\textsc{sa}-dance-\textsc{pst}\\
\glt ‘All the people danced.’
\z
\z

As I will show, these two quantifiers display a number of syntactic and semantic differences. Though I focus on the two universal quantifiers, I will also compare their behavior to other adnominal quantifiers in Logoori, such as -\textit{lla} ‘one, some’ \REF{ex:landman:2a} and -\textit{iinge} ‘many, much’ \REF{ex:landman:2b}.


\ea\label{ex:landman:2} 
\ea\label{ex:landman:2a} 
\gll   vaandu  va-\textbf{lla}    va-syeev-i\\
       2person	2-many            \textsc{sa}-dance-\textsc{pst}\\
\glt ‘Some people danced.’


\ex\label{ex:landman:2b} 
 \gll vaandu  v-\textbf{iinge}  va-syeev-i\\
     2person  2-many  2\textsc{sa}-dance-\textsc{pst}\\
\glt ‘Many people danced.’\\
\z
\z

My main aims for this paper are two. First, to present compositional analysis of (universally) quantified nominals in Logoori; and second, to add to our knowledge of how quantification is expressed across languages (cf. \citealt{Matthewson2013}), and in Bantu specifically \citep{ZerbianKrifka2008}. As the latter authors point out in their recent survey of Bantu quantification, more work is needed in this area:

\begin{quote}
A literature review on quantification in (whatever) Bantu languages reveals that few studies exist which touch upon quantification... The huge variety found among the Bantu languages as well as the gaps in documentation necessitate further detailed work on aspects of quantification. \citep[383, 412]{ZerbianKrifka2008}
\end{quote}

The remainder of this paper is organized as follows. In \sectref{sec:landman:2}, I introduce the basic differences between the two universal quantifiers. In \sectref{sec:landman:3}, I present a compositional analysis of the quantifiers, taking as a starting point work by \citet{Matthewson2013} on quantification across languages. In \sectref{sec:landman:4}, I consider some additional patterns that fall outside the scope of the proposed analysis; and in \sectref{sec:landman:5}, I conclude the paper and articulate some questions for further research.

\section{Basic differences between the two universal quantifiers}\label{sec:landman:2}
\subsection{The main semantic difference: Distributivity} 

The main semantic difference between \textit{vuri} and -\textit{oosi} regards distributivity (see \citealt{Gil1995}; \citealt{Vendler1962}; among others). While -\textit{oosi} is non-distributive (i.e., it permits distributive or collective interpretations), \textit{vuri} is necessarily distributive. Consider, e.g., \REF{ex:landman:3}, which is ambiguous. On its distributive reading, \REF{ex:landman:3} is true just in case each person individually lifted a car. On its collective reading, \REF{ex:landman:3} is true just in case all the people together lifted a car.

\ea\label{ex:landman:3}
\gll vaandu  v-oosi    va-geeng-i    mudoga\\
     2person  2-all    2\textsc{sa}-lift-\textsc{pst}    car\\
\glt ‘All the people lifted a car.’  {\checkmark}distributive, {\checkmark}collective
\z

\textit{Vuri}, in contrast, only permits a distributive reading:

\ea\label{ex:landman:4}
\gll vuri  muundu  a-geeng-i    mudoga\\
     every  1-person  1\textsc{sa}-lift-\textsc{pst}    car\\
\glt ‘Every person lifted a car.’  {\checkmark}distributive, {\texttimes}collective
\z

Accordingly, when a collective reading is forced, for example by adding the adverb \textit{halla} ‘together’ as in \REF{ex:landman:5}, -\textit{oosi} is grammatical \REF{ex:landman:5a} while \textit{vuri} is not \REF{ex:landman:5b}.    

\ea\label{ex:landman:5} 
\ea\label{ex:landman:5a} 
\gll vaandu    v-oosi  va-geeng-i    mudoga  \textbf{halla}\\
     2person  2-all  2\textsc{sa}-lift-\textsc{pst}  car    together\\
\glt ‘All the people lifted a car together.’
\ex\label{ex:landman:5b}
\gll *vuri  muundu  a-geeng-i  mudoga  \textbf{halla}\\
     every  1-person  1\textsc{sa}-lift-\textsc{pst}  car    together\\
\z
\z

Similarly, inherently collective predicates such as \textit{kuvugaana} ‘to gather, meet’ may combine with -\textit{oosi} \REF{ex:landman:6a} but not with \textit{vuri} \REF{ex:landman:6b}.

\ea\label{ex:landman:6}
\ea\label{ex:landman:6a}
\gll vaandu  v-oosi    va-vugaan-i\\
     2person  2-all    2\textsc{sa}-gather-\textsc{pst}  \\
\glt ‘All the people gathered.’
\ex\label{ex:landman:6b}
\gll *vuri  muundu  a-vugaan-i  \\
     every  1-person  1\textsc{sa}-gather-\textsc{pst}  \\
\z
\z

This semantic difference (i.e., distributivity) is also apparent in the types of nominals each quantifier may combine with. As is typical for a distributive universal quantifier (cf. \citealt{Gil1995}), \textit{vuri} most naturally combines with singular count nouns \REF{ex:landman:7a}. If \textit{vuri} combines with a plural \REF{ex:landman:7b} or mass noun \REF{ex:landman:7c}, individuated readings result, e.g., groups of books and bottles of water. In contrast, -\textit{oosi} may naturally combine with singular count nouns \REF{ex:landman:8a}, plural count nouns \REF{ex:landman:8b}, or mass nouns \REF{ex:landman:8c}.

\ea\label{ex:landman:7}
\ea\label{ex:landman:7a}
\textsc{sg} count\\
\gll soom-i      vuri    ki-tabu    \\
     1\textsc{sg.sa}.read-\textsc{pst}  every    7-book  \\
\glt ‘I read every book.’
\ex\label{ex:landman:7b}
\textsc{pl} count\\
\gll soom-i      vuri    vi-tabu    \\
     1\textsc{sg.sa}.read-\textsc{pst}  every    8-book\\
\glt ‘I read every group of books.’
\ex\label{ex:landman:7c}
\textsc{mass}\\
\gll ngur-i      vuri    ma-zi    \\
     1\textsc{sg.sa}.buy-\textsc{pst}    every    6-water\\
\glt ‘I bought every (bottle of) water.’
\z
\z

\ea\label{ex:landman:8}
\ea\label{ex:landman:8a}
\textsc{sg} count\\
\gll soom-i      ki-tabu  ch-oosi    \\
     1\textsc{sg.sa}.read-\textsc{pst}  7-book  7-all\\
\glt ‘I read all of the book.’ or ‘I read the whole book.’
\ex\label{ex:landman:8b}
\textsc{pl} count\\
\gll soom-i      vi-tabu  vy-oosi  \\
     1\textsc{sg.sa}.read-\textsc{pst}  8-book  8-all\\
\glt ‘I read all the books.’
\ex\label{ex:landman:8c}
\textsc{mass}\\
\gll ngur-i      ma-zi    g-oosi  \\
     1\textsc{sg.sa}.buy-\textsc{pst}    6-water  6-all\\
\glt ‘I bought all the water.’
\z
\z

\subsection{Syntactic and morphological differences}

\textit{Vuri} and -\textit{oosi} also display a number of syntactic and morphological differences. I observe five here.

First, -\textit{oosi} is post-nominal (9a-b), while \textit{vuri} is pre-nominal (10a-b). 

\ea\label{ex:landman:9}
\ea
\gll vaandu    v-oosi    va-syeev-i    \\
     2person    2-all    2\textsc{sa}-dance-\textsc{pst}       \\
\glt ‘All the people danced.’
\ex\label{ex:landman:9b}
\gll \textup{*}v-oosi    vaandu  va-syeev-i  \\
     2-all    2person  2\textsc{sa}-dance-\textsc{pst}  \\
\z
\z

\ea\label{ex:landman:10}
\ea
\gll vuri    muundu  a-syeev-i    \\
     every    1person  1\textsc{sa}-dance-\textsc{pst}       \\
\glt ‘Everyone danced.’
\ex\label{ex:landman:10b}
\gll \textup{*}muundu  vuri    a-syeev-i  \\
     1person    every    1\textsc{sa}-dance-\textsc{pst}    \\
\z
\z

In this respect, -\textit{oosi} patterns with all other adnominal modifiers (such as adjectives, numerals, possessives, demonstratives, and relative clauses). These also canonically appear post-nominally (11a-b). 

\ea\label{ex:landman:11}
\ea
\gll vaandu  \{yavo/va-lla/va-vere/va-nene\}    va-syeevi  \\
     2person  \{2those/2-one/2-two/2-important\} 2\textsc{sa}-dance-\textsc{pst}\\
\glt ‘\{Those/some/two/important\} people danced.’
\ex\label{ex:landman:11b}
\gll *\{yavo/va-lla/va-vere/va-nene\}  vaandu   va-syeevi\\
     \{2those/2-one/2-two/2-important\}  2person  2\textsc{sa}-dance-\textsc{pst}\\
\z
\z

A second syntactic difference regards co-occurrence with a pronoun. While -\textit{oosi} may co-occur with a pronoun \REF{ex:landman:12a}, \textit{vuri} may not \REF{ex:landman:12b}. Here too, -\textit{oosi} patterns with other adnominal modifiers, which also may co-occur with a pronoun \REF{ex:landman:12c}.

\ea\label{ex:landman:12}
\ea\label{ex:landman:12a}
\gll kunyi  v-oosi    ku-syeev-i      \\
     we  2-all    1\textsc{pl}.\textsc{sa}-dance-\textsc{pst}\\
\glt 'We all danced.'
\ex\label{ex:landman:12b}
\gll *vuri  kunyi    \{a-syeev-i / ku-syeevi\}  \\
     every  we    \{1\textsc{sa}-dance-\textsc{pst} / 1\textsc{pl}.\textsc{sa}-dance-\textsc{pst\}}\\
\glt (Intended interpretation: ‘We each danced.’)
\ex\label{ex:landman:12c}
\gll kunyi \{va-lla/va-vere/va-nene/v-iinge\}     ku-syeev-i\\
     we    \{2-one/2-two/2-important/2-many\} 1\textsc{pl}.\textsc{sa}-dance-\textsc{pst}\\
\glt ‘We \{some/two/important/many\} danced.’
\z
\z

A third difference regards co-occurrence with a null head noun. While -\textit{oosi} may appear on its own, i.e., with a null head noun \REF{ex:landman:13a}, \textit{vuri} may not \REF{ex:landman:13b}.\footnote{Examples \REF{ex:landman:13a} and \REF{ex:landman:13c} can only be used when it is clear from the context what the head noun refers to, e.g., in answer to a question like ‘How many people danced?’} Here, too, -\textit{oosi} patterns with all other adnominal expressions, which may also appear on their own \REF{ex:landman:13c}.

\ea\label{ex:landman:13}
\ea\label{ex:landman:13a}
\gll v-oosi    va-syeev-i      \\
     2-all    2\textsc{sa}-dance-\textsc{pst}\\
\glt ‘All danced.’
\ex\label{ex:landman:13b}
\gll *vuri    a-syeev-i  \\
     every    1\textsc{sa}-dance-\textsc{pst}\\
\glt (Intended interpretation: ‘Everyone danced.’)
\ex\label{ex:landman:13c}
\gll \{va-lla/va-vere/va-nene/v-iinge\}  va-syeev-i\\
     \{2-one/2-two/2-important/2-many\}  1\textsc{pl}.\textsc{sa}-dance-\textsc{pst}\\
\glt ‘\{Some/two/important/many\} danced.’
\z
\z

  Finally, -\textit{oosi} must agree in noun class with the head noun, as the paradigm in \tabref{tab:Landman:1} shows. \textit{Vuri}, in contrast, is morphologically invariant.

%%please move \begin{table} just above \begin{tabular} 
\todo{reduce width of table and distance between columns}
\begin{table}
\caption{Noun class agreement paradigm for -\textit{oosi}}
\label{tab:Landman:1}
\begin{tabularx}{\textwidth}{XX}
\lsptoprule
\textsc{singular} & \textsc{plural}\\
\midrule
1  \textit{woosi} & 2  \textit{voosi}\\

3  \textit{gwoosi} & 4  \textit{joosi}\\

5  \textit{rioosi} & 6  \textit{goosi}\\

7  \textit{choosi} & 8  \textit{vyoosi} \\

9  \textit{yoosi} & 10  \textit{zyoosi} \\

11  \textit{ruwoosi} & 12  \textit{koosi}\\

13  \textit{twoosi} & 14  \textit{vwoosi}\\

& 10  \textit{gwoosi}\\
\lspbottomrule
\end{tabularx}
\end{table}


In this respect, too, -\textit{oosi} behaves like all other adnominal modifiers, which also must agree with the head noun.\footnote{However, -\textit{oosi} displays the same agreement morphology as demonstratives, rather than adjectives. This sets -\textit{oosi} apart from the other two Logoori quantifiers, -\textit{lla} ‘one, some’ and -\textit{iinge} ‘many’, which do agree like adjectives. A similar pattern is observed by \citet{Krifka1995} and \citet{ZerbianKrifka2008} for Swahili -\textit{ote} ‘all’.
} 

\subsection{Summary of differences between the two universal quantifiers}

\tabref{tab:Landman:2} provides a summary of the differences between -\textit{oosi} and \textit{vuri}. In brief, their semantic properties accord with familiar differences between non-distributive and distributive quantifiers. Syntactically speaking, we see a divide that will factor into the analysis developed below.

%%please move \begin{table} just above \begin{tabular
\begin{table}
\caption{Summary of differences between -\textit{oosi} and \textit{vuri}}
\label{tab:Landman:2}


\begin{tabularx}{\textwidth}{p{5cm}XX} 
\lsptoprule
& \textit{oosi} &   \textit{vuri}\\
\midrule
{Distributive only}& yes   & no \\
{Combines with \textsc{sg} count}& yes   &  no\\
{Combines with \textsc{pl} count}& yes   & if individuated\\
{Combines with \textsc{mass}}& yes   & if individuated\\
Post-nominal& yes   & no \\
Co-occurs with a pronoun& yes   & no \\
Occurs on its own& yes   & no \\
Agrees in noun class & yes   & no  \\
\lspbottomrule
\end{tabularx}
\end{table}

\section{Analysis}\label{sec:landman:3}

In this section, I present a compositional analysis of \textit{vuri} and -\textit{oosi}. I take as a starting point the cross-linguistic generalizations for different types of universal quantifiers observed by \citet{Matthewson2013}. 

\subsection{Cross-linguistic generalizations for universal quantifiers}

\citet{Matthewson2013} presents a preliminary typology of quantifiers. She looks at 37 languages from 25 different families and finds that while there is variation in the syntactic behavior of different quantifiers, the syntax/semantics correspondence is not random. Specifically, she reports the following two tendencies for universal quantifiers. First, she observes that distributive universal quantifiers such as English \textit{every} tend to ``combine directly with NP, while other quantifiers do not." \citep[36]{Matthewson2013}. That is, distributive quantifiers tend to be determiner quantifiers (henceforth D-quantifiers) (as in \citealt{BarwiseCooper1981}). Syntactically, a D-quantifier heads a DP and combines directly with NP (see e.g. \citealt[146]{HeimKratzer1998}): 

\ea\label{ex:landman:}
$[_{DP} [_{D} every] [_{NP} person]]$
\z

Semantically, a D-quantifier combines with an NP predicate, type <\textit{e}, \textit{t}>, to form a generalized quantifier, type <\textit{e}, \textit{t}>, \textit{t}> \citep{Montague1973,BarwiseCooper1981}. Accordingly, as a universal quantifier, \textit{every} can be assigned the lexical denotation in \REF{ex:landman:15}:
 
\ea\label{ex:landman:15}
$⟦every⟧ = [{\lambda}f_{<e,t>} . [{\lambda}g_{<e,t>}. {\forall}x [f(x) \to g(x)]]]$
\z

This denotation would yield a distributive interpretation for \textit{every}, stipulating that quantification is over atomic individuals.\footnote{ Distributivity may alternatively come from another source; this is not crucial to my analysis.}

Matthewson's second generalization regards universal quantifiers translated as ‘all’. These quantifiers, she observes, tend to combine with a full DP. For example, English \textit{all} can be analyzed syntactically as a Q (henceforth, a Q-quantifier), which combines with a full DP to form a QP \citep{Matthewson2001}:  

\ea\label{ex:landman:}
$[_{QP} [_{Q} all] [_{DP} {the\ people}]]$
\z

Semantically, \textit{all} combines with an individual-denoting DP (such as a definite plural), type \textit{e}, to produce a generalized quantifier, type <\textit{e},\textit{ t}>, \textit{t}>. I adopt the formalism of \citet{Zimmermann2014} here, which is based on \citet{Matthewson2001}:

\ea\label{ex:landman:}
$⟦{all}⟧ = [{\lambda}y{_{e}} . [{\lambda}g_{<e,t>} . {\forall}x [ x {\leq}{ y} \to g(x)]]]$
\z

Because \textit{all} quantifies over subparts (\textit{x} ${\leq}$ \textit{y}) of the individual denoted by DP, distributive and collective interpretations are both possible. In the case that the subparts are atomic, a distributive interpretation results, and in the case that there is only one subpart (i.e., \textit{x} = \textit{y}), a collective interpretation results.

In the next subsection, I look at whether Matthewson's generalizations hold for the two Logoori universal quantifiers. As \citet{Zimmermann2014} points out, African languages are under-represented in Matthewson's survey, representing just four of the thirty-seven languages: Igbo (Igoboid), Koromfe (Gur), Fongbe (Kwa), and Xhosa (Bantu). \citet{Zimmermann2014} additionally supports Matthewson's generalizations with data from the West African languages Hausa \citep{Zimmermann2013} and Wolof.

\subsection{Do Matthewson's generalizations hold for the Logoori universal quantifiers?}
\subsubsection{\textit{Vuri} as a D-quantifier}

The D-quantifier analysis can naturally be extended to \textit{vuri}. By this account, \textit{vuri} would have the syntax in \REF{ex:landman:18} and the semantics in \REF{ex:landman:19}.

\ea\label{ex:landman:18}
$[_{DP} [_{D} {\text{\textit{vuri}}}] [_{NP} {\text{\textit{muundu}}}]]$
\z

\ea\label{ex:landman:19}
$⟦\text{\textit{vuri}}⟧ = [{\lambda}{f}_{<e,t>} . [{\lambda}g_{<e,t>} . {\forall}x [ {f(x)} \to g(x)]]]$
\z

This analysis fares well with the properties observed for \textit{vuri} above (summarized in \tabref{tab:Landman:2}). That \textit{vuri} is distributive is accounted for, again stipulating that quantification is over atomic individuals in \REF{ex:landman:19}. That \textit{vuri} most naturally combines with singular count nouns would be expected, assuming that singular count nouns denote properties of atomic individuals. In the case that \textit{vuri} combines with plurals or mass nouns (which, under standard assumptions, do not denote atomic individuals, see e.g. \citealt{Link1983}), semantic coercion would yield individuated readings. That \textit{vuri} is necessarily pre-nominal is expected, assuming that DP is head-initial, as phrasal categories in Logoori generally are. That \textit{vuri} may not co-occur with a pronoun would be accounted for if pronouns occupy D, i.e., \textit{vuri} and pronouns occupy the same position. Independent evidence that Logoori pronouns do occupy D is provided by examples like \REF{ex:landman:20}, in which pronouns may co-occur with an overt head noun (see \citealt{Postal1966}, among others, for relevant arguments that such co-occurring pronouns are in D).

\ea\label{ex:landman:20}
\gll kunyi  vaana    ku-syeev-i\\
     we  2child    1\textsc{pl}.\textsc{sa}-dance-\textsc{pst}  \\
\glt ‘We children danced.’
\z

  This leaves just two properties of \textit{vuri} unaccounted for: (a) that \textit{vuri} may not occur on its own; and (b) that it agrees with the head noun. However, neither of these properties provides evidence against the D-quantifier analysis, either; they are consistent with it, though unaccounted for. Thus, on the whole the D-quantifier account of \textit{vuri} is a good fit.

\subsubsection{-\textit{Oosi} as a Q-quantifier}

The status of -\textit{oosi} is most interesting here given Matthewson's generalizations, as it is less clear that it behaves like a Q-quantifier. By the Q-quantifier account, -\textit{oosi}, like English \textit{all}, would have the syntax in \REF{ex:landman:21} and the semantics in \REF{ex:landman:22}.

\ea\label{ex:landman:21}
$[_{QP} [_{ DP} \text{\textit{vaandu}}] [_{Q} \text{\textit{voosi}}]]$
\z


\ea\label{ex:landman:22}
$⟦{\text{-\textit{oosi}}}⟧ = [{\lambda}y{_{e}} . [{\lambda}g_{<e,t>}{}  . {\forall}x [ x {\leq}{ y} \to g(x)]]]$
\z


For the most part, a Q-quantifier account is consistent with the properties summarized for -\textit{oosi} in \tabref{tab:Landman:2}. That -\textit{oosi} allows for distributive or collective interpretations, and combines with singular, plural, or mass nouns would be accounted for, given its lexical denotation in \REF{ex:landman:22}. That -\textit{oosi} may co-occur with a pronoun would be accounted for if pronouns occupy D, since the two would appear in distinct positions. That -\textit{oosi} may appear alone, without the head noun, is also expected, if the head noun can be null. Finally, that -\textit{oosi} agrees in noun class with the head noun would follow if Q agrees. What would not be expected on this account is that QP would be head-final, since Logoori is otherwise head-initial; this is reason to consider an alternative, and arguably simpler, account. Such an account is detailed in the next section.

\subsubsection{An alternative: -\textit{oosi} as a DP-internal modifier}

A clear alternative to analyzing -\textit{oosi} as a Q-quantifier is to analyze it instead as a DP-internal adnominal modifier, given the range of properties that -\textit{oosi} shares with all other adnominal modifiers. As a DP-internal modifier, -\textit{oosi} would have the syntax in \REF{ex:landman:23}.\footnote{\\
 This, in fact, is what \citet{ZerbianKrifka2008} suggest for Swahili -\textit{ote} ‘all, any’. Moreover, they report that “Bantu languages have few genuine quantifiers. Rather, these languages display a range of adnominal modification with quantitative interpretation.” (p. 401)} 

\ea\label{ex:landman:23}
$[_{DP} [_{ NP} {\text{\textit{vaandu}}}] [_{AP} {\text{\textit{voosi}}}]]$
\z

There are, I believe, several points in favor of a DP-internal syntactic analysis for Logoori -\textit{oosi}. First, -\textit{oosi} has the same syntactic distribution as all other adnominal expressions, as observed above. Other adnominal modifiers also are post-nominal, may co-occur with a pronoun, may appear on their own, agree in noun class with the head noun, and combine with singular count, plural count, or mass nouns. All of these properties are consistent with the structure in \REF{ex:landman:23}. 

A second point in favor of treating -\textit{oosi} as internal to the DP is illustrated by \REF{ex:landman:24}, which shows that \textit{vuri} and -\textit{oosi} may actually co-occur within the same nominal phrase. Here, -\textit{oosi} appears to be interpreted within the scope of \textit{vuri}, thus also suggesting that it is positioned within the DP (at least on this interpretation).

\ea\label{ex:landman:24}
\gll soom-i      vuri    ki-tabu  ch-oosi\\
     1\textsc{sg.sa}.read-\textsc{pst}  every    7-book  7-all \\
\glt ‘I read every whole book.’
\z

A third indication that -\textit{oosi} is positioned within DP is that it may precede DP-internal modifiers, such as adjectives or numerals. Although -\textit{oosi} can follow possessives, demonstratives, adjectives, and numerals \REF{ex:landman:25a}, it may also precede all of them \REF{ex:landman:25b}. This suggests that -\textit{oosi} is internal to DP (or at the very least can be). 

\ea\label{ex:landman:25} 
\ea\label{ex:landman:25a} 
\gll vaana  \textup{\{}vaange\textup{/}yavo\textup{/}va-nene/vya Chazima/va-vere\}  v-oosi  va-gon-aa\\
     2children  \{2my/2those/2-important/2of Chazima/2-two\}  2-all  2\textsc{sa}-sleep-\textsc{prs}\\
\glt `All \{my/those/important/of Chazima's/two\} children are sleeping.'  
\ex\label{ex:landman:25b}
\gll vaana  v-oosi\textup{   \{}vaange\textup{/}yavo\textup{/}va-nene/vya Chazima/va-vere\}  va-gon-aa\\
     2children  2-all   \{2my/2those/2-important/2of Chazima/2-two\}  2\textsc{sa}-sleep-\textsc{prs}\\
\glt     `All \{my/those/important/of Chazima's/two\} children are sleeping.'  \\
\z
\z

  Summarizing, given that -\textit{oosi} (a) behaves syntactically just like all other adnominal modifiers, and (b) may co-occur with \textit{vuri}, the simplest analysis of -\textit{oosi} would be to treat it as a DP-internal modifier \REF{ex:landman:23}.

Positioning -\textit{oosi} internal to the DP raises the question, however, of how best to analyze it semantically. As sister to NP, -\textit{oosi} is expected to combine with an NP predicate, type <\textit{e},\textit{ t}>. I see two possibilities for a compositional analysis here. The first is suggested by \citet{ZerbianKrifka2008} for Swahili -\textit{ote} ‘all, any’, which they propose “can be analyzed as constructing the sum individual of all the entities that fall under the noun it applies to (cf. \citealt{Link1983})” (p. 401). -\textit{Oosi} might also be analyzed as mapping predicates to sum individuals. Alternatively, Brisson’s (1998; 2003) account of English \textit{all} could be extended to Logoori -\textit{oosi}. Brisson argues that English \textit{all} is not a quantifier, but rather restricts the domain of a covert distributive operator on the VP. Though Brisson analyzes English \textit{all} syntactically as a DP adjunct, her semantics could be extended to Logoori -\textit{oosi}. 

\subsubsection{Summary of the analysis}

This subsection summarizes the analysis. \textit{Vuri} is a D-quantifier. This fits Matthewson's generalization for distributive quantifiers. It is also consistent with Zerbian \& Krifka's observation that in Bantu, “...marked formatives are used for the expression of the universal quantifier `every'” (p. 401), as \textit{vuri} is the only quantificational expression in Logoori that is pre-nominal and does not display noun class agreement. 

-\textit{Oosi}, in contrast, is a DP-internal modifier. Its semantics can be modeled either as mapping sets to sum individuals (as \citealt{ZerbianKrifka2008} suggest for Swahili -\textit{ote} ‘all’), or as a domain restrictor \citep{Brisson1998,Brisson2003}. 

The syntactic and semantic properties of -\textit{oosi} are interesting given Matthewson's cross-linguistic generalizations for universal quantifiers translated as ‘all’. Matthewson presents her generalizations as tendencies, and not absolutes, but it is interesting that -\textit{oosi} does not seem to fit the observed Q-quantifier pattern for ‘all’-type quantifiers. 

\section{Other patterns regarding \textit{vuri} and -\textit{oosi}}\label{sec:landman:4}

In this section I document several other patterns regarding \textit{vuri} and -\textit{oosi} in an effort to lay the groundwork for future research on quantification in Logoori. In \sectref{sec:landman:4.1} I look at how the two universal quantifiers interact scopally with other quantificational nominals (albeit preliminarily, since judgments for scope are difficult to obtain in fieldwork contexts). In \sectref{sec:landman:4.2} I look at how the universal quantifiers interact scopally with negation (again, preliminarily). In \sectref{sec:landman:4.3} I observe a range of additional interpretations available for -\textit{oosi}, beyond just ‘all’, which are not accounted for by the proposed analysis.

\subsection{Scope}\label{sec:landman:4.1} 

In this subsection I look at how \textit{vuri} and -\textit{oosi} interact scopally with other nominal quantifiers. The aim is to more comprehensively understand the semantic properties of each quantifier.

Both \textit{vuri} and -\textit{oosi} interact scopally with bare nouns. Consider \REF{ex:landman:26a}, for example, in which the bare noun \textit{ridisha} ‘window’ is subject and \textit{vuri murumu} ‘every room’ is object. \REF{ex:landman:26a} is scopally ambiguous. On the surface scope reading, \textit{ridisha} ‘window’ scopes over \textit{vuri murumu} ‘every room’. In this case the sentence is true just in case there is one particular window that is in every room (the pragmatically odd reading here). On the inverse scope reading, \textit{vuri murumu} 'every room' scopes above \textit{ridisha} 'window'. In this case the sentence is true just in case every room has a (potentially different) window. The same ambiguity is available for -\textit{oosi}, as \REF{ex:landman:26b} illustrates.

\ea\label{ex:landman:26} 
\ea\label{ex:landman:26a}
\gll ri-dirisha  ri-vey-e  vuri  mu-rumu  \\
     5-window  5-have-\textsc{prs}  every  3-room\\
\glt i. ‘There is one particular window that is in every room.’  \\
ii. ‘Every room has a potentially different window.’  
\ex\label{ex:landman:26b}
\gll ri-dirisha  ri-vey-e  mu  zi-rumu  zy-oosi\\
     5-window  5-have-\textsc{prs}  with  3-room  3-all\\
\glt i. ‘There is one particular window that is in all rooms.’ \\
ii. ‘All rooms have a potentially different window.’        
\z
\z

\textit{Vuri} and -\textit{oosi} also interact scopally with nomimals modified by -\textit{lla} ‘one, some’. For example, \REF{ex:landman:27a} is scopally ambiguous. On the surface scope reading, \textit{ridisha llara} ‘one window’ scopes over \textit{vuri murumu} ‘every room’. In this case, the sentence is true just in case there is one particular window that is in every room. On the inverse scope reading, \textit{vuri murumu} ‘every room’ scopes above \textit{ridisha llara} ‘one window’, so that the sentence is true just in case every room has one (potentially different) window. The same ambiguity is again available for -\textit{oosi}, as \REF{ex:landman:27b} illustrates.

\ea\label{ex:landman:27} 
\ea\label{ex:landman:27a}
\gll ri-dirisha     lla-ra  ri-vey-e  vuri  mu-rumu  \\
     5-window   5-one  5-have-\textsc{prs}  every  3-room\\
\glt i. ‘There is one particular window that is in every room\textit{.}’  \\
ii. ‘Every room has one potentially different window.’        
\ex\label{ex:landman:27b}
\gll ri-dirisha   lla-ra   ri-vey-e         mu     zi-rumu  zy-oosi\\
     5-window  5-one   5-have-\textsc{prs}    with   3-room  3-all\\
\glt i. ‘There is one particular window that is in all rooms.’ \\
ii. ‘All rooms have one potentially different window.’    \textsc{}  
\z
\z

Summarizing this subsection, both \textit{vuri} and -\textit{oosi} interact scopally with bare nominals and nominals modified by -\textit{lla} 'one, some' (i.e., existentially quantified nominals). In particular, both \textit{vuri} and -\textit{oosi} permit inverse scope interpretations with respect to these nominals.

\subsection{Negation}\label{sec:landman:4.2}

In this section I look at how \textit{vuri} and-\textit{oosi} interact with negation. Like many other Bantu languages (\citealt{ZerbianKrifka2008}), there is no counterpart to the negative English determiner \textit{no} in Logoori. There are, instead, different ways of expressing propositions such as ‘No one danced.’ One option often volunteered by my consultant is to use -\textit{oosi} in combination with the clausal negation \textit{mba} ‘\textsc{neg}’ and the morpheme \textit{ku}:\footnote{\\
 I have glossed \textit{ku} here as \textsc{ku} because I am unsure of its semantics; see, however, \citet{BowlerGluckman2015} for an account of the semantics of \textit{ku}.}

\ea\label{ex:landman:28}
\gll muundu   woosi  a-syeev-i    ku  mba\\
     1person\textsc{}    1all    3\textsc{sa}-dance-\textsc{pst}  \textsc{ku}  \textsc{neg} \\
\glt ‘No one danced.’
\z

  \textit{Vuri} and -\textit{oosi} behave differently in negated sentences. \textit{Vuri} may occur as subject of a negated sentence \REF{ex:landman:29a}, in which case it must scope above negation. It is judged ungrammatical as object \REF{ex:landman:29b}.

\ea\label{ex:landman:29} 
\ea\label{ex:landman:29a}
\gll vuri  muundu  a-nyar-a    ku  mba  \\
     every  1person  1\textsc{sa}-mess.up-\textsc{pst}  \textsc{ku}  \textsc{neg}\\
\glt ‘Every person did not mess up.’        
\ex\label{ex:landman:29b}
\gll \textup{*}ya-yaanz-a  muundu   vuri    mba \\
     1\textsc{sa}-like-\textsc{prs}  1person     every    \textsc{neg}\\
\z
\z

  In contrast, -\textit{oosi} is interpreted as an existential (negative polarity item) in negated sentences, whether subject \REF{ex:landman:30a} or object \REF{ex:landman:30b}.

\ea\label{ex:landman:30} 
\ea\label{ex:landman:30a}
\gll muundu  w-oosi    a-nyar-a    ku  mba  \\
     1person  2-all    1\textsc{sa}-mess.up-\textsc{pst}  \textsc{ku  neg}\\
\glt ‘No one messed up.’        
\ex\label{ex:landman:30b}
\gll ya-yaanz-a  vaandu   v-oosi    mba \\
     1\textsc{sa}-like-\textsc{prs}  2person     2-all    \textsc{neg}\\
\glt ‘He doesn't like anyone.’
\z
\z

\subsection{A range of interpretations for -\textit{oosi}}\label{sec:landman:4.3}

Though I have focused on the ‘all’ interpretation of -\textit{oosi}, there are a number of other interpretations available for -\textit{oosi} in Logoori. I review these briefly here.

  First, as observed earlier, -\textit{oosi} can mean ‘whole’ when it modifies a singular count noun:

\ea\label{ex:landman:}
\gll a-syoom-i  ki-tabu  ch-oosi\\
     1\textsc{sa}-read-\textsc{pst}    7-book \textsc{7-}all \\
\glt ‘She read the whole book.’
\z

  Second, as also observed above, -\textit{oosi} is interpreted as an existential (negative polarity item) in a negated sentence:  

\ea\label{ex:landman:} 
\gll muundu  w-oosi    a-nyar-a    ku  mba  \\
     1person  2-all    1\textsc{sa}-mess.up-\textsc{pst}  \textsc{ku}  \textsc{neg}\\
\glt ‘No one messed up.’        
\z

\ea\label{ex:landman:}
\gll ya-yaanz-a  vaandu   v-oosi    mba \\
     1\textsc{sa}-like-\textsc{prs}  2person     2-all    \textsc{neg}\\
\glt ‘He doesn't like anyone.’
\z

  Finally, -\textit{oosi} permits free choice interpretations in intensional or modal contexts \REF{ex:landman:33}.

\ea\label{ex:landman:33}
\gll muundu  w-oosi    a-nyar-a  ku-syeev-a\\
     1person  1-any    1\textsc{sa}-can-\textsc{prs}  \textsc{inf-}dance-\textsc{prs} \\
\glt ‘Anyone can dance.’
\z

It is possible that a single semantic analysis of -\textit{oosi} accounts for all of its possible interpretations; I leave this issue for future research. 

\section{Conclusion}\label{sec:landman:5}

In this paper, I have documented and analyzed universally quantified nominals in Logoori. Specifically, I have analyzed \textit{vuri} ‘every’ as a D-quantifier, and -\textit{oosi} ‘all’ as a DP-internal modifier. More broadly, the study has added to our knowledge of how quantification is expressed in Bantu, as well as how the Logoori patterns relate to previous cross-linguistic work on quantification, both on Bantu (\citealt{ZerbianKrifka2008}) and across languages more generally \citep{Matthewson2013}. The study has articulated the following questions for future research: (i) what exactly are, and what accounts for, scope interactions among Logoori quantifiers; and (ii) can a unified account of the range of interpretations available for -\textit{oosi} be achieved?

\section*{Acknowledgements}

I am very grateful to my Logoori consultant Isaac K. Tomas for sharing his language with me. Special thanks also to Michael Diercks for extensive input and advice. Many thanks also to the editors of this volume and three anonymous reviewers, whose suggestions have improved this paper greatly, to Mary Paster and Michael Marlo for useful comments, and to Malte Zimmermann, whose plenary talk at ACAL 45 provided a model for this work. This research was supported by the Department of Linguistics and Cognitive Science at Pomona College and by a NSF Collaborative Research Grant (Structure and Tone in Luyia: BCS-1355749).

\subsection*{Abbreviations}

\begin{tabularx}{.45\textwidth}{lX}
\textsc{comp} &  complementizer  \\
\textsc{neg} &  negation  
\end{tabularx}
\begin{tabularx}{.45\textwidth}{lX}
\textsc{pst} &  past \\
\textsc{sa}  & subject agreement\\
\end{tabularx}

\medskip
Numerals indicate Bantu noun class.

\printbibliography[heading=subbibliography,notkeyword=this]

\end{document}