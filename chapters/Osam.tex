\documentclass[output=paper]{langsci/langscibook} 
\title{Valency changing processes in {Akan} }
\author{%
E. Kweku Osam\affiliation{University of Ghana} 
}
% \chapterDOI{} %will be filled in at production




\abstract{}

\maketitle
\begin{document}

\begin{abstract}
Valency has been considered as both a semantic and syntactic notion. Semantically it is used to refer to the participants in an event; as a syntactic notion it is used to indicate the number of arguments in a construction. In Akan, a Kwa language spoken in Ghana, we can identify various transitivity classes of verbs: verbs that are strictly intransitive; those that are strictly transitive; and verbs that are used ditransitively. Apart from these, there are verbs that can be used both intransitively and transitively. Critical to the study of the notion of valency in Akan is the fact that there is clear evidence for grammatical relations in the language. As is the case in many languages, Akan possesses morphosyntactic means through which the valency of verbs can be adjusted. The application of these morphosyntactic processes reduces or increases the valency of verbs. This paper examines these processes in Akan. The critical valency-reducing processes in Akan are reflexivization, reciprocals, anticausative/inchoative constructions, impersonal constructions, object omission constructions, and unspecified object constructions. Valency-increasing processes include causativization and agentivization through serialization.
\end{abstract}

Keywords: \emph{Akan, valency, transitivity, grammatical relations, causative, arguments}

\chapter{1 Introduction}
\begin{styleBodyTextIndent}
The linguistics literature is replete with studies on valency—those that are theoretical, cross-linguistic, and others that focus on the study of valency in specific languages. The interest in understanding the notion of valency has resulted in a number of volumes dedicated to the subject. Notable among these are Comrie \& \citet{Polinsky1993}, Dixon \& \citet{Aikhenvald2000a}, Malchukov \& Comrie (in press). The papers in Malchukov \& Comrie (in press) came out of the Leipzig Valency Classes Project and the Conference on Valency Classes in the World's Languages held in 2011. Worthy of mention is also \citet{NicholsEtAl2004} that put forward the idea of the basic valency orientation of languages. As in the case of the Leipzig Project, a number of scholars adopting the methodology of Nichols and associates, have examined the basic valency orientation of various languages (see, for example, Narrog 2009, Plank \& Lahiri 2009, van Gelderen 2011, and Luraghi 2012).  
\end{styleBodyTextIndent}

The goal of this paper is to examine the valency of Akan verbs and explore the morphosyntactic processes that apply to modify the valency of verbs. It is expected that the paper will help expand further our understanding of the behavior of verbs in Akan, building on what has been reported in \citet{Osam2008a}. 

This paper is based on the Akan Verbs Database project which was implemented in the Department of Linguistics, University of Ghana, between 2009 and 2011, and for which I was the Principal Investigator. The project was funded by the University of Ghana Research Fund. By the end of the project, a database of over 3,500 verb stems and other verb forms had been created.

The data come from various sources, including published narratives, translated works and, in some cases, constructed examples based on my knowledge as a native speaker.

The paper is structured as follows: §2 provides an overview of the verbal morphology, focusing on the tense-aspects distinctions of the language and clause structure/grammatical relations. This section also covers some aspects of serial verb constructions in the language since they have a bearing on valency adjustment in the language. §3 deals with Akan verb classes. In §4 the morphosyntactic processes that reduce the valency of verbs are discussed; §5 focuses on the processes that increase valency. §6 is the summary and conclusion.

The label Akan is used to refer to a group of closely related dialects spoken in Ghana, and partially in the south eastern parts of Cote d’Ivoire. It belongs to the Kwa sub-family of Niger-Congo. The name also refers to the people who speak the language. The dialects of Akan include: Agona, Ahafo, Akuapem, Akwamu, Akyem, Asante, Assin, Bono, Denkyira, Fante, Kwahu, and Wassa. Generally, except Fante, all the other dialects tend to be classified as Twi in terms of Akan dialectology. Current speaker population is estimated at nearly ten million.

\chapter[2  Some morphosyntactic features of Akan]{2  Some morphosyntactic features of Akan}
\begin{styleBodyTextIndent}
In this section I provide an overview of tense/aspect and clause structure in the language. I should point out that Akan is a two tone language and one of the outstanding phonological features of the language is the presence of Tongue Root Harmony in all the dialects and rounding harmony in the Fante dialect. The discussion throughout the paper will draw examples from the Fante (Fa) dialect and in some cases from the Asante (As) and Akuapem (Ak) dialects as well. Examples drawn from Fante will generally not be identified as such; those from Asante and Akuapem will be indicated as As and Ak, respectively, in parentheses generally at the end of a translation.
\end{styleBodyTextIndent}

\section{2.1  Verbal affixes}

In Akan verbal constructions are \textstyleNummerierungszeichen{generally} formed with the verb and its person, tense, aspect, mood, and polarity affixes. There are also verbal prefixes to mark motion towards or away from a deictic centre.

%\setcounter{itemize}{0}
\begin{itemize}
\item \ea\label{ex:}
\langinfo{}{}{Subject Marking}\\
\end{itemize}
     \emph{        }\emph{\textup{Subject }}\emph{\textup{Prefixes}}\\
     \textsc{1sg  }\emph{\textup{      }}\emph{me-}\\
     \textsc{2sg  }\emph{      wo-, i- }\emph{\textup{(Fa)}}\\
     \textsc{3sg  }\emph{      ɔ{}-}\\
     \textsc{3sg}\emph{\textup{ (inanimate)}}\emph{    ɛ{}- }\emph{\textup{(in Ak, As and some Fa subdialects)}}\\
     \textsc{1pl  }\emph{      yɛ-}\\
     \textsc{2pl  }\emph{      mo-/wɔ- }\emph{\textup{(Fa)}}\emph{  }\\
     \textsc{3pl  }\emph{      wɔ-/yɛ-}\\
The pre-verbal affixes in Akan include the tense/aspect markers. As I have argued elsewhere (Osam 1994a, 2008b), Akan can be said to have a Future Tense and the following aspectual forms: Completive (\textsc{compl}), Perfect (\textsc{perf}), Progressive (\textsc{prog}), Habitual (\textsc{hab}), Continuative (\textsc{cont}), and Consecutive (\textsc{consec}).

The Future tense is coded by the prefix \emph{bɛ-}. The realization of the vowel is determined by vowel harmony, especially in the Fante dialect (see Dolphyne 1988).

\begin{itemize}
\item \end{itemize}
\gll \emph{a.}  \'{I}y\'{i}  bɛ-m\`{a}    k\'{o}\'{r}-yɛ    \`{a}-b\`{a}    h\'{o}\'{m}    \'{n}t\'{a}m\'{u}.\\
       this  \textsc{fut}{}-make  one-be    \textsc{cons}{}-come  \textsc{2pl.poss}  middle\\
\glt ‘This will bring unity among you.’ \citep[79]{Krampah1970}
\z

\gll \emph{b.  }D\`{a}\'{a}ky\'{e} y\`{e}-b\'{e}-ny\'{a}    b\`{i}  \'{a}-k\'{a}    \'{a}-ky\`{e}rɛ    \`{a}-f\'{o}f\'{o}r\'{o}.  \\
       future   \textsc{1pl.subj}{}-\textsc{fut}{}-get  some  \textsc{cons}{}-say\textsc{   cons}{}-show  \textsc{pl}{}-new\\
\glt    ‘In future we will have something to tell others.’ (Adi 1973: 34; Ak)
\z

What I consider the Completive is what in the general literature on Akan is referred to as the Past tense. However, I have shown in various places (Osam 1994, 2004, 2008) that this form is more of an aspect than tense. The Completive is a suffix in the affirmative; but a prefix in the negative.  

\begin{itemize}
\item \end{itemize}
\gll \textup{a.}  N\'{e}  \`{n}-s\`{e}\`{w}-n\'{o}\'{m}    b\`{e}\`{e}n\'{u}   ny\`{i}n\'{a}  yɛ-ɛ    \`{e}dz\`{i}b\`{a}\'{n}.\\
     \textsc{\textup{  }}\textsc{\textup{3sg}}   \textsc{poss.pl}{}-in law-\textsc{col}  two   all  make-\textsc{compl}  food\\
\glt ‘Each of his two in-laws cooked.’ \citep[57]{Krampah1970}
\z

\gll \textup{b.  }N\'{e}  \`{n}-s\`{e}\`{w}-n\'{o}\'{m}    b\`{e}\`{e}n\'{u}   ny\`{i}n\'{a}  \`{a}-\`{n}-yɛ      \`{e}dz\`{i}b\`{a}\'{n}.\\
     \textsc{  3sg}   \textsc{poss.pl}{}-in law-\textsc{col}  two  all  \textsc{compl-neg}{}-make  food\\
\glt ‘His two in-laws did not cook.’
\z

The Perfect aspect is realized by the prefix \emph{a-}, and generally agrees with the vowel of the verb root in ATR harmony.

\begin{itemize}
\item \end{itemize}
\gll \textup{a.}  B\`{a}ny\'{i}\'{n} n\'{o}  \'{a}-tɔ     \`{a}s\`{a}\'{a}s\'{e}.\\
       man  \textsc{def}  \textsc{perf}{}-buy  land\\
\glt ‘The man has bought a piece of land.’
\z

\gll \textup{b.  }\`{M}-b\`{o}fr\'{a}  n\'{o}  \'{a}-b\`{a}    f\'{i}\'{e}.\\
     \textsc{  pl}{}-child  \textsc{def}  \textsc{perf}{}-come  home\\
\glt ‘The children have come home.’
\z

The Progressive is a prefix, \emph{re-}.

\begin{itemize}
\item \end{itemize}
\gll \textup{a.  }M\`{a}\`{a}m\'{e}   n\'{o}  r\`{e}-hw\`{e}-hwɛ     n\'{e}    b\'{a}   n\'{o}.\\
       woman    \textsc{def}  \textsc{prog-redu}{}-look  \textsc{3sg.poss}  child  \textsc{def}\\
\glt ‘The woman is looking for her child.’
\z

\gll b.  \`{A}b\`{o}fr\'{a}  n\'{o}  r\`{u}-t\`{u}-t\'{u}    \`{n}-w\'{u}r\'{a}    n\'{o}.\\
       child  \textsc{def}  \textsc{prog-redu} -uproot  \textsc{pl}{}-weed  \textsc{def}\\
\glt ‘The child is removing the weeds.’
\z

Akan also has a Habitual aspect which is realized by tone (see Dolphyne 1988 for more discussion on this).

\begin{itemize}
\item \end{itemize}
\gll a.  \`{O}k\`{u}\`{a}f\'{o}  n\'{o}   kɔ    h\`{a}b\'{a}\'{n}  m\`{u}  \`{a}nɔp\'{a}    b\'{i}\'{a}r\'{a}.\\
       farmer  \textsc{def}  go;\textsc{hab}    farm  in  morning  every\\
\glt ‘The farmer goes to the farm every morning.’
\z

\gll b.  Akosua    tɔ\`{n}     \`{n}dzɛ\'{m}b\'{a}  wɔ  g\'{u}\'{a}-m\'{u}.\\
       Akosua  sell;\textsc{hab}  things    in  market-in\\
\glt ‘Akosua sells in the market.’
\z

Two of the aspects, the Continuative and Consecutive, are treated as derived aspects. The Continuative is used for stative verbs where the Progressive is used for dynamic verbs. The Consecutive aspect applies to non-initial verbs in a serial construction in which the initial verb is in either the Progressive aspect or the Future tense. Generally, the Consecutive is realized as a low tone \emph{à-}.

\begin{itemize}
\item \end{itemize}
\gll a.  \`{A}b\`{o}fr\'{a}  n\'{o}  ts\`{e}    d\`{u}\'{a}  n\'{o}  \'{a}s\'{e}.\\
       child  \textsc{def}  sit/\textsc{cont}  tree  \textsc{def}  under\\
\glt ‘The child is sitting under the tree.’
\z

\gll b.  Kwesi  gy\`{i}n\`{a}    f\'{i}\'{e}  n\'{o}  \'{e}ny\'{i}\'{m}.\\
       Kwesi  stand/\textsc{cont}  house  \textsc{def}  front\\
\glt ‘Kwesi is standing in front of the house.’
\z

\begin{itemize}
\item \end{itemize}
\gll a.  B\`{a}ny\'{i}\'{n}  n\'{o}  bɛ-s\'{a}\'{w}   \`{a}-ky\`{e}rɛ    dɔ\'{m}  n\'{o}.\\
       man  \textsc{def}  \textsc{fut}{}-dance  \textsc{cons}{}-show  crowd  \textsc{def}\\
\glt   ‘The man will dance for the crowd.’
\z

\gll   b.  Kofi  r\`{e}-n\'{a}\'{n}ts\`{e}\`{w}  \`{a}-kɔ    sk\'{u}\`{u}l.\\
       Kofi  \textsc{prog}{}-walk  \textsc{cons}{}-go  school\\
\glt   ‘Kofi will walk to school.’
\z

There are verbal constructions in Akan which sometimes have two prefixes, \emph{bɛ-} and \emph{kɔ-}. These prefixes arise from the verbs for ‘come’ and ‘go’, respectively. I have referred to these as Motional prefixes \citep{Osam2002}. They reflect the movement towards or away from a deictic centre where an event takes place. Movement towards the deictic centre is marked by the ‘come’ verb; and away from the centre is marked by the ‘go’ verb. 

\begin{itemize}
\item \end{itemize}
\gll a.  Ama  bɔ-sɔ-ɔ      gy\'{a}  wɔ  d\'{a}\'{n}    n\'{o}  \'{e}ky\`{i}\`{r}.\\
       Ama  come-light-\textsc{compl}  fire  at  building  \textsc{def}  back\\
\glt   ‘Ama came and lit a fire behind the building.’
\z

\gll  b.  Esi  kɛ-f\'{a}-\`{a}      \`{e}k\`{u}t\'{u}    n\'{o}  b\'{a}-\`{a}    f\'{i}\'{e}.\\
       Esi  go-take-\textsc{compl}  orange    \textsc{def}  come-\textsc{compl}  home\\
\glt   ‘Esi went and brought the things home.’
\z

Even though it is too early to say that Akan has a prospective aspect, the language has the means to express prospective meaning. This is done through a combination of the Progressive and the ‘come’ Motional prefix.

\begin{itemize}
\item \end{itemize}
\gll a.  Ɔm\`{a}\`{n}p\`{a}ny\'{i}\'{n}  r\`{e}-bɔ-sɔ\'{r}.\\
       president  \textsc{prog}{}-come-stand\\
\glt   ‘The president is about to rise.’
\z

\gll  b.  Hyɛ\'{n}    n\'{o}  r\`{e}-b\'{e}-gy\'{i}n\'{a}.\\
       vehicle    \textsc{def}  \textsc{prog}{}-come-stop\\
\glt   ‘The vehicle is about to stop.’
\z

The language makes a two way distinction in terms of mood — the indicative and the Imperative. The Imperative has two manifestations; what we may call Imperative proper and the Optative. Whereas the Imperative has no segmental representation, the Optative is realized through the use of a homorganic nasal with a high tone, \emph{\'{n}-}. 

\begin{itemize}
\item \end{itemize}
\gll a.  Gy\`{a}\`{e}!\\
       stop/\textsc{imp}\\
\glt   ‘Stop it.’
\z

\gll  b.  Yɛ-\'{n}-kɔ.\\
       \textsc{1pl}{}-\textsc{opt}{}-go\\
\glt   ‘Let’s go.’
\z

The language uses a homorganic nasal prefix to express negation.

\begin{itemize}
\item \end{itemize}
\gll a.  M\`{i}-\`{n}-ny\'{i}\`{m}     \`{e}dw\'{u}m\'{a}  â   ɔ{}-yɛ.\\
       \textsc{1sg.subj}{}-\textsc{neg}{}-know  work    \textsc{rel}  \textsc{3sg.subj}{}-do\\
\glt   ‘I don’t know what work she does.’
\z

\gll  b.  M\`{a}\`{a}m\'{e}    n\'{o}  \`{a}-\`{n}-k\'{a}       \`{a}sɛ\'{m}  n\'{o}  \`{a}-\`{n}-ky\`{e}rɛ\\
       woman    \textsc{def}  \textsc{compl}{}-\textsc{neg}{}-say  matter  \textsc{def  compl}{}-\textsc{neg}{}-show\\
\gll   n\'{e}    \'{m}-b\'{a}    n\'{o}.\\
       \textsc{3sg.poss}  \textsc{pl}{}-child  \textsc{def}\\
\glt   ‘The woman didn’t tell her children about the case.’  
\z

\section{2.2  Verbal reduplication}

Many verbs in Akan are subject to the morphological process of reduplication. Generally, verbal reduplication is required where the event is iterative and/or where either the Subject or Direct Object NPs or both are plural.

\begin{itemize}
\item \end{itemize}
\gll a.  \'{A}f\'{e}\'{i}  \`{n}-n\`{a}\`{m}f\`{o}   y\'{i}   ky\`{i}\`{n}-ky\'{i}\'{n}-\'{n}     \`{a}-dɔf\'{o}     k\`{a}kr\'{a} \\
       now  \textsc{pl}{}-friend  these  \textsc{redu}{}-roam-\textsc{compl}  \textsc{pl}{}-lover  few  \\
\gll   â  w\'{o}-n\'{i}\'{m}     wɔ\'{n}     n\'{o}   s\`{o} \\
       \textsc{rel}  \textsc{3pl.subj}{}-know  \textsc{3pl.obj}  \textsc{dcm}  on\\
\gll   kr\'{a}-kr\`{a}-\`{a}       wɔ\`{n}.\\
       \textsc{redu}{}-bid farewell-\textsc{compl}  \textsc{3pl.obj}\\
\glt   ‘Now these two friends visited some of their friends to bid them farewell.’ (Adi 1973: 41; Ak)
\z

\gll  b.  N\'{e}    n\`{a}    n\`{a}  n-\'{e}gy\`{a}      \'{e}-w\'{u}-w\'{u}.\\
       \textsc{3sg.poss}  mother    and  \textsc{3sg.poss} -father  \textsc{perf-redu}{}-die\\
\glt   ‘Both his mother and father are dead.’ \citep[24]{Martin1936}
\z

\section{2.3  Akan clause structure}

The syntax of Akan distinguishes grammatical relations, as has been discussed in various studies (Osam 1994, 1996, 1997, 2000, \& 2004). The language has Subject and Direct Object with Nominative Accusative marking that is realized through word order. Word Order in the language is strictly SVO, with very little variation.

\subsection{2.3.1  Transitive constructions}

The prototypical Akan transitive clause has A (as Subject) and O (as Direct Object) as core arguments. The A argument precedes the O; and each can be realized as full NP or as a pronominal element.

%\setcounter{itemize}{0}
\begin{itemize}
\item \end{itemize}
\gll a.  Akosua    ky\`{e}-\`{e}    \`{a}k\'{o}kɔ    n\'{o}.\\
       Akosua  catch-\textsc{compl}  chicken  \textsc{def}\\
\glt   ‘Akosua caught the chicken.’ (As)
\z

\gll  b.  \`{O}-k\`{u}\`{a}-f\'{o}    n\'{o}  \'{a}-d\'{a}\'{a}-d\`{a}\`{a}    ɔ{}-bɔ-f\'{o}    n\'{o}.\\
       \textsc{sg}{}-farm-\textsc{idm}    \textsc{def}  \textsc{perf-redu}{}-deceive  \textsc{sg}{}-hit-\textsc{idm}  \textsc{def}\\
\glt   ‘The farmer has deceived the hunter.’
\z

If the O argument is inanimate, it is not represented by a pronominal form unless some other clausal element comes after it.

\begin{itemize}
\item \end{itemize}
\gll a.  \`{O}-w\'{i}\`{a}-\`{a}    n\'{a}\'{m}  n\'{o}.\\
       \textsc{3sg.subj}{}-steal-\textsc{compl}  fish  \textsc{def}\\
\glt   ‘He stole the fish.’
\z

\gll  b.  \`{O}-w\'{i}\`{a}-\`{e}      Ø.\\
       \textsc{3sg.subj}{}-steal-\textsc{compl}\\
\glt   ‘He stole it.’
\z

\gll  c.  \`{O}-w\'{i}\`{a}-\`{a}      n\`{o}    \'{n}d\'{e}d\`{a}.\\
       \textsc{3sg.subj}{}-steal-\textsc{ compl}  \textsc{3sg.obj}  yesterday\\
\glt   ‘He stole it yesterday.’
\z

A transitive clause could also have optional Oblique elements expressed by locational (both spatial and temporal) phrases. It could also be a postpositional phrase. Generally, obliques would occur in clause-initial or clause-final positions. 

\begin{itemize}
\item \end{itemize}
\gll a.  M\`{a}\`{a}m\'{e}    n\'{o}  s\'{i}-\`{i}    d\'{a}\'{n}  \'{m}f\'{e}d\`{a}.\\
       woman    \textsc{def}  build-\textsc{compl}  house  last year\\
\glt   ‘The woman built a house last year.’
\z

\gll  b.  \'{M}f\'{e}d\`{a}    m\`{a}\`{a}m\'{e}    n\'{o}  s\'{i}\`{i}    d\'{a}\'{n}.\\
       last.year  woman    \textsc{def}  build-\textsc{compl}  house\\
\glt   ‘Last year the woman built a house.’
\z

\subsection{2.2.2  Intransitive constructions}

The single argument of an intransitive clause, the S argument, always precedes the predicate. 

%\setcounter{itemize}{0}
\begin{itemize}
\item \end{itemize}
\gll a.  \`{E}dz\`{i}b\`{a}\'{n}  n\'{o}  \'{a}-b\`{e}\`{n}.  \\
       food    \textsc{def}  \textsc{perf}{}-be cooked\\
\glt   ‘The food is cooked.’  
\z

\gll  b.  Ɔ{}-sɔ-f\'{o}      n\'{o}  w\'{u}-\`{u}    \'{n}d\'{e}d\`{a}.\\
       \textsc{sg}{}-pray-\textsc{idm}    \textsc{def}  die-\textsc{compl}  yesterday\\
\glt   ‘The priest died yesterday.’
\z

\subsection{2.2.3  Ditransitive constructions}

Akan has ditransitive constructions in which there are three core arguments, \textsc{agent, benefactive} and \textsc{theme}. In ditransitive constructions, the NP in the immediate post-verbal position is grammatically the Direct Object and semantically the \textsc{benefactive}. The entity that is transferred, the \textsc{theme}, occurs after the \textsc{benefactive} NP. I have referred to this as the \emph{Asymmetrical Object} \citep{Osam2000}.

%\setcounter{itemize}{0}
\begin{itemize}
\item \end{itemize}
\gll a.  M\`{a}\`{a}m\'{e}    n\'{o}  m\'{a}-\`{a}    \`{m}-b\`{o}fr\'{a}  n\'{o}  \`{e}dz\`{i}b\'{a}\'{n}.\\
       woman    DEF  give-COMPL  PL-child  DEF  food\\
\glt   ‘The woman gave the children food.’
\z

\gll  b.  P\`{a}p\'{a}  n\'{o}  kyɛ-ɛ    hɔ\`{n}    s\`{i}k\'{a}.\\
       man  DEF  gift-COMPL  3PL OBJ  money\\
\glt   ‘The man gave them money.’  
\z

\subsection{2.2.4  Serial verb constructions}

One feature of Akan syntax crucial to valency discussions is serial verb constructions (\textsc{SVC}). Akan serialization has been studied extensively (see, for example, Lord 1973, Schachter 1974, Essilfie 1977, Forson 1990, Osam 1994a, 1994b, 1997, 2004, 2014, Agyeman 2002, Hellan et al. 2003, Kambon 2012). Without going into the details of Akan SVCs, it is important to identify some salient features.

\begin{styleListei}
\emph{Subject marking:} the subject may be a NP or a pronominal form that occurs on only the initial verb. 
\end{styleListei}

%\setcounter{itemize}{0}
\begin{itemize}
\item \end{itemize}
\gll a.  Yaakwa  n-\'{a}b\'{a}k\'{a}\'{n}     gy\'{i}n\`{a}-\`{e}   yɛ-ɛ     ky\`{e}\'{a}m\'{e}\\
       Yaakwa  \textsc{3sg.poss}{}-first.born  stand-\textsc{compl}  be-\textsc{compl}   spokesperson\\
\gll   bɔ-ɔ     f\'{i}\'{e}   \`{a}m\`{a}\`{n}dzɛɛ  ky\'{e}rɛ-ɛ   \`{a}-hɔh\'{o}    n\'{o}.\\
       tell-\textsc{compl}  home  news    show-\textsc{compl}  \textsc{pl}{}-visitor  \textsc{def}\\
\glt   ‘Yaakwa’s eldest son stood as the spokesperson and briefed the visitors.’ \citep[83]{Krampah1970}
\z

\gll  b.  Wɔ-tw\'{e}-\`{e}    n\'{o}  gy\`{i}n\'{a}-\`{a}  \`{n}kyɛ\'{n}.\\
       \textsc{3pl.subj}{}-pull\textsc{{}-compl}  \textsc{def}  stand-\textsc{compl}  aside\\
\glt   ‘They pulled him aside.’ (Adi 1973: 31; Ak)
\z

Generally, there is uniformity in tense/aspect coding, as shown in the examples above. However, mixed tense/aspect is possible in some SVCs. 

\begin{styleListei}
\emph{Negation Marking:} across all dialects of Akan, each verb in the series takes the negation prefix when the sentence is negated, as in (20). 
\end{styleListei}

\begin{itemize}
\item \end{itemize}
\gll M\`{a}\`{a}m\'{e}  n\'{o}  \`{a}-\`{n}-tɔ      \`{e}dz\`{i}b\`{a}\'{n}  \`{a}-\`{m}-m\'{a}  \\
     woman  \textsc{def}  \textsc{compl}{}-\textsc{neg}{}-buy  food    \textsc{compl}{}-\textsc{neg}{}-give\\
\gll n\'{e}    \'{m}-b\'{a}    n\'{o}.\\
     \textsc{3.sg.poss}  \textsc{pl}{}-child  \textsc{def}\\
\glt ‘The woman did not buy food for her children.’
\z

In some serial constructions, the initial verb is \emph{de/dze}. This is a form with reduced verbal properties. The \emph{de }is used in the Twi dialects and \emph{dze} is used in Fante. 

\begin{itemize}
\item \end{itemize}
\gll a.  Wɔ-d\`{e}    \`{a}sɛ\'{m}  n\'{o}  kɔ-ɔ    \`{a}h\`{e}mf\'{i}\'{e}.\\
       \textsc{3pl.subj}{}-take  matter  \textsc{def}  go-\textsc{compl}  palace\\
\glt   ‘They took the case to the palace.’ (Ak)
\z

\gll  b.  Ɔ{}-dz\`{e}    n-\'{a}h\`{o}ɔdz\'{e}\'{n}    ny\`{i}n\'{a}  yɛ-ɛ     \`{e}dw\'{u}m\'{a} \\
       \textsc{3sg.subj}{}-take  \textsc{3sg.poss}{}-strength  all  do-\textsc{compl}  work\\
\gll   ny\'{a}-\`{a}    s\`{i}k\'{a}.\\
       get-\textsc{compl}  money\\
\glt   ‘She worked very hard and made money.’
\z

The de serialization is also relevant in the expression of agentive arguments in the case of verbs of spatial location. This will be further discussed in §5.2.

\chapter[3  Akan verb classes]{3  Akan verb classes}

Based on their argument structure, various transitivity classes of verbs can be identified in Akan. Some are strictly intransitive (§3.1), some are strictly transitive (§3.2), some are ditransitive (§3.3), and some have varying expressions of arguments. Verbs of the last type are addressed in §4 and §5.

\section{3.1  Strictly intransitive/monovalent}

These verbs occur with a single argument; the S argument only.

%\setcounter{itemize}{0}
\begin{itemize}
\item \end{itemize}
\gll \textup{  }b\'{e}\'{n}   \textup{‘be cooked’}     b\`{e}\`{r}   \textup{‘be ripe, be fair in complexion’}   \\
\gll dw\'{o}   \textup{‘cool’ }      f\'{e}   \textup{‘become soft/smooth’}  \\
\gll h\'{o}\'{n}   \textup{‘swell’}     hw\'{e}\'{m}   \textup{‘blow one’s nose’} \\
\gll hw\`{e}\`{n}ts\`{i}   \textup{‘sneeze’}     p\'{e}\'{r}   \textup{‘struggle, roll around during sleep’}\\
\gll prɔ   \textup{‘rot’}       h\'{u}\'{r}   \textup{‘boil’} \\
\begin{itemize}
\item \end{itemize}
\gll a.  \`{E}dz\`{i}b\`{a}\'{n}  n\'{o}  \'{a}-b\`{e}\`{n}.\\
       food    \textsc{def}  \textsc{perf}{}-cook\\
\glt   ‘The food is cooked.’
\z

\gll  b.  \`{A}\`{n}k\`{a}\'{a}    n\'{o}  \'{a}-prɔ.\\
       oranges  \textsc{def}  \textsc{perf}{}-rot\\
\glt   ‘The oranges are rotten.’
\z

\section{3.2  Strictly transitive/bivalent}

There are verbs that require two arguments, A and O.

\begin{itemize}
\item \end{itemize}
\gll b\`{o}\`{r}   \textup{‘beat’}    d\`{a}\`{a}d\`{a}\`{a}   \textup{‘deceive’}  hy\`{i}r\`{a}   \textup{‘bless’}    \\
\gll k\'{a}     \textup{‘bite’}    k\'{u}     \textup{‘kill’}    ky\'{e}   \textup{‘catch’}\\
\gll nyɛ\'{n}   \textup{‘rear’}    p\`{a}\`{a}     \textup{‘curse’}    s\`{a}n\`{e}   \textup{‘infect’}    \\
\gll s\`{i}\`{e}     \textup{‘bury’}    t\'{a}\'{n}     \textup{‘hate’}\\
\begin{itemize}
\item \end{itemize}
\gll a.  Ɔ{}-bɔ-f\'{o}ɔ  n\'{o}  k\'{u}-\`{u}     ɔs\'{o}n\'{o}.\\
       \textsc{sg}{}-hit-\textsc{idm}  \textsc{def}  kill-\textsc{compl}  elephant\\
\glt   ‘The hunter killed an elephant.’  
\z

\gll  b.  Abam  s\'{i}\`{e}-\`{e}     n\'{e}    m\`{a}\`{a}m\'{e}.\\
       Abam  bury-\textsc{compl}  \textsc{3sg.poss}  mother\\
\glt   ‘Abam buried his mother.’
\z

\section{3.3  Ditransitive}

There are verbs that are ditransitive or trivalent, requiring three core arguments.

\begin{itemize}
\item \end{itemize}
\gll kyɛ   \textup{‘to gift, give as a gift’  }  gy\'{e}   \textup{‘charge’}\\
\gll m\'{a}     \textup{‘give’}        ky\`{e}rɛ   \textup{‘teach, show’}\\
\begin{itemize}
\item \end{itemize}
\gll a.  \'{N}ts\'{i}  ɔ{}-m\'{a}-\`{a}      n\`{o}    \`{a}-hɛ\'{n}    \`{e}b\`{i}\'{a}s\'{a}.  \\
       so  \textsc{3sg.subj}{}-give-\textsc{compl}  \textsc{3sg.obj}  \textsc{pl}{}-boat    three\\
\glt ‘So she gave him three boats.’ \citep[8]{Martin1936}
\z

\gll  b.  Esi  kyɛ-ɛ    p\`{a}p\'{a}  n\'{o}   s\`{i}k\'{a}.\\
       Esi  gift-\textsc{compl}  man  d\textsc{ef}  money\\
\glt   ‘Esi gave the man money.’
\z

\chapter[4   Decreasing valence ]{4   Decreasing valence }

There are morphosyntactic processes that reduce verb valence. In various languages, morphological derivations are utilized in reducing the valence of a verb. But as has been pointed out in the literature, verbs can also manifest a change in the valence structure without the application of any morphological processes: “Alternations in a verb’s valency pattern are not necessarily the result of a morphological derivational process. Verbs or whole classes of verbs may have alternate valency patterns without any change in their formal makeup . . .” (Haspelmath \& Müller-Bardey 2004: 1131).

\section{ \%/}

Reflexivization in Akan is marked by the use of a possessive (\textsc{poss}) pronoun and the morpheme \emph{ho} ‘self’.

%\setcounter{itemize}{0}
\begin{itemize}
\item \end{itemize}
\gll a.  Kofi  \'{e}-k\`{u}    Yaw.\\
       Kofi  \textsc{perf}{}-kill  Yaw\\
\glt ‘Kofi has killed Yaw.’
\z

\gll  b.  Kofi    \'{e}-k\`{u}    n\'{o}    h\'{o}.\\
       Kofi    \textsc{perf}{}-kill  \textsc{3sg.poss}  self\\
\glt ‘Kofi has killed himself.’
\z

In reflexivization, the notion of coreference is crucial. \citet[44]{Kemmer1993} notes that “Coreference . . .  means that two participants in a single event frame designate the same entity in the described situation.”  It requires that the A and O arguments have the same referent. Reflexive constructions in Akan involving bivalent verbs have the coreferential O argument replaced by the POSS+\emph{ho}.  The reduction in the valency of the verb lies in the fact that there is no semantic differentiation between the A and O arguments. 

Apart from the direct reflexive, certain verbs, specifically grooming, or body care actions \citep{Kemmer1993} are used reflexively and, consequently, demonstrate (semantic) valency decreasing properties. Examples of such verbs in Akan are: \emph{pepa} ‘wipe’, \emph{sera} ‘smear/use lotion or oil on the body’, \emph{twutwuw} ‘wash (as with a sponge/washcloth)’, \emph{siesie} ‘dress up’. The actions coded by these verbs can be carried out by an \textsc{agent} on a \textsc{patient} entity or the \textsc{agent} can carry it out on themselves. 

\begin{itemize}
\item \end{itemize}
\gll a.  Aba  tw\`{u}tw\'{u}\`{w}-\`{w}    Ekua  n\'{o}    h\'{o}.\\
       Aba  wash.wash-\textsc{compl}  Ekua  \textsc{3sg.poss}  self\\
\glt   ‘Aba washed Ekua.’
\z

\gll  b.  Aba  tw\`{u}tw\'{u}\`{w}-\`{w}    n\'{o}    h\'{o}.\\
       Aba  wash.wash-\textsc{compl}  \textsc{3sg.poss}  self\\
\glt   ‘Aba washed someone/herself.’
\z

\begin{itemize}
\item \end{itemize}
\gll a.  Efua  s\'{e}r\`{a}-\`{a}    Kwesi    n\'{o}    h\'{o}.\\
       Efua  smear-\textsc{compl}  Kwesi    \textsc{3sg.poss}  self\\
\glt   ‘Efua used lotion on Kwesi.’
\z

\gll  b.  Efua  s\'{e}r\`{a}-\`{a}    n\'{o}    h\'{o}.\\
       Efua  smear-\textsc{compl}  \textsc{3sg.poss}  self\\
\glt   ‘Efua used lotion on someone/herself.’
\z

In (29a) and (30a), the entities in subject position, as the \textsc{agents}, carry out the activities on certain other individuals, the entities in the \textsc{patient} role. In (29b) and (30b), the replacement of the full NP with the reflexive pronoun in the post-verbal position creates ambiguities. In each sentence, the referent of the reflexive pronoun could be the entity in the subject position, the \textsc{agent}; or it could be individual already mentioned in the context of the discourse; that is Ekua in (29a) and Kwesi in (30a).

Where the referents of the reflexive pronouns in (29b) and (30b) are the \textsc{agents} in the subject positions in the two sentences, we can argue for a reduction in the valency of the verbs \emph{sera} ‘smear/use lotion or oil on the body’ and \emph{twutwuw} ‘wash (as with a sponge/washcloth)’, on the basis of coreferentiality and non-individuation of the participants involved in the situation. 

In addition to the reduction in the valency of the verb \emph{sera} through reflexivization, its valency can also be reduced through its use in an intransitive construction as (31b) shows.

\begin{itemize}
\item \end{itemize}
\gll a.  Efua  s\'{e}r\`{a}-\`{a}    n\'{o}    h\'{o}.\\
       Efua  smear-\textsc{compl}  \textsc{3sg.poss}  self\\
\glt   ‘Efua used lotion on herself.’
\z

\gll  b.  Efua  s\'{e}r\`{a}-\`{e}.\\
       Efua  smear-\textsc{compl}\\
\glt   ‘Efua used lotion (on herself).’
\z

There is another feature of the verb \emph{sera} pertaining to valency adjustment that will be discussed in §4.5.2.

\section{4.2  Reciprocals}

The reciprocal in Akan is formed in ways similar to the reflexive. It also uses the morpheme \emph{ho}. But unlike the reflexive, the possessive pronoun that combines with \emph{ho} is in the plural (32). Similar to the reflexive, the reciprocal creates reduced valency due to coreferentiality and lack of individuation.

\begin{itemize}
\item \end{itemize}
\gll a.  Hwɛ,  \'{e}m\'{i}     n\`{a}   w\'{o}-\'{e}gy\`{a}     y\`{e}-h\'{u}\'{n}-\`{n}       \\
       look  \textsc{1sg.emph}  and  \textsc{2sg.poss}{}-father  \textsc{1pl.subj}{}-see-\textsc{compl}  \\
\gll   hɛ\'{n}    h\'{o}   b\'{e}\'{r}   â  m\'{i}-dz\'{i}-\`{i}     \`{m}-f\'{e}  \\
       \textsc{1sg.poss}  self  time  \textsc{rel}  \textsc{1sg.subj}{}-eat-\textsc{compl}  \textsc{pl}{}-year\\
\gll   \`{e}d\`{u}\`{o}n\'{u}   \`{a}n\'{a}\'{n}     n\'{o}.     \\
       twenty    four    \textsc{dcm}\\
\glt ‘Your father and I got to know each other when I was twenty-four years.’ \citep[9]{Martin1936}
\z

\gll  b.  \`{M}-b\`{e}r\'{a}\'{n}ts\'{e}  n\`{a}  \`{n}-k\`{a}t\`{a}\'{a}s\'{i}\'{a}  n\'{o}   ny\'{e}     Araba Akɔm  \\
       \textsc{pl}{}-man    and  \textsc{pl}{}-woman  \textsc{def}  accompany  Araba Akɔm\\
\gll   h\'{a}m-\`{e}\`{e}      y\`{e}-y\'{a}\`{w}-\`{w}    hɔ\'{n}    h\'{o}  m\`{a}\\
       quarrel-\textsc{compl}    \textsc{redu}{}-insult-\textsc{compl}  \textsc{3pl.poss}  self  that\\
\gll   ɔ{}-yɛ-ɛ        \`{a}sɛ\'{m}  wɔ  sk\'{u}\`{u}l.\\
       \textsc{3sg.subj}{}-make-\textsc{compl}  issue  in  school\\
\glt ‘The young men and young women quarreled with Araba Akɔm and insulted each other such that it became an issue in the school.’ \citep[21]{Martin1936}  
\z

\section{4.3  Anticausative/Inchoative}

The anticausative (or decausative, inchoative, spontaneous, pseudopassive, being the various ways in which this type of construction has been labelled) works by removing the \textsc{agent} argument in the construction. In his characterization of the causative/inchoative alternation, Haspelmath states that:

\begin{quote}
An inchoative/causative verb pair is defined semantically: it is a pair of verbs which express the same basic situation (generally a change in state, more rarely a going-on) and differ only in that the causative verb meaning includes an Agent participant who causes the situation, whereas the inchoative verb meaning excludes a causing Agent and presents the situation as occurring spontaneously . . . Inchoative verbs are generally intransitive and causative verbs are transitive . . . \citep[90]{Haspelmath1993}
\end{quote}

As pointed out by Haspelmath \& Müller-\citet[1132]{Bardey2004}: 

\begin{quote}
In many languages there is a strong requirement for all sentences to have subjects. When in such languages a valency-changing category removes the agent argument from the subject position, the patient argument must take up the subject position instead.
\end{quote}

Akan is a typical example of such languages. When a sentence has a single argument, that argument is always in the subject position. In Akan anticausatives the \textsc{theme} argument is the subject of the sentence. 

Unlike some other languages where there is a derivational process to indicate either the demotion of the \textsc{agent} argument or its introduction, in Akan there is no change in the morphology of the verb stem to reflect the process of anticausative. This puts Akan into what \citet[91]{Haspelmath1993} describes as non-directed alternation, that is, where “. . . neither the inchoative nor the causative verb is derived from the other.”

The verbs in the language that can be used in the anticausative construction include: \emph{bɔ} ‘break’, \emph{bu} ‘break’, \emph{bue} ‘open’, \emph{butu} ‘overturn’, \emph{dum} ‘put out’, \emph{hyew} ‘burn’, \emph{koa} ‘bend’, \emph{kyea} ‘bend’, \emph{nane} ‘melt’, \emph{sԑe} ‘destroy’, \emph{te} ‘tear, pluck’, \emph{tuei} ‘puncture’, \emph{woso} ‘shake, vibrate’. 

Example (33a) and (34a) are causative and (33b) and (34b) are anticausative/inchoative.

\begin{itemize}
\item \end{itemize}
\gll a.  n\`{a}  wɔ-b\`{o}-bɔ-ɔ       \`{n}-k\`{u}r\`{a}b\'{a}   n\'{o}  â \\
       and  \textsc{3pl.subj}{}-\textsc{redu}{}-break\textsc{{}-compl}  \textsc{pl}{}-jar    \textsc{def}  \textsc{rel}  \\
\gll   ɔ{}-dz\'{e}-dz\`{e}    hɔ\'{n}    \'{n}s\'{a}-m\'{u}    n\'{o}.   \\
             \textsc{3pl.subj}{}-\textsc{redu}{}-hold  \textsc{3pl.poss}  hand-in  \textsc{dcm}\\
\glt ‘. . .  and they broke the jars that were in their hands.’ (Judges 7:19 Fante Bible)
\z

\gll  b.  \`{n}-k\`{u}r\`{a}b\'{a}   n\'{o}  â  ɔ{}-dz\'{e}-dz\`{e}    hɔ\'{n}  \\
       \textsc{pl}{}-jar    \textsc{def}  \textsc{rel}  \textsc{3pl.subj}{}-\textsc{redu}{}-hold  \textsc{3pl.poss}\\
\gll   \'{n}s\'{a}-m\'{u}    n\'{o}    b\`{o}-bɔ-\`{e}.\\
       hand    \textsc{dcm}    \textsc{redu}{}-break-\textsc{compl}\\
\glt ‘. . .  and the jars in their hands broke.’
\z

\begin{itemize}
\item \end{itemize}
\gll a.  n\`{a}  Moses    hy\'{e}\`{w}-\`{w}   ts\'{i}\'{r}  n\'{o}.\\
       and   Moses     burn-\textsc{compl}  head  \textsc{def}\\
\glt ‘. . .  and Moses burned the head.’ (Leviticus 8:20 Fante Bible)
\z

\gll  b.  n\`{a}  ts\'{i}\'{r}   n\'{o}  hy\'{e}\`{w}-\`{e}\`{e}.\\
       and  head  \textsc{def}  burn-\textsc{compl}  \\
\glt “. .  . and the head burned.’
\z

\section{ \%/}

Another means of valency decrease is through impersonal constructions. The notion of impersonal construction adopted here follows Siewierska (2008; 2011) and Malchukov \& \citet{Ogawa2011}. These are constructions that do not have a referential subject. Malchukov \& \citet{Ogawa2011}, following \citet{Keenan1976}, argue that the subject in an impersonal construction deviates from the prototype subject. In Keenan’s (1976) approach, the canonical subject is expected to have the following properties (Malchukov \& Ogawa 2011:23):

\begin{styleNoSpacing}
a referential argument
\end{styleNoSpacing}

\begin{styleNoSpacing}
a definite NP
\end{styleNoSpacing}

\begin{styleNoSpacing}
topical
\end{styleNoSpacing}

\begin{styleNoSpacing}
animate
\end{styleNoSpacing}

\begin{styleNoSpacing}
agentive
\end{styleNoSpacing}

Based on cross-linguistic studies, various coding strategies of impersonal constructions have been isolated. One of these is the pronominal impersonal \citep{Siewierska2011}. In some languages this involves the use of a regular personal pronoun as the subject of the construction. Akan does this by using the regular 3 person plural subject pronoun, as illustrated in (35).

\begin{itemize}
\item \end{itemize}
\gll a.  Wɔ-\'{a}-ky\`{e}\`{r}    \`{e}w\`{i}-f\'{o}    n\'{o}.\\
       \textsc{3pl.subj}{}-\textsc{perf}{}-catch  thief-\textsc{pl}  \textsc{def}\\
\glt ‘They have arrested the thieves.’/‘The thieves have been arrested.’
\z

\gll  b.  Wɔ-\'{a}-t\`{o}    \'{e}s\'{i}ky\`{i}r\'{e}  n\'{o}    b\'{o}  m\'{u}.\\
       \textsc{3pl.subj}{}-\textsc{perf}{}-raise  sugar    \textsc{3sg.poss}  price  in\\
\glt   ‘They have increased the price of sugar.’/‘The price of sugar has been increased.’
\z

The subject pronouns in (35) are non-referential and non-individuated (Hopper \& Thompson 1980). Example (36) below is taken from the Apostles’ Creed of the Christian faith. 

\begin{itemize}
\item \end{itemize}
\gll Wɔ-bɔ-ɔ       N\`{o}    \`{m}b\`{e}\`{a}m\'{u}d\'{u}\'{a}  m\`{u}, \\
     \textsc{3pl.subj}{}-hit\textsc{{}-compl}  \textsc{3sg.obj}  cross    in\\
\gll \`{O}-w\'{u}-\`{i},     w\`{o}-s\'{i}\'{e}-\`{e}     N\`{o}.\\
     \textsc{3sg.subj}{}-die  \textsc{3pl.subj}{}-bury\textsc{{}-compl}  \textsc{3sg.obj}\\
\glt ‘He was crucified, dead and buried.’ (Source: Christian Asɔr Ndwom 1937)
\z

Even though in the constructions in (35) and (36) there are two arguments, A and O, functionally, there is reduced valency because the subjects are not prototypical subjects. The sentences have \textsc{agent} subjects. Nonetheless, the \textsc{agents} involved are not distinct from the \textsc{patients} because the pronominal form used is non-referential. This makes it impossible to identify the referent, an obvious way of downgrading the \textsc{agent} argument.

\section{4.5  Object omission/suppression }

Decreasing valency can also involve verbs that potentially have A and O arguments. However, the O argument remains suppressed either because it is understood or it is known that any one of a range of entities can fill that argument position.

\subsection{4.5.1  Understood object}

Verbs in this group, falling in the category of Inherent Complement Verbs (\emph{dwanse} ‘urinate’, \emph{bow} ‘be drunk’, \emph{nye} ‘defecate’, \emph{dɔr} ‘be fatty’, and sa ‘dance’) can occur with both overt A and O arguments. However, in many instances, the O argument is not expressed because speakers know what it is, thereby reducing the grammatical valence of the verb. The verbs and their inherent complements are listed below:

%\setcounter{itemize}{0}
\begin{itemize}
\item \end{itemize}
\gll a.  dw\`{a}\`{n}s\`{e}   dw\'{a}\'{n}s\'{e}\\
\glt   ‘urinate’  ‘urine’
\z

\gll  b.  b\`{o}\`{w}    \`{n}s\'{a}\\
\glt   ‘be drunk’  ‘drink/alcohol’
\z

\gll  c.  ny\`{e}    b\'{i}\'{n}\\
\glt   ‘defecate’  ‘faeces’
\z

\gll  d.  dɔr\`{e}    s\`{e}r\`{a}d\'{e}ԑ\\
\glt   ‘be fatty’  ‘fat’
\z

\gll  e.  s\`{a}    \`{a}s\'{a}\\
\glt   ‘dance’    ‘a dance’
\z

\begin{itemize}
\item \end{itemize}
\gll a.  B\`{a}ny\'{i}\'{n}  n\'{o}  \'{a}-b\`{o}\`{w}    \`{n}s\'{a}.\\
       man  \textsc{def}  \textsc{perf}{}-be drunk  alcohol\\
\glt ‘The man is drunk (with alcohol).’
\z

\gll  b.  B\`{a}ny\'{i}\'{n}  n\'{o}  \'{a}-b\`{o}\`{w}.\\
       man  \textsc{def}  \textsc{perf}{}-be drunk\\
\glt ‘The man is drunk.’
\z

\begin{itemize}
\item \end{itemize}
\gll a.  \`{A}pɔ\`{n}ky\'{e}  n\'{o}  \'{a}-dɔr\`{e}      s\`{e}r\`{a}d\'{e}ԑ\'{ }.\\
       goat    \textsc{def}  \textsc{perf}{}-be fatty    fat\\
\glt   ‘The goat is fatty.’ (Lit. ‘The goat is fatty with fat.’) (As)
\z

\gll  b.  \`{A}pɔ\`{n}ky\'{e}  n\'{o}  \'{a}-dɔr\'{e}.      As\\
       goat    \textsc{def}  \textsc{perf}{}-be fatty  \\
\glt   ‘The goat is fatty.’
\z

In (38a) and (39a) the verbs are used with their inherent complements. However, in (38b) and (39b), there are no overt complements, reflecting a reduction in the grammatical valence of the verbs \emph{bow} ‘be drunk’ and \emph{dɔre} ‘be fatty’.

\subsection{4.5.2  Unspecified Object}

Akan has a limited number of verbs that demonstrate varying valency: monovalent, bivalent, and trivalent. The reduction in valence of these verbs revolves around the non-expression of the O argument in all instances. This means that the single argument of the intransitive construction is an \textsc{agent}. It is more appropriate then to talk about the suppression of the O argument. Another feature of these verbs is that they tend to take various items as the O argument. Examples of the verbs are: \emph{soa} ‘carry’, \emph{hyɛ} ‘wear, dress up’, \emph{sera} ‘apply body lotion, smear’, \emph{sa} ‘administer enema’, \emph{son} ‘apply herbal nasal drop’, \emph{tua} ‘douche’.

%\setcounter{itemize}{0}
\begin{itemize}
\item \end{itemize}
\gll a.  Araba  s\'{e}r\`{a}-\`{a}    \`{n}k\'{u}.\\
       Araba  smear-\textsc{compl}  shea.butter\\
\glt ‘Araba applied shea butter on her body.’
\z

\gll  b.  Araba  s\'{e}r\`{a}-\`{e}.\\
       Araba  smear-\textsc{compl}\\
\glt ‘Araba applied (some substance) to her body.’
\z

In the first example above, the A and O arguments are expressed. However in the second example the O argument remains unexpressed; the single argument, S, is the \textsc{agent}. The ditransitive use of these verbs is illustrated in (41):

\begin{itemize}
\item \end{itemize}
\gll a.  Araba  s\'{e}r\`{a}-\`{a}    \`{a}b\`{o}fr\'{a}  n\'{o}  \`{n}k\'{u}.\\
       Araba  smear-\textsc{compl}  child  \textsc{def}  shea.butter\\
\glt ‘Araba applied shea butter on the child.’
\z

\gll  b.  Araba  s\'{e}r\`{a}-\`{a}    \`{a}b\`{o}fr\'{a}  n\'{o}.\\
       Araba  smear-\textsc{compl}  child  \textsc{def}\\
\glt ‘Araba applied (some substance) on the child.’
\z

In (41a), the verb has three arguments: A, E, and O, in that order. In (41b), the O argument is unspecified, leaving the A and E arguments.

\chapter[5  Increasing valency]{5  Increasing valency}

There are processes that lead to an increase in the arguments of a verb. There are two main ways in Akan that this happens: the introduction of an \textsc{agent} through serialization and causativization.

\section{5.1  Causativization}

Causative constructions increase valence by adding a causing \textsc{agent} to an event. As seen in §4.3, Akan has verbs that permit the anticausative construction. The causative variant would have an \textsc{agent} present, as in (42a) and (43a). In (42b) and (43b), the change in state of the \textsc{patient} entities — chains and stick, respectively — is captured without the specification of the responsible \textsc{agent}. In (42a) and (43a), on the other hand, the events are presented with the causing \textsc{agents} overtly stated. Effectively, in (42a) and (43a) where we have the causative constructions, the valency of the verbs has been increased by the addition of the causing \textsc{agents}.

%\setcounter{itemize}{0}
\begin{itemize}
\item \ea\label{ex:}
\langinfo{}{}{a.  Causative}\\
\end{itemize}
\gll N\`{a}  \`{o}-b\`{u}-b\'{u}-\`{u}      hɔ\'{n}    \`{m}p\`{o}ky\`{e}rɛ  m\`{u}   \`{e}s\`{i}\`{n}-\'{e}s\'{i}\'{n}.\\
     and  \textsc{3sg.subj}{}-\textsc{redu}{}-break-\textsc{compl}  \textsc{3pl.poss}  chains    in  \textsc{redu}{}-piece\\
\glt ‘. . . and he broke their chains in pieces.’ (Fante Bible Psalm 107:14)
\z

\ea\label{ex:}
\langinfo{}{}{b.  Anticausative}\\
\gll Hɔ\'{n}  \`{m}p\`{o}ky\`{e}rɛ   m\`{u}  b\`{u}-b\'{u}-\`{u}     \`{e}s\`{i}\`{n}-\'{e}s\'{i}\'{n}.\\
     \textsc{3pl.poss}  chains    in  \textsc{redu}{}-break-\textsc{compl}  \textsc{redu}{}-piece\\
\glt ‘Their chains broke into pieces.’
\z

\begin{itemize}
\item \ea\label{ex:}
\langinfo{}{}{ a.  Causative}\\
\end{itemize}
\gll Kofi  \'{a}-ky\'{e}\'{a}    \`{a}b\`{a}\'{a}  n\'{o}.\\
     Kofi  \textsc{perf}{}-bend  stick  \textsc{def}\\
\glt ‘Kofi has bent the stick.’
\z

\ea\label{ex:}
\langinfo{}{}{b.  Anticausative}\\
\gll \`{A}b\`{a}\'{a}  n\'{o}  \'{a}-ky\'{e}\'{a}.\\
     stick  \textsc{def}  \textsc{perf}{}-bend\\
\glt ‘The stick is bent.’
\z

\section{5.2  Agent introduction through serialization}
\subsection{5.2.1  Agentive Argument for Verbs of Spatial Configuration}

There are verbs that code the location or spatial configuration of an entity. Examples  include: \emph{da} ‘lie, be at’, \emph{twer} ‘lean’, \emph{bea} ‘lie’, \emph{sɛn} ‘hang’, \emph{hyɛ} ‘ be in’, \emph{gu} ‘be in’, \emph{tar} ‘paste, stick’, \emph{fam} ‘stick’, and \emph{si} ‘stand’. For verbs like these, the introduction of an agentive NP requires the use of a serial construction. Even though the resulting construction cannot be said to increase the valency of the verbs, it shows how an agentive argument can be introduced through the syntactic strategy of serialization.

%\setcounter{itemize}{0}
\begin{itemize}
\item \end{itemize}
\gll a.  N\`{a}  ɔ{}-dz\`{e}     \`{m}-p\`{o}m\'{a}   n\'{o}  hy\`{e}-hyɛ-ɛ \\
             and  \textsc{3sg.subj}{}-take  \textsc{pl}{}-pole    \textsc{def}  \textsc{redu}{}-put-\textsc{compl}  \\
\gll   \`{a}d\'{a}k\'{a}     n\'{o}    h\'{o}  \\
       box    \textsc{3sg.poss}  self  \\
\glt    ‘And he took the poles and put them on the ark.’/‘And he put the poles on the ark.’ (Fante Bible, Exodus 40: 20)
\z

\gll  b.  \`{M}-p\`{o}m\'{a}   n\'{o}   hy\`{e}-hyɛ   \`{a}d\'{a}k\'{a}  n\'{o}    h\'{o}.\\
       \textsc{pl}{}-pole    \textsc{def}  \textsc{redu-cont}  box  \textsc{3sg.poss}  self  \\
\glt   ‘The poles are on the ark.’
\z

\begin{itemize}
\item \end{itemize}
\gll a.   Wɔ-d\`{e}    wɔ\'{n}    \'{a}-kyɛ\'{m}   s\`{e}\`{n}-sɛn\`{e} \\
       \textsc{3pl.subj}{}-take  \textsc{3pl.poss}  \textsc{pl}{}-shield  \textsc{redu}{}-hang/\textsc{hab}\\
\gll   w-\`{a}-f\'{a}s\'{u}\'{o}    h\'{o}   hy\`{i}\'{a}.    \\
       \textsc{3sg.poss-pl}{}-wall  self  meet/\textsc{hab}  \\
\glt ‘They hang their shields around your walls.’ (Asante Bible, Ezekiel 27: 11)
\z

\gll  b.  Wɔ\'{n}    \'{a}-kyɛ\'{m}    s\`{e}\`{n}-sɛn\`{e}     w-\`{a}-f\'{a}s\'{u}\'{o} \\
       \textsc{3pl.poss}  \textsc{pl}{}-shield  \textsc{redu}{}-hang/\textsc{hab}  \textsc{3sg.poss-pl}{}-wall\\
\gll   h\'{o}  hy\`{i}\'{a}.\\
       self  meet/\textsc{hab}\\
\glt ‘Their shields hang around your walls.’ (As)
\z

\begin{itemize}
\item \end{itemize}
\gll a.  K\`{a}\`{n}dz\'{e}\'{a}  n\'{o}  s\`{i}    p\'{o}\'{n}  n\'{o}  d\'{o}.\\
       lantern    \textsc{def}  stand/\textsc{cont}  table  \textsc{def}  on\\
\glt ‘The lantern is on the table.’
\z

\gll  b.  Ato  dz\`{e}  k\`{a}\`{n}dz\'{e}\'{a}  n\'{o}  s\'{i}-\`{i}    p\'{o}\'{n}  n\'{o}  d\'{o}.\\
       Ato  take  lantern    \textsc{def}  stand-\textsc{compl}  table  \textsc{def}  on\\
\glt ‘Ato placed the lantern on the table.’
\z

\chapter[6  Valence Adjustment through Reduplication  ]{6  Valence Adjustment through Reduplication  }

One valency adjusting morphological process in Akan with limited application is verbal reduplication. So far, only two verbs have been identified in the language that change their valence when reduplicated. The verbs are \emph{da} ‘sleep’ and \emph{di} ‘eat’. Their reduplicated forms are \emph{deda} ‘put to sleep’ and \emph{didi} ‘eat’.

%\setcounter{itemize}{0}
\begin{itemize}
\item \end{itemize}
\gll a.  Ama  d\`{a}-\`{a}    \`{a}w\`{i}\'{a}.\\
       Ama  sleep-\textsc{compl}  afternoon\\
\glt   ‘Ama slept in the afternoon.’ (As)
\z

\gll  b.  Ama  d\`{e}-d\'{a}-\`{a}    \`{a}b\`{o}fr\'{a}    n\'{o}  \`{a}w\`{i}\'{a}.\\
       Ama  \textsc{redu}{}-sleep\textsc{{}-compl}  child    \textsc{def}  afternoon\\
\glt   ‘Ama put the child to sleep in the afternoon.’ (As)
\z

As shown in (47b), the reduplication of the verb \emph{da} ‘sleep’ is a means by which an \textsc{agent} argument is introduced.

The behavior of \emph{di} when reduplicated is the reverse of the reduplication of \emph{da}. When \emph{di} is reduplicated, it loses the capacity to have a \textsc{patient} argument.

\begin{itemize}
\item \end{itemize}
\gll a.  Kofi  d\'{i}-\`{i}    \`{a}d\`{u}\'{a}\'{n}    n\'{o}\\
       Kofi  eat-\textsc{compl}  food    \textsc{def}\\
\glt   ‘Kofi ate the food.’ (Ak)
\z

\gll  b.  Kofi  d\`{i}-d\'{i}-\`{i}.\\
       Kofi  \textsc{redu}{}-eat-\textsc{compl}\\
\glt   ‘Kofi ate.’ (Ak)
\z

\gll  c.  *Kofi  d\`{i}-d\'{i}-\`{i}      \`{a}d\`{u}\'{a}\'{n}    n\'{o}.\\
       Kofi  \textsc{redu}{}-eat-\textsc{compl}  food    \textsc{def}\\
\glt   ‘Kofi ate the food.’
\z

There is a polysemous use of \emph{didi} where it takes a postpositional phrase in the post-verbal position (49). But the use of the reduplicated form of the verb \emph{di} in this situation is metaphorical and does not contradict the case made about the verb.

\begin{itemize}
\item \end{itemize}
\gll a.  \`{O}b\'{i}\'{a}r\'{a}    d\`{i}-d\'{i}    n-\`{a}dw\'{u}m\'{a}    h\'{o}.\\
       everybody  \textsc{redu}{}-eat  \textsc{3sg.poss}{}-work  self\\
\glt   ‘Everybody benefits from their work.’ (Ak)
\z

\gll  b.  \`{N}krɔf\'{o}  n\'{o}  d\`{i}-d\'{i}-\`{i}      ɔh\'{e}\'{n}   n\'{o}  \'{a}s\'{e}.\\
       people  \textsc{def}  \textsc{redu}{}-eat-\textsc{compl}  chief  \textsc{def}  under\\
\glt   ‘The people sabotaged the chief.’ (Ak)
\z

\chapter[7  Summary and conclusion]{7  Summary and conclusion}

The notion of valency has received extensive treatment in the linguistics literature. I set out in this paper to examine the valency of Akan verbs and to investigate the morphosyntactic ways in which the valency of verbs can be modified. 

I have shown in the preceding discussion that we can identify verbs in the language that are invariably monovalent — that is, verbs that take only one core argument, the S argument; those that are bivalent — requiring two core arguments, A and O; and those that are trivalent, needing three core arguments, that is, A, O, and E.

Apart from the verbs with invariant argument structure, there are many verbs that exhibit variations in the expression of their arguments. I have shown in the paper that overall, the morphosyntactic mechanisms by which the valency of verbs is modified in Akan fit into various cross-linguistic patterns. Akan is not known as a language with complex morphology. Consequently, the valency adjusting processes tend to be more syntactic than morphological. Verbs in the language that can undergo a reduction in the expression of their arguments do so through reflexivization, the use of reciprocals, anticausatives, impersonal constructions, and various forms of object suppression. It has been shown in the paper that where there is valency decrease resulting in only argument being expressed, the single argument is always the S argument.  Increase in verb valency is achieved through causativization, and agentivization through serialization. It has also been demonstrated that the language uses reduplication in a very limited way to adjust verb valency. As stated in the paper, the use of reduplication applies to only two verbs in the language.

Dixon \& \citet[25-27]{Aikhenvald2000b}, in ending their paper, identify some topics that need to be investigated regarding the notion of valency cross-linguistically:

\begin{quote}
Our preliminary impression is that, across the languages of the world, there tend to be more valency-increasing derivations (comitative and applicative) than valency-reducing derivations (passive, antipassive, reflexive, reciprocal, etc.). This needs to be verified, through study of a large representative sample of languages; if it is true, linguists should seek an explanation. (Dixon \& \citet[26]{Aikhenvald2000b}  
\end{quote}

From what has been presented in this paper, it is obvious that in Akan there are more valency decreasing morphosyntactic strategies than those used to increase valency. Based on my knowledge of the languages that are genetically related to Akan, for example, Ga and Ewe, one would expect a similar tendency. However, this is an issue that needs to be investigated. More broadly, the tools that have been developed by the Leipzig Project and by Nichols and associates need to be applied to Akan and related languages in order to contribute to our further understanding of the notion valency.

\chapter{Abbreviations}

\textsc{col}   collective

\textsc{compl}  completive

\textsc{cons}  consecutive

\textsc{cont}  continuative

\textsc{dcm}   dependent clause marker

\textsc{def}   definite

\textsc{emph}  emphatic

\textsc{fut}  future

\textsc{hab}  habitual

\textsc{idm}  identity marker

\textsc{imp}  imperative

\textsc{neg}  negation

\textsc{obj}  objective

\textsc{opt}  optative

\textsc{perf}  perfect

\textsc{pl}  plural

\textsc{poss}  possessive

\textsc{prog}  progressive

\textsc{redu}  reduplication

\textsc{rel  }relativizer

\textsc{sg  }singular

\textsc{subj  }subject

\chapter{References}

Adi, Kwabena. 1973. \emph{Brako}. Accra-Tema: Ghana Publishing Company.

Agyeman, Nana Ama. 2002. \emph{Serial verb constructions in Akan.} Trondheim: NTNU. (Master’s thesis.)

Bible Society of Ghana. 1964. \emph{Twerɛ Kronkron}\emph{\textup{ (}}\emph{Asante Bible}\emph{\textup{). Accra: Bible Society of Ghana.}}

Bible Society of Ghana. 1974. \emph{Nwoma Krɔnkrɔn}\emph{\textup{ (}}\emph{Fante Bible}\emph{\textup{). Accra: Bible Society of Ghana.}}

Comrie, Bernard \& Polinsky, Maria (eds.). 1993. \textit{Causatives and transitivity.} Amsterdam: John Benjamins.

Denteh, A. Crakye. 1989. \emph{Duabataboɔ}. Accra: Baafour Educational Enterprises Ltd.

Dixon, R. M. W. \& Aikhenvald, Alexandra Y. (eds.). 2000a. \textit{Changing valency: Case studies in transitivity.} Cambridge: Cambridge University Press.

Dixon, R. M. W. \& Aikhenvald, Alexandra Y. 2000. Introduction. In Dixon, R. M. W. \& Aikhenvald, Alexandra Y. (eds.), \textit{Changing valency: Case studies in transitivity}, 1-29. Cambridge: Cambridge University Press.

Dophyne, Florence A. 1988. \textit{The Akan (Twi-Fante) language: Its sound systems and tonal structure.} Accra: Ghana Universities Press.

Essilfie, Thomas. 1977. \textit{Serialization in Fante}. Leeds: Leeds University. (Doctoral thesis.)

Forson, Barnabas. 1990. On the morphology of Akan serial constructions. In Hutchison, J. P. \& Manfredi, V. (eds.), \textit{Current approaches to African linguistics}, 63-66. Dordrecht: Foris.

Haspelmath, Martin. 1993. More on the typology of inchoative/causative verb alternations. In Comrie, Bernard \& Polinsky, Maria (eds.), \textit{Causatives and transitivity}, 87-120. Amsterdam: John Benjamins.

Haspelmath, Martin \& Müller-Bardey, Thomas. 2004. Valency change. In Booij, Geert, Lehmann, Christian, Mugdan, Joachim \& Skopeteas, Stavros (eds.), \textit{Morphology: An international handbook on inflection and word-formation}, vol. 2, 1130-1145. Berlin: Walter de Gruyter Mouton.

Hellan, Lars, Beermann, Doroth\'{e}e \& Andenes, Eli S. 2003. Towards a typology of serial verb constructions in Akan. In Kropp Dakubu, Mary Esther \& Osam, E. Kweku (eds.), \textit{Studies in the languages of the Volta basin}, vol.1, 61-86. Legon: University of Ghana.

Hopper, Paul \& Thompson, Sandra. 1980. Transitivity in grammar and discourse. \textit{Language} 56. 251–299.

Kambon, Obadele. 2012. \textit{Serial verb nominalization in Akan.} Legon: University of Ghana. (Doctoral thesis.)

Keenan, Edward L. 1976. Towards a universal definition of subject. In Li, Charles (ed.), \textit{Subject and topic}, 303-334. New York: Academic Press.

Kemmer, Suzanne. 1993. \textit{The middle voice}. Amsterdam: John Benjamins.

Krampah, D. E. K. 1970. \emph{Mbofra mfa adwen}. Accra: Bureau of Ghana Languages.

Lord, Carol. 1973. Serial verbs in transition. \textit{Studies in African Linguistics} 4. 269-296.

Luraghi, Silvia. 2012. Basic valency orientation and the middle voice in Hittite. \textit{Studies in Language} 36. 1-32.

Malchukov, Andrej \& Comrie, Bernard (eds.). In press. \textit{Valency classes in the world’s languages}. Berlin: Walter de Gruyter Mouton.

Malchukov, Andrej \& Ogawa, Akio. 2011. Towards a typology of impersonal constructions: A semantic map approach. In Malchukov, Andrej \& Siewierska, Anna (eds.), \textit{Impersonal constructions: A cross-linguistic perspective}, 19-56. Amsterdam: John Benjamins.

Martin, Henry James. 1936. \emph{Ekumfi Benada}. Accra: Bureau of Ghana Languages.

Methodist Church Ghana. 1937. \textit{Christian Asɔr Ndwom }(\textit{The Fante Hymnbook}). Cape Coast: Methodist Book Depot Ltd.

Narogg, Heiko. 2009. Synchrony and diachrony in transitivity pairs. (Paper presented at the 8th Association of Linguistic Typology conference, University of California, Berkeley, July 23-26, 2009.)

Nichols, Johanna, Peterson, David \& Barnes, Jonathan. 2004. Transitivizing and detransitivising languages. \textit{Linguistic Typology} 8. 149–211.

Osam, E. Kweku. 1994a. \textit{Aspects of Akan grammar: A functional perspective}. Eugene: University of Oregon. (Doctoral dissertation.)

Osam, E. Kweku. 1994b. From serial verbs to prepositions and the road between. \textit{Sprachtypologie und Universsalienforschung} 47. 16-36.

Osam, E. Kweku. 1996. The object relation in Akan. \textit{Afrika und Übersee} 79. 57-83.

Osam, E. Kweku. 1997. Verb serialization and grammatical relations in Akan. In Giv\'{o}n, T. (ed.), \textit{Grammatical relations: A functionalist perspective}, 255-281. Amsterdam: John Benjamins.

Osam, E. Kweku. 2000. The status of the indirect object in Akan. \textit{Journal of Asian and African Studies }59. 171-177.

Osam, E. Kweku. 2002. The ingressive in Akan: A re-consideration. In Ameka, Felix \& Osam, E. Kweku (eds.), \textit{New directions in Ghanaian linguistics}, 113-126. Accra: Black Mask.

Osam, E. Kweku. 2004. \textit{The Trondheim Lectures. An introduction to Akan: Its verbal and multiverbal systems}. Legon: University of Ghana.

Osam, E. Kweku. 2008a. Verbal alternations in Akan. \textit{Journal of African Languages and Linguistics} 29. 97-118.

Osam, E. Kweku 2008b. Akan as an aspectual language. In Ameka, Felix \& Kropp Dakubu, Mary Esther (eds.), \textit{Aspect and modality in Kwa languages}, 69-89. Amsterdam: John Benjamins.

Osam, E. Kweku. 2014. Descriptive features of serial verb constructions in Akan. In Yankah, Kwesi, Saah, Kofi K. \& Amfo, Nana Aba Appiah (eds.), \textit{A Legon reader in Ghanaian linguistics}, 11-23. Oxfordshire: Ayebia Clarke Publishing.

Payne, Thomas E. 2006. \textit{Exploring language structure: A student’s guide}. Cambridge: Cambridge University Press.

Plank, Franz \& Lahiri, Aditi. 2009. Macroscopic and microscopic typology: Basic valence orientation. (Paper presented at the 8th Association of Linguistic Typology conference, University of California, Berkeley, July 23-26, 2009.)

Schachter, Paul. 1974. A non-transformational account of serial verbs. \textit{Studies in African Linguistics} 5. 253-270.

Siewierska, Anna. 2008. Introduction: Impersonalization: An agent-based vs. a subject-based perspective. \textit{Transactions of the Philological Society} 106. 115–137.

Siewierska, Anna. 2011. Overlap and complementarity in reference impersonals: Man-constructions vs. third person plural-impersonals in the languages of Europe. In Malchukov, Andrej \& Siewierska, Anna (eds.), \textit{Impersonal constructions: A cross-linguistic perspective}, 57-89. Amsterdam: John Benjamins.

van Gelderen, Elly. 2011. Basic valency and language change. (Paper presented at the 9th Association of Linguistic Typology conference, University of Hong Kong, Hong Kong July 21-24, 2011.)


\section*{Abbreviations}
\section*{Acknowledgements}

{\sloppy
\printbibliography[heading=subbibliography,notkeyword=this]
}
\end{document}