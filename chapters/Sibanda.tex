\documentclass[output=paper]{langsci/langscibook} 
\title{The Ndebele applicative construction} 
\author{%
 Galen Sibanda \affiliation{Michigan State University} 
}
% \chapterDOI{} %will be filled in at production
\abstract{
Applicatives have been shown to be generally valence adding, and the Ndebele applicative construction is no exception. While change in argument structure is discussed in the article, the main focus is on the variation in thematic roles of the participants that the applicative introduces. The thematic roles associated with a given Ndebele verb are predictable from the semantic class of that verb, including those of participants introduced by applicative \textit{-el}. A number of different semantic classes are considered. The discussion raises theoretical questions about verb meaning and that of the Ndebele applicative suffix \textit{-el}. The precise meaning of \textit{-el} can be best captured by appealing to the notion of underspecification. The thematic roles \textit{-el} introduces in all semantic verb classes are \textsc{reason} and \textsc{location}, and not always \textsc{beneficiary} which has often been highlighted in other studies of applicatives. 
}

\maketitle
\begin{document}
% Key words: Ndebele, applicative, semantic classes, semantic role, thematic role, underspecification, argument structure
\section{Introduction}\label{sec:sibanda:1}
\todo{check sectioning, probably there are some subsections which have not been recognized}
There have been many studies of applicative constructions, covering a wide range of languages around the world (e.g. \citealt{Peterson2007}) including Bantu languages (e.g. \citealt{Ngonyani1996a,Mabugu2011,Jerro2016} (\& this volume). Of particular interest in most of these studies has been how the argument structure of the verb is altered in applicative constructions. Applicatives have been shown to be generally valence adding, and the Ndebele applicative construction considered here is no exception. 

 While change in Ndebele argument structure is discussed in this article, specifically in \sectref{sec:sibanda:3}, the main focus is on the variation in the thematic roles of participants that the applicative introduces. A number of verbs in different semantic classes are considered in \sectref{sec:sibanda:4} although neither the list of verbs nor semantic classes is exhaustive. As such, this work does not provide statistics, but the goal is to establish some generalizations about the thematic roles of participants introduced by the applicative suffix and to show how these relate to semantic classes of verbs. The semantic classification of verbs follows from the works of, for example, \citet{Chafe1970,Dowty1987,Dowty1991,FoleyVanValin1984}, and \citet{Payne1997}. Thematic roles appealed to are based on the works of \citet{Frawley1992,Fillmore1968,Fillmore1977} and \citet{Halliday1970}, among others. The discussion raises important questions about verb meaning and the precise meaning of the applicative suffix \textit{-el} (normally followed by default final vowel \textit{-a} or another suffix). \sectref{sec:sibanda:5} is the conclusion.

\section{Background}\label{sec:sibanda:}

The discussion in this chapter revolves around three key issues: thematic roles, semantic classes and applicative constructions. It is therefore necessary to provide some background to these before delving into the main investigation of the Ndebele applicative construction. 

\section{Thematic roles}\label{sec:sibanda:}

Thematic role lists date back to the work of the Sanskrit grammarian, Panini, in the 6\textsuperscript{th} century BCE (\citealt[548]{Dowty1991}; \citealt[19-20]{Srikumar2013}), and structuralists such as \citet{Blake1930}. They were brought to prominence in linguistic theory by \citet{Gruber1965,Fillmore1968,Fillmore1977}, and \citet{Jackendoff1972,Jackendoff1976}. Commonly known as \textit{semantic roles}, thematic roles have also been referred to by other names. For instance, Panini called them \textit{karakas} (\citealt[548]{Dowty1991}; \citealt[19-20]{Srikumar2013,Cardona1974}); \citet{Fillmore1968} called them semantic \textit{cases} or \textit{deep cases}; \citet{Gruber1965,Gruber1976} and \citet{Jackendoff1972} used the term \textit{thematic relations}; for \citet{Stowell1981} they were \textit{theta-grids}; and for yet others \textit{theta roles} \citep{Chomsky1981,Marantz1984}. 

While the significance of thematic roles in linguistic theory, particularly in studies concerned with the syntax-semantics interface, cannot be refuted, identifying or labeling thematic roles has had challenges, traces of which may still be evident in this article. Some of the important issues are discussed in the works of, for example, \citet{Dowty1991}, \citet{Jackendoff1987}, \citet[689]{Newmeyer2010} and \citet[6]{KittilaEtAl2011}. The number of proposed roles varies from a few to an almost limitless list. For instance, in his localist approach, \citet{Anderson1971} proposed only three roles (\textsc{source, location} and \textsc{goal}), and all non-local values then derive from these. In his original work, \citet{Fillmore1968} claimed that thematic roles (what he called \textit{cases}) formed a finite set including \textsc{agentive, instrumental, dative, factitive, locative} and \textsc{objective,} although he made it clear this was not a complete list. With more research the number later grew to over two thousand roles, also referred to as \textit{Frame Elements} \citep{Fillmore1985}. \citet{Blake1930} listed 87 temporal or locative roles and 26 other roles. To aggravate the situation, as \citet[548-549]{Dowty1991} observes, “new candidates for thematic roles are being proposed all the time, e.g. \textsc{figure} and \textsc{ground} in \citealt{Talmy1985a}, \textsc{neutral} in \citealt{Rozwadowska1988}, \textsc{landmark} in \citealt{Jackendoff1982}, even \textsc{subject} in \citealt{Baker1985}.” Perhaps the most extreme case which, however, seems to have been ignored by many, is the HPSG one where each verb assigns its own peculiar thematic roles, different from the roles of any other verb \citep{PollardSag1994}. In HPSG the verb \textit{love} would, for example, assign two thematic roles: \textsc{lover} and \textsc{lovee}. As \citet{Dowty1989} notes, in this approach, there would be no thematic role types but only individual thematic roles, and important semantic generalizations are lost. \citet[52]{Payne1997} hammers the same point in stating that “an infinitely long list of sematic roles is as useless as no list at all”.

A related problem is the issue of delimiting boundaries between roles. Depending on theoretical approach, some roles are further subdivided into more specific roles; different names are sometimes used for the same role concept; and definitions of some roles overlap. For instance, \citet[553]{Dowty1991} illustrates “role fragmentation” with \textsc{agent,} which he says has been divided into \textsc{agent} and \textsc{actor} \citep{Jackendoff1983}, \textsc{agent} and \textsc{effector} (van \citealt{Valin1990}), \textsc{volitive}, \textsc{effective}, \textsc{initiative}, and \textsc{agentive} \citep{Cruse1973}, while \citet{Lakoff1977} proposes up to fourteen different characteristics. As another instance, the role called \textsc{direction} (‘towards/away from’) is sometimes used as a cover term for \textsc{source} and \textsc{goal.} \citet{Anderson1977} uses \textsc{theme} for what is now widely taken to be a \textsc{patient}, and others use \textsc{patient} and \textsc{theme} interchangeably. The \textsc{patient} versus \textsc{theme} issue partly reflects a definitional problem (\citealt[548-549]{Dowty1991}; \citealt[113]{Lobner2002}). \citet[5]{PalmerEtAl2010} explain that while it is difficult to draw clear boundaries between \textsc{patients} and \textsc{themes,} the commonly held view is that a \textsc{patient} undergoes a change of state whereas a \textsc{theme} simply changes location.

Another problem is that some participants have been claimed to take more than one role. Following \citet{CulicoverWilkins1986} and \citet{Talmy1985}, \citet[395]{Jackendoff1987} proposes two tiers where \textsc{agent} and \textsc{patient} occupy the “action tier” while other roles dealing with motion and location (e.g. \textsc{source, theme, goal}) occupy the “thematic tier”, such that a single participant could have both \textsc{patient} and \textsc{goal} roles but on different tiers, or both \textsc{patient} and \textsc{theme}, etc.

 Observing all these problems, \citet{Dowty1991} moves away from positing many roles and proposes defining Prototypical \textsc{agents} and \textsc{patient}\textsc{s} such that each one covers an array of different finer types. He argues that thematic roles should not be viewed as semantic primitives or discrete categories, but must be defined in terms of entailments so that they are seen as prototypes where there may be different degrees of membership. Thus some roles will be more agent-like or patient-like depending on the number of \textsc{agent} or \textsc{patient} Proto-role properties they fulfill. This would seem to solve the problems of “role fragmentation” and boundaries. However, his proposal has also been criticized \citep{LevinHovav1996}.

Another criticism regarding thematic role lists, especially when they first gained popularity, was that they were often unstructured. As a result they generally could not capture important commonalities and differences across roles. To address this issue, a number of hierarchies, many of which make reference to animacy, frequency in the world’s languages, and subjects and objects, have since been proposed (\citealt{Fillmore1968};  \citealt[12]{SaintDizierViegas1995};    \citealt[334]{KiyosawaGerdts2010}), \citet{BresnanKanerva1989}, though they often differ in their details. Of relevance to this study, Bentley (1994: vii) mentions that thematic role hierarchies have sometimes been used to explain differences in the behavior of objects, as well as the relative prominence of arguments in events. \citet[129]{Mchombo2004}, for example, has argued for the thematic hierarchy in \REF{ex:sibanda:3} which attempts to explain the morphosyntactic behavior of different applied objects. (See also \citealt{NgonyaniGithinji2006}.)

\ea
\textsc{agent > beneficiary > goal/experiencer > instrument > patient/theme > location > malefactive > circumstantial}\\
\z

The ordering of roles in this hierarchy can explain why a \textsc{beneficiary} and a \textsc{circumstantial}, for example, take different object positions in a Bantu sentence. Generally, roles on the left of the hierarchy, often associated with animate objects, surface closer to the verb while those to the right are more peripheral and are usually associated with inanimate objects. It will be shown in the next sections that Ndebele does not depart much from this.

 Despite the criticisms noted above, thematic roles remain necessary in capturing important semantic and syntax-semantics generalization, including the behavior of applicative constructions. However, clear definitions of semantic roles are necessary before getting into detailed discussion. The definitions of roles introduced by the Ndebele applicative \textit{-el} and others used in this study (some of which are combinations of definitions) and their sources are provided in \tabref{tab:sibanda:1}.\footnote{One semantic notion associated with the Ndebele applicative construction, but largely left out of the discussion, is ‘in the presence of’ or ‘witnessed by.’ Sometimes this notion has been mistaken for a \textsc{beneficiary} or \textsc{malaficiary}, but the (un)fortunatness of a situation is sometimes simply implied by the verb root itself. An example is \textit{Umntwana uyangigul}\textbf{\textit{el}}\textit{a,} literally ‘The child is sick “for me”’ where ‘me’ is clearly not \textsc{beneficiary} as no one can be sick on behalf of another person. The sentence should be interpreted as ‘The child is sick in my presence’. That this is to my detriment can be inferred from the verb \textit{gula} ‘be sick’ itself, not from the applicative suffix \textit{-el}.} 

%%please move \begin{table} just above \begin{tabular
\begin{table}
\caption{Definitions of thematic roles}
\label{tab:sibanda:1}
\small
\begin{tabularx}{\textwidth}{>{\scshape}lXp{2cm}}
\lsptoprule
 \textup{Thematic Role} & {Definition} & {Reference}\\
\midrule
 {{agent (a)}} & {animate and volitional perceived instigator/initiator of the action or event} & {\citet{Fillmore1968}; \citet[49]{Payne1997}}\\
\tablevspace
 {{beneficiary (b)}} & {participant for whose benefit the action expressed by the verb is performed} & {\citet[11]{SaintDizierViegas1995}; \citet[4]{PalmerEtAl2010}}\\
\tablevspace
 {{emitter (em)}} & {entity that produces or emits a sound, smoke, fumes, gas, dust, etc.} & \\
\tablevspace
 {{experiencer (e)}} & {(animate) participant that is informed of something or that experiences perception, feeling or some psychological state expressed by the predicate (e.g. first argument of \textit{love}, second argument of \textit{annoy})} & {\citet[11]{SaintDizierViegas1995}; \citet[113]{Lobner2002}}\\
\tablevspace
 {{goal (g)}} & {entity towards which a movement is directed (e.g. second argument of \textit{reach}, \textit{arrive}), or the motivation of an action} & {\citet[11]{SaintDizierViegas1995}}\\
\tablevspace
 {{location (l)}} & {place in which the action or state described by the predicate takes place (e.g. second argument of \textit{fall})} & {\citet[11]{SaintDizierViegas1995}}\\
\tablevspace
 {{maleficiary (m)}} & {participant to whose detriment the action expressed by the verb is performed}  & \citep[5]{KittiläZúñiga2010}\\
\tablevspace
 {{patient (p)}} & {participant undergoing the action and that is affected by it – usually undergoes a physical, visible change in state (e.g., second argument of \textit{kill}, \textit{eat})}  & {Saint-Dizier \& \citet[11]{Viegas1995}; \citet[4]{PalmerEtAl2010}; \citet[51]{Payne1997}}\\
\tablevspace
 {{reason (r)}} & {motivational source of a predication or event} & {\citet[225]{Frawley1992}}\\
\tablevspace
 {{stimulus (sti)}} & {causer of an emotional reaction} & {\citet[13]{PalmerEtAl2010}; \citet{Dowty1991}}\\
\tablevspace
 {{theme (t)}} & {entity that is moving or changing location, condition, or state or being in a given state or position (e.g. the second argument of \textit{give}, the argument of \textit{walk}, \textit{die})} & {Saint-Dizier \& \citet[11]{Viegas1995}; \citet[4]{PalmerEtAl2010}}\\
\lspbottomrule
\end{tabularx}
\end{table}

\section{Semantic verb classes}\label{sec:sibanda:}

Verb classes structure the lexicon. One of the most influential studies is Levin’s (1993) study of English verbs, which is based on syntactic alternations. Her investigation shows correlations between some aspects of the semantics and the syntactic behavior of English verbs. Similar studies have been done for Spanish \citep{VazquezEtAl2000} and German \citep{Schumacher1986,SchulteimWaldeBrew2002}. Other approaches to identifying semantic verb classes include elements of Lexical Conceptual Structure \citep{Gruber1967,Jackendoff1983,Jackendoff1990} and sematic roles \citep{Chafe1970,Cook1979,Longacre1976,FoleyVanValin1984,VanValin1993}. No attempt will be made in this paper to identify Ndebele verb semantic classes using such methods, but this study utilizes some classes proposed elsewhere in the literature.

\section{Applicative construction} \label{sec:sibanda:}

Studies of Bantu applicative constructions have arrived at some interesting conclusions on a number of issues such as object symmetry, the \textsc{beneficiary} thematic role, animacy and thematic hierarchies. 

Most studies of applicative constructions in Bantu languages have concentrated on syntactic properties, especially regarding the behavior of the applied object (\textsc{ao}) versus that of the base or “logical object” (\textsc{lo}). Of particular interests has been the comparison between symmetrical and asymmetrical type languages often involving the subjection of the objects to various syntactic tests such as passivization, object agreement and word order have been applied to distinguish them \citep{BresnanMosh1990,NgonyaniGithinji2006,Pylkkanen2000,Machobane1989,AlsinaMchombo1993,Baker1988,Harford1993,Ngonyani1996}. While both object types generally display “true” or “primary object” syntactic properties in a symmetrical type language, only one of the two objects behaves like a “true” or “primary” object in an asymmetrical language. Only a few studies have paid more attention to the semantics of the applicative suffix. 

One of the main semantic questions has concerned the issue of \textsc{benefactive} versus other thematic roles. In particular, studies of applicative constructions have claimed that the \textsc{beneficiary} role is the one most commonly associated with the applied object cross-linguistically (e.g., \citealt{Peterson2007,Polinsky2008,KiittilaZuniga2010}). \citet[40]{Peterson2007} concludes that “if a language has a construction which could be characterized as an applicative it is most common that the semantic role of the applicative object will be that of a \textsc{recipient} and/or \textsc{beneficiary}/\textsc{maleficiary.”} While acknowledging the presence of other roles of the \textsc{ao} such as \textsc{recipient}, \textsc{maleficiary}, \textsc{reason} and \textsc{instrument}, \citet{Schadeberg2003}, \citet[101]{deKindBostoen2012} and  \citet[1]{MartenKula2014} also view the \textsc{beneficiary} as the most widespread and productive role associated with the applied object in Bantu. 

Also regarding semantics, a number of studies have been concerned with tracing the original or underlying meaning associated with the applicative suffix. For instance, \citet[3]{CannMabhugu2007} argue that in Shona “all the primary meanings associated with [the] applicative suffix can be derived from an underspecified generalized \textsc{goal} relation.”   \citet{DeKindBostoen2012} also argue for an underlying \textsc{goal} function of the applicative in \textit{ciLubà}. According to \citet{Trithart1983}, earlier scholars such as \citet{Endemann1876,vanEeden1956} and  \citet{KahlerMayer1966} proposed an original \textsc{locative} use of the applicative, a view also taken by \citet[74]{Schadeberg2003}. However, \citet[75]{Trithart1983} suggests an original \textsc{benefactive} function of the applicative in Bantu and in Niger-Congo languages in general. 

While all these observations might be true to some extent, the different claims may be due to the fact that these studies have not systematically analyzed the behavior of verbs from different semantic classes in the applicative construction. Although semantic classes are sometimes mentioned, in most studies there has been no clear demonstration that the conclusions have been arrived at after examining verbs from different semantic classes, rather than just choosing commonly used or random verbs. As will be illustrated with Ndebele examples below, the applicative construction behaves differently with verbs from different semantic classes. It will be shown that with many classes, the \textsc{beneficiary} does not feature at all. Thus, blanket statements like those cited above about \textsc{beneficiary} in applicative constructions are inaccurate, and they may be valid only with regards to particular semantic verb classes.

\section{General properties of the Ndebele applicative construction}\label{sec:sibanda:}

As already alluded to, the Ndebele Applicative is marked by the derivational suffix \textit{\textit{-el}} plus the verbal default final vowel \textit{-a} or another derivational or inflectional suffix. It is valence adding, as illustrated in \REF{ex:sibanda:2} with a divalent verb root \textit{phek} ‘cook’ which normally takes only one object. 

\ea\label{ex:sibanda:2}
\settowidth\jamwidth{\textsc{a p (l)}}
{Divalent verb root} \\
\ea{\label{ex:sibanda:2a}
\gll U-mama u-ø-phek-a i-lambazi.\\
 1a-mother 1a-\textsc{tns}-cook-a 5-porridge\\\jambox{\textsc{a p}} 
}
\glt ‘Mother is cooking porridge.’

\ex{\label{ex:sibanda:2b}
\gll U-mama u-ø-phek-\textbf{el}-a um-ntwana i-lambazi. \\
 1a-mother 1a-\textsc{tns}-cook-\textsc{app}-a 1-child 5-porridge\\\jambox{\textsc{a b p}} 
}
\glt ‘Mother is cooking the child porridge.’

\ex{\label{ex:sibanda:2c}
\gll U-mama u-ø-phek-\textbf{el}-a i-lambazi pha-ndle.\\
 1a-mother 1a-\textsc{tns}-cook-\textsc{app}-a 5-porridge 16-outside\\\jambox{\textsc{a p l}}
}
\glt ‘Mother is cooking the porridge outside.’

\ex{\label{ex:sibanda:2d}
\gll U-mama u-ø-phek-\textbf{el}-a i-lambazi in-dlala.\\
 1a-mother 1a-\textsc{tns}-cook-\textsc{app}-a 5-porridge 9-hunger\\ \jambox{\textsc{a p r}} 
}
\glt ‘Mother is cooking the porridge due to hunger.’

\ex{\label{ex:sibanda:2e}
\gll *U-mama u-ø-phek-\textbf{el}-a i-lambazi.\\
 1a-mother 1a-\textsc{tns}-cook-\textsc{app}-a 5-porridge \\ \jambox{\textsc{a p}} 
}
\glt ‘Mother is cooking the porridge for/at …’

\ex{\label{ex:sibanda:2f}
\gll U-mama u-ø-phek-a i-lambazi (pha-ndle).\\
 1a-mother 1a-\textsc{tns}-cook-a 5-porridge (16-outside)\\ \jambox{\textsc{a p (l)}} 
}
\glt ‘Mother is cooking the porridge (outside).’
\z
\z

As seen in \REF{ex:sibanda:2a}, a root such as \textit{phek} ‘cook’ only requires an \textsc{agent} (A) and a \textsc{patient} (P) as arguments. However, suffixing \textit{-el}, as in \REF{ex:sibanda:2b}, introduces a new argument with a \textsc{beneficiary} (B) participant role and the verb ends up with two objects. Note that the \textsc{beneficiary} intepretation can be replaced by a \textsc{maleficiary} one if a different root such as \textit{w} ‘fall’ or \textit{chem} ‘urinate’ is used, or if the pragmatic situation yields a negative interpretation. The applicative versions of these two roots would most likely be interpreted as \textit{wel} ‘fall on’ or \textit{chemel} ‘urinate on’, not ‘fall for (on behalf of)’ or ‘urinate for (on behalf of)’. Thus wherever reference is made to a \textsc{beneficiary} role, there is need to keep in mind that a \textsc{maleficiary} interpretation would most likely apply in the same situation if a different verb encoding misfortune is used or if the same verb is used in a negative pragmatic context; hence it is not necessary to discuss the \textsc{maleficiary} interpretation further in this article. 

 The same applicative can introduce a \textsc{locative} (L) sematic role, as in \REF{ex:sibanda:2c}. While an unapplicativized form of the verb can optionally take the \textsc{locative}, as in \REF{ex:sibanda:2f} so that the sentence carries roughly the same meaning, the \textsc{locative} in \REF{ex:sibanda:2c} behaves as a core argument since its omission makes the sentence incomplete or ungrammatical. It should be noted that while the \textsc{beneficiary} must immediately follow the verb, the \textsc{locative} comes after the \textsc{patient}\textit{,} perhaps reflecting a thematic hierarchy: A > B > P > L. 

 The core argument introduced by the applicative can also be \textsc{reason} (R), as in \REF{ex:sibanda:6d}. Like an applied \textsc{locative}, an applied \textsc{reason} behaves as a core argument since it cannot be left out without making the sentence ungrammatical. 

 Example \REF{ex:sibanda:2e} illustrates that suffixing the applicative \textit{-el} without introducing a third argument (whose role may be \textsc{beneficiary}, \textsc{locative} or \textsc{reason}) is ungrammatical. The only exception would be a case of “definite null instantiation”, i.e. an identifiable referent expressed by zero \citep{Fillmore1986,FillmoreKay1999}, where the second of the two overt arguments of the applicativized verb would be a \textsc{beneficiary}, \textsc{locative} or \textsc{reason}, not a \textsc{patient}.

 The facts are similar with regards to intransitive monovalent verb roots such as \textit{khal} ‘cry’, a sound emission root, illustrated in \REF{ex:sibanda:3}. 

\ea\label{ex:sibanda:3}
\settowidth\jamwidth{\textsc{a p (l)}}
{Monovalent verb root with \textsc{emitter} (Em) role}\\
 (Similar roots: \textit{dum} ‘make a sound, thunder’, \textit{lil} ‘cry, moan’, \textit{bhong} ‘roar’ \textit{bhons} ‘low (of cattle)’ \textit{bubul} ‘groan’, \textit{bovul} ‘bellow’, \textit{klabalal} ‘shout loudly’)\\
\ea{\label{ex:sibanda:3a}
\gll U-sane lu-ya-khal-a.\\
 11-baby 11-\textsc{tns}-cry-a \\\jambox{\textsc{em}}
}

\glt ‘The baby is crying.’

\ex{\label{ex:sibanda:3b}
\gll U-sane lu-ø-khal-\textbf{el-}a u-chago. \\
 11-baby 11-\textsc{tns}-cry-\textsc{app}-a 11-milk\\\jambox{\textsc{em r}}
}

\glt ‘The baby is crying for milk.’ 

\ex{\label{ex:sibanda:3c}
\gll U-sane lu-ø-khal-\textbf{el-}a pha-ndle.\\
 11-baby 11-\textsc{tns}-cry-\textsc{app}-a 16-outside\\\jambox{\textsc{em l}}

}
\glt ‘The baby is crying outside.’


\ex{\label{ex:sibanda:3d}
\gll *U-sane lu-ø-khal-\textbf{el-}a \\
 11-baby 11-\textsc{tns}-cry-\textsc{app}-a \\ \jambox{\textsc{em}} 
}
\glt ‘The baby is crying for/at…’
\z
\z

Example \REF{ex:sibanda:3a} shows that \textit{khal} ‘cry’ has only one argument, realized as the subject and with an \textsc{emitter} (\textsc{em}) sematic role. When the verb is applicativized, a new argument is introduced and its sematic role can be \textsc{reason} as in \REF{ex:sibanda:3b} or \textsc{location} as in \REF{ex:sibanda:3c}. Without a second argument the applicativized verb is ungrammatical, as shown in \REF{ex:sibanda:3d}.

 An additional argument is required even when a three-place verb root, such as \textit{ph} ‘give’ is applicativized, as exemplified in \REF{ex:sibanda:4}. 

\ea\label{ex:sibanda:4}
\settowidth\jamwidth{\textsc{a p (l)}}
{Trivalent verb root}\\
\ea{\label{ex:sibanda:4a}
\gll U-Sihle u-ø-pha u-sane uku-dla. \\
 1a-Sihle 1a-\textsc{tns}-give 11-baby 15- food \\\jambox{\textsc{a g t}}

}
\glt ‘Sihle is giving the baby food.’

\ex{\label{ex:sibanda:4b}
\gll U-Sihle u-ø-ph-\textbf{el}-a u-mama u-sane uku-dla. \\
 1a-Sihle 1a-\textsc{tns}-give-\textsc{app}-a 1a-mother 11-baby 15- food \\\jambox{\textsc{a b g t}} 
}
\glt ‘Sihle is giving the baby food for mother.’

\ex{\label{ex:sibanda:4c}
\gll U-Sihle u-ø-ph-\textbf{el}-a u-sane uku-dla pha-ndle \\
 1a-Sihle 1a-\textsc{tns}-give-\textsc{app}-a 11-baby 15- food 16-outside\\\jambox{\textsc{a g t l}}

}
\glt ‘Sihle is giving the baby food outside.’

\ex{\label{ex:sibanda:4d}
\gll U-Sihle u-ø-ph-\textbf{el}-a u-sane uku-dla in-dlala \\
 1a-Sihle 1a-\textsc{tns}-give-\textsc{app}-a 11-baby 15- food 9-hunger\\\jambox{\textsc{a g t r}}
} 
\glt ‘Sihle is giving the baby food for hunger.’

\ex{\label{ex:sibanda:4e}
\gll *U-Sihle u-ø-ph-\textbf{el}-a u-sane uku-dla \\
 1a-Sihle 1a-\textsc{tns}-give-\textsc{app}-a 11-baby 15- food \\\jambox{\textsc{a g t}}

}
\glt ‘Sihle is giving the baby food for/at …’
\z
\z


The three arguments of the verb root \textit{ph} ‘give’ are normally associated with the roles \textsc{agent}, \textsc{goal} (G) and \textsc{theme} (T), as in \REF{ex:sibanda:4a}. When the applicative \textit{-el} is suffixed, the additional argument may be \textsc{beneficiary} \REF{ex:sibanda:8b}, \textsc{location} \REF{ex:sibanda:4c}, or \textsc{reason} \REF{ex:sibanda:4d}. Leaving out the fourth argument is unacceptable \REF{ex:sibanda:4e}. Note that although nouns are used for \textsc{reason} in \REF{ex:sibanda:2}-\REF{ex:sibanda:4}, these can be replaced by a phrase or clause beginning with \textit{ukuthi…/ukuba…/ukuze …} ‘because …; so that ….’ In fact, out of context \textsc{reason} is usually expressed more clearly with such a phrase. Ignoring the \textsc{emitter} which might, perhaps, be ranked high like the \textsc{agent}, the thematic hierarchy drawn from the examples above and based on the relative prominence of arguments in events (the more prominent occurring in the subject position, or if they are objects, closer to the applicativized verb stem) can now be hypothesized as A > B > G > T/P > L/R. 

\section{The applicative with verbs in different semantic classes}\label{sec:sibanda:}

It is clear from the example sets in preceding sections that suffixing \textit{-el} always entails introducing a new argument to the clause. However, the sematic role of the new argument varies from verb to verb. This section looks at what is predictable in this semantic variation.

\section{Verbs of motion} \label{sec:sibanda:}

Verbs of motion are considered first. \citet[171]{Frawley1992} notes that motion involves either positional change or the displacement of some entity and that a complete semantic characterization of motion events “requires the specification of eight semantic properties in addition to displacement itself.” These properties are captured in the meaning of the roles \textsc{theme}, \textsc{source}, \textsc{goal}, \textsc{path} including direction, \textsc{site} and medium, \textsc{instrument} or conveyance, \textsc{manner}, and \textsc{agent}. In \REF{ex:sibanda:5} we see what happens when applicative \textit{-el} is suffixed to monovalent voluntary motion verb roots. (Although an instrument can indeed be associated with motion in Ndebele, it is not introduced by the applicative suffix but by a \textit{nga-} ‘by/with’ phrase.)

\ea\label{ex:sibanda:5}
\settowidth\jamwidth{\textsc{a p (l)}}
{Monovalent verb root + voluntary motion \textit{gijim} ‘run’}\\
 (Similar roots: \textit{hamb} ‘walk, move’, \textit{tshitsh} ‘walk fast’, \textit{phaph} ‘fly’, \textit{khas} ‘crawl’, \textit{eq} ‘jump’, \textit{ntshez} ‘swim’)\\
 
\ea{\label{ex:sibanda:5a}
\gll U-Themba u-ya-gijim-a.\\
 1a-Themba 1a-\textsc{tns}-run-a \\\jambox{\textsc{t}}
}

\glt ‘Themba is running.’

\ex{\label{ex:sibanda:5b}
\gll U-Themba u-ø-gijim-\textbf{el-}a u-nina. \\
 1a-Themba 1a-\textsc{tns}-run-\textsc{app}-a 1a-mother\\\jambox{\textsc{t b}}
}

\glt ‘Themba is running for his mother.’ 

\ex{\label{ex:sibanda:5c}
\gll U-Themba u-ø-gijim-\textbf{el-}a u-nina. \\
 1a-Themba 1a-\textsc{tns}-run-\textsc{app}-a 1a-mother\\\jambox{\textsc{t g}}
}

\glt ‘Themba is running to his mother’

\ex{\label{ex:sibanda:5d}
\gll U-Themba u-ø-gijim-\textbf{el-}a e-nkundleni. \\
 1a-Themba 1a-\textsc{tns}-run-\textsc{app}-a \textsc{loc}-stadium\\\jambox{\textsc{t g}}
}

\glt ‘Themba is running to the stadium’

\ex{\label{ex:sibanda:5e}
\gll U-Themba u-ø-gijim-\textbf{el-}a u-nina. \\
 1a-Themba 1a-\textsc{tns}-run-\textsc{app}-a 1a-mother\\\jambox{\textsc{t r}}
}

\glt ‘Themba is running because of his mother’

\ex{\label{ex:sibanda:5f}
\gll U-Themba u-ø-gijim-\textbf{el-}a e-nkundleni. \\
 1a-Themba 1a-\textsc{tns}-run-\textsc{app}-a \textsc{loc}-stadium\\\jambox{\textsc{t l}}
}

\glt ‘Themba is running in the stadium’
\z
\z

Verbs of voluntary motion with monovalent verb roots show that the applicative may introduce the \textsc{beneficiary} \REF{ex:sibanda:5b}, \textsc{goal} (\ref{ex:sibanda:5c}-d), \textsc{reason} \REF{ex:sibanda:5e} or \textsc{location} \REF{ex:sibanda:5f}. In \REF{ex:sibanda:5} \textit{Themba} is treated as a \textsc{theme}, not an \textsc{agent} because he is definitely the entity that is moving or changing location. While he is animate and can act with volition like an \textsc{agent,} it is not clear if he is the initiator of the running and whether or not he is actually acting volitionally. In \REF{ex:sibanda:5d} and \REF{ex:sibanda:5f} the prefix \textit{e-} is traditionally treated as a locative marker but \REF{ex:sibanda:5d} shows that the location can also be a \textsc{goal} that an object moves towards.

 Verbs of involuntary motion with monovalent verb roots, exemplified in \REF{ex:sibanda:6}, show that the applicative introduces the same thematic roles as in \REF{ex:sibanda:5} except for the \textsc{beneficiary,} as none of the actions implied by each of the verbs in this subclass can be done on behalf of another person or thing.

\ea\label{ex:sibanda:6}
\settowidth\jamwidth{\textsc{a p (l)}}
{Monovalent verb root + involuntary motion \textit{balek} ‘flee’}\\
 (Similar roots: \textit{-w} ‘fall, drop’, \textit{gelez} ‘flow’, \textit{ntshaz} ‘squirt’)\\
\ea{\label{ex:sibanda:6a}
\gll U-Themba u-ya-baleka-a. \\
 1a-Themba 1a-\textsc{tns}-flee-a \\\jambox{\textsc{t}}
}

\glt ‘Themba is fleeing.’


\ex{\label{ex:sibanda:6b}
\gll U-Themba u-ø-balek-\textbf{el-}a in-yoka. \\
 1a-Themba 1a-\textsc{tns}-flee-\textsc{app}-a 9-snake\\\jambox{\textsc{t r}}
}

\glt ‘Themba is fleeing from the snake.’ 

\ex{\label{ex:sibanda:6c}
\gll U-Themba u-ø-balek-\textbf{el-}a e-ndlini. \\
 1a-Themba 1a-\textsc{tns}-flee-\textsc{app}-a \textsc{loc}-house\\\jambox{\textsc{t g}}
}

\glt ‘Themba is fleeing into the house.’

\ex{\label{ex:sibanda:6d}
\gll U-Themba u-ø-balek-\textbf{el-}a e-ndlini. \\
 1a-Themba 1a-\textsc{tns}-run-\textsc{app}-a \textsc{loc}-house\\\jambox{\textsc{t l}}
}

\glt ‘Themba is fleeing in the house.’
\z
\z

Themba is again here treated as a \textsc{theme} for the same reasons as in \REF{ex:sibanda:5}, and the locative prefix can still introduce a \textsc{goal} \REF{ex:sibanda:6c}.

 Possible thematic roles added by the applicative are even fewer with verbs of motion whose roots are divalent, as seen in \REF{ex:sibanda:7} where only \textsc{beneficiary} and \textsc{reason} are permissible. 

\ea\label{ex:sibanda:7}
\settowidth\jamwidth{\textsc{a p (l)}}
{Verb of motion with divalent verb root \textit{y} ‘go to’} \\
 (Similar roots: \textit{z} ‘come’, \textit{suk} ‘depart, leave’, \textit{fik} ‘arrive’)\\
\ea{
\gll U-Themba u-ø-ya e-sitolo. \\
 1a-Themba 1a-\textsc{tns}-go-a \textsc{loc}-store\\\jambox{\textsc{t g}}
}

\glt ‘Themba is going to the store.’

\ex{
\gll U-Themba u-ø-y-\textbf{el-}a{\rmfnm} u-mama e-sitolo. \\
 1a-Themba 1a-\textsc{tns}-go-APP-a 1a-mother \textsc{loc}-store\\\jambox{\textsc{t b g}}
}

\glt ‘Themba is going to the store for mother.’

\ex{
\gll U-Themba u-ø-y-\textbf{el-}a u-mama e-sitolo / e-sitolo u-mama.\\
 1a-Themba 1a-\textsc{tns}-go-APP-a 1a-mother \textsc{loc}-store / \textsc{loc}-store 1a-mother\\\jambox{\textsc{t r g / g r}}
}

\glt ‘Themba is going to the store for (because of) mother.’

\ex{
\gll U-Themba u-ø-y-\textbf{el-}a uku-sebenza e-sitolo / e-sitolo uku-sebenza \\
 1a-Themba 1a-\textsc{tns}-go-APP-a 15-work \textsc{loc}-store / \textsc{loc}-store 15-work\\\jambox{\textsc{t r g / g r}}
}

\glt ‘Themba is going to the store (in order) to work.’
\z
\z

\footnotetext{I do not include the ‘defecate on/at’ metaphorical meaning of yela.}

Each of these motion verbs with divalent roots already has a \textsc{goal} or \textsc{source} role as a base argument. The root \textit{suk} ‘depart’ is one example with a \textsc{source} rather than a \textsc{goal}.

\section{Verbs of surface contact through motion}\label{sec:sibanda:}

Another interesting verb class is that of surface contact through motion. Some verbs in this class have more than one base argument frame as their roots may subcategorized for a noun or locative object (besides a phrase beginning with \textit{ukuze/ukuthi} ‘so that’). For most verbs in this category, \textit{-el} introduces a \textsc{beneficiary}, \textsc{goal}, \textsc{reason} or \textsc{location} if the object of the unapplicativized verb is expressed as a noun, as illustrated in \REF{ex:sibanda:8}. 

\ea\label{ex:sibanda:8}
\settowidth\jamwidth{\textsc{a p (l)}}
{Divalent or trivalent With noun object}\\
 (Similar roots: \textit{esula} ‘wipe’, \textit{sunduz} ‘push’, \textit{fuq} ‘push’, \textit{dons} ‘pull’, \textit{nind} ‘smear’, \textit{gcob} ‘smear’)\\
\ea{\label{ex:sibanda:8a}
\gll U-Musa u-ø-thanyel-a izi-bi. \\
 1a-Musa 1a-\textsc{tns}-sweep-a 8-trash\\\jambox{\textsc{a t}}
}

\glt ‘Musa is sweeping the trash.’

\ex{\label{ex:sibanda:8b}
\gll U-Musa u-ø-thanyel-\textbf{el-}a u-nina izi-bi. \\
 1a-Musa 1a-\textsc{tns}-sweep-\textsc{app}-a 1a-mother 8-trash\\\jambox{\textsc{a b t}}
}

\glt ‘Musa is sweeping the trash for her mother.’

\ex{
\gll U-Musa u-ø-thanyel-\textbf{el-}a u-nina izi-bi. \\
 1a-Musa 1a-\textsc{tns}-sweep-\textsc{app}-a 1a-mother 8-trash\\\jambox{\textsc{a g t}}
}

\glt ‘Musa is sweeping the trash to her mother.’

\ex{
\gll U-Musa u-ø-thanyel-\textbf{el-}a u-nina izi-bi. \\
 1a-Musa 1a-\textsc{tns}-sweep-\textsc{app}-a 1a-mother 8-trash\\\jambox{\textsc{a r t}}
}

\glt ‘Musa is sweeping the trash because of her mother.’

\ex{
\gll U-Musa u-thanyel-\textbf{el-}a izi-bi pha-ndle. \\
 cl.1a-Musa cl.1a-\textsc{tns}-sweep-\textsc{app}-a 8-trash 15-outside\\\jambox{\textsc{a t l}}
}

\glt ‘Musa is sweeping the trash outside.’

\ex{
\gll U-Musa u-thanyel-\textbf{el-}a izi-bi pha-ndle. \\
 cl.1a-Musa cl.1a-\textsc{tns}-sweep-\textsc{app}-a 8-trash 15-outside\\\jambox{\textsc{a t g}}
}

\glt ‘Musa is sweeping the trash (to the) outside.’
\z
\z

Without a derivational suffix such as \textit{–el}, only the verb roots \textit{thanyel} ‘sweep’, \textit{esul} ‘wipe’ and \textit{hlikihl} ‘wipe, scrab’ may subcategorize for a locative object, as shown in \REF{ex:sibanda:9a}. When on a verb root that has a root-determined locative argument, \textit{-el} can only introduce an applied argument with the role of \textsc{beneficiary} or \textsc{reason}(\ref{ex:sibanda:9b}-c). 

\ea\label{ex:sibanda:9}
\settowidth\jamwidth{\textsc{a p (l)}}
{Divalent verb roots with locative object} \\
 (Similar roots: \textit{esula} ‘wipe’ and \textit{hlikihl} ‘wipe, scrab’; null instantiation: \textit{nind} ‘smear’ \& \textit{gcoba} ‘smear’)\\
\ea{\label{ex:sibanda:9a}
\gll U-Musa u-thanyel-a pha-nsi. \\
 1a-Musa 1a-\textsc{tns}-sweep-a 16-down\\\jambox{\textsc{a l}}
}

\glt ‘Musa is sweeping the floor.’

\ex{\label{ex:sibanda:9b}
\gll U-Musa u-ø-thanyel-\textbf{el-}a u-nina pha-nsi.\\
 1a-Musa 1a-\textsc{tns}-sweep-\textsc{app}-a 1a-mother 16-down\\\jambox{\textsc{a b l}}
}

\glt ‘Musa is sweeping the floor for her mother.’

\ex{\label{ex:sibanda:9c}
\gll U-Musa u-ø-thanyel-\textbf{el-}a uku-lala pha-nsi.\\
 1a-Musa 1a-\textsc{tns}-sweep-\textsc{app}-a 15-sleep 16-down\\\jambox{\textsc{a r l}}
}

\glt ‘Musa is sweeping the floor (in order) to sleep.’
\z
\z

Note that roots such as \textit{nind} ‘smear’ and \textit{gcob} ‘smear’ can take a null-instantiated locative object because they are actually three place verbs. Null-instantiation is possible due to the fact that these verb roots are also acceptable with divalent argument frames, although the omitted object, usually with a \textsc{theme} sematic role, can be recovered through logical reasoning. Their normal behavior when \textit{-el} is suffixed is illustrated in \REF{ex:sibanda:10} where \textit{-el} introduces the \textsc{beneficiary}, \textsc{reason} or \textsc{location}. The \textsc{goal} role is determined by the base verb root itself.

\ea\label{ex:sibanda:10}
\settowidth\jamwidth{\textsc{a p (l)}}
{Trivalent root}\\
\ea{\label{ex:sibanda:10a}
\gll U-Musa u-ø-gcob-a u-sana ama-futha. \\
 1a-Musa 1a-\textsc{tns}-smear-a 11-baby 6-oil\\\jambox{\textsc{a g t}}
}

\glt ‘Musa is smearing the oil onto the baby.’/\\
 ‘Musa is smearing the baby with oil.’

\ex{
\gll U-Musa u-ø-gcoba-\textbf{el-}a u-nina u-sana ama-futha. \\
 1a-Musa 1a-\textsc{tns}-smear-\textsc{app}-a 1a-mother 11-baby 6-oil\\\jambox{\textsc{a b g t}}
}

\glt ‘Musa is smearing the oil onto the baby for mother.’ /\\
‘Musa is smearing the baby with oil for mother.’

\ex{
\gll c U-Musa u-ø-gcoba-\textbf{el-}a u-nina u-sana ama-futha. \\
 1a-Musa 1a-\textsc{tns}-smear-\textsc{app}-a 1a-mother 11-baby 6-oil\\\jambox{\textsc{a r g t}}
}

\glt ‘Musa is smearing the oil onto the baby because of mother.’ /\\
 ‘Musa is smearing the baby with oil because of mother.’

\ex{
\gll U-Musa u-ø-gcob-\textbf{el-}a u-sana ama-futha pha-ndle. \\
 1a-Musa 1a-\textsc{tns}-smear-\textsc{app}-a 11-baby 6-oil 15-outside\\\jambox{\textsc{a g t l}}
}

\glt ‘Musa is smearing the oil onto the baby outside’ / \\
‘Musa is smearing the baby with oil outside’ 
\z
\z

Where there is a \textsc{goal,} as in \REF{ex:sibanda:10a}, it is also possible to drop it if that \textsc{goal} and the \textsc{agent} are co-referential (i.e. \textit{UMusa ugcoba uMusa amafutha} is normally expressed as \textit{UMusa ugcoba amafutha ‘}Musa is smearing the oil onto herself‘/‘Musa is smearing herself with oil’). However, applicativization is odd in the absence of the \textsc{goal.}

\section{Verbs of surface contact}\label{sec:sibanda:}

Verbs of surface contact are similar to those of surface contact through motion except that with the former the applicative suffix generally does not introduce the \textsc{goal} sematic role. For most roots in this class, including\textit {mukul} ‘slap’ \textit{wakal} ‘slap’ and \textit{tshay} ‘hit’, the \textsc{goal} is excluded because neither the \textsc{agent} nor \textsc{patient} moves towards any specific entity. Examples are provided in \REF{ex:sibanda:11} with the verb root \textit{tshay} ‘slap’.

\ea\label{ex:sibanda:11}
\settowidth\jamwidth{\textsc{a p (l)}}
{Surface-contact root \textit {tshay} ‘slap’} \\
 (Similar roots: \textit{tshay} ‘hit’, \textit{khab} ‘kick’, \textit{ang} ‘kiss’, \textit{qabul/qabuj/qabuz} ‘kiss’, \textit{mukul, makal/wakal/waqaz} ‘slap’\textit{, bhansul} ‘slap lightly’, \textit{thint} ‘touch’)\\
\ea{
\gll U-Themba u-ø-tshay-a in-yoka. \\
 1a-Themba 1a-\textsc{tns}-hit-a 9-snake\\\jambox{\textsc{a p}}
}

\glt ‘Themba is hitting a snake.’

\ex{
\gll U-Themba u-ø-tshay\textit{-el}-a u-Musa in-yoka. \\
 1a-Themba 1a-\textsc{tns}-hit-\textsc{app}-a 1a-Musa 9-snake\\\jambox{\textsc{a b p}}
}

\glt ‘Themba is hitting the snake for Musa.’ 

\ex{
\gll U-Themba u-ø-tshay\textit{-el}-a in-yoka umu-thi\\
 1a-Themba 1a-\textsc{tns}-hit-\textsc{app}-a 9-snake 4-medicine\\\jambox{\textsc{a p r}}
}

\glt ‘Themba is hitting the snake for medicine.’ 

\ex{
\gll U-Themba u-ø-tshay\textit{-el}-a in-yoka pha-ndle \\
 1a-Themba 1a-\textsc{tns}-hit-\textsc{app}-a 9-snake 16-outside\\\jambox{\textsc{a p l}}
}

\glt ‘Themba is hitting the snake outside.’ 
\z
\z

For a few roots such as \textit{khab} ‘kick’ and \textit{waqaz} ‘slap’, the \textsc{patient} object may be treated as a \textsc{theme} if it is viewed as a moving entity. As seen with the root \textit{khab} ‘kick’ in \REF{ex:sibanda:12e}, the locative object introduced by \textit{-el} is then treated as a \textsc{goal}. Examples \REF{ex:sibanda:12d} and \REF{ex:sibanda:12e} actually show that there are two separate argument framesː \textsc{agent-patient-locative}, and \textsc{agent-theme-goal}. It seems we get a \textsc{theme} and \textsc{goal} reading in \REF{ex:sibanda:12e} because such verbs have dual membership. They also become members of the class of verbs of surface contact through motion if more force is exerted on the object and the object yields. In other words, whether or not they have a \textsc{goal} role is dependent on the amount of force exerted and the weakness of the entity to which force is being applied.

\ea\label{ex:sibanda:12}
\settowidth\jamwidth{\textsc{a p (l)}}
{Exceptions (Dual membership), e.g. \textit{khab} ‘kick’}\\
 (Similar roots: \textit{waqaz} ‘slap’, \textit{khahlel} ‘kick’, \textit{gqubul/gqikil} ‘head butt’, \textit{hlankal/muhluz} ‘slap’)\\
\ea{\label{ex:sibanda:12a}
\gll U-Themba u-ø-khab-a um-duli. \\
 1a-Themba 1a-\textsc{tns}-kick-a 4-wall\\\jambox{\textsc{a p}}
}

\glt ‘Themba is kicking the wall.’

\ex{\label{ex:sibanda:12b}
\gll U-Themba u-ø-khab\textit{-el}-a u-Musa um-duli. \\
 1a-Themba 1a-\textsc{tns}-kick-\textsc{app}-a 1a-Musa 4-wall\\\jambox{\textsc{a b p}}
}

\glt ‘Themba is kicking the wall for Musa.’

\ex{\label{ex:sibanda:12c}
\gll U-Themba u-ø-khab\textit{-el}-a um-duli i-mali\\
 1a-Themba 1a-\textsc{tns}-kick-\textsc{app}-a 4-wall 9-money\\\jambox{\textsc{a p r}}
}

\glt ‘Themba is kicking the wall for money.’

\ex{\label{ex:sibanda:12d}
\gll U-Themba u-ø-khab\textit{-el}-a um-duli pha-ndle \\
 1a-Themba 1a-\textsc{tns}-hit-\textsc{app}-a 4-wall 16-outside\\\jambox{\textsc{a p l}}
}

\glt ‘Themba is kicking the wall outside.’

\ex{\label{ex:sibanda:12e}
\gll U-Themba u-ø-khab\textit{-el}-a um-duli pha-ndle \\
 1a-Themba 1a-\textsc{tns}-hit-\textsc{app}-a 4-wall 16-outside\\\jambox{\textsc{a t g}}
}

\glt ‘Themba is kicking the wall (to the) outside.’
\z
\z

With roots that inherently imply little force, such as \textit{bhansul} ‘slap lightly (at the back)’, \textit{thint} ‘touch’ and \textit{-anga} ‘kiss’, there is never a \textsc{theme} and \textsc{goal} reading.

\section{Involuntary processes (no \textsc{beneficiary})}\label{sec:sibanda:}

Verbs of involuntary processes can be divided into two subgroups: those that involve motion and those that do not. Verbs in both subcategories take a \textsc{patient} subject. Where motion is involved, the subject could also be viewed as a \textsc{theme}. However, there is no complete change of location since the subject does not totally leave the point of origin. On the basis of there being no complete change in location from the point of origin, I treat the subject as a \textsc{patient}. In fact, the subject can be seen more as coming out changed from the actions of a \textsc{causer} than from movement. For instance, what is at the fore in \REF{ex:sibanda:13a} is that something is causing the tree to grow (changing it from small to big), not to move (from point A to B). To the verbs of involuntary processes that encode motion, the applicative supplies a \textsc{goal}, \textsc{location} or \textsc{reason} as in \REF{ex:sibanda:13b}, \REF{ex:sibanda:13c} and \REF{ex:sibanda:13d}, respectively.

\ea\label{ex:sibanda:13}
\settowidth\jamwidth{\textsc{a p (l)}}
 {+ Motion}\\
 (Similar root: \textit{ncibilik} ‘melt’)\\
\ea{\label{ex:sibanda:13a}
\gll Isi-hlahla si-ya-khula. \\
 7-tree 7-\textsc{tns}-grow-a \\\jambox{\textsc{p}}
}

\glt ‘The tree is growing’

\ex{\label{ex:sibanda:13b}
\gll Isi-hlahla si-khul\textit{-el}-a ko-makhelwane. \\
 7-tree 7-\textsc{tns}-grow-\textsc{app}-a \textsc{loc}-neighbor \\\jambox{\textsc{p g}}
}

\glt ‘The tree is growing towards/into the neighbor’s’

\ex{\label{ex:sibanda:13c}
\gll Isi-hlahla si-khul\textit{-el}-a ko-makhelwane. \\
 7-tree 7-\textsc{tns}-grow-\textsc{app}-a \textsc{loc}-neighbor \\\jambox{\textsc{p l}}
}

\glt ‘The tree is growing at the neighbor’s’ 

\ex{\label{ex:sibanda:13d}
\gll Isi-hlahla si-khulela uku-thi si-dl-iw-e.\\
 7-tree 7-\textsc{tns}-grow-\textsc{app}-a 15-that 7-eat-\textsc{pass-tns}\\\jambox{\textsc{p r}}
}

\glt ‘The tree is growing so that it will be eaten.’ 
\z
\z

Although the root \textit{ncibilik} ‘melt’ belongs to this class where the single argument undergoes a change of state, it has dual class membership as it can also be treated like a \textsc{theme-goal} verb of motion (contrasting with the \textsc{patient-goal} frame similar to \REF{ex:sibanda:13b}). This is, however, only possible when motion, not change of state, is at the fore and after applicativazation, unlike with the verb root \textit{y} ‘go to’, for example, where the \textsc{theme-goal} reading occurs before applicativization. An example of the \textsc{theme-goal} semantic frame is in \REF{ex:sibanda:14} where \textit{phansi} can refer to the ground of floor. 

\ea{\label{ex:sibanda:14}
\settowidth\jamwidth{\textsc{a p (l)}}
{Verb of motion: root \textit{ncibilik} ‘melt’}\\
\gll U-ngqwaqwane w-a-ncibilik\textit{-el}-a pha-nsi\\
 1a-ice 1a-\textsc{tns}-melt-\textsc{app}-a 16-down \\ \jambox{\textsc{t g} (or \textsc{m})}
}
\glt ‘The ice melted onto the ground/floor’
\z

 Without motion, verbs of involuntary processes introduce the same roles as \REF{ex:sibanda:14} except for the \textsc{goal.} This is illustrated \REF{ex:sibanda:15}.

\ea\label{ex:sibanda:15}
\settowidth\jamwidth{\textsc{a p (l)}}
{- Motion} \\
 (Similar root: \textit{f} ‘die; break; break down’)\\
\ea{
\gll In-hlama i-y-om-a. \\
 9-dough 9-\textsc{tns}-dry-a \\\jambox{\textsc{p}}
}

\glt ‘The dough dries.’


\ex{
\gll In-hlama y-om\textit{-el}-a pha-ndle. \\
 9-dough 9-\textsc{tns}-dry-\textsc{app}-a 15-outside \\\jambox{\textsc{p l}}
}

\glt ‘The dough dries outside.’

\ex{
\gll In-hlama y-om\textit{-el}-a ukuthi i-langa li-ya-tshisa \\
 9-dough 9-\textsc{tns}-dry-\textsc{app}-a that 5-sun 5-\textsc{tns}-hot\\\jambox{\textsc{p r}}
}

\glt ‘The dough dries because the sun is hot.’ 
\z
\z

While the literature supports \textsc{beneficiary} as the most common role associated with applicative constructions, it is clear from \REF{ex:sibanda:13} and \REF{ex:sibanda:15} that \textit{-el} does not introduce this role to verbs of involuntary process. The \textsc{beneficiary}/\textsc{maleficiary} role may arguably only be inferred in very specific circumstances that also involve something good or bad happening to a \textsc{location} or \textsc{goal} as a result of the process.” That is, a \textsc{beneficactive}/\textsc{maleficactive} reading can be inferred for verbs of this class only if they also have membership in another class, as in \REF{ex:sibanda:14}.

\section{State verbs (no \textsc{beneficiary)}}\label{sec:sibanda:}

Example \REF{ex:sibanda:16} shows that state verbs have a \textsc{patient} subject and behave exactly like those represented by \REF{ex:sibanda:15} when the applicative \textit{-el} is suffixed. The single argument of the verb here is a \textsc{patient} rather than a \textsc{theme} because there is no clear movement or change of location, but the participant in these stative verbs may change state, for example, from hot to cold in (\ref{ex:sibanda:16b}-c) even if ‘hot’ is not mentioned. 

\ea\label{ex:sibanda:16}
\settowidth\jamwidth{\textsc{a p (l)}}
{State root \textit{qanda} ‘be cold’}\\
 (Similar roots: \textit{phil} ‘be alive’, \textit{khudumal} ‘be warm’, \textit{bil} ‘boil’)\\
\ea{\label{ex:sibanda:16a}
\gll Ama-nzi a-ya-qand-a. \\
 6-water 6-\textsc{tns}-cold-a\\\jambox{\textsc{p}}
}

\glt ‘The water is cold.’

\ex{\label{ex:sibanda:16b}
\gll Ama-nzi a-qand\textit{-el}-a e-mbiz-eni. \\
 6-water 6-\textsc{tns}-cold-\textsc{app}-a \textsc{loc}-pot-\textsc{loc}\\\jambox{\textsc{p l}}
}

\glt ‘The water is/becomes cold in the pot.’

\ex{\label{ex:sibanda:16c}
\gll Ama-nzi a-qand\textit{-el}-a uku-thi a-se-friji-ni. \\
 6-water 6-\textsc{tns}-cold-a \textsc{15}-that 6-\textsc{loc}-refrigerator-\textsc{loc} \\\jambox{\textsc{p r}}
}

\glt ‘The water is/becomes cold because it is in the refrigerator.’
\z
\z

As can be seen, when \textit{-el} is suffixed to the root, the thematic role of the new participant can only be \textsc{location} \REF{ex:sibanda:16b} or \textsc{reason} \REF{ex:sibanda:16c}, not \textsc{beneficiary}. In fact Ndebele verbs exemplified by \REF{ex:sibanda:16}, although not unambiguously verbs of involuntary process, can be better classified with those in \REF{ex:sibanda:15} since they also often involve a change state when \textit{-el} is suffixed to the root.

\section{Verbs of feeling (no \textsc{beneficiary)}}\label{sec:sibanda:}

The last class considered contains verbs of feeling that normally surface with two arguments with the sematic roles of \textsc{experiencer} (E) and \textsc{theme} (\textsc{or stimulus}\textit {Sti}). 

\ea\label{ex:sibanda:17}
\settowidth\jamwidth{\textsc{a p (l)}}
{Verbs of feeling: Verb root \textit{esab} ‘be scared/afraid’}\\
 (Similar root: -\textit{enyany} ‘be disgusted (by something)’)\\
\ea{
\gll u-Themba w-e-sab-a in-yoka. \\
 1a-Themba 1a-\textsc{tns}-afraid-a 9-snake\\\jambox{\textsc{e t}}
}

\glt ‘Themba was afraid/scared of the snake.’

\ex{
\gll u-Themba w-e-sab\textit{-el}-a in-nyoka uku-luma. \\
 1a-Themba 1a-\textsc{tns}-afraid-APP-a 9-snake 15-bite\\\jambox{\textsc{e t r}}
}

\glt ‘Themba got scared of the snake because it bites.’

\ex{
\gll u-Themba w-e-sab\textit{-el}-a in-nyoka e-gusw-ini. \\
 1a-Themba 1a-\textsc{tns}-afraid-APP-a 9-snake \textsc{loc}-forest-\textsc{loc}\\\jambox{\textsc{e t l}}
}

\glt ‘Themba got scared of the snake in the forest.’
\z
\z

Example \REF{ex:sibanda:17} shows that for verbs of feeling, the argument introduced by the applicative may have a \textsc{reason} or \textsc{location} sematic role. Again here the \textsc{beneficiary} is completely excluded. 

\section{Summary: The applicative with verbs in different semantic classes}\label{sec:sibanda:}

\tabref{tab:sibanda:2} presents argument frames associated with applicative constructions formed from roots of different semantic classes. Thematic roles in the argument fames are presented in plain type, and those associated with the argument introduced by applicative \textit{-el} are in bold and vary across B, G, L and R. 

%%please move \begin{table} just above \begin{tabular
\begin{table}
\caption{Argument frames by verb class}
\label{tab:sibanda:2}

\begin{tabularx}{\textwidth}{XX}
\lsptoprule
{Verb class} & {Thematic role frames}\\
\midrule
{1. motion} & {a. Monovalent root + Voluntary motion} {\textsc{t}\textbf{\textsc{b}} \textsc{t}\textbf{\textsc{g}} \textsc{t}\textbf{\textsc{l}} \textsc{t}\textbf{\textsc{r}}}{b. Monovalent root + Involuntary motion} {\textsc{t}\textbf{\textsc{g}} \textsc{t}\textbf{\textsc{l}} \textsc{t}\textbf{\textsc{r}}}{c. Divalent root} {\textsc{tb}\textbf{\textsc{g}} \textsc{t}\textbf{\textsc{r}}\textsc{g/tg}\textbf{\textsc{r}} \textsc{}}\\
{2. surface contact through motion} & {a. Divalent root with applied noun object} {\textsc{a}\textbf{\textsc{b}}\textsc{t a}\textbf{\textsc{g}}\textsc{t a}\textbf{\textsc{r}}\textsc{t at}\textbf{\textsc{l}} \textsc{at}\textbf{\textsc{g}}}{b. Divalent root with applied Locative Object} {\textsc{a}\textbf{\textsc{b}}\textsc{l a}\textbf{\textsc{r}}\textsc{l}}{c. Trivalent root} {\textsc{a}\textbf{\textsc{b}}\textsc{gt a}\textbf{\textsc{r}}\textsc{gt agt}\textbf{\textsc{l}}}\\
{3. surface contact (no motion)} & {\textsc{a}\textbf{\textsc{b}}\textsc{p ap}\textbf{\textsc{r}} \textsc{ap}\textbf{\textsc{l}}}\\
{4. involuntary processes}  & {\textsc{p}\textbf{\textsc{g}} \textsc{p}\textbf{\textsc{l}} \textsc{p}\textbf{\textsc{r}}}\\
{5. state} & {\textsc{p}\textbf{\textsc{l}} \textsc{p}\textbf{\textsc{r}}}\\
{6. feeling} & {\textsc{et}\textbf{\textsc{r}} \textsc{et}\textbf{\textsc{l}}}\\
{7. sound emission}  & {\textsc{EM.}\textbf{\textsc{R}} \textsc{EM.}\textbf{\textsc{L}}}\\
\lspbottomrule
\end{tabularx}

\end{table}

As can be seen, \textit{-el} can introduce \textsc{reason} and \textsc{location} thematic roles in all the classes we have seen above. \textsc{Reason} can also be introduced in all subclasses, which translates to all verbs. Note the absence of \textsc{location} when a verb of motion has a divalent root (case 1c in \tabref{tab:sibanda:2}). The \textsc{beneficiary} role does not feature at all in the last four classes in \tabref{tab:sibanda:2}. However, although the first two classes involve motion and the last four do not, it would be premature to conclude that the \textsc{beneficiary} thematic role occurs only in classes where motion is involved since this study has not exhaustively covered all semantic verb classes.

\section{Conclusion}\label{sec:sibanda:}

The discussion has shown that there is clear variation across semantic classes and sometimes also a little variation within classes in terms of thematic roles added by the applicative. While the variation across sematic classes can be captured by identifying the different thematic roles assigned to arguments in each class and those that applicative \textit{-el} introduces, the variations within a given semantic class may be due to differences in transitivity, the number of participants associated with the roots, and whether or not the action associated with the verb is voluntary. Also, some classes overlap, resulting in some verbs not being the best representatives of their classes. In spite of the variation within classes, thematic roles of participants associated with a given Ndebele verb are generally predictable from the semantic class of the root including those of participants introduced by applicative \textit{-el}. It has been shown that \textsc{reason} and \textsc{location} are the thematic roles \textit{-el} introduces in all semantic classes, and not \textsc{beneficiary} counter to many other studies of applicatives (e.g., \citealt{Schadeberg2003,Peterson2007,Polinsky2008,KiittilaZuniga2010,deKindBostoen2012}; and \citealt{MartenKula2014}). The Ndebele applicative was also shown to introduce arguments with roles often thought of as either participants (for example, \textsc{beneficiary}, \textsc{source}, \textsc{goal}) or nonparticipants (\textsc{location}, \textsc{reason}). 

 An issue that arises from the discussion is whether there is a precise or “single” meaning of the applicative suffix \textit{-el}. Due to English influence, it is tempting to conclude that the applicative \textit{-el} is a polysemous suffix\footnote{For an argument in support of polysemy, though accepting underspecification to some extent, see \citegen{Mabugu2011}’s analysis of the Shona applicative construction.} that takes various “prepositional” meanings such as ‘for’, ‘at’, ‘in’, ‘on’, and so on, which serve as cues for thematic roles. For example English \textit{at} is a cue for \textsc{location} and \textit{for} signals \textsc{beneficiary} or \textsc{reason}. However, a close examination of the use of -\textit{el} suggests that it is an underspecified suffix\footnote{\citet{Marten2002} also advances an argument for underspecification, although his analysis does not focus on semantic classes, but is driven by the Relevance-Theoretic notion of concept strengthening.} whose “prepositional” information, such as ‘for’, ‘at’, ‘in’ and ‘on’ is largely determined by the verb class. It appears that \textit{-el} encodes a very general relationship such as ‘extra argument’, and further semantic specificity is available from the verb itself if we know its semantic class. For example, \textit{i}n Ndebele native speakers already know which semantic classes are compatible with \textsc{beneficiary} and \textsc{reason} roles, so there is no need to specify any specific preposition-like meaning inhering “in” the applicative morpheme. If semantic class information is known then there is no need for detailed information in the applicative. 

 In short, the discussion above has shown that the Ndebele Applicative Construction suffixes the applicative \textit{-el} to the verb which adds a new argument, and the thematic role of the argument is constrained by the semantic class of the verb and context. The results of this study can subsequently be used as a test tool for evaluating class membership of additional verbs, since verbs of the same class take similar thematic roles. If the applicative argument triggers an irregular thematic role then the verb does not belong to the expected class or, at best, has dual membership.

\section*{Acknowledgments}

I would like to thank members of ACAL 46 who attended my presentation, reviewers and editors for their generous and insightful comments which helped me improve this article to its current state.

\section*{Abbreviations}

\begin{tabularx}{.45\textwidth}{lX}
\textsc{a} & \textsc{agent} \\
\textsc{app} & applicative \\
\textsc{ao} & applied object  \\
\textsc{b} & \textsc{beneficiary} \\
\textsc{c} & \textsc{causer} \\
\textsc{e} & \textsc{experiencer} \\
\textsc{em} & \textsc{emitter} \\
\end{tabularx}
\begin{tabularx}{.45\textwidth}{lX}
\textsc{g} & \textsc{goal} \\
\textsc{l} & \textsc{location} \\
\textsc{lo} & logical object  \\
\textsc{loc} & locative affix \\
\textsc{m} & \textsc{maleficiary} \\
\textsc{p} & \textsc{patient} \\
\textsc{pur} & \textsc{purpose}  \\
\end{tabularx}

\begin{tabularx}{.45\textwidth}{lX}
\textsc{r} & \textsc{reason} \\
\textsc{s} & \textsc{source} \\
\textsc{sti} & \textsc{stimulus} \\
\end{tabularx}
\begin{tabularx}{.45\textwidth}{lX}
\textsc{t} & \textsc{theme} \\
\textsc{tns} & tense  \\
-a & default final vowel for verbs \\
\end{tabularx}

1, 1a, 2, 2a, 3, 4, 5, 6, 7, 8, 9, 10, 11, 14, 15, 16: noun class prefixes or agreement markers (e.g. 1a = noun class 1a prefix or agreement marker)

\begin{verbatim}%%move bib entries to localbibliography.bib

%Levin, Beth. to appear. Verb classes within and across languages. In Comrie, Bernard & Malchukov, Andre (eds.), Valency classes: A comparative handbook. Berlin: De Gruyter Mouton. [Not cited in main text]

\end{verbatim}

\printbibliography[heading=subbibliography,notkeyword=this]

\end{document}