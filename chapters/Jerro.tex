\documentclass[output=paper]{langsci/langscibook} 
\title{The locative applicative and the semantics of verb class in Kinyarwanda} 
\author{%
 Kyle Jerro \affiliation{University of Texas at Austin } 
}
% \chapterDOI{} %will be filled in at production


\abstract{
This paper investigates the interaction of verb class and the locative applicative in Kinyarwanda (Bantu; Rwanda). Previous analyses of applicative morphology have focused almost exclusively on the syntax of the applied object, assuming that applicativization adds a new object with a transparent thematic role (e.g. \citealt{kissable:1977,baker:1988,bresnan:1990,alsina:1993,mcginnis:2001,jeong:2007, jerro:2015}, \emph{inter alia}). I show instead that the interpretation of the applied object is contingent upon the meaning of the verb, with the applied object having a {\sc path, source,} or {\sc goal} semantic role with motion verbs from different classes. The general {\sc locative} role discussed in previous work appears with non-motion verbs. I outline a typology of the interaction of the locative applicative with four different verb types and provide a semantic analysis of applicativization as a paradigmatic constraint on the lexical entailments of the applicativized variant of a particular verb. }

\maketitle
\begin{document}

\newcommand{\comment}{\bf\color{blue}}
\newcommand{\p}{\leavevmode\llap{(}}
\newcommand{\odd}{\leavevmode\llap{\#}}
\newcommand{\odder}{\leavevmode\llap{?\#}}
\newcommand{\q}{\leavevmode\llap{?}}
\newcommand{\hm}{\footnotesize\leavevmode\llap{?(?)}}
\newcommand{\qq}{\leavevmode\llap{??}}
\newcommand{\bad}{\leavevmode\llap{*}}
\newcommand{\huh}{\leavevmode\llap{?*}}

\def\emdash{\hbox{~--\ }}
\def\endash{\hbox{--}}


% \captionsetup[table]{labelfont=bf}
% \setcitestyle{citesep={;}}

% \frenchspacing

 
 
 


% \noindent \textbf{Keywords:} Applicative Morphology, Argument Structure, Syntax, Lexical Semantics, Directed Motion, Bantu Languages, Kinyarwanda 

 
\section{Introduction}%%%%%%%%%%%%%%%%%%%%%%%%%%%%%%%%%%%%%%%%%%%%%%%%%%%%%%%%%%%%%%%%%%%%%%%%%%%%%%%%%%%%%%%%%%%%%%%%%%%%%%%%%%%%%%%%%%%%%%%%%%%%%%%%%%%%%%%%%%%%%%%%%%%%%%%%%%%%%%%%%%%%%%%%%%%%%%%%%%%%%%%%%%%%%%%%%%%%%%%%%%%%%%%%%%%%%%%%%%%% 

 The applicative has been traditionally analyzed as a valency-increasing morpheme which adds a new object and associated thematic role to the argument structure of a given verb (see \citealt{dixon:1997} and \citealt{peterson:2007} for a typological overview of valency-changing morphology cross-linguistically). An example from Kinyarwanda (Bantu; Rwanda) is given in (\ref{apples}b), with the applicative morpheme \emph{-ir}:\footnote{All Kinyarwanda data in this paper come from elicitations conducted by the author.}

\begin{exe}
\ex\label{apples}
	\begin{xlist}
          \ex\gll Umu-gabo a-ra-ndik-a in-kuru.\\
          {\sc 1}-man {\sc 1S-pres}-write-{\sc imp} {\sc
            9-}story\\
          \glt `The man is writing the story for the child.'
          \ex\gll Umu-gabo a-ra-ndik-ir-a umw-ana in-kuru.\\
          {\sc 1}-man {\sc 1S-pres}-write-{\sc appl-imp} {\sc 1-}child {\sc 9-}story\\
          \glt `The man is writing the child the story.'
	\end{xlist}
\end{exe} 
%
Whereas the transitive verb \emph{kwa-ndika} `write' in (\ref{apples}a) licenses a subject and one object, the applicativized variant in (\ref{apples}b) has two post-verbal NPs.\footnote{Bantu applicative morphemes usually immediately follow  the verb stem, and the vowel is often conditioned by the preceding stem vowel. For Kinyarwanda, when the vowel in the preceding syllable is mid (i.e. [e] or [o]), the allomorph is \emph{-er}, else it is \emph{-ir}. Because \emph{-ir} is the more general form, I use it as the citation for the duration of this paper.}  Applicative morphology often licenses objects that are assigned one of various different thematic role types, such as {\sc beneficiary, reason/motive,} and {\sc locative}. In the case of (\ref{apples}b), the applied object  \emph{umwana} `child' is a {\sc beneficiary}.

 A majority of the literature on applicative morphology has focused on the syntactic properties of the applied object, examining whether the thematic object and the applied object have equal access to objecthood diagnostics like passivization and object marking on the verb. These approaches assume that applicativization transparently adds a new object participant with a specific thematic role to the verb's argument structure.  On this view, the applicative morpheme lexically or syntactically licenses a new object that may or may not block --- contingent upon the language ---  the thematic object from having its default object properties (see \S2 for an overview of the literature on object symmetry).

 However, work outside Bantu has shown that the meaning of particular verb classes affects argument realization patterns (\citealt{fillmore:1970,levin:1993, lrh:2008a, beavers:2011a}, \emph{inter alia}). The hypothesis for languages with applicatives is that verb meaning should affect which thematic roles can assigned to the applied object. In this paper I show that verb class indeed affects the argument realization of the applicative morpheme, and I outline a four-way typology based on the nature of location denoted by the base verb (see \citet{sibanda:2016} for evidence for the interaction between verb class and applicative morphology in Ndebele). Furthermore, I suggest that applicativization is not a productive operation of argument addition. Instead, I analyze applicativization as a paradigmatic constraint which requires that the lexical entailments associated with a particular argument in the predicate be stricter than those in the corresponding non-applicativized predicate.
 
 
In the next section, I discuss previous approaches to applicative morphology. In \S3 I show that there is variation in the use of the locative applicative in Kinyarwanda, and in \S4 I show that these uses are tied to four separate verb classes. In \S5 I outline an analysis based on the lexical entailments of the verb which accounts for both the traditional use of an applicative as well as the semantic uses described in this paper. In \S6 I conclude the discussion.
	 
 

\section{Previous approaches} %%%%%%%%%%%%%%%%%%%%%%%%%%%%%%%%%%%%%%%%%%%%%%%%%%%%%%%%%%%%%%%%%%%%%%%%%%%%%%%%%%%%%%%%%%%%%%%%
 

 Previous work on applicative morphology has looked almost exclusively at the syntactic nature of the applied object in relation to the thematic object (i.e. the object licensed by the verb). The mainstay of research on applied objects has looked at the syntax of the applied object, analyzing applicativization as an operation that adds an object to the argument structure of the verb (\citealt{baker:1988, bresnan:1990, alsina:1992, alsina:1993, marantz:1993, pylkkanen:2008, mcginnis:2001, mcginnis:2003, baker:2006, zeller:2006, zellerngoboka:2006, peterson:2007, jeong:2007, bakeretal:2012, jerro:2015}).  
  
However, these approaches do not capture the full empirical range of uses of applicative morphology. For example, cases exist in which the applicative affects the meaning of the thematic object instead of adding a new syntactic object \citep{marten:2003, creissels:2004, cannmabugu:2006, bond:2009}. For example, \citet{marten:2003} notes a use of the applicative in Swahili that indicates a pragmatically noteworthy property of the verbal object, as in (\ref{martendress}b) where there is a pragmatically salient property that is absent with the non-applicativized variant.

\begin{exe}
\ex\label{martendress}Swahili \begin{xlist}
	\ex \gll Juma a-li-va-a kanzu\\
Juma 1S-{\sc pst}-wear-{\sc fv} kanzu\\
	\glt `Juma was wearing a Kanzu.'
	\ex\gll Juma a-li-val-i-a nguo rasmi.\\
Juma {\sc 1S-pst}-wear-{\sc appl-fv} clothes official\\
	\glt `Juma was dressed up in official/formal clothes.'
	\ex\gll \#Juma a-li-val-i-a kanzu.\\
Juma {\sc 1S-pst}-wear-{\sc appl-fv} kanzu\\
	\glt Intended: `Juma was wearing a Kanzu.' (Marten 2003: 9) %(14)
	\end{xlist}
\end{exe}
%
In (\ref{martendress}b), there is no additional object that is not present with the non-applicativized verb in (\ref{martendress}a); the only difference between the two is that (\ref{martendress}b) provides additional pragmatic information about the object. Specifically, Marten argues for what he terms \emph{concept strengthening}, where the applicative is used to make a narrower claim about the object of the sentence than in the non-applied sentence. This use of the applicative is a semantic/pragmatic use which lies outside of the standard analysis of applicativization as an object-adding operation. 

A separate literature on the typology of directed motion has argued that applicatives in certain Bantu languages can be used to license {\sc goals} \citep{schaefer:1985,sitoe:1996}. For example, consider the data in (\ref{tswana}a) and (\ref{tswana}b) from Setswana (Bantu; Botswana); in (\ref{tswana}a), there is a locative phrase which describes a general location. In (\ref{tswana}b), the presence of the applicative \emph{-\c{\`e}l} indicates a {\sc goal}  reading of the locative that is not present in (\ref{tswana}a).\footnote{The glosses and English translations of the data in (\ref{tswana}) are copied from the original examples in \citet{schaefer:1985}.}

\begin{exe}
	\ex\label{tswana}Setswana\begin{xlist}
		\ex  \gll m\`o-s\'im\`an\'e \'o-k\'ib\'itl-\`a f\'a-tl\`as\'e g\'a-d\`i-tlh\`ar\`e.\\
			1-boy 1S-run.heavily-{\sc imp} {\sc nearby-}under {\sc loc-8-}tree\\
			\glt `The boy is running with heavy footfall under the trees.'
		\ex\gll m\`o-s\'im\`an\'e \'o-k\'ib\'itl-\c{\`e}l-\`a kw\'a-tl\`as\'e g\'a-th\`ab\`a.\\
			1-boy 1S-run.heavily-to-{\sc imp} {\sc distant}-under {\sc loc-}mountain\\
			\glt `The boy is running with heavy footfall to under the mountain.'  (Schaefer 1985: Tables VI-VII)
	\end{xlist}
\end{exe}
%
Contra analyses where the locative applicative is assumed to add an object that is assigned a {\sc locative} thematic role, data such as that in (\ref{tswana}) suggest that the semantic role of the locative applicative is not always a general {\sc locative}.

Few studies have investigated the effect of the semantics of verb class on the realization of applicative morphology. An exception is \citet{rugemalira:1993}, who gives a detailed account of the interaction of locatives with 500 different verbs in Runyambo (Bantu; Tanzania).\footnote{See also \citet{cannmabugu:2006} for discussion of the interaction of verb class and applicative in Shona and \citet{sibanda:2016} for a related discussion of applicatives in Ndebele.} Rugemalira describes a four-way typology of locatives with different verbs: verbs that require an applicative to license a location, verbs that disallow the applicative with a locative phrase, verbs where the applicative changes the interpretation of the location, and verbs where the applicative and locative prefix are in complementary distribution.\footnote{Unlike the present study, Rugemalira rejects the notion that there are semantically-defined verb classes such as `motion verbs'. The data below from Kinyarwanda do suggest unity of verbs across classes.}
The interaction with verb meaning has not been central in work since Rugemalira's dissertation. In this paper, I bring this perspective back to the fore, exploring the interaction of verb meaning and locative applicatives in Kinyarwanda.

\section{Typology of locative meanings}%%%%%%%%%%%%%%%%%%%%%%%%%%%

In this section, I describe four kinds of locative meanings that may be added to a verb by the applicative: {\sc locative, goal, path}, and {\sc source}. I employ different morphosyntactic tests to motivate the syntactic and semantic differences among the uses.  I assume a typology of motion where a complete motion event involves an agent moving from a {\sc source}, along a {\sc path}, and ending at a {\sc goal} (cf. \citealt{talmy:1975,slobin:1996,zlatevy:2004,beaversetal:2009}, \emph{inter alia}).\footnote{My use of the term \emph{path} here is what \citet{ashersablay:1995} refer to as \emph{strict internal path}, i.e. the portion of a path which does not include the {\sc source} and {\sc goal}.} Of course, a particular motion event may not syntactically license  all of these elements simultaneously, and --- as I show below --- different verbs in Kinyarwanda categorize syntactically and/or semantically for different components of the motion event. 

%\subsection{

The first category is verbs where the applicative adds a general {\sc locative} role, i.e. the location where the event took place.\footnote{This applicative form contrasts with the locative applicative described in \citet{kimenyi:1980,zeller:2006,zellerngoboka:2006}, i.e. \emph{--ho}. For all the speakers I have consulted, \emph{-ir} is the locative applicative, while \emph{--ho} is one of a class of locational clitics (cf. \S4).} For the verb \emph{ku-vuga} `to talk', the applicative is obligatory in (\ref{talkapp}b) for licensing the argument with the {\sc locative} role. 
	\begin{exe}
		\ex\label{talkapp}\begin{xlist}
		\ex\gll Yohani a-ri ku-vug-a.\\	
				John 1S-{\sc be} {\sc inf}-talk-{\sc imp}\\
				\glt `John is talking.'
		\ex\gll Yohani a-ri ku-vug-ir-a mu n-zu.\\
			John 1S-{\sc be} {\sc inf}-talk-{\sc appl-imp} 18 9-house\\
				\glt `John is talking in the house.'
	\end{xlist}
	\end{exe}
 %
 In (\ref{talkapp}a), the verb \emph{ku-vuga} `to talk' is intransitive, while in (\ref{talkapp}b), there is a new argument that is licensed by the applicative morpheme. Note that locative applicative sentences differ from other sentences with applicatives in that the {\sc locative} applied object often requires a locative prefix \emph{mu}.\footnote{Though for some verbs, the locative prefix is omitted. I assume that whether the applied object is marked with a locative varies on a verb-by-verb basis. In fact, \citet{rugemalira:2003} shows that there is considerable variation across and within languages as to whether the locative prefix is required.} Other applied objects in Kinyarwanda, such as {\sc beneficiaries} in (\ref{apples}b), do not require any prefixes. Crucially, the locative prefixes found throughout this paper are class markers and not prepositions, contrary to the Kinyarwanda orthographic convention of writing the prefix separately (as demonstrated in Appendix A). Thus for the duration of the paper I treat phrases which are preceded by locatives such as \emph{ku} and \emph{mu} as arguments and not obliques.  

In the second type, the applicative adds a {\sc goal} to the event described by the verb. This appears with verbs such as \emph{kw-iruka} `to run', \emph{gu-tembera} `to go about', \emph{ku-jya} `to go', and  \emph{gu-simbuka} `to jump'.
	\begin{exe}
		\ex\label{runbase}\begin{xlist}
		\ex\gll Yohani a-ri kw-iruk-a.\\	
				John 1S-be {\sc inf}-run-{\sc imp}\\
				\glt `John is running.'
		\ex\gll Yohani a-ri kw-iruk-ir-a kw' i-soko.\\
				John 	1S-{\sc be} {\sc inf-}run-{\sc appl-imp} 17 5-market \\
				\glt `John is running to the market.'
		\end{xlist}
		\end{exe}
%
 In (\ref{runbase}b), the new location licensed by the applicative is not a general description of where the event took place, but rather the {\sc goal}  of the running event.
 
 Third, the applied object may be a {\sc path}, as in (\ref{inalco}b). Unlike the verb \emph{ku-vuga} `to talk' in (\ref{talkapp}), the non-applicativized variant of the verb \emph{kw-injira} `to enter' permits a {\sc locative} object in the non-applicativized variant.  Other verbs that pattern like \emph{kw-injira} `to enter' are \emph{gu-sohoka} `to exit', \emph{ku-manuka} `to descend', \emph{kuzamoka} `to ascend', and \emph{ku-rira} `to climb'.

\begin{exe}
	\ex\label{inalco}\begin{xlist}
		\ex\gll N-di kw-injir-a mu n-zu.\\
				1{\sc sg-be} {\sc inf-}enter-{\sc imp} 18 9-house\\
				\glt `I am entering the house.'
		\ex\gll N-di kw-injir-ir-a mu n-zu mu mu-ryango.\\
				1{\sc sg-be} {\sc inf-}enter-{\sc appl-imp} 18 9-house 18 3-door\\
				\glt `I am entering the house through the door.'
	\end{xlist}
\end{exe}
%
 Here, the applied object describes the {\sc path}  through which the motion event occurs. Note that in (\ref{inalco}b), the applicative is obligatory.%you also get things like `iburyo'  right on climbing mountains 

 Finally, the applied object may be a {\sc source}, as in (\ref{pika}) where the applicative attaches to the verb \emph{kw-ambuka} `to cross'.
 \begin{exe}
	\ex\label{pika}\begin{xlist}
		\ex \gll Karemera y-$\emptyset$-ambuts-e in-yanja.\\
			Karemera 1S-{\sc pst}-cross-{\sc perf} 9-ocean\\
			\glt `Karemera crossed the ocean.'
		\ex\gll Karemera y-$\emptyset$-ambuk-iy-e  i Mombasa (mu) n-yanja.\\
			Karemera 1S-{\sc pst}-cross-{\sc appl-perf}  {\sc 23} Mombasa \phantom{(}18 9-ocean\\
			\glt `Karemera crossed the ocean from Mombasa.'
		\end{xlist}
\end{exe}
In this example, the applied object is obligatorily interpreted as the {\sc source} of the motion event.\footnote{The perfective morpheme \emph{-(y)e} often has phonological ramifications for the final consonant of the stem. Here, the infinitive \emph{kw-ambuka} `to cross' changes to \emph{-ambuts} when the perfective suffix is present.} 

 



\subsection{Evidence for the typology}%%%%%%%%%%%%%%%%%%%%%%%%%%%%%%%%%%%%%%%%%%%%%%%%%%%%%%
%%%%%%%%%%%%%%%%%%%%%%%%%%%
%%%%%%%%%%%%%%%%%%%%%%%%%%%
%%%%%%%%%%%%%%%%%%%%%%%%%%%


\subsubsection{Interpretive differences}%%%%%%%%%%%%%%%%%%%%%%%%%%%

 
 One indication of the differences between the applicativized and non-applicativized variants is the interpretive difference of the locational phrase in the applicativized and non-applicativized sentences. For example, consider the following context: Karemera is cooking, and he is talking about needing to run back to the store to get some things he forgot. I leave the room, but when I get back, and he has gone. In this context, I could ask the question in (\ref{intorun}), to which an interlocutor could respond with (\ref{kuti}a).
	\begin{exe}
		\ex\label{intorun}\gll [Ikibazo]: Karemera y-a-gi-ye he?\\
				problem: Karemera 3S-{\sc pst}-go-{\sc perf} where\\
				\glt `Question: Where did Karemera go? 		
		\ex\label{kuti}\begin{xlist}
			\ex\gll Y-iruk-iy-e kw' isoko.\\
					{\sc 1S-}run-{\sc appl-perf} 17 store\\
					\glt `He ran to the store.' \hfill 
			\ex\gll\odd Y-irutse-e kw' isoko.\\
					{\sc 1S-}run-{\sc perf} 17 store\\
					\glt (`He ran to the store.') (on intended reading)
			\end{xlist}
	\end{exe}
%
In this context it is infelicitous to use (\ref{kuti}b), where the locative is understood as describing the general location of the running event (e.g. a context where someone is running inside of a store), and not the {\sc goal}  of the subject's motion. The sentence in (\ref{kuti}a), on the other hand, describes the {\sc goal}  to which the running event is directed. 

\iffalse
cut this?: Consider another scenario: I'm standing in front of my house, and I see a lion. I want to run inside to get away from it. 

	\begin{exe}
		\ex\label{intorun}\begin{xlist}
			\ex\gll N-iruk-iy-e mu n-zu.\\
					i-run-{\sc appl-perf} 18 9-house\\
					\glt `I ran into the house.'
			\ex\gll \#N-iruts-e mu nzu.\\
					i-run-{\sc perf} 18 9-house\\
					\glt `I ran into the house.'	(on intended reading)				
		\end{xlist}
	\end{exe}
%

\fi

 Positional verbs also have a {\sc goal} object with the locative applicative, where the applied object is understood as the place where the subject is intending to sit.  Locative phrases can be used with both variants, but the interpretations are crucially distinct. The non-applicativized variant in (\ref{sitwater}a) is a general {\sc locative}, describing where the subject is sitting, while in (\ref{sitwater}b) it describes the sub-location on which the subject is sitting (by accident).\footnote{Consultants have noted that the applicativized positional verbs indicate a {\sc goal}  that is construed as being accidentally sat upon. I treat these verbs as a subclass of the manner of motion verbs (see below), but with this class there is an implicature that arrival at the {\sc goal}  is accidental.}
	\begin{exe}
		\ex\label{sitwater}\begin{xlist}
		\ex\gll N-icay-e ku mazi.\\
				1{\sc sgS-}sit-{\sc perf} 17 6-water\\
				\glt `I sat in the water.' (e.g. in a lake or a pool)
		\ex\gll N-icar-iy-e ama-zi.\\
				1{\sc sgS-}sit-{\sc appl-perf} 6-water\\
				\glt `I sat in the water.' (e.g. a puddle of water on a bench after it rained)
	\end{xlist}
	\end{exe}
%
%The meaning contrasts between the applied and non-applied sentences in this section show what a non-applied verb means and how the addition of an applicative morpheme changes the meaning. %Furthermore, the applied object has a different interpretation in (\ref{kuti}) than in (\ref{sitwater}), supporting the claim that applicativization with different verbs brings about different meanings. 
The data in this section show that while in certain cases a locative can appear with both a base verb and applicativized verb (though this property is verb-specific; see Appendix A), the interpretation of the location differs between the two. 

\subsubsection{Locational clitics}%%%%%%%%%%%%%%%%%%%%%%%%%%%

Another diagnostic for making precise the different locations that are selected for by different verbs with and without locative applicatives is the locative clitic. Kinyarwanda has three locative clitics that  replace locative phrases (for discussion on cognate clitics in other languages, see \citet{diercks:2011} for Lubukusu and \citet{simango:2012} for Chiche\^wa). 	Crucially, (and distinct from the function of locative clitics in Chiche\^wa), the locative clitics are in complementary distribution with the locative object, behaving similarly to a pronoun. 
%
\begin{table}[ht]
	\begin{center}	
 \caption{\textbf{Kinyarwanda locative clitics}}
	\begin{tabular}[t]{llc}\\\hline
		Clitic			& Meaning				& Class \\\hline
			 \emph{=ho}  & at or on something	& 16 \\
			
		 \emph{=yo} 	&  at or to a place 		& 17 \\

		 \emph{=mwo/mo}  &  inside of something & 18\\
	
	\hline
	 \end{tabular}
\label{tab:clitic} 
	 		 \end{center}
\end{table}
%
To date there is no detailed semantic discussion of the meanings of the locative clitics in Kinyarwanda, but the intuitive definitions in Table \ref{tab:clitic} are suitable for the current discussion. I assume that these three clitics correspond to the class 16, 17, and 18 locative class prefixes, cf. Appendix A.  
%I leave a detailed semantic analysis of the Kinyarwanda clitics to future research.

The use of a locative clitic is conditioned by two factors. First, the clitic replaces a locative phrase that is selected for by the verb (or applicative) and behaves as a syntactic object. Second, the semantics of the clitic must be compatible with the specific kind of motion conveyed in the sentence.  For example, in (\ref{runclitic}a) the locative clitic \emph{=yo} is not licit since there is no {\sc goal}  selected for by the non-applicativized version of the verb \emph{kw-iruka} `to run'. 
 
 
\begin{exe}
	\ex\label{runclitic}{\sc kwiruka} : to run \begin{xlist}
		\ex\gll *N-iruts-e=yo.\\
				1{\sc sg-}run-{\sc perf=17.loc}\\
				\glt (`I ran there.') \hfill *Goal
%		\ex\gll N-iruts-e=ho.\\
		%		{\sc 1sg}-run{\sc -perf=loc}\\
	%			\glt `I ran on something (e.g. treadmill).'
		\ex\gll N-iruts-e=mwo. \\
				1{\sc sg-}run-{\sc perf=18.loc}\\
				\glt `I ran inside of somewhere.' (e.g. a house)\hfill General Location
		%not the 'into' kind of in!!
	\end{xlist}
	\ex\label{runtoclitic}{\sc kwiruk-ir-a} : to run to\begin{xlist}
		\ex\gll N-iruk-iy-e=yo.\\
				1{\sc sg}-run-{\sc appl-perf=17.loc}\\
				\glt `I ran (to) there.'\hfill Goal
		\ex\gll N-iruk-iy-e=mwo.\\
				1{\sc sg}-run-{\sc appl-perf=18.loc}\\
				\glt `I ran into there.' \hfill Goal
	\end{xlist}
\end{exe}
In (\ref{runtoclitic}a), however, the locative clitic is permissible because the applicativized verb selects for a {\sc goal}  locative object (cf. (\ref{kuti}), above). The clitic \emph{=mwo} is permissible for both applicativized and  non-applicativized variants of the verb \emph{kw-iruka} `to run' but, crucially, with different interpretations. With the non-applicativized verb in (\ref{runclitic}b), the clitic is a general location inside which the event is taking place (e.g. inside a house). In (\ref{runtoclitic}b), on the other hand,  the clitic is the location into which the subject moves (i.e. the {\sc goal} ).  

The locative clitic is sensitive to whether a verb permits a locative object and the kind of location that that object describes. The data in (\ref{runclitic}) and (\ref{runtoclitic}) show that the verb \emph{kw-iruka} `to run' optionally allows a general {\sc locative}, but with the applicative, the locative object is a {\sc goal}.


 Another example is with the verb \emph{kw-injira} `to enter', where the referent of the locative clitic differs in interpretation between the applicativized and non-applicativized verbs. With the bare verb, \emph{=mwo} refers to the {\sc goal} (\ref{enterclitic}a), while with the applicativized variant the clitic refers to the {\sc path} through which the event took place (\ref{enterthclitic}a). The clitic \emph{=yo} presents a similar pattern; in (\ref{enterclitic}b), the clitic refers to the {\sc goal}, while in (\ref{enterthclitic}b), it refers to the {\sc path}.
 

\begin{exe}
	\ex\label{enterclitic} {\sc kwinjira} : to enter \begin{xlist}
		\ex\gll N-di kw-injir-a=mwo.\\
				1{\sc sg-be} {\sc inf-}enter-{\sc imp=18.loc}\\
				\glt `I am entering.' (e.g. the house)
		\ex\gll N-di kw-injir-a=yo.\\
				1{\sc sg-be} {\sc inf-}enter-{\sc imp=17.loc}\\
				\glt `I am entering.' (e.g. a country)
	\end{xlist}
	\ex\label{enterthclitic} {\sc kwinjir-ir-a} : to enter through \begin{xlist}
		\ex\gll N-di kw-injir-ir-a=mwo.\\
				1{\sc sg-be} {\sc inf-}enter-{\sc appl-imp=18.loc}\\
				\glt `I am entering through somewhere.' (e.g. a window)
		\ex\gll N-di kw-injir-ir-a=yo.\\
				1{\sc sg-be} {\sc inf-}enter-{\sc appl-imp=17.loc}\\
				\glt `I am entering through somewhere.' (e.g. Canada en route to America)
		\end{xlist}
		\end{exe}
%
The data in (\ref{enterclitic}) and (\ref{enterthclitic}) allow us to draw two conclusions regarding the argument structure of the verb \emph{kw-injira} `to enter'. First, in its non-applied form in (\ref{enterclitic}), the verb selects for a grammatical object which is the {\sc goal}. Second, in the applicativized variant in (\ref{enterthclitic}), the applied object is a {\sc path}. Crucially, the semantic role of the locative arguments that are selected by both applicativized and  non-applicativized variants of \emph{kw-injira} `to enter' are distinct from those selected by the verb \emph{kw-iruka} `to run' in (\ref{runclitic}) and (\ref{runtoclitic}). 


\iffalse
 Positional verbs show a distinct pattern: the locative clitics in the non-applicativized variant are grammatical, as shown in (\ref{sitclitic}), and all three clitics are possible depending on context. With the applicative, on the other hand, the clitics are disallowed.
 
\begin{exe}
	\ex\label{sitclitic} {\sc kw-icara} : to sit \begin{xlist}
		\ex\gll N-di kw-icar-a=yo.\\
				1{\sc sg-be} {\sc inf-}sit{-\sc imp=17.loc}\\
			\glt `I am sitting there (e.g. at my house).'
		\ex\gll N-di kw-icar-a=mwo. \\
1{\sc sg-be} {\sc inf-}sit{-\sc imp=18.loc}\\
			\glt `I am sitting there (e.g. inside my car).'
		\ex\gll N-di kw-icar-a=ho.\\
				{\sc 1sg-be} {\sc inf-}sit{-\sc imp=16.loc}\\
	\glt `I am sitting there (e.g. on a chair).'
	\end{xlist}
	\ex\label{sitonclitic} {\sc kw-icar-ir-a} : to sit on\begin{xlist}
		\ex\gll *N-di kw-icar-ir-a=yo.\\
				1{\sc sg-be} {\sc inf-}sit{-\sc imp=17.loc}\\
			\glt `I am sitting there (e.g. in my house).'
		\ex\gll *N-di kw-icar-ir-a=mwo. \\
1{\sc sg-be} {\sc inf-}sit{-\sc imp=18.loc}\\
			\glt `I am sitting there (e.g. inside my car).'
		\ex\gll *N-di kw-icar-ir-a=ho.\\
				1{\sc sg-be} {\sc inf-}sit{-\sc imp=16.loc}\\
			\glt `I am sitting there (e.g. on a chair).'

\end{xlist}

\end{exe}
%
The reason of the ungrammaticality of (\ref{sitonclitic}) is not clear, though a possible explanation is based on the pragmatics of using the locative clitics. Because the applicative in these cases introduces a ``surprising'' entity that is sat upon, it can be assumed that a sentence of the type in (\ref{sitclitic}) is making an informative claim about the object. In (\ref{sitonclitic}), the PP that carries the surprising information is replaced with a pro-form locative clitic. Because proforms are used as placeholders for old information, the use of a proform in this pragmatic context is infelicitous.\footnote{Thanks to Eric Campbell for originally suggesting this style of explanation.} Crucially for the current discussion, the semantics of location between the applicativized and  non-applicativized verbs are distinct, and the verb \emph{kw-icara} `to sit' behaves differently than other applicativized verbs discussed in this section. {\comment say more...? maybe just cut this? Reviewer two says: "it may be due to the deletion of the applied object from the surface syntax. The author should consult A\&M 1993 already cited in this article"}
\fi

\section{Verb class interactions} %%%%%%%%%%%%%%%%%%%%%%%%%%%%%%%%%%%%%%%%%%%%%%%%%%%%%%
In the previous section I showed that verbs differ in the kinds of locative arguments they take, with four general classes: verbs where the applied object is a general {\sc locative}, a {\sc goal}, a {\sc path}, or a {\sc source}.  In this section I show that the four interpretations of the locative applicative described in the previous section do not appear arbitrarily, but rather the interpretation of the applied argument is contingent upon the semantic class of the verb. 

 
 The {\sc goal} applied object is reserved for verbs of manner of motion, such as \emph{kw-iruka} `to run', \emph{gu-tembera} `to go about', and \emph{gu-simbuka} `to jump'. Evidence of the underlying {\sc goal}  with these verbs comes from the fact that the {\sc goal} can be licensed by the verb without an applicative.
		\iffalse
		\begin{exe}
			\ex\gll Yohani y-iruts-e i-ruhanda.\\
					John 	{\sc 1-}run-{\sc perf} 5-road\\
					\glt `John ran on the road.'
			\end{exe}
			\fi %what in the literal fuck is this here for? this shows that there's a path....? but i forget why that's important
		 Many manner of motion verbs are ambiguous between a static location reading and a change of location reading. For example, \emph{gu-simbuka} `to jump' in (\ref{jumpmfer}) can be coerced into having a {\sc goal}  reading independently of the applicative.		
		\begin{exe}
			\ex\label{jumpmfer}\gll Yohani y-a-simbuts-e mu ma-zi.\\
				John 	1-{\sc pst-}jump-{\sc perf} 18 6-water\\
				\glt `John jumped while in the water.'
				\glt `John jumped into the water.'
		\end{exe}
%
 The applicativized variant in (\ref{cello}), however, requires that the locative is a {\sc goal}:
			\begin{exe}
			\ex\label{cello} \gll Yohani y-a-simbuk-iy-e mu ma-zi.\\
				John 1-{\sc pst}-jump-{\sc appl-perf} 18 6-water\\
				\glt `John jumped into the water.'
				\glt \bad (`John jumped while in the water.')
			\end{exe}
	%
The ability for the locative in (\ref{jumpmfer}) to be interpreted as a {\sc goal}  is a pattern attested in several unrelated languages where manner of motion verbs which convey or entail displacement are coercible to have a {\sc goal}  reading \citep{nikitina:2008,thametal:2012,bassa:2013}.\footnote{An anonymous reviewer asks if there are other possibilities besides these patterns, specifically questioning whether the applied object with a verb like \emph{kw-iruka} `to run' could be a {\sc source} and not a {\sc goal}, i.e. to mean something like `run away from X'. Consultants rejected this reading. What consultants do allow, however, is that an applicativized manner of motion predicate can have a `toward' interpretation. For example, a sentence like that in (\ref{cello}) could mean `John jumped toward the water' instead of `John jumped into the water'. I assume that the {\sc goal}  in these cases is prospective; the subject is entailed to move in the direction of the {\sc goal} , but it is not necessarily the case that the subject arrives, which subsumes both the `to' and `towards' readings.} 
 		
 When the applicative licenses a {\sc path} , it is with so-called path verbs, such as \emph{kw-injira} `to enter', \emph{gu-sohoka} `to exit', \emph{ku-manuka} `to descend', \emph{kuzamuka} `to ascend', and \emph{ku-rira} `to climb'. The {\sc source} applied object is restricted to the verb of traversal \emph{kw-ambuka} `to cross'.


With verbs that encode no location in their meaning, the applicative licenses a general location, and the applicative morpheme is obligatory for licensing a {\sc locative} with such verbs. For example, consider the verb \emph{ku-vuga} `to talk'; this verb does not license a location, as shown by the inability of the non-applicativized verb to appear with locative clitics, as in (\ref{twigs}b). 

\begin{exe}
	\ex\label{twigs}\begin{xlist}
		\ex\gll Habimana a-ri ku-vug-a.\\	
				Habimana 1S-{\sc be} {\sc inf}-talk-{\sc imp}\\
				\glt `Habmiana is talking.'
	\ex\gll Habimana  a-ri ku-vug-a(=*ho/yo/mo).\\
						Habimana {\sc 1-cop} {\sc inf-}talk-{\sc imp=loc}\\
						\glt `Habimana is talking (there).
\end{xlist}
\end{exe}
%
The applicativized variant, however, does in fact permit locative clitics, as in (\ref{sade}). The use of different locatives is contingent upon context; in this example, \emph{=mo} would mean that the subject is speaking inside of a location (e.g. his house), while \emph{=yo} means that he is speaking at a general location (e.g. a park).\footnote{The clitic \emph{=ho} is also permissible with the applicativized variant of \emph{ku-vuga} `to talk', but it does not have a literal locational interpretation. Rather, it means that the subject is using something to talk, such as his cellphone.}

\begin{exe}
		\ex\label{sade}\gll Habimana a-ri ku-vug-ir-a(=mo/yo).\\
			Habimana 1S-{\sc be} {\sc inf}-talk-{\sc appl-imp=loc}\\
				\glt `Habimana is talking there.' 
	\end{exe}

In this section, I have shown that the different meanings encoded by the locative clitic are conditioned by the verb to which the applicative attaches. Table \ref{tab:verb} summarizes the thematic role of applied objects which are present with the different verbs. Note that this typology reflects all of the logically possible components of a motion event ({\sc source, path, and goal}), though which component of the motion event is brought out as the applied object crucially depends on the class of verb. When there is no motion in the meaning of the verb, the default interpretation is that the {\sc locative} is a general location. 
\begin{table}
\begin{center}\caption{\textbf{Verb classes and corresponding applied object meaning}}
\begin{tabular}[t]{lll}\\\hline
 Role of the Applied Object		& Verb Type		& Example \\\hline

{\sc goal}					& manner of motion 	& \emph{kw-iruka} `to run' \\

{\sc path} 					& change of location	& \emph{kw-injira} `to enter'\\

{\sc source}			 	& traversal			& \emph{kw-ambuka} `to cross' \\

{\sc  locative}		& no location encoded by verb & \emph{ku-vuga} `to talk' \\\hline

\end{tabular}
	 
\label{tab:verb} 
	 		 \end{center}
\end{table}





 


\section{Towards an analysis of locative applicatives}%%%%%%%%%%%%%%%%%%%%%%%%%%%%%%%%%%%%%%%%%%%%%%%%%%%%%%
%%%%%%%%%%%%%%%%%%%%%%%%%%%
%%%%%%%%%%%%%%%%%%%%%%%%%%%
%%%%%%%%%%%%%%%%%%%%%%%%%%%

 

  The previous two sections have shown that the semantic role of a locative applied object is contingent upon the meaning of the verb. Most previous approaches assume that the applicative adds a new applied object with a corresponding {\sc locative} thematic role and thus do not have an obvious means for capturing the various locative semantic roles found with different classes of motion verbs.

I suggest instead an analysis of applicative morphology as a paradigmatic constraint which requires that the applicativized variant of a given verb has monotonically stricter set of lexical entailments than the non-applicativized variant, as proposed for various argument alternations in \citet{ackermanmoore:2001} and \citet{beavers:2010b}.\footnote{By \emph{monotonic} I mean that a new meaning is added without removing any prior meaning in the base predicate \citep{akg:2007, akg:2012}.} There are two ways that this semantic narrowing is present in sentences with a locative applicative. The first is that the applicative can add a wholesale new syntactic object absent from the meaning of the verb, as is the case with verbs that do not license a {\sc locative} in their non-applied form, such as \emph{ku-vuga} `to talk'  in (\ref{sade}) above. Second, the applicative can realize an argument that is selected for semantically but not realized syntactically by the non-applicativized verb. In the former case, the entailments of the applicativized variant are narrower by virtue of specifying a particular location where the event takes place; in the latter case, the entailments are narrower by virtue of naming a specific location that is semantically entailed to exist (but not syntactically licensed) in the non-applicativized variant. 
 
 Thus the applicativized variant should include all of the information in the non-applicativ- ized variant with additional semantic information pertaining to an argument of the predicate. In order to make this distinction precise, I assume a neo-Davidsonian style semantics, where each participant is linked to the event by a specific thematic role ({\sc agent, theme, path, goal,} etc.). Due to restrictions of length, I do not give a fully articulated formalization of the mapping between syntax and semantics, but crucially, any entity linked to a thematic role is a syntactically realized argument. For example, the notation \emph{ag(john', e)} states that  \emph{john} is the {\sc agent} of the event $e$.
	 
 For verbs that do not have a {\sc locative} in the non-applicativized variant, a {\sc locative} is encoded by the applied object, narrowing the truth-conditional content by describing a location at which the event took place. As shown above in (\ref{sade}), the verb \emph{ku-vuga} `to talk' does not select for a {\sc locative}. From this, I assume the denotation in (\ref{feminist}) for \emph{kuvuga} `to talk'.
 			\begin{exe}
			\ex\label{feminist} $\exists e.[talk'(e)\ \&\ ag(john',e)]$
			\end{exe}
	%
	This denotation states that there is a talking event and that John is the {\sc agent} of that event. The applicativized variant licenses a locative object, which specifies a location at which the event takes place. 
		\begin{exe}
		\ex\begin{xlist}
		\ex\label{whatsapp}\gll Yohani a-ri ku-vug-ir-a mu nzu.\\
			John 1S-{\sc be} {\sc inf}-talk-{\sc appl-imp} 18 house\\
				\glt `John is talking in the house.'
		\ex $\exists e.[talk'(e)\ \&\ ag(john',e)\ \&\ loc(house',e) ]$
	\end{xlist}
	\end{exe}
	%
In (\ref{whatsapp}) --- repeated from (\ref{talkapp}b) --- the sentence has the same truth-conditions as in (\ref{feminist}), but specifies an additional locative participant. For considerations of space, I do not provide a fully-articulated analysis of how the applicativized variant is derived from the non-applicativized variant.\footnote{See \citet{jerro:2016}, Chapter 3, for a more fully articulated analysis of locative applicatives.} The crucial point here is that the lexical entailments of the applicativized predicate are narrower than those of the non-applicativized predicate by virtue of the additional locative participant.	

 Verbs which denote a location in their denotation (i.e. verbs of directed motion, as discussed above) do not add a general locative applied object. In these sentences, the applicative is used to bring out a locative participant present in the meaning of the non-applicativized verb, which has the effect of narrowing the truth-conditional content by naming a specific location. For example, the verb \emph{kw-injira} `to enter' denotes a {\sc path}, though this is not realized syntactically with the base verb, as shown in (\ref{enter}) where the {\sc  path} is existentially bound.\footnote{Technically, all verbs of directed motion entail the presence of a {\sc source, path,} and {\sc goal}. Why a particular component of motion is preferred with distinct verbs is left for future research. In the present discussion, I assume that verbs of different classes lexically specify a given component of a motion event that is brought out by applicativization.}  %Here, I notate the presence of a semantically implicit location information by introducing a location (\emph{l}) with an existential quantifier. %When an element is syntactically overt, it is bound off with a specified location (in the case of (\ref{mackwell}), a path).   


\begin{exe}
	\ex\label{enter}\begin{xlist}
		\ex\gll Yohani a-ri kw-injir-a mu nzu.\\
				John 1{\sc-be} {\sc inf-}enter-{\sc imp} 18 house\\
				\glt `John entered the house.'
		\ex $\exists e\exists l.[enter'(e) \ \&\ ag(john', e)  \ \&\ th(house', e)  \ \&\ path(l,e)]$
		\end{xlist}
		\end{exe}
		%
With the applicativized variant of  \emph{kw-injira} `to enter', the {\sc path}  participant which is existentially bound in (\ref{enter}) is  instead licensed as a syntactic argument, as in (\ref{mackwell}) where  the {\sc path}  is realized overtly as the applied object.
	 		%		
	\begin{exe}
	\ex\label{mackwell}\begin{xlist}
		\ex\gll Yohani a-ri kw-injir-ir-a mu muryango mu nzu.\\
				John 1{\sc-be} {\sc inf-}enter-{\sc appl-imp} 18 door 18 house\\
				\glt `John entered through the door.'
	 \ex $\exists e.[enter'(e) \ \&\ ag(john', e)  \ \&\ th(house', e)  \ \&\ path(door', e) ]$
	\end{xlist}
\end{exe}
%
%{\comment a better example is the jump example from above}

%%%%%%%%%%%%%%%%%%%%%%%%%%%%%%%%%%%%%%%%%%%%%%%%%%%%%%%%%%%%%%%%%%%%%%%%%%%%%%%%%%%%%%%%%%%%%%%%%%%%%%%%%%%%%%%%%%%%%%%%%%%%%%%%%%%%%%%%%%%%%%%%%%%%%%%%%%%%%%%%%%%%%%%%%%%%%%%%%%%%%%%%%%%%%%%%%%%%%%%%%%%%%%%%%%%%%%%%%%%%%%%%%%%%%%%%%%%%%%%%%%%%%%%%%%%%%%%%%%%%%%%%%%%%%%%%%%%%%%%%%%%%%%%%%%%%%%%%%%%%%%%%%%%%%%%%%%%%%%%%%%%%%%%%%%%%%%%%%%%%%%%%%%%%%%%%%%%%%%%%%%%%%%%%%%%%%%%%%%%%%%%%%%%%%%%%%%%%%%%%%%%%%%%%%%%%%%%%%%%%%%%%%%%%%%%%%%%%%%%%%%%%%%%%%%%%%%%%%%%%%%%%%%%%%%%%%%%%%%%%%%%%%%%%%%%%%%%%%%%%%%%%%%%%%%%%%%%%%%%%%%%%%%%%%%%%%%%%%%%%%%%%%%%%%%%%%%%%%%%%%%%%%%%%%%%%%%%%%%%%%%%%%%%%%%%%%%%%%%%%%%%%%%%%%%%%%%%%%%%%%%%%%%%%%%%%%%%%%%%%%%%%%%%%%%%%%%%%%%%%%%%%%%%%%%%%%%%%%%%%%%%%%%%%%%%%%%%%%%%%%%%%%%%%%%%%%%%%%%%%%%%%%%%%%%%%%%%%%%%%%%%%%%%%%%%%%%%%%%%%%%%%%%%%%%%%%%%%%%%%%%%%%%%%%%%%%%%%%%%%%%%%%%%%%%%%%%%%%%%%%%%%%%%%%%%%%%%%%%%%%%%%%%%%%%%%%%%%%%%%%%%%%%%%%%%%%%%%%%%%%%%%%%%%%%%%%%%%%%%%%%%%%%%%%%%%%%%%%%%%%%%%

The analysis presented so far has shown that an applicativized variant requires stricter truth-conditional content on an argument, which is satisfied by either adding an new locative object or syntactically licensing a participant that is only semantically entailed by the meaning of the verb. This analysis subsumes the object-adding function that has been the focus of the mainstay of research on applicatives and additionally provides a framework of analysis for discussing the applied objects found with particular motion verbs. A further possibility in this semantically-oriented analysis is that the applicativized variant need not necessarily license an additional object, but may change the semantic nature of an existing thematic object of the verb, provided there is a stricter semantic meaning in the applicativized variant, akin to the paradigmatic argument alternations discussed in \citet{ackermanmoore:2001} and \citet{beavers:2010b}. I turn now to a brief discussion of this use.

First, I propose that the thematic roles of {\sc goal} and {\sc recipient} are in the appropriate relation of restricted truth conditions; namely, a {\sc recipient} has all the entailments of a {\sc goal}, but with the additional meaning of prospective change of possession. Consider the definitions of {\sc goal} and {\sc recipient} in (\ref{define}).

 \begin{exe}
 	\ex\label{define}\begin{xlist} 
		\ex {\sc goal}: a place to which motion is directed 
		\ex {\sc recipient}: a place to which motion is directed + prospective change of possession
	\end{xlist}
\end{exe}
With these definitions in mind, the theory developed so far predicts that with verbs that license a {\sc goal}, it should be possible to satisfy the output condition by  `narrowing' the {\sc goal} to {\sc recipient} --- without modifying the argument structure of the verb. Consider the verb \emph{gu-tera} `to throw', which is ditransitive in its non-applied form and has a {\sc goal} indirect object.\footnote{I use the term \emph{indirect object} to describe the {\sc goal/recipient} object of a double object construction (see \citealt{beavers:2011a} for comparable terminology).}

\begin{exe}
\ex\label{pelt}\gll Karemera y-a-tey-e i-buye Nkusi.\\
	Karemera 1-{\sc pst-}throw-{\sc perf} 5-rock Nkusi\\
	\glt `Karemera threw the rock at Nkusi.'
\end{exe}
 The sentence in (\ref{pelt}) has the reading that Karemera is throwing a rock directly at Nkusi, possibly trying to harm him and, crucially, without the intention of giving Nkusi possession of the rock. The denotation of this sentence is as follows:
\begin{exe}
\ex\label{bernie} $\exists e.[throw'(e) \ \&\ ag(karemera', e) \ \&\ th(rock', e) \ \&\ goal(nkusi', e)]$ 

\end{exe}

Given the relationship of {\sc goals} and {\sc recipients} assumed above, the constraint that the applicativized variant has stricter truth conditions is satisfied by the change of the {\sc goal} participant in (\ref{bernie}) to a {\sc recipient}, given (\ref{define}).  For the verb \emph{gu-tera} `to throw', this is precisely the meaning of the applicativized variant, as shown in (\ref{recipe}). 

\begin{exe}
	\ex\label{recipe}\gll Karemera y-a-ter-ey-e i-buye Nkusi.\\
					Karemera 1-{\sc pst-}throw-{\sc appl-perf}	5-ball Nkusi\\
					\glt `Karemera threw the rock to Nkusi.'
\end{exe}
%
 Crucially, the required reading in (\ref{recipe}) is one where Karemera is attempting to give Nkusi possession of the rock and not that Karemera is throwing the rock at Nkusi. In this sentence, Nkusi is not just the {\sc goal} of the throwing event, but also the {\sc recipient} of a prospective change of possession. The denotation of the sentence in (\ref{recipe}) is that in (\ref{sanders}).

\begin{exe}
\ex\label{sanders}  $\exists e.[throw'(e) \ \&\ ag(Karemera', e) \ \&\ th(rock', e) \ \&\ rec(Nkusi', e)]$ 
\end{exe}

Evidence for the meaning difference between the two comes from the fact that prospective catching can only be modified when the applied object has been narrowed to a {\sc recipient}, as in (\ref{danny}a), where the conjunction \emph{ariko} `but' is used to contrastingly deny the reception of the ball.  In (\ref{danny}b), on the other hand, where there is no applicative, the object is not a {\sc recipient}; thus it is infelicitous to modify any notion of Nkusi attempting to catch the rock. 

\begin{exe}
	\ex\label{danny}\begin{xlist}
	\iffalse
		\ex\gll Karemera y-a-tey-e i-buye Nkusi, ariko Karemera a-ra-mu-hush-a.\\
				Karemera {\sc 1S-pst}-throw-{\sc perf} 5-rock Nkusi, but Karemera {\sc 1S-pst-1O}-miss-{\sc imp}\\
				\glt `Karemera threw the rock at Nkusi, but Karemera missed.'
	\fi
		\ex\gll  Karemera y-a-ter-ey-e i-buye Nkusi, ariko Nkusi nti-y-a-ri-fash-e.\\
				Karemera 1S{\sc-pst-}throw-{\sc appl-perf} 5-rock Nkusi, but Nkusi {\sc neg-1S-pst-5O}-catch-{\sc perf}\\
		\glt `Karemera threw the rock to Nkusi, but Nkusi didn't catch it.'
	\ex\gll \#Karemera y-a-tey-e i-buye Nkusi, ariko Nkusi nti-y-a-ri-fash-e.\\
			Karemera {\sc 1sg-pst-}throw-{\sc perf} 5-rock Nkusi but Nkusi {\sc neg-1S-pst-5O}-catch-{\sc perf}\\	
		\glt `Karemera threw the rock at Nkusi, but Nkusi didn't catch it.'
\end{xlist}
\end{exe}

Further evidence is that in the presence of another applied object (such as {\sc locative}), only the {\sc goal} reading is possible with the applicativized variant of \emph{gu-tera} `to throw'; here, the applicative is used to license a {\sc locative} object, leaving the lexical entailments of the indirect object unchanged. Consider the example in (\ref{manchester}). Given that the applicative licenses the {\sc locative}, it is expected that in (\ref{manchester}), \emph{Nkusi} is the {\sc goal} and not the {\sc recipient} since the applicative is not being used to narrow the entailments associated with the {\sc goal}. 

\begin{exe}
\ex\label{manchester}\gll Karemera y-a-ter-ey-e i-buye Nkusi mu nzu.\\
		Karemera {\sc 1S-pst-}throw-{\sc appl-perf} 5-rock Nkusi in 9.house\\
		\glt `Karemera threw the rock at Nkusi in the house.'\\
%			$\to$ Interpretation: All participants are in the house, and Nkusi is not the recipient.
	\end{exe}
	%
	Thus in (\ref{manchester}) the reading is that the rock is thrown \emph{at} Nkusi, and not that it is thrown \emph{to} him.
		
	\iffalse  It could be argued that the applicative is not being used to license the locative, as I claim, but rather that the applicative licenses the goal in this sentence.  Evidence that the applicative is being used to license the locative comes from the scope of the locative phrase. In some sentences, there is a contrast between the scope of the location with applicativized and  non-applicativized variants of the same sentence. For example, in (\ref{x}a), the locative PP scopes over the object NP, and optionally the subject. When the applicative is used, as in (\ref{x}b), however, both the subject and the object are obligatorily in the location described by the PP.

\begin{exe}
\ex\label{x}\begin{xlist}
\ex\gll Umuyobozi y-ubats-e inzu mu mujyi.\\
		chief {\sc 1S-}build-{\sc perf} house in town\\
		\glt `The chief built the house in town, (and the chief's location is not explicit).'

\ex\gll Umuyobozi y-ubak-iy-e inzu mu mujyi.\\
		chief		{\sc 1S}-build-{\sc appl} house in town\\
		\glt `The chief built the house in town, (and the chief is obligatorily in the town).'
\end{xlist}
\end{exe}

 The scope differences in (\ref{x}) arise because of the syntactic structure; in (a), the PP \emph{mu mujyi} `in town' modifies the NP \emph{inzu} `house', while in (b), it is an object of the verb. As a complement to the verb, it takes scope over the entire sentence. Evidence that the locative phrase in (\ref{manchester}) is a true applied object to the verb is that the only interpretation is one where all participants are in the house, showing that \emph{mu nzu} `in the house' is the object to the verb. The fact that this is the interpretation in (\ref{manchester}) ---  that all participants are in the house  --- shows that the applicative is being used to license the locative, which correctly predicts that the direct object is obligatorily a goal, and not a recipient. 
\fi

\iffalse
  

\subsection{A Potential Counterexample}

 

 In Lubukusu (Bantu, Western Kenya), there is an alternation with the applicativized and  non-applicativized variants of the verb where applicativized variant changes the directionality of the location. With the non-applicativized verb in (\ref{send}a), the object PP is understood as the goal, while in the applicativized variant in (\ref{send}b), the PP is understood as the source.\footnote{I am grateful to Justine and Hellen Sikuku for their native speaker judgments on the Lubukusu data.}

\begin{exe}
\ex\label{send} 
\begin{xlist}
\ex\gll $\emptyset$-a-rum-a ebarwa mu posta.\\
			{\sc 1S-pst-}send-{\sc imp} letter in post.office\\
			\glt `S/he sent the letter \emph{to} the post office.'
\ex\gll $\emptyset$-a-rum-er-a ebarwa mu posta.\\
			{\sc 1S-pst-}send-{\sc loc-imp} letter in post.office\\
			\glt `S/he sent the letter \emph{from} the post office.'\hfill (Lubukusu)
			\end{xlist}
\end{exe}
%
 {\sc source} and {\sc goal}  locatives (\emph{from} and \emph{to}) do not lie in the same relationship as recipients and goals discussed above, but the data in (\ref{send}) show that the applied sentence indicates a source, while the non-applied sentence indicates a goal. The potential problem is that \emph{from} is not a semantically narrowed interpretation of \emph{to}, in the way that a recipient was argued above to be narrowed to a goal. This is potentially problematic for the analysis of semantic narrowing analysis because there is no semantic narrowing with the applied sentence. 
 
 However, an alternative analysis is that the applicativized variant in (\ref{send}b) is creating the same scope over the subject as in (\ref{x}b) above; the {\sc source} reading difference arises not due to semantic narrowing, but rather because the locative, when applied, has scope over the entire sentence, i.e. the subject is in the post office as well as the letter. This analysis naturally captures the data in (\ref{send}) by analyzing this situation as a change in the scope of the locative.
 \fi
 \section{Conclusion}%%%applicativized variant
 
  
 In this paper, I have shown that the semantic role of an applied object may be contingent upon the meaning of the verb, and I have described a case in Kinyarwanda where the use of the applicative does not license a new argument at all, but rather modifies the semantic role of an existing argument. In order to capture these facts, I proposed an analysis of applicativization as sensitive to a paradigmatic output condition. 
 
 Specifically, in \S3 I showed that there exist three classes of motion predicates where the applied object is assigned the role of {\sc source, path,} or {\sc goal,} respectively, and in \S4 I showed that  these applied objects appear with verbs of traversal, path verbs, and manner of motion verbs, respectively. In \S5 I provided a preliminary account of applicativization as a paradigmatic output condition on the applicativized variant of a given verb where the predicate of the applicativized verb must have stricter lexical entailments associated with a particular argument than the non-applicativized verb. This analysis captures the typology of predicates presented in \S\S3-4 and makes the further prediction that certain verbs may not add a new argument at all under applicativization. I have not attempted an exhaustive account of applicativization in Kinyarwanda, but rather I have shown that applicativization cannot be analyzed simply as an operation which adds a whole new argument with an associated semantic role. Instead, I have proposed a framework for discussing applicatives which provides a more empirically predictive analysis of the uses of applicative morphology in Bantu. 
 
 
 \section*{Appendix A: Prepositions or class prefixes?}
 
 Bantu languages are well known for their gender class agreement, a class of prefixes that indicate singular and plural as well as other semantic distinctions such as animacy. Relevant to the discussion above is that there are classes reserved in many Bantu languages for locative expressions. For example, Chiche\^was has \emph{pa--}, \emph{ku--}, and \emph{mu--} (traceable to Proto-Bantu), which are referred to as classes 16, 17, and 18 in the literature \citep{bresnan:1989,bresnan:1994,bresnanmchombo:1995,maho:1999,marten:2010}. While Kinyarwanda orthographic conventions (which are adopted above) separate \emph{ku} and \emph{mu} from the following noun, an empirical question arises as to the status of these locatives. I show in this appendix that the locatives \emph{ku} and \emph{mu} in Kinyarwanda are in fact locative class markers and that nouns marked with locative class prefixes are arguments and not prepositional phrases (see Jerro 2013 and Jerro \& Wechsler 2015 for general discussion of agreement in Kinyarwanda).
 
 
First, locatives can appear as the subject of a passive, a position reserved for arguments. In (\ref{locpas}), the locative phrase \emph{mw' ishymaba} is the subject of a passivized verb, triggering subject agreement (cf. \citealt{bresnan:1989,bresnan:1994}). It is important to note that the subject agreement marker is from class 16 (the class for inherent locatives, such as \emph{aha-ntu} `a place' and \emph{ah-irengeye} `a high place'), and in Kinyarwanda and various other languages such as Kesukuma (\citealt{batibo:1985}: 245), any noun marked from class 16, 17, 18, or 23 (i.e. locative classes) triggers a subject marker from class 16  \citep{maho:1999}.


\begin{exe}

	\ex\label{locpas}\gll Mw' i-shyamba h-a-tem-e-w-e igi-ti n' umu-higi.\\
			{\sc 18} 5-forest 16S-{\sc pst}-cut-{\sc appl-pass-perf} 7-tree by 1-hunter\\
			\glt `In the forest was cut the tree by the hunter.'

\end{exe}
%
 Of crucial importance is that the subject triggers agreement on the verb, an agreement relation reserved for arguments.  



Furthermore, locative phrases can be object-marked on verbs, as shown in (\ref{locom}), where the class 16 object marker \emph{ha--} replaces the locative phrase. 

\begin{exe}
	\ex\begin{xlist}
	\ex\label{loco}\gll N-a-bon-ey-e umw-ana mw' i-shyamba.\\
					1{\sc sg}S-{\sc pst}-see-{\sc appl-perf} 1-child 18 5-forest\\
					\glt `I saw the child in the forest.'
					
	\ex\label{locom}\gll N-a-ha-bon-ey-e umw-ana.\\
			1{\sc sg}S-{\sc pst-16O}-see-{\sc appl-perf} 1-child\\
			\glt `I saw the child there.'
	\end{xlist}
\end{exe}

 
 
 The final piece of evidence that locative phrases are arguments is that they cannot appear productively across predicates, which would be expected if the locative prefixes were in fact prepositions that license oblique phrases. For example, in (\ref{stravinsky}) the locative phrase \emph{mu nzu} `in the house' cannot be used with the verb \emph{ku-vuga} `to talk'.\footnote{Some verbs do select a locative argument, but I this is selection is on a case-by-case basis and not productive across verbs.}
 \begin{exe}
\ex\label{stravinsky}\gll Habimana a-ri ku-vug-a (*mu n-zu).\\
			Habimana 1S-{\sc be} {\sc inf}-talk-{\sc imp} \phantom{(*}18 9-house\\%)
				\glt `Habimana is talking (*in the house).'
\end{exe}
In order to have a locative phrase such as \emph{mu nzu} `in the house' with the verb \emph{ku-vuga} `to talk', the applicative is obligatory, as in (\ref{stravinsky}).
\begin{exe}
\ex\label{stravinsky}\gll Habimana a-ri ku-vug-ir-a mu n-zu.\\
			Habimana 1S-{\sc be} {\sc inf}-talk-{\sc appl-imp} 18 9-house\\
				\glt `Habimana is talking in the house.'
\end{exe}
 From these diagnostics, I conclude that  locative phrases in Kinyarwanda are class-marked arguments and not oblique phrases. 
 
\iffalse 
  As a final note, investigating the semantics of verbs may shed light on the complex question of  object symmetry mentioned in \S2. 
 
  Preliminary data from Lubukusu show that ingestive verbs pattern differently than other verb classes with respect to symmetry. 
 
  The general pattern appears to be that morphological causatives are asymmetrical, where the thematic object cannot undergo objecthood tests such as passivization.
 
  
 \begin{exe}
%
\ex\label{anything2}
\begin{xlist}
\ex\gll Omu-khangarani $\emptyset$-a-p-isi-bw-a li-sisi ne omw-ekesi.\\
	{\sc 1-}warrior {\sc 1S-pst-}hit-{\sc caus-pass-imp} {\sc
          5-}wall by {\sc 1}-teacher\\
	\glt `The warrior was made to hit the wall by the teacher.'

\ex \gll {\bad}Li-sisi ly-a-p-isi-bw-a omu-khangarani ne omw-ekesi.\\
		{\sc 5}-wall {\sc 5S-pst}-hit-{\sc caus-pass-imp} {\sc 1-}warrior by {\sc 1-}teacher\\
		\glt `The wall was made to be hit by the warrior by the teacher.'
		

\end{xlist}
\end{exe}
%
 However, with ingestives such as \emph{kunywa} `drink' the pattern is
symmetrical:
%
\begin{exe}
\ex\label{beer}\begin{xlist}
\ex\gll Kyle $\emptyset$-a-nyw-esy-ebw-a kamalwa ne Mama Leo.\\
		Kyle {\sc 1S-pst-}drink-{\sc caus-pass-imp} beer by Mama Leo\\
		\glt `Kyle was made to drink the beer by Mama Leo.'
\ex\gll Kamalwa k-a-nyw-esy-ebw-a Kyle ne Mama Leo.\\
			beer {\sc 6S-pst-}drink-{\sc caus-pass-imp} Kyle by Mama Leo\\
			\glt `The beer was made to be drunk to Kyle by Mama Leo.'
\end{xlist}
\end{exe}

 These data show that verb class can have an affect on the symmetry patterns within a specific language, which may answer many of the unresolved questions in the literature on object symmetry. 

\fi
  

 
 
\iffalse 
appl constraint: 
if $\beta$ is the sum of the lexical entailments of the arguments \{x_1 ... x_n\} of a verb, and $\phi$ = $\alpha$($\beta$), $\phi$ = 1 iff $\phi \noteq \beta$ and 
\fi 



\subsection*{Acknowledgements}
I am grateful to John Beavers, Eric Campbell, Bernard Comrie, Michael Diercks, Daniel Hieber, Larry Hyman, Scott Myers, Stephen Wechsler, the audiences of ACAL 46 and the UC Santa Barbara NAIL Roundtable, two anonymous reviewers, and the volume editors for helpful discussion on the points developed in this paper. I also express my deepest gratitude to all Rwandese people who have shared their intuitions on Kinyarwanda with me over the years, and especially Gilbert and F\'elicit\'e Habarurema for their judgments on most of the data presented here. This paper would not have been possible without their patience. All errors remain the fault of the author.


\subsection*{Abbreviations}
\begin{tabular}{ll}
 1 \emph{to} 23 &  gender class prefixes\\
 1{\sc sg} &  first-person singular\\
 {\sc asp} & aspect\\
 {\sc appl} & applicative\\
 {\sc ben}  & benefactive applicative\\
 {\sc fv} & final vowel\\
 {\sc imp} &  imperfective\\
 {\sc inf} &  infinitive\\
 {\sc loc} &  locative\\
  {\sc O} &  object marker\\
  {\sc perf} &  perfective\\
  {\sc pst} &  past tense\\
  {\sc pl} & plural\\
  {\sc pres} & present tense\\
  S  & subject prefix\\
  {\sc sg} & singular\\
\end{tabular}
% \bibliography{dissrefscopy}





\section{Benefactive and Recipient Applicatives} %%%%%%%%%%%%%%%%%%%%%%%%%%%%%%%%%%%%%%%%%%%%%%%%%%%%%%

 
 The applicative can be interpreted as a recipient or beneficiary when the noun is animate, though the interpretation is constrained based on the semantics of the verb. 

% \underline{Generalization}: The applicative upgrades a goal to a recipient or adds a new argument (interpretable as a recipient or benefactive, depending on the verb). 

 Verbs that do not semantically encode a recipient interpret the object as a beneficiary.

\begin{exe}
	\ex\label{break} {\sc kumena}: break
	\begin{xlist}
	\ex\gll N-a-menn-ye igikombe.\\
			{\sc 1sgS-pst-}break-{\sc perf} cup\\
			\glt `I broke the cup.'
	\ex\gll N-a-men-ey-e igikombe John.\\
			{\sc 1sgS-pst-}break-{\sc appl-perf} cup John\\
			\glt *`I broke the cup to John.'\hfill *recipient
			\glt `I broke the cup for John.'\hfill $\surd$ benefactive
\end{xlist}
\end{exe}

 Verbs that semantically have a recipient can optionally encode the animate object as a beneficiary or recipient.
\begin{exe}
	\ex{\sc kujugunya}: throw%%%%%
\begin{xlist}
		\ex\gll N-a-juguny-e umupira.\\
				{\sc 1sg-pst}-throw-{\sc perf}  ball\\
				\glt `I threw the ball'\hfill base
		\ex\gll N-a-jugun-iy-e John umupira.\\
				{\sc 1sg-pst}-throw-{\sc appl-perf} Mary ball\\
				\glt `I threw Mary the ball.'\hfill $\surd$recipient
				\glt `I threw the ball for Mary.'\hfill $\surd$benefactive
		\end{xlist}
\ex {\sc koherera}: send
\gll N-$\emptyset$-oher-er-eje ibarwa Mary.\\
		{\sc 1sgS-pst-}send-{\sc appl-perf} letter Mary\\
		\glt `I sent the letter to Mary.'\hfill $\surd$recipient 
		\glt `I sent the letter for Mary.'\hfill $\surd$benefactive 
\end{exe}


 If the non-applicativized variant verb semantically selects for a goal, that goal can be upgraded to a recipient:

\begin{exe}
\ex\label{rock}{\sc gutera}: throw
	\begin{xlist}
\ex\gll Kyle yateye ibuye John.\\
	Kyle throw rock John\\
	\glt `Kyle threw the rock at John.'\hfill base
	\ex\label{recipe}\gll Kyle yater-ey-e umupira John.\\
	Kyle throw-appl	ball John\\
	\glt `Kyle threw the ball to John.'\hfill $\surd$recipient
	\glt  `Kyle threw the ball for John.'	\hfill	$\surd$benefactive
\end{xlist}
\end{exe}

		 

			 Evidence that the goal is underlyingly present for \emph{gu-tera} `throw' is that on the benefactive interpretation, there is still a goal. 
			
			\begin{exe}
				\ex\begin{xlist}
					\ex\gll N-da-gu-ter-er-a kuri John.\\
							1{\sc sgS-pres-2sgO}-throw-{\sc appl-imp} to John\\
							\glt `I'll throw (it) to John for you.'
				\ex\gll N-da-wu-mu-gu-ter-er-a.\\
					1{\sc sgS-pres-cl3O-3sgO-2sgO}-throw-{\sc appl-imp}\\
					\glt `I'll throw it (e.g. the ball) to him for you.
			\end{xlist}
			\end{exe}
		
		 

 \emph{guha} `give' seems to have a narrower requirement to have a recipient on the bare verb; the applicative can only add a beneficiary. 

% It needs to be motivated that the non-applied goal is actually a goal, with tests on (1) whether ``John" in (16b) can be inanimate, and (2) whether it can be denied that the subject still has the ball. 


\begin{exe}
\ex	{\sc guha}: give%%%%%
\begin{xlist}
	\ex\gll N-a-ha-ye John igitabo.\\
		{\sc 1sg-pst-}give-{\sc perf} John book\\
		\glt `I gave John the book.'\hfill base
	\ex\gll U-$\emptyset$-m-p-er-e John iterefone.\\
		{\sc 2sg-pst-1sgO-}give-{\sc appl-perf} John telephone\\
		\glt `Give John the telephone for me.'\hfill $\surd$benefactive
		\glt *`Give the telephone to John.' \hfill *recipient
	\end{xlist}
\end{exe}


\section*{Abbreviations}
\section*{Acknowledgements}

\printbibliography[heading=subbibliography,notkeyword=this]

\end{document}