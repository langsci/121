\documentclass[output=paper]{langsci/langscibook} 
\title{Variation in the expression of information structure in eastern {Bantu} languages} 
\author{%
Steve Nicolle\affiliation{Canada Institute of Linguistics / SIL International} 
}
% \sectionDOI{} %will be filled in at production






\abstract{}

\maketitle
\begin{document}


\begin{abstract}
Despite significant typological similarities, eastern Bantu languages differ in how information structure is expressed, but much of this variation only becomes apparent when discourse considerations are taken into account. Using data from narrative texts in eleven eastern Bantu languages I highlight three parameters of variation. First, in some languages surveyed all topics must be left dislocated, while in others certain kinds of topics in specific discourse contexts may be right dislocated. Second, all languages surveyed express argument focus on non-subjects through cleft constructions and right dislocation. However, while argument focus can be expressed through right dislocation of the subject in most languages surveyed, it can be expressed only through the use of cleft constructions in three languages, one of which allows right dislocation of subjects in response to content questions. Third, whilst all of the languages have thetic (topicless) sentences with VS constituent order, most also allow SV constituent order in the orientation sections of narratives, and one language allows SV thetic sentences elsewhere as well.
\end{abstract}

\begin{itemize}
\item \section{1  Introduction}\end{itemize}

The eastern Bantu languages\footnote{The term \textit{e}\textit{astern} here indicates the geographical distribution of these languages within Africa rather than a genetic affiliation.} are very similar as far as general morphosyntactic parameters are concerned: they all exhibit typical Bantu noun class systems with obligatory subject marking on verbs, optional object marking, and S{\textgreater}V{\textgreater}IO{\textgreater}DO{\textgreater}X constituent order in pragmatically neutral clauses. However, they also exhibit various kinds of morphosyntactic variation (Marten, Kula \& Thwala 2007; van der Wal \& Biberauer 2014) including variations in constituent order due to different instantiations of information structure (Zerbian 2006; Buell, Riedel \& van der Wal 2011; Yoneda 2011; Downing \& Hyman forthcoming). Most studies of information structural variation in Bantu languages to date have relied predominantly on data collected through elicitation. Such data is well suited to isolating salient interpretational effects in individual sentences, but is less appropriate when investigating the effect of larger stretches of discourse on constituent order variation. This methodological bias is found in studies of African languages more generally, so that, for example, Güldemann, Zerbian \& \citet[157]{Zimmermann2015} in their review of research into information structure (IS) in African languages, stated, “we largely restrict the discussion to IS phenomena that hold within a sentence and exclude those in discourse units above this level.” In part, this situation reflects a lack of available text corpora. Mwamzandi’s (2014) study of variation in adnomial demonstratives and reciprocal constructions in Swahili is one of the few investigations into information structure in a Bantu language that is based on an extensive text corpus (the Helsinki Corpus of Swahili, consisting of 12.5 million words from written Swahili texts).

In this paper, I attempt to complement existing research into information structure in Bantu languages in two ways. First, this study is based on the expression of information structure in narrative texts rather than in elicited data.\footnote{Unless otherwise stated, all examples are from texts that were either published in the sources listed in \tabref{tab:1} or used in their preparation. All examples are in the orthographies used in the published sources.} There are a number of limitations but also benefits to be derived from using narrative texts as data sources. The most important limitations are that a) not every type of construction (contrastive topic, given topic, contrastive focus, corrective focus, verum focus, etc.) is attested in each text corpus, and texts representing other genres could well reveal additional constructions and parameters of variation; b) corpus data do not yield negative evidence – that is, textual data only reveals what is possible, not what is impossible (that is, ungrammatical or pragmatically implausible); and c) prosody is only available in cases where a narrative was transcribed from an oral source and that source is still available (punctuation only represents prosody indirectly and imperfectly, and there is no guarantee that prosody will be appropriately reproduced when written narratives are read aloud). The main benefits of a textual approach are that it reveals patterns beyond the sentence, and it is not dependent on speakers’ judgements, invented contexts, or translation. The second way in which this study is designed to complement existing research is that it describes variation in information structure in eleven Bantu languages, thereby complementing more detailed studies of individual languages.\footnote{The current study is part of a larger investigation into narrative discourse in eastern Bantu languages \citep{Nicolle2015b}.} These languages and the associated data sources are listed in \tabref{tab:1}.

\begin{tabular}{lllll}
\lsptoprule
 Language & Classification & Location & Data & Source\\
 Fuliiru & [flr] DJ63 & South Kivu, Democratic Republic of Congo & 153 texts studied; data drawn from 13 (1,000 clauses approximately) & Van Otterloo (2011; 2015)\\
 Digo & [dig] E73 & Kwale District, Kenya & 5 lightly edited oral texts and 2 written texts (864 clauses) & \citet{Nicolle2015a}\\
 Jita & [jit] EJ25 & Mara Rural District, Tanzania & 10 lightly edited oral texts (1,096 clauses) & Pyle \& \citet{Robinson2015}\\
 Kwaya & [kya] EJ251 & Mara Rural District, Tanzania & 10 lightly edited oral texts (1,015 clauses) & \citet{Odom2015}\\
 Suba-Simbiti & [ssc] EJ403 & Mara Rural District, Tanzania & 8 lightly edited oral texts (513 clauses) & \citet{Masatu2015}\\
 Kabwa & [cwa] EJ405 & Mara Rural District, Tanzania & 9 lightly edited oral texts and 2 written texts (530 clauses) & \citet{Walker2011}\\
 Rangi (Langi) & [lag] F33 & Kondoa District, Tanzania & 66 texts (3,200 clauses) & \citet{Stegen2011}\\
 Mwani & [wmw] G403 & Cabo Delgado Province and Quirimba archipelago, Mozambique & 7 texts (number of clauses not known) & \citet{Floor2005}\\
 Bena & [bez] G63 & Wanging’ombe District and Njombe District, Tanzania & 3 written and 3 lightly edited oral texts (674 clauses) & \citet{Broomhall2011}; \citet{Eaton2015a}\\
 Malila & [mgq] M24 & Mbeya Rural District, Tanzania & 10 lightly edited oral texts and 4 written texts (755 clauses) & \citet{Eaton2015b}\\
 Makonde & [kde] P23 & Makonde Plateau, Mozambique & 8 lightly edited oral texts (585 clauses) & \citet{Leach2015}\\
\lspbottomrule
\end{tabular}
%%please move \begin{table} just above \begin{tabular
\begin{table}
\caption{Languages included in this study}
\footnote{Languages are classified using the ISO 639-3 language code as cited in \textit{Ethnologue} (Lewis, Simons and Fennig 2015) and Maho’s (2003) updated version of Guthrie’s (1967-71) classification of the Bantu languages.}
\label{tab:1}
\end{table}

\begin{itemize}
\item \section{2  Information structural generalizations}\end{itemize}

Information structure concerns the way in which an utterance or text is structured to accommodate the (assumed) knowledge state of the addressees, thereby helping addressees to arrive at a coherent interpretation of the utterance or text. Information structure in eastern Bantu narrative texts is primarily expressed through variations in the relative order of subject, verb, object and oblique constituents in a sentence, although intonation, pauses and – in certain languages – focus markers also play a role. Three parameters of variation will be discussed in this paper:

\begin{itemize}
\item the position of the topic, specifically whether any topics can be right dislocated;
\item how argument focus on the subject is expressed;
\item how thetic sentences are expressed, specifically whether SV constituent order is possible.
\end{itemize}

Before looking in detail at how each of these parameters is instantiated in the data, I shall define the terms topic, focus and thetic sentence, and related concepts.

Although topic and focus are complementary notions, they belong to separate dichotomies (De Cat 2007: 66; Erteschik-Shir 2007: 42). The first dichotomy is topic-comment (corresponding to what \citet{Halliday1967} and Dik (1981; 1989) call \textit{theme-rheme}): the (conceptual) topic of a sentence is what a sentence provides information about, and the comment is whatever is predicated about the topic. Topic expressions\footnote{Henceforth \textit{topic} will be used to designate a topic expression (i.e. a linguistic form) and \textit{conceptual topic} will be used where the referent rather than the referring expression is intended.} may be syntactically integrated with the comment (internal topics) or syntactically non-integrated (external topics, also called \textit{topic frames} or \textit{themes}\footnote{Dik (1981: 129–144) makes a distinction between \textit{theme}, which is defined as an expression designating “a domain or universe of discourse with respect to which it is relevant to pronounce the following predication” (ibid. 130), and \textit{topic}, which is part of that predication.}). Both possibilities are illustrated in example (1) below. The context is that the animals have dug a well, but Hare did not help; Hare has repeatedly stolen water, so Tortoise plans to catch Hare and cut off his tail as a punishment. The topic of each clause is \emph{oyo} ‘that’, referring to Hare. In the first clause this is a non-integrated (external) topic, and in the second clause it is an integrated (internal) topic, as it is coreferential with the object prefix \emph{mu}\emph{\nobreakdash-}.

\ea\label{ex:}
\langinfo{}{}{Jita (‘Well’ text, line 33)}\\
\gll [Oyo]\textsubscript{TOPIC} [munane era ripanga,]\textsubscript{COMMENT}\\
that\_one give(\textsc{pl}).me just machete \\


\gll [oyo]\textsubscript{TOPIC} [enimugwata ara.]\textsubscript{COMMENT}\\
that\_one I.will.him.catch just \\


\glt \textup{‘}\textup{That one, just give me a machete, that one, I’ll just catch him.’}
\z

The second dichotomy is focus-presupposition: The presupposition of a sentence is the grammatically or lexically expressed information “which the speaker assumes the hearer already knows or is ready to take for granted at the time the sentence is uttered” \citep[52]{Lambrecht1994}. The focus is defined in contrast to the presupposition as the grammatically or lexically expressed information “which cannot be taken for granted at the time of speech” \citep[207]{Lambrecht1994} and which, moreover, is considered by the speaker to be the most important information in the sentence \citep[277]{Dik1989}. Within this broad definition, focus is often correlated with new information \citep[39]{Good2010}, where information can be new either in relation to the discourse as a whole or in relation to the predicate in question (referred to in Gundel \& Fretheim 2004 as referential givenness-newness and relational givenness-newness respectively), but also includes contrastive and identificational focus (see Gibson, Koumbarou, Marten \& van der Wal, to appear).\footnote{For a critical evaluation of the notion of focus as a universal linguistic category, see Matić \& \citet{Wedgwood2013}.}

Finally, thetic sentences are sentences in which neither the topic-comment dichotomy nor the focus-presupposition dichotomy is applicable. Thetic sentences are typically used to introduce new participants into a narrative, and to report events or situations in which neither the action nor the participants can be taken for granted, for example, in answer to a general question such as “What happened?” As such, they do not contain a topic. It has be argued that thetic sentences contain neither topic nor focus expressions \citep[755]{Yoneda2011}, however following Lambrecht (1994, 2000) and \citet[55]{Nicolle2015b}, we will assume that, since the whole sentence is informationally prominent (that is, it is asserted), a thetic sentence exhibits sentence focus.

\begin{itemize}
\item \begin{itemize}
\item \section{3  Obligatory and optional left dislocation of topics}\end{itemize}
\end{itemize}

The pre-verbal domain is restricted to lexical topics and non-focus subjects in all of the eastern Bantu languages surveyed; that is, focus elements may not occur pre-verbally. This may be true for most eastern Bantu languages (see Zerbian 2006; van der Wal 2009; Yoneda 2011), except for languages such as Kikuyu [E51], which has a pre-verbal focus marker that is derived historically from a copula construction (Schwarz 2003; 2007; van der Wal 2014). Pre-verbal topics are said to be \textit{left dislocated}, meaning that they occur outside of the clause nucleus. However, there is variation concerning whether all topic expressions must be left dislocated or whether certain topics may be right dislocated in specific discourse contexts.

\textit{Left dislocation} is used in a broad sense (as in Shaer, Cook, Frey \& Maienborn 2009) which subsumes both left dislocation proper and topicalization.\footnote{In this paper, \textit{topicalizat}\textit{ion} refers to a structural property and does not entail that the constituent exhibiting this property also functions as a topic in discourse. In this paper?/In the literature? the term \textit{fronting} is also used to refer to both topicalization and left dislocation proper (cf. Cohen 2009: 313).} Left dislocation proper involves a resumptive element, such as the class 9 object marker \textit{i}\textit{\nobreakdash-} in example (2) below, which is coreferential with the topic \textit{barabara} ‘road’. Topicalization involves a non-resumptive (‘gap’) construction, as in (3) where there is no object prefix corresponding to the topic \textit{pesa} ‘money’.

\ea\label{ex:}
\langinfo{}{}{ Digo \citep[237]{Nicolle2013}}\\
\gll [Barabara]\textsubscript{TOPIC} ndipho [a-ka-\textbf{i}{}-rich-a]\textsubscript{COMMENT}\\
     9.road then \textsc{3sg-seq}{}-9.\textsc{om}{}-leave-\textsc{fv}\\
\glt \textup{‘}\textup{The road, then, he left it.’}
\z

\ea\label{ex:}
\langinfo{}{}{ Digo \citep[228]{Nicolle2013}}\\
\gll [Pesa]\textsubscript{TOPIC} sino [hu-na-Ø-hew-a]\textsubscript{COMMENT}\\
10.money us \textsc{1pl-pres}{}-Ø-be.given-\textsc{fv} \\


\glt \textup{‘}\textup{Money, us we are given.’}
\z

The topics in the examples above correspond to the grammatical object of each clause. In Makonde, Fuliiru, Mwani, and possibly in other languages (but not in Digo), left dislocation of the object can also be used to give prominence to the final constituent of the sentence. In (4) below, moving the object \textit{kirya kijumba} ‘that box’ out of the comment emphasizes the fact that it was laid on the bed (as opposed to being opened, laid elsewhere, etc.). The object is left dislocated but follows the subject, which is also left dislocated, giving two topics and giving focal prominence to the comment.\footnote{For discussion of a similar construction in Matengo, see Yoneda (2011: 756–758).}

\ea\label{ex:}
\langinfo{}{}{ Fuliiru (Van Otterloo 2011: 350)}\\
\gll [Ulya munyere]\textsubscript{TOPIC1} [kirya kijumba]\textsubscript{TOPIC2} [a-na-ki-gwejez-a ku ngingo]\textsubscript{COMMENT}\\
1.that 1.girl 7.that 7.box \textsc{3sg-seq}{}-7-lay-\textsc{fv }on bed\\


\glt \textup{‘}\textup{That girl, that box, she laid it on the bed.’}
\z

Most topics, however, are subjects, and such topic-comment constructions conform to the canonical SVO constituent order. Nevertheless, subjects in topic position can be separated from the verb by a non-core element such as an exclamative or adverbial phrase, such as \textit{bhuri rusiku} ‘every day’ in the following example:

\ea\label{ex:}
\langinfo{}{}{ Jita (Pyle \& Robinson 2015: 32)}\\
\gll Eyo mw=ibhara [wamembe]\textsubscript{TOPIC} [bhuri rusiku :aa-jaga\footnotemark{}\textsuperscript{ }mu=mugunda gwaye]\textsubscript{COMMENT}\\
there in=forest hyena every  day he.\textsc{pst}{}-went in=field his{\rmfnm}\\


\glt \textup{‘}\textup{There in the forest, Hyena, every day he went to his field.’}
\z


\footnotetext{In examples from Jita and Kwaya, the symbol {\textless}:{\textgreater} at the beginning of a verb indicates far past tense and the symbol {\textless}\^{}{\textgreater} at the beginning of a verb indicates narrative tense with 3\textsc{sg} subjects. The past anterior and the far past, and the 3\textsc{sg} form of the narrative and the 1\textsc{sg} form of the anterior are only distinguished through tone, which is not marked in the orthography. Thus, these symbols differentiate the forms.}
\footnotetext{In this and subsequent examples glossing will be by word rather than by morpheme as a morpheme-based gloss is not necessary for the analysis of constituent order. In long examples, for reasons of space, only a free translation will be provided.}

Topics may also be extracted from their host clause(s), which then intervene between the topic and the comment. In (6), the topic \textit{rhibuyi eryo} ‘that rock’ is the object of a purpose clause, which in turn is the complement of a possessive clause, which is the complement of ‘see’ in a temporal clause. (The embedded clauses are indicated in square brackets. Extraction from multiple embedded clauses such as this may not be possible in all languages surveyed.)

\ea\label{ex:}
\langinfo{}{}{ Jita (Pyle \& Robinson 2015: 32)}\\
\gll Ribhuyi eryo\textsubscript{i} [ejire arora [atari na [ja kwasisya Ø\textsubscript{i}]]] eeganirisya muno.\\
rock that when he.saw he.is.not with of to.break he.thought much\\

\glt \textup{‘}\textup{That rock}\textsubscript{i}\textup{, [when he saw [he has nothing [to break (it}\textsubscript{i}\textup{)]]], he thought a}\textup{ lot.’}
\z

In all of the languages surveyed, topics are left dislocated at points of textual discontinuity. Textual discontinuity occurs at episode and paragraph breaks, when events are presented in a non-iconic order (including elaborations and parenthetical material), and when the topic is a \textit{switch topic}. A switch topic (also called \textit{shifted topic} and \textit{link} in Vallduví 1992: 109–10) occurs when the conceptual topic of the sentence under consideration is different from the conceptual topic of the immediately preceding sentence. In the following example, the topic changes from \textit{yuno} ‘that one’, referring to the speaker’s brother, to \textit{go mafuha ga taa} ‘that lamp oil’.

\ea\label{ex:}
\langinfo{}{}{ Digo (‘Sababu ya Bahari Kuhenda Munyu’ text, line 11)}\\
\gll Pho [yuno]\textsubscript{TOPIC} [kanipha vitu]\textsubscript{COMMENT}\textsubscript{ }ananipha mafuha ga taa bahi,\\
so that.one he.does.not.give.me things he.gives.me oil of lamp only\\


\gll [go mafuha ga taa]\textsubscript{TOPIC} ndo [n’yarya]\textsubscript{COMMENT}?\\
that oil of lamp \textsc{excl} I.go.eat\\

\glt \textup{‘So t}\textup{hat one doesn’t give me anything, he only gives me lamp oil, that lamp oil – can I g}\textup{o eat it?’}
\z

A switch topic is distinct from a continued topic, which is a topic that was also the topic of the previous sentence. Both switch topics and continued topics that occur at points of discontinuity are typically expressed lexically, and always in a pre-verbal position (with the exception of renewed topics in Mwani and temporary topics in Rangi, to be discussed below). Continued topics at points of discontinuity include conceptual topics that continue across paragraph and episode boundaries, and topics that are repeated when describing events narrated out of sequential order, such as elaborations. In the following example, the conceptual topic remains the man who is mentioned in the first sentence (indicated in bold). The sentence after the direct quotation starts a new paragraph, indicated by the use of the past tense \textit{a}\textit{\nobreakdash-} in \textit{w}\textit{a}\textit{rima} ‘he farmed’ as opposed to the consecutive tense \textit{chi}\textit{\nobreakdash-} in \textit{a}\textit{chi}\textit{amba} ‘he said’ in the previous sentence. The sentence initiating the new paragraph starts with the continued topic \textit{yuya bwana} ‘that gentleman’ in pre-verbal position:

\ea\label{ex:}
\langinfo{}{}{ Digo \citep[65]{Nicolle2015a}}\\
\textbf{Yuy}\textbf{a mlume} achiamba, “Mino rivyo nchirima tsula n’naphaha, phahi ndarima dza phapha na ko Mwamtsola, ili niphahe vitu vinji vyanjina niguze nigule ng'ombe.”\\
\textbf{Yuya bwana} warima munda uchifika dza Mazera, na hiku uchifika dza Malindi ela kaguwire hata tsere mwenga.\\

\glt \textbf{\textup{That man}} \textup{said}\textup{ [}\textup{cons}\textup{], “Me when I farmed a termite mound I was getting (a lot of food), so I shall farm from here to Mwamtsola, so that I will get lots of things to sell so I can buy a cow.”}

\glt \textbf{\textup{That gentleman}} \textup{farmed}\textup{ [}\textup{pst}\textup{] a field as large as from from Mazeras to Malindi, but he didn’t get even one grain of maize.’}
\z

The majority of continued topics do not occur at points of discontinuity, and as most topics are subjects, continued topics are typically expressed using just a subject prefix on the verb. Since the subject prefix is obligatory in all of the languages surveyed, strictly speaking there is no overt \textit{lexical} topic expression in such clauses, though there is an understood conceptual topic. Occasionally, however, a continued topic is expressed lexically even when there is no discontinuity in the text. When this occurs, eastern Bantu languages differ in how such topics are expressed. In most eastern Bantu languages, topics must occur pre-verbally, whereas in others, topics in certain discourse contexts may occur post-verbally.

In most languages surveyed, all topics are left dislocated, and any post-verbal element is interpreted as the focus. The following examples from Jita and Fuliiru texts illustrate left dislocated continued topics in which the topic (\textit{wamembe} ‘hyena’ in Jita and \textit{wandare} ‘lion’ in Fuliiru) is a subject (preserving the canonical SV constituent order); the following verbs contain subject agreement and are inflected for narrative tense (Jita) and sequential tense (Fuliiru):

\ea\label{ex:}
\langinfo{}{}{ Jita (‘Hare and Hyena’ text, lines 8-10, Allison Pyle \& Holly Robinson, p.c.)}\\
\gll Wamembe nayomba, wamembe nasurumbara, wamembe najira obhuramusi,\\
     hyena he.spoke hyena he.lamented hyena he.got decision\\
\gll bhwokuja-otema obhurembo, natura mumugundu gwaye.\\
he.went-cut sap/birdlime he.put in.field his\\


\upshape
‘Hyena spoke, hyena lamented, hyena made up his mind, he went and cut birdlime (tree sap), (and) he put (it) in his field.” (Free translation: SN)
\z

\ea\label{ex:}
\langinfo{}{}{Fuliiru (Van Otterloo 2011: 541)}\\
\gll Wandare anayuvwa kwâkola mulirira umwana. Wandare anabwîra uyo mushaaja...\\
lion he.heard(\textsc{seq}) that it.is.crying.for child lion he.told(\textsc{seq}) that old\_man  \\


\glt \textup{‘}\textup{Lion heard that it [the cow] is crying for (its) child. Lion told that old man...’}
\z

In Digo, Mwani and Rangi, topics may be right dislocated under certain conditions. It should be noted that right dislocated topics are distinct from post-verbal subjects which may occur in thetic sentences (see §5 below). Right dislocated topics refer to specific, identifiable participants, are often marked as such (for example, they are modified by demonstratives or are proper names), and are separated from the verb by a pause and sometimes by non-core elements. Like left dislocated constituents, right dislocated constituents are outside of the clause nucleus. In contrast, post-verbal subjects are grammatical subjects in their own right and are never separated from the verb by a pause or by non-core elements.

In Digo and Mwani, continued topics are right dislocated when there is textual continuity, that is, when events are presented in sequence within a single thematic unit (i.e. a paragraph). In the following example, the continued topic \textit{mutu yuyu} ‘this person’ is right dislocated. This is possible because there is no change of topic and no discontinuity; the use of the consecutive tense marker \textit{chi}\textit{\nobreakdash-} in \textit{a}\textit{chi}\textit{nyamala} ‘he stayed silent’ indicates a sequential action within a single paragraph:

\ea\label{ex:}
\langinfo{}{}{Digo (Nicolle 2015a: 27–28)}\\
Achidziuza mwakpwe rohoni, “Pho munda, nkauhenda mkpwulu na sikaphaha hata tsere mwenga, kpwani nini?” Lakini [achinyamala]\textsubscript{COMMENT} [mutu yuyu]\textsubscript{TOPIC} wala kayagomba na mutu.\\
\glt \textup{‘}\textup{He}\textup{ asked himself [}\textsc{cons}\textup{] in his heart, “That field, I have made it big and I have not got even a single maize cob, but why?” But }\textup{he stayed silent}\textup{~[}\textsc{cons}\textup{] }\textbf{\textup{this man}}\textup{, neither did he speak with anyone.’}
\z

In (12), from Mwani, the conceptual topic does not change but is referred to using a noun phrase and proper names in the last clause. There is no discontinuity in the text as the order of events is iconically represented: they were called and then they came, and so the topic is right dislocated.

\ea\label{ex:}
\langinfo{}{}{Mwani \citep[10]{Floor2005}}\\
\gll Wakati waifikire sumana yawasikizane, wakitíwa,\\
time when.it.arrived week that.they.agree they.were.called\\


\gll [wakíja]\textsubscript{COMMENT} [wó-wawiri,]\textsubscript{TOPIC} Anli na Ntendaji.\\
they.came those-two \textsc{name }and \textsc{name}\\


\glt \textup{‘}\textup{When the week that they agreed upon arrived, they were called, they came those two, Anli and Ntendaji.’}
\z

In addition, in Mwani a topic may also be right dislocated if it is a renewed topic, that is, an element that has previously functioned as a topic but not in the immediately preceding clause \citep[10-11]{Floor2005}. (It is not clear whether this is always the case or is an option.) Example (13) shows the difference between a right dislocated renewed topic, \textit{muk}\textit{a ire} ‘that woman’, which was last mentioned two clauses earlier, and a left dislocated switch topic, \textit{vinu vire} ‘those things’, which has not previously functioned as a topic.

\ea\label{ex:}
\langinfo{}{}{Mwani \citep[10-11]{Floor2005}}\\
\gll Sambi [akikála]\textsubscript{COMMENT} [muka ire]\textsubscript{TOPIC}\\
now she.sat woman that \\


\gll na [vinu vire]\textsubscript{TOPIC} [akipíka akiwápa wanu]\textsubscript{COMMENT}\\
and things those she.cooked she.gave.to.them people \\


\glt \textup{‘}\textup{Now she sat down that woman and those things she cooked (them) and gave to the people.’}
\z

In Rangi, a switch topic may be right dislocated if it is only temporary; that is, if it functions as the topic of a single clause but the previous conceptual topic is resumed immediately after. This is optional, however, as not all temporary topics are right dislocated. In (14), the subject twice changes from the elder to the boys and immediately back to the elder. The first clause involving the boys has SV order but the second has VS order. The boys are a temporary topic whereas the elder is more permanent: he is introduced formally in the first clause and it is he who speaks at the end of this passage; these are typical features of major participants (Dooley \& Levinsohn 2001: 119).

\ea\label{ex:}
\langinfo{}{}{Rangi \citep[532-3]{Stegen2011}}\\
Aho kalɨ kwijáa na \textbf{moosi }\textbf{ʉ}\textbf{mw}\textbf{ɨ}, afuma iʉndii, maa \textbf{vatavana} navo viyokʉʉja, maa akavasea, “Mpokeri isɨrɨ raanɨ.” [Maa vakasiita]\textsubscript{COMME}\textsubscript{NT} [\textbf{vara vatavana}.]\textsubscript{TOPIC} Maa [\textbf{ʉ}\textbf{ra moosi}]\textsubscript{TOPIC} [akasea...]\textsubscript{COMMENT}\\
\glt \textup{‘}\textup{In times of old there was }\textbf{\textup{one elder}}\textup{, he came from the field, and }\textbf{\textup{the boys}}\textup{ and they, they are coming, and he told them, “Carry my hoe.” And }\textup{they refused, }\textbf{\textup{those boys}}\textup{. And }\textbf{\textup{that elder}}\textup{ said...’}
\z

In other languages, temporary topics are left dislocated. In the Bena example in (15), \emph{inyama} ‘meat’ is a left dislocated temporary topic; \textit{u}\textit{\nobreakdash-Mbwa} ‘Dog’ occurs after the verb because it receives argument focus (see §~4).

\ea\label{ex:}
\langinfo{}{}{Bena \citep[34]{Eaton2015a}}\\
\gll U-Duuma aaheliye kwa mwipwave kuhungila kivembo.\\
Leopard he.went to uncle to.greet misfortune\\


\gll [Inyama]\textsubscript{TOPIC} [aalekile,]\textsubscript{COMMENT} [iloleela]\textsubscript{PRESUPPOSITION} [u-Mbwa]\textsubscript{FOCUS}\\
meat he.left he.looks.at Dog\\


\glt \textup{‘Leopard went to console his uncle for his bereavement. He left}\textup{ the meat, Dog was looking after it}\textup{.’}
\z

Given the available data, it seems that the majority of the languages surveyed pattern like Jita and Fuliiru in that all topics are obligatorily left dislocated. However, it is possible that evidence may emerge of specific discourse contexts that trigger right dislocation of topics in other languages.

\begin{itemize}
\item \begin{itemize}
\item \section{4  The expression of subject argument focus}\end{itemize}
\end{itemize}

\textit{Argument} (or \textit{term}) focus arises when non-predictable information is expressed by a noun phrase. It is found in declarative sentences when a certain event or situation is presupposed, but the speaker assumes that the addressee does not know the identity of one of the participants in that event or situation. In (15) above, it is presupposed that Leopard will not leave his meat unattended and the post-verbal subject \textit{u}\textit{\nobreakdash-Mbwa} ‘Dog’ identifies who has been left to guard it.

Argument focus in Bantu languages may be expressed in-situ, immediately after the verb (IAV), or in clause-final position (Yoneda 2011: 761–2; Gibson et al., to appear). In most eastern Bantu languages surveyed it appears that argument focus is associated with clause-final constituents, an exception being Makonde which appears to use the IAV position, in line with most other Bantu languages exhibiting a conjoint/disjoint distinction (Gibson et al., to appear). There is evidence that elements in this position are right dislocated in at least some cases, as the occurrence of the adverbial \textit{woori} ‘now’ between the presupposition and focus elements in (16) suggests. The context for this example is that the animals have dug a well, but Hare is coming at night and stealing the water. The animals plan to post a guard by the well, and this constitutes the presupposition; the focus identifies who was chosen.

\ea\label{ex:}
\langinfo{}{}{Jita (Pyle \& Robinson 2015: 18)}\\
\gll [Mbamuta-ko]\textsubscript{PRESUPPOSITION} woori [nyawatare]\textsubscript{FOCUS}\\
they.put.him-there now lion\\


\glt \textup{‘}\textup{Now they put lion there.’}
\z

Makonde does not allow both an object and subject to occur after the verb, and so when there is a post-verbal subject the object is left dislocated \citep[91]{Leach2015}. In Example (17) below, \textit{shakulya} ‘staple food’ is always served with \textit{imbogwa} ‘sauce’ and so it is assumed that someone will provide this; the post-verbal subject identifies this person as the speaker.

\ea\label{ex:}
\langinfo{}{}{Makonde \citep[92]{Leach2015}}\\
\gll Paukile ndawika kukaja kumwaulila ndyagwe do:\\
when he.went and.arrived he.told.her his.wife \textsc{quot}\\


\gll “Ndyangu, taleka shakulya, [imbogwa namanya]\textsubscript{PRESUPPOSITION} [nimwene]\textsubscript{FOCUS}”\\
my.wife cook staple.food sauce I.will.know myself\\


\glt \textup{‘When he got home he told his wife,}\textup{ “Get some food ready for me, wife – but as for the meat sauce, I’ll deal with that.”}\textup{’}
\z

Argument focus can also be expressed through cleft constructions in all languages surveyed. In such constructions, the focused element and the presupposition – often in the form of a relative clause or verbless predicate – are connected using a copula or focus marker. The orders focus{\textgreater}presupposition and presupposition{\textgreater}focus are both found:

\ea\label{ex:}
\langinfo{}{}{Kabwa \citep[25]{Walker2011}}\\
\gll Kumbe [omukari wunu]\textsubscript{FOCUS} ng’we [yankorera eng’ana yinu]\textsubscript{PRE}\textsubscript{SUPPOSITION}\\
\textsc{excl} woman this \textsc{cop} she.did.to.me thing this\\


\upshape
‘Gosh, it was this woman who did this thing to me.’
\z

\ea\label{ex:}
\langinfo{}{}{Suba-Simbiti \citep[28]{Masatu2015}}\\
\gll [Omoremo ghono yaamanyirë]\textsubscript{PRESUPPOSITION} [no-bhötëghi ubhwa sinswe]\textsubscript{FOCUS}\\
work \textsc{rel }he.knew \textsc{foc}{}-trapping of fish\\


\upshape
‘The work which he knew is fishing.’
\z

All languages surveyed use both clause-final (or IAV) position and cleft constructions for non-subject argument focus, but there is cross-linguistic variation with subject focus marking. The fact that subject focus behaves differently from other kinds of argument focus is not surprising. What Güldemann et al. (2015: 159) refer to as the “default topic-hood of subjects/agents” probably underlies the fact that focus marking on subjects sometimes differs from focus marking on other constituents, for example by requiring explicit focus marking even when a non-subject focus element is unmarked (ibid. 170; see also Fiedler et al. 2010 for a discussion of this phenomenon in Gur, Kwa and West Chadic languages).

In all languages surveyed, argument focus on the subject can be indicated by cleft constructions. Argument focus on the subject can also be indicated by clause-final position in Jita, Kabwa, Kwaya, Suba-Simbiti, Bena, Malila and Rangi, and by IAV position in Makonde. The Jita example in (20) illustrates both strategies.\footnote{The precise differences in interpretation between different focus positions could not be determined on the basis of narrative corpus data alone.} The post-verbal subject in the first clause identifies ‘women only’ as those who were living in the land (that the land was inhabited by someone is a presupposition); the cleft construction (with a copula clitic) in the second clause identifies these women as the ones who had stolen Mariro’s cows, an event of which the audience is already aware.

\ea\label{ex:}
\langinfo{}{}{Jita (Pyle \& Robinson 2015: 34)}\\
\gll Echaaro echo [:bhaariga bheekaye-mo]\textsubscript{PRESUPPOSITION} [abhagasi era,]\textsubscript{FOCUS}\\
land that they.were living-there women only\\


\gll [ni=bho]\textsubscript{FOCUS} [:bhaariga bheebire jing’a ja Mariro]\textsubscript{PRESUPPOSITION}\\
\textsc{cop=3pl} they.were they.had.stolen cows of Mariro\\


\glt \textup{‘}\textup{In that land were living women only, it was they who had stolen Mariro’s cows.’}
\z

 Digo and Fuliiru do not allow post-verbal subjects to receive argument focus. In both languages, subject argument can only be expressed using cleft constructions. In Fuliiru, the cleft construction is formed with a ‘focus copula’ which is cliticized to the following verb:

\ea\label{ex:}
\langinfo{}{}{Fuliiru (Van Otterloo 2015: 345)}\\
\gll [Yàbá bágénì]\textsubscript{FOCUS} [bó=bàgírá yìbì]\textsubscript{PRESUPPOSITION}\\
these guests \textsc{foc}=they.do these\\


\glt \textup{‘}\textup{These guests, they (are the ones who) did these things.’}
\z

In Digo, the cleft construction consists of the copula prefix \textit{ndi} (\textit{si} in the negative) plus a referential marker, and is typically, although not always, followed by a relative clause:

\ea\label{ex:}
\langinfo{}{}{Digo \citep[55]{Nicolle2015a}}\\
\gll Ndipho atu achimanya kukala [iye]\textsubscript{FOCUS} [ndi=ye ariyehenda mambo higo]\textsubscript{PRESUPPOSITION}\\
then people they.knew that she \textsc{cop=1.ref} who\textsc{.}did things those\\


\glt \textup{‘}\textup{Then people knew that it was her who did those things.’}
\z

Finally, in Mwani, argument focus is post-verbal in response to a content question, but when there is no prior question a cleft construction is used (Floor p.c. 8 April 2014; see also Floor 2005: 9):

\ea\label{ex:}
\langinfo{}{}{Mwani (Floor p.c.)}\\
\gll Kitabu atwarire nani? [Katwala]\textsubscript{PRESUPPOSITION} [Saidi]\textsubscript{FOCUS}\\
book 3\textsc{sg.rel.pst}.take who(\textsc{subject}) 3\textsc{sg.pst}.take Saidi\\


\glt \textup{‘}\textup{Who took the book? Saidi took it.’}
\z

\ea
\gll [Atwarire]\textsubscript{PRESUPPOSITION} ndi [Saidi]\textsubscript{FOCUS}\\
\textsc{3sg.rel.pst}.take \textsc{cop }Saidi\\


\glt \textup{‘}\textup{It was Saidi who took it.’}
\z

\section{5  VS and SV thetic sentences}
As we have seen, the pre-verbal domain in eastern Bantu languages is restricted to topics and non-focus subjects, and so a sentence with canonical SV constituent order will normally be interpreted as expressing topic-comment sentence articulation. It is therefore not surprising that thetic sentences, in which there is no topic, exhibit VS constituent order. These post-verbal subjects are not right dislocated; post-verbal subjects in thetic sentences are grammatical subjects in their own right, they are not topics (and so are generally non-specific or non-established), and they are never separated from the verb by a pause or by non-core elements. The following examples illustrate this.
\ea\label{ex:}
\langinfo{}{}{Kwaya \citep[29]{Odom2015}}\\
\gll Woori bhunu :aariga acheeganiirisha mmbe \^{}n-aa-j-a waarukerwe.\\
now while he.was he.still.be\_thinking so he.came frog\\
‘Now while he was still continuing to think, came a frog.’
\z

\ea\label{ex:}
\langinfo{}{}{Digo (‘Mhegi wa Mihambo’ text, line 25b)}\\
\gll ratuluka fisi, rina chitswa dza cha mutu...\\
it.emerged hyena it.has head of like person\\
\upshape
‘...a hyena emerged, it had a head like a person’s...’
\z

A common function of thetic sentences is to introduce participants into a narrative. When participants are introduced using a verb of arrival (‘come’, ‘emerge’, ‘appear’, etc.), agreement is with the agent noun phrase in all languages (see the examples above) except Fuliiru. However, when participants are introduced using existential verbs, the verb agrees with the noun phrase in Makonde,\footnote{Van der \citet{Wal2008} reports that languages which distinguish conjoint and disjoint verb forms – which includes Makonde – differ concerning which form is used in thetic sentences; for example, Sesotho [S32/33] uses a conjoint verb form, whereas Makhuwa [P31] uses a disjoint form. Makonde patterns like Makhuwa.} Bena, Malila, Jita, Kabwa, Kwaya, Suba-Simbiti and Ekoti, but with a locative noun class in Fuliiru, Digo and Rangi (see Nicolle 2015b: 17–20).

In thetic sentences, only VS constituent order is found in the Digo, Jita and Kabwa text corpora. However, a few texts in the other languages surveyed begin with SV clauses where the subject is not a topic. Most of the subjects in these SV clauses are either well-known folk-tale or animal characters – as in (26) below from Bena – or they refer to non-specific participants, such as ‘children’ in the Kwaya example (27) and ‘one man’ in the Fuliiru example (28). This suggests that at the start of a narrative, where no established referents are available to be topics, SV constituent order may be used in thetic sentences when the subject is known to the audience or is (currently) non-specific.

\ea\label{ex:}
\langinfo{}{}{Bena \citep[54]{Eaton2015a}}\\

\gll Pa vutalilo [u-Mbwa nu Duuma]\textsubscript{S} [vaali]\textsubscript{V} nu wunyalumwinga.\\
at start dog and leopard they.were with unity\\
\upshape
‘In the beginning Dog and Leopard were together.’
\z

\ea\label{ex:}
\langinfo{}{}{Kwaya \citep[60]{Odom2015}}\\
\gll Rusuku rumwi [abhaana]\textsubscript{S} [mbaja]\textsubscript{V} okureebha.\\
day one children they.came to.herd\\
\upshape
‘One day children went to herd.’
\z


\ea\label{ex:}
\langinfo{}{}{Fuliiru (Van Otterloo 2015: 104)}\\
\gll [Mushoshi muguma]\textsubscript{S} [akagira]\textsubscript{V} lusiku likulu ha’mwage.\\
man one he.held day great at.home\\
\upshape
‘One man had a feast at his house.’
\z

Although VS thetic sentences are the norm in most languages surveyed, with SV thetic sentences occasionally at the start of narratives, this is not the case in the Suba-Simbiti text corpus. Only one Suba-Simbiti text starts with a VS thetic sentence; of the other texts, three start with SVO thetic sentences, one with a VO thetic sentence, and one with a subject in a copula construction followed by a past form of the verb\emph{ rë} ‘be’:


\ea\label{ex:}
\langinfo{}{}{Suba-Simbiti \citep[16]{Masatu2015}}\\

\gll [Musimbëtë na Mohaasha]\textsubscript{S} m=bhaana abha enda ëmwë [bha-a-rë]\textsubscript{V}\\
Musimbiti and Mohaasha \textsc{cop}=2.children 2.\textsc{ass }9.stomach 9.one \textsc{3pl-pst}{}-be\\


‘Msimbiti and Mohaasha were siblings, they were.’
\z
Unusually for an eastern Bantu language, Suba-Simbiti also allows SV thetic sentences after the start of a narrative, as in (30). No buffalo has been mentioned previously in the text, and so a topic-comment reading is ruled out. (Two topic-comment sentences follow, in which the buffalo and then the youth function as topics.)

\ea\label{ex:}
\langinfo{}{}{Suba-Simbiti \citep[44]{Masatu2015}}\\
\gll Bhoono hano yaarëësyanga urusikö urwöndë,\\
now when he.was.herding day another\\
\gll {}[eng’era]\textsubscript{S} [ekaasha]\textsubscript{V} mu-rihisho irya waabho riyo.\\
buffalo it.came in-group of their.place that\\
\upshape
‘Another day when he was herding, a buffalo came among their herd.’
\z
\section{6  Conclusions}

Based on the available narrative texts, a number of generalizations can be made concerning information structure in the eastern Bantu languages surveyed.

\section{Obligatory and optional left dislocation of topics}

All languages surveyed have left dislocated (pre-verbal) topics, and topic-comment sentence articulation is very common. When there is textual discontinuity (switch topics, episode or paragraph breaks, non-iconic order of events, etc.), topics in all languages are left dislocated. Moreover, all topics are left dislocated in Jita and Fuliiru, and probably in most other eastern Bantu languages. In Jita and Fuliiru, left dislocation of topics is never overridden by other discourse factors. However, certain topics subtypes are right dislocated in a few languages. Continued topics are right dislocated in Digo and Mwani when there is textual continuity (i.e. within a paragraph); renewed topics may be right dislocated in Mwani; and temporary topics are optionally right dislocated in Rangi. In Digo, Mwani and Rangi, therefore, the default left dislocation of topics can be overridden by discourse factors.

\section{The expression of subject argument focus}

Clause-final or IAV constituents other than the subject may express argument focus in all languages. In Jita, Kabwa, Kwaya, Suba-Simbiti, Bena, Malila, Makonde and Rangi, argument focus on the subject can be expressed both through clause{}-final or IAV position and through cleft constructions. However, in Fuliiru and Digo, argument focus on the subject can only be expressed using a cleft construction; this appears to be a grammatical constraint and is not tied to specific discourse contexts. In Mwani, argument focus involves right dislocation in response to a content question but a cleft construction is used where there is no prior question.

\section{VS and SV thetic sentences}

Post-verbal subjects are the norm in thetic sentences and are the only possibility in Digo, Jita and Kabwa. However, SV thetic sentences occur in Fuliiru, Kwaya, Suba-Simbiti, Rangi, Bena, Malila, and Makonde for the presentation of new participants at the start of narratives. The Suba-Simbiti corpus also includes SV thetic sentences elsewhere. This suggests that in most eastern Bantu languages surveyed, the default VS constituent in thetic sentences can be overridden by discourse factors.

\section{Abbreviations}

\textsc{1pl}  1\textsuperscript{st} person plural

\textsc{3pl}  3\textsuperscript{rd} person plural

\textsc{3sg}  3\textsuperscript{rd} person singular

\textsc{cop}  copula

\textsc{excl}  exclamation

\textsc{foc}  focus marker

\textsc{fv}  final vowel

\textsc{om}  object marker

\textsc{pres}  present tense

\textsc{quot}  quotative marker

\textsc{ref}   referential marker

\textsc{rel}  relative marker

\textsc{seq}  sequential tense

\section{References}

Broomhall, Elizabeth. 2011. Bena text linguistics\textit{.} Nairobi: Africa International University. (Unpublished ms.)

Buell, Leston C. \& Riedel, Kristina \& van der Wal, Jenneke. 2011. What the Bantu languages can tell us about word order and movement. \textit{Lingua} 121. 689–701.

Cohen, Ariel. 2009. Fronted quantificational adverbs. In Shaer, Benjamin \& Cook, Philippa \& Frey, Werner \& Maienborn, Claudia (eds.), \textit{Dislocated elements in discourse: Syntactic, semantic, and pragmatic perspectives}, 312–327\textit{.} New York \& London: Routledge.

Eaton, Helen. 2015a. Bena narrative discourse. \textit{SIL Language and Culture Documentation and Description} 25. Dallas: SIL International.

De Cat, Cécile. 2007. \textit{French dislocation: Interpretation, syntax, acquisition.} Oxford: Oxford University Press.

Dik, Simon C. 1981. \textit{Functional grammar}. 3\textsuperscript{rd} edn. Dordrecht: Foris.

Dik, Simon C. 1989. \textit{The theory of functional grammar, part 1: The structure of the clause}. Dordrecht: Foris.

Dooley, Robert A. \& Levinsohn, Stephen H. 2001. \textit{Analyzing discourse: A manual of basic concepts.} Dallas: SIL International.

Downing, Laura J. \& Hyman, Larry M. Forthcoming. Information structure in Bantu. In Féry, Caroline \& Ishihara, Shinichiro (eds.), \textit{The Oxford handbook of information structure}, Oxford Handbooks Online.

Eaton, Helen. 2015a. Bena narrative discourse. \textit{SIL Language and Culture Documentation and Description} 25. Dallas: SIL International.

Eaton, Helen. 2015b. Malila narrative discourse. \textit{SIL Language and Culture Documentation and Description} 27. Dallas: SIL International.

Erteschik-Shir, Nomi. 2007. \textit{Information structure: The syntax-discourse interface.} Oxford: Oxford University Press.

Fiedler, Ines \& Hartmann, Katharina \& Reineke, Brigitte \& Schwarz, Anne \& Zimmermann, Malte. 2010. Subject focus in West African languages. In Zimmermann, Malte \& Féry, Caroline (eds.), \textit{Information structure: Theoretical, typological, and experimental perspectives}, 234–257. Oxford: Oxford University Press.

Floor, Sebastian J. 2005. Marked word-order and focus in KiMwani. (Unpublished ms).

Gibson, Hannah \& Koumbarou, Andriana \& Marten, Lutz \& van der Wal, Jenneke. To appear. Locating the Bantu conjoint/disjoint alternation in a typology of focus marking. In van der Wal, Jenneke \& Hyman, Larry M. (eds.), \textit{The conjoint/disjoint alternation in Bantu.} Berlin: De Gruyter.

Good, Jeff. 2010. Topic and focus fields in Naki. In Fiedler, Ines \& Schwarz, Anne (eds.), \textit{The expression of information structure: A documentation of its diversity across Africa}, 35–67. Amsterdam/Philadelphia: John Benjamins.

Güldemann, Tom \& Zerbian, Sabine \& Zimmermann, Malte. 2015. Variation in information structure with special reference to Africa. \textit{Annual Review of Linguistics} 1. 155–178.

Gundel, Jeanette K. \& Fretheim, Thorstein. 2004. Topic and focus. In Horn, Laurence R. \& Ward, Gregory (eds.), \textit{The handbook of pragmatics}, 175–196. Oxford: Blackwell.

Guthrie, Malcolm. 1967-71. \textit{Comparative Bantu. An introduction to the comparative linguistics and prehistory of the Bantu languages.} (4 vols.) Farnborough: Gregg International.

Halliday, M.A.K. 1967. Notes on transitivity and theme in English II. \textit{Journal of Linguistics} 3. 199–244.

Lambrecht, Knud. 1994. \textit{Information structure and sentence form. }Cambridge: Cambridge University Press.

Lambrecht, Knud. 2000. When subjects behave like objects. \textit{Studies }\textit{in Language} 24. 611-682.

Leach, Rhoda Martyn. 2015. Makonde narrative discourse. \textit{SIL Language and Culture Documentation and Description} 29. Dallas: SIL International.

Lewis, M. Paul \& Simons, Gary F. \& Fennig, Charles D. (eds.). 2015. \emph{Ethnologue: Languages}\emph{ of the World. }\emph{18}\emph{th}\emph{ edn. }Dallas, Texas: SIL International. Online version: \href{http://www.ethnologue.com/}{http://www.ethnologue.com}.

Maho, Jouni. 2003. A classification of the Bantu languages: An update of Guthrie’s referential system. In Nurse, Derek \& Philippson, Gérard (eds.), \textit{The Bantu languages}, 639–651. London \& New York: Routledge.

Marten, Lutz \& Kula, Nancy C. \& Thwala, Nhlanhla. 2007. Parameters of morphosyntactic variation in Bantu. \textit{Transactions of the Philological Society} 105. 253–338.

Masatu, Amanda. 2015. Suba-Simbiti narrative discourse. \textit{SIL Language and }\textit{Culture Documentation and Description} 31. Dallas: SIL International.

Matić, Dejan \& Wedgwood, Daniel. 2013. The meanings of focus: The significance of an interpretation-based category in cross-linguistic analysis. \textit{Journal of Linguistics} 49. 127–163.

Mwamzandi, Mohamed Y. 2014. \textit{Swahili word order choices: Insights from information structure.} Arlington: University of Texas at Arlington. (Doctoral dissertation.)

Nicolle, Steve. 2013. \textit{A grammar of Digo (a Bantu language of Kenya and Tanzania).} Dallas: SIL International.

Nicolle, Steve. 2015a. Digo narrative discourse. \textit{SIL Language and Culture Documentation and Description }26. Dallas: SIL International.

Nicolle, Steve. 2015b. A comparative study of eastern Bantu narrative texts. \textit{SIL Electronic Working Papers }2015–003. Dallas: SIL International.

Odom, Shannon Ronit. 2015. Kwaya narrative discourse. \textit{SIL Language and Culture Documentation and Description} 30. Dallas: SIL International.

Pyle, Allison \& Robinson, Holly. 2015. Jita narrative discourse. \textit{SIL Language and C}\textit{ulture Documentation and Description} 28. Dallas: SIL International.

Schwarz, Florian. 2003. Focus marking in Kikuyu. \textit{ZAS Papers in Linguistics} 30. 41-118.

Schwarz, Florian. 2007. Ex-situ focus in Kikuyu. In Aboh, Enoch O. \& Hartmann, Katharina \& Zimmermann, Malte (eds.), \textit{Focus strategies in African languages: the interaction of focus and grammar in Niger-Congo and Afro-Asiatic}, 139-160. Berlin: Mouton de Gruyter.

Shaer, Benjamin \& Cook, Philippa \& Frey, Werner \& Maienborn, Claudia. 2009. Introduction. In Shaer, Benjamin \& Cook, Philippa \& Frey, Werner \& Maienborn, Claudia (eds.), \textit{Dislocated elements in discourse: Syntactic, semantic, and pragmatic}\textit{ perspectives}, 1–27\textit{.} New York \& London: Routledge.

Stegen, Oliver. 2011. \textit{In quest of a vernacular writing style for the Rangi of Tanzania: Assumptions, processes, challenges.} Edinburgh: University of Edinburgh. (Doctoral dissertation). \url{http://www.sil.org/resources/archives/42655}

Szűcs, Peter. 2014. Information structure and the English left periphery. In Butt, Miriam \& Holloway King, Tracy (eds.), \textit{Proceedings of the LFG14 conference}, 545–565. Stanford CA: CSLI.

van der Wal, Jenneke. 2008. Agreement in thetic VS sentences in Bantu and Romance. In De Cat, Cecile \& Demuth, Katherine (eds.), \textit{The Bantu-Romance connection. A comparative investigation of verbal agreement, DPs and information structure}, 323–350. Amsterdam/Philadelphia: John Benjamins.

van der Wal, Jenneke. 2009. \textit{Word order and information structure in Makhuwa-Enahara}. Utrecht: LOT (Landelijke Onderzoekschool Taalwetenschap) Publications.

van der Wal, Jenneke. 2014. Morphological marking of term focus. (Paper presented at Information Structure in Head-Marking Languages workshop, Max Planck Institute Nijmegen, 28-29 March 2014.)

van der Wal, Jenneke \& Biberauer, Theresa. 2014. From macroparameters to nanoparameters: A comparative Bantu case study. (Paper presented at ComSyn meetings, Leiden, 5 February 2014.)

Van Otterloo, Roger. 2011. \textit{The Kifuliiru language, vol. 2: A descriptive grammar}. Dallas: SIL International.

Van Otterloo, Roger. 2015. Fuliiru narrative discourse. \textit{SIL Language and Culture Documentation and Descriptio}\textit{n }32. Dallas: SIL International.

Walker, John R. 2011. Discourse features of Kabwa and accompanying texts. (Unpublished ms.)

Yoneda, Nobuko. 2011. Word order in Matengo (N13): Topicality and informational roles. \textit{Lingua} 121. 754–771.

Zerbian, Sabine. 2006. \textit{Expression of information structure in Northern Sotho.} Berlin: Humboldt University. (Doctoral dissertation.)



\section*{Abbreviations}
\section*{Acknowledgements}

{\sloppy
\printbibliography[heading=subbibliography,notkeyword=this]
}
\end{document}