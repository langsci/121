\documentclass[output=paper]{langsci/langscibook} 
\title{The productivity of the reversive extension in Standard Swahili} 
\author{% 
 Deo Ngonyani\affiliation{Michigan State University}\lastand 
 Nancy Jumwa Ngowa \affiliation{Pwani University}
}
% \chapterDOI{} %will be filled in at production


\abstract{
This paper is a study of the Swahili reversive verb derivation \textit{-ul}, which is also known as conversive or separative. Attachment of this suffix to a root X derives a verb with the meaning ‘undo X.’ The paper explores the productivity of the reversive using data generated in two elicitation tests: (a) a coinage test and (b) an acceptability judgment test. For the coinage test, we created a questionnaire of 20 nonsense verbs from which subjects were instructed to coin their reversives. Subjects were able to create the reversive forms without difficulty. For the second questionnaire, we created reversive forms using 20 most frequently used verbs from SALAMA of Helsinki Corpus of Swahili. For each of these verbs, a reversive verb was created. Subjects were asked to identify each derived verb as either (a) an existing word, (b) a possible word or (c) impossible. Only three words polled as existing in the language. This revealed that the main constraint for the 20 verbs is that the reversive can attach only to verbs with semantic meaning that may be un-done or reversed. From these tests, we conclude that although the reversive applies to a restricted set of verbs, it is productive and available for creation of new words.
}

\maketitle
\begin{document}

%\textbf{Keywords:} Bantu morphology, productivity, reversive, verb extensions

\section{Introduction}\label{sec:ngonyaningowa:1}

This is a study of the reversive (\textsc{rev}) derivation in Standard Swahili, a verbal derivation that is illustrated in the contrast between \REF{ex:ngonyaningowa:1a} and \REF{ex:ngonyaningowa:1b}.

\ea\label{ex:ngonyaningowa:1}
\ea\label{ex:ngonyaningowa:1a}  
\gll M-toto    a-li-fung-a    dirisha. \\
1-child    1\textsc{sm-pt}-shut-\textsc{fv}  5.window \\
\glt ‘The child shut the window.’

\ex\label{ex:ngonyaningowa:1b}
\gll M-toto    a-li-fung-u-a      dirisha. \\
1-child    1\textsc{sm-pt}-shut-\textsc{rev-fv}  5.window \\
\glt ‘The child opened the window.’
\z
\z

The suffix \textit{-u} appears on the root \textit{fung} ‘shut’ in \REF{ex:ngonyaningowa:1b} with the semantic effect of reversing the action denoted by the root  in \REF{ex:ngonyaningowa:1a}. Therefore, \textit{fung} ‘shut’ is reversed into \textit{fungu} ‘open’. This derivation is also known as a separative \citep{Schadeberg2003}, conversive \citep{Ashton1947}, or inversive \citep{Doke1935}.

The reversive is part of the system of derivations in Bantu languages, which  Bantuists refer to as “verb extensions” \citep{Guthrie1962}. There are three types of derivational suffixes:

\begin{itemize}
\item Valence-increasing suffixes (applicative, causative)
\item Valence-reducing suffixes (passive, reciprocal, stative)
\item Non-valence-changing suffixes (reversive, contactive, static)
\end{itemize}

While valence-increasing suffixes and valence-reducing suffixes are considered productive, non-valence-changing suffixes are regarded as not productive \citep{Shepardson1986}. Most of the meanings of such suffixes seem to be fossilized.

Apart from the reversive reading, the reversive extension exhibits a range of idiosyncratic meanings, as well as lexicalized forms in Swahili. Its idiosyncrasies in meaning, and its supposed unproductivity, have led dictionary makers to list reversive verbs as separate entries, not related to the roots or not derived like forms involving other derivational affixes \citep[for example][]{TUKI2001,TUKI2004,Mdeeetal2009}. Although the reversive words are not listed as derivations in the dictionaries, the consistency and transparency of the semantics of many reversive words is so obvious that it is impossible to ignore. With respect to productivity, the status of the reversive is not clear. \citet{Shepardson1986} considers the reversive as not productive, while \citet{Schadeberg1973} claims that it is productive. 

The objectives of this paper are two-fold. The first objective is to describe the reversive derivation, its manifestation, and its semantics. The second objective is to explore its productivity. We argue that the reversive in Swahili is a productive affix that can be used by speakers to create new words. We demonstrate that the apparent relative unproductivity is due to structural restrictions.

In \sectref{sec:ngonyaningowa:2}, we provide an overview of productivity theory and methodology, while in \sectref{sec:ngonyaningowa:3} we present the basic phonological, morphological, syntactic, and semantic facts about the reversive. The data for this section are words from dictionaries of Standard Swahili. In \sectref{sec:ngonyaningowa:4}, the methodology for measuring productivity used in this study is described, and results are discussed in \sectref{sec:ngonyaningowa:5}. In \sectref{sec:ngonyaningowa:6}, we make concluding remarks.

\section{Morphological productivity} \label{sec:ngonyaningowa:2}

Studies of morphological productivity are based on the intuition that some word-formation processes or rules are used a great deal more than others. For example, in English \textit{-ness} and \textit{-ity} both can be used to derive nouns from adjectives. Consequently, we get \textit{happy → happiness} and\textit{ sensitive → sensitivity}. The suffix \textit{-ness} is used a lot more than \textit{-ity.} Words like \textit{sensitiveness} may be recognizable, but *\textit{happity} sounds not English. In this section, we define productivity, identify its factors and constraints, and discuss how productivity is measured. 

Several definitions of productivity have been advanced. Among them are: 

\begin{quote}
The productivity of a word-formation process can be defined as its general potential to be used to create new words and as the degree to which this potential is exploited by the speakers. \citep[127]{Plag2006}
\end{quote}

\begin{quote}
The productivity of a morphological process (whether inflectional or derivational) has to do with how much (or, in the limiting case, whether) it is used in the creation of forms which are not listed in the lexicon. \citep[315]{Bauer2005}
\end{quote}

\begin{quote}
A morphological rule or pattern is said to be productive if (and to the extent that) it can be applied to new bases and new words can be formed by it. \citep[114]{HaspelmathSims2010}
\end{quote}

These and several other definitions converge on two senses of productivity, namely, the potential to create new words and the extent to which speakers use the word-formation process or rule to create new words (\citealt{Corbin1987} cited in \citealt{Bauer2001}). If a rule or process can be used in the creation of new words at a particular point in time, it is said to be available. A rule that is available is considered productive (\citealt[205]{Bauer2001}; \citealt{Plag2006}). Speakers can use the affix or rule to create new words. If an affix or rule can no longer be used for creation of new words, it is not available. Such a rule is not productive. A rule that is available should be used by the speech community rather than just in a single person’s idiosyncratic language. Availability is a \textit{yes}-\textit{no} issue. That is to say, a rule is either available or not available. Two English suffixes \textit{-ion} (as in \textit{action}) and \textit{-ment} (as in \textit{judgment}) illustrate this contrast. Although both are widely attested with many words, \textit{-ment} is not available for creation of new words while \textit{–ion} can still be used \citep{Bauer2001}. 

In contrast, the extent to which a rule or process is used is not a \textit{yes}-\textit{no} issue. In productivity studies, this is known as profitability. “The profitability of a morphological process reflects the extent to which its availability is exploited in language use, and may be subject unpredictably to extra-systemic factors” \citep[211]{Bauer2001}. In this sense, profitability is attested in a continuum. For example, the suffixes cited earlier, \textit{-ness} and \textit{-ity,} are both used today in the creation of new words. Measurements have revealed that \textit{-ness} is more frequently used than \textit{-ity}. Therefore, \textit{-ness} is more profitable than \textit{-ity} \citep{Plag2006}. Our concern in this study is primarily on the availability (i.e.\ productivity rather than profitability) of the reversive to create new words. 

Three features are considered factors that promote productivity. They are (a) \textsc{transparency}, (b) the \textsc{frequency of the base}, and (c) the \textsc{usefulness} of the word \citep{Bauer2001,Lieber2009}. \citet{Bauer2001} calls them prerequisites of productivity. Transparency refers to clear segmentation and one-to-one meaning-form correspondence. That is, one form appears everywhere in the same shape with the same meaning. For example, \textit{candidness}, \textit{commonness}, and\textit{oddness} all have the same form of the affix \textit{-ness} with the mean ‘state of being X’. This has a better chance of being more productive than \textit{-ity}, as in \textit{timidity}, \textit{locality}, and \textit{oddity}. The suffix \textit{-ity} triggers some phonological changes in the pronunciation. For example, [\textipa{"}tɪmɪd] → [tə\textipa{"}mɪdəti]. Moreover, the resultant meaning is not always ‘the state of being X’; for example, \textit{oddity} is not ‘the state of being odd’, neither is \textit{locality} ‘the state of being local’. This makes \textit{-ity} less productive than \textit{-ness}. The frequency of the base is the extent to which bases are available for the rule. If an affix can only be attached to a small set of bases, it is likely to be less productive than an affix that can attach to a wide range of bases. For example, when we compare \textit{-ness} to the deadjectival \textit{-en}, as in \textit{soften} and \textit{blacken}, we see the former affix is used in the derivation of a wide range of adjectives. The latter, however, applies only to a small set of adjectives that are monosyllabic. The bases for \textit{-ness} are more frequent compared to the bases for \textit{-en}. To the extent that a particular process creates words that are needed by the speech community, the word is said to be useful. Neologisms are created to serve a certain naming need \citep{Stekauer2005}. A process that creates such neologisms is productive. We demonstrate in this study that in spite of its transparency and usefulness, the reversive in Swahili does not have too many bases to which it can attach. 

In fact, bases and affixes may be subject to several restrictions. These are extensively discussed by, among others, \citet{Bauer2001}, \citet{HaspelmathSims2010}, \citet{Lieber2009}, \citet{Rainer2005}, and \citet{Plag2006}. One kind of restriction that affects the productivity of a rule is phonological. An affix or a rule may be sensitive to certain phonological conditions. For example, the English suffix \textit{-eer} appears on bases that end in /t/, such as \textit{profiteer} and \textit{racketeer}, and not in other environments. Therefore, there is no \textit{*gaineer} or \textit{*fraudeer}. Another kind of restriction is morphological, such as the presence of some morpheme that may attract some other specific affix. In English, for example, verbs with the affix \textit{-ise} attract the affix \textit{-ation} as a deverbal derivation \citep{Fernandez-Dominguez2013}. Syntactic restrictions also restrict productivity. The applicability of affixes and morphological processes may be sensitive to the word class. In specifying the productivity of the suffix \textit{-able}, for example, we must state that it attaches to transitive verbs. Etymology or the origin of the base may play a crucial role in restricting the extent to which a rule is used. In English, the adjective-forming suffix \textit{-ic} is restricted, because it attaches to words that are loans from Latin or French, such as, \textit{specific} and \textit{eclectic}. Another restriction is known as semantic blocking. The existence of a word may block the derivation of another word with the same meaning that uses a particular morphological process. For example, we have \textit{pig→piglet,} but the suffix \textit{–let} is not available for \textit{cow→*cowlet}. In this case, it is believed that since the word \textit{calf} already exists in the lexicon, it blocks the derivation that would yield \textit{cowlet}. Some rules, affixes, or processes are constrained by pragmatic or sociolinguistic conditions. \citet[135]{Bauer2001} cites an example from Dyrbal in which the suffix \textit{-ginary} ‘covered with, full of’ is only used for something that is unpleasant, for example, \textit{guna-ginay} ‘covered in faeces’. 

In recent years, several empirical measurements and methods have been developed to gauge the productivity of rules, processes, and affixes. \citet[143-161]{Bauer2001} and \citet{Fernandez-Dominguezetal2007} present useful surveys of measurements proposed by various scholars. As noted earlier in the definition of productivity, the measurements can be linked to the two approaches to productivity, namely, productivity as availability of a rule and productivity as profitability or the degree to which a rule is used. Work that focuses on determining availability is said to take a qualitative approach. The qualitative approach  takes into consideration the limitations and constraints of a process. The quantitative approach, on the other hand, assumes that a more productive process will be able to coin more lexemes. Productivity is measured by counting or quantifying attested forms with higher frequencies. In the quantitative approach, productivity is scalar, ranging from very productive to not productive at all. Let it be noted that the characterization of the two approaches does not mean that the qualitative approach does not use any statistical methods \citep[434-437]{Fernandez-Dominguez2013}. Our study of the reversive takes into account various qualitative aspects while focusing on whether or not the reversive is available to contemporary speakers of Swahili. 

Proposals for measuring productivity can be summarized under three models \citep[35]{Fernandez-Dominguezetal2007}: 
(a) frequency models, 
(b) probabilistic models, and 
(c) onomasiological models. Frequency models are based on the intuition that if an affix is found on many words, it is very productive. Such counting of attested words is done by \citet{Plag1999} with data from the \textit{Oxford English Dictionary}. A major issue with this model is that the measures deal with attested words with a particular affix. The nominal suffix \textit{-ment} in English, for instance, is often cited as an example of the flaw in frequency measurements. It appears in many words, such as in \textit{punishment}, \textit{entertainment}, \textit{appointment}, and \textit{commitment}. However, this affix is no longer available for the creation of new words. All the words it appears on were formed between mid-16\textsuperscript{th} and mid-18\textsuperscript{th} century \citep[181]{Bauer2001}. It is, therefore by definition, not productive. Probabilistic models statistically measure the probability of finding a new word by a given morphological process. Productivity is measured by counting neologisms. One such measure is hapax legomenon or unique formations. The insight behind this is that productive rules will produce more unique formations while unproductive rules will not produce such formations \citep{Baayen1992}. The onomasiological model is attributed to \v{S}tekauer who sought to focus studies of productivity on meaning first, rather than form first \citep{Stekauer2005}. The model is based on the idea that processes and rules come into play when there is a need for creating a word  for what \v{S}tekauer calls naming units. A word-formation process that results in creating a new naming unit is therefore considered productive. 

Accordingly, there are three main sources of data for studying productivity \citep{Plag2006,SchroederMuehleisen2010,Bolozky1999}. The first source is dictionaries from which one can search for neologism and words of particular word-formation processes or affixes. For example, \citet{Fernandez-Dominguez2013} makes use of the \textit{Oxford English Dictionary}. However, words may appear in the speech of the community that are not yet in the dictionary, because dictionaries tend to use formal language that is often old. The second source are language corpora in which naturally occurring words appear. Statistical methods of calculating frequencies of forms or tokens make use of such databases. Baayen’s index of hapax legomena, or words occurring only once in a corpus, is based on the notion of the probability of creation of new words based on neologisms in a corpus \citep{Baayen1992}. \citet{Plag2006} and \citet{Fernandez-Dominguez2013} use data from the British National Corpus. The third source of data is elicitation tests. There are two kinds of elicitation tests. One elicits acceptability judgments on existing or non-existing forms \citep{AronoffSchvaneveldt1978}. The other prompts subjects to coin new words within the parameters of some word-formation processes. Aronoff and his associates have sought to investigate productivity as a search for possible words, just as syntax studies investigate possible sentences and phrases \citep{Aronoff1976}. The present study pursues this idea of potentiality and the search for possible words.

Before we describe our test, let us describe the morphological phenomenon we investigate. In the following section, we present a description of the reversive derivation in Swahili. 

\section{The reversive derivation} \label{sec:ngonyaningowa:3}

In this section, we describe the reversive in terms of its phonological, morphological, syntactic, and semantic features. We also examine constraints with respect to the semantics of the bases to which the reversive may attach. All examples presented here are from three dictionaries of Standard Swahili: \textit{Kamusi ya Kiswahili-Kiingereza,} ‘Swahili-English Dictionary’ \citep{TUKI2001}; \textit{Kamusi ya Kiswahili Sanifu}, ‘A Dictionary of Standard Swahili’ \citep{TUKI2004}, and \textit{Kamusi Kamili ya Kiswahili}, ‘A Complete Swahili Dictionary’ \citep{Mdeeetal2009}.

As in other Bantu languages, the verb in Swahili exhibits a complex morphological structure consisting of inflectional prefixes, derivational suffixes and inflectional suffixes. Inflectional prefixes include subject marker, object marker, tense, negation, and relative marker. Inflectional suffixes include mood and negation. Derivational suffixes appear between the root and the inflectional suffixes. The following two examples illustrate some of the inflections and derivations.

\ea\label{ex:ngonyaningowa:2}
\ea\label{ex:ngonyaningowa:2a}
\gll wa-li-tu-lip-i-a \\
  2\textsc{sm}-\textsc{pt}-us-pay-\textsc{app}-\textsc{fv} \\
\glt  ‘they paid for us’ \\

\ex\label{ex:ngonyaningowa:2b}  
\gll wa-lip-ish-eni \\
  2\textsc{om}-pay-\textsc{caus}-\textsc{imp} \\
\glt  ‘(you all) make them pay’ 
\z
\z 

The root for these two examples is \textit{lip} ‘pay.’ In \REF{ex:ngonyaningowa:2a}, the derivational suffix is the applicative \textit{-i} and in \REF{ex:ngonyaningowa:2b} the derivational suffix is the causative \textit{-ish.} Other derivations include the reciprocal, the passive, and the stative. The final vowel \textit{-a} appears after the derivational suffixes, even in citation forms. The final vowel often stands in contrast to the subjunctive mood marker \textit{-e}. It is also replaced by \textit{-eni} in plural imperative. For this reason, \textit{-a} is often considered the indicative marker. There is also a negative suffix \textit{-i} for present tense negation and the question marker \textit{-je}.

The reversive is part of the complex system of verbal derivations, which display varying degrees of productivity. It is realized in four allomorphs, namely, -\textit{u, -o, -ul} and \textit{-ol.} The first two are illustrated in the following set of examples.

\ea\label{ex:ngonyaningowa:3}
\ea\label{ex:ngonyaningowa:3a}  
\tabme{zib-a}{zib-u-a}{‘plug’}{‘unplug’}
\ex\label{ex:ngonyaningowa:3b}
\tabme{teg-a}{teg-u-a}{‘set a trap’}{‘disassemble a trap’}
\ex\label{ex:ngonyaningowa:3c}
\tabme{pang-a}{pang-u-a}{‘arrange’}{‘disarrange’} 
\ex\label{ex:ngonyaningowa:3d}
\tabme{fung-a}{fung-u-a}{‘shut’}{‘open’}
\ex\label{ex:ngonyaningowa:3e}
\tabme{chom-a}{chom-o-a}{‘stab’}{‘pull out’}
\z
\z

The examples presented here do not include the inflectional prefixes. The verb roots in \REF{ex:ngonyaningowa:3} all end in consonants. The final vowel \textit{-a} is used for the citation forms of the verbs. 

The allomorph \textit{-o} in \REF{ex:ngonyaningowa:3e} has a restricted distribution. It appears when the root has the vowel /o/. The allomorphs \textit{-ul} and \textit{-ol} appear on verbs that also carry other extensions such as causative, applicative, and passive in addition to the reversive. The following examples show these allomorphs of the reversive followed by the applicative derivation.

\ea\label{ex:ngonyaningowa:4}
\ea\label{ex:ngonyaningowa:4a}
\tabme{zib-u-a}{zib-ul-i-a}{‘unplug’}{‘unplug with/for’}
\ex\label{ex:ngonyaningowa:4b}
\tabme{teg-u-a}{teg-ul-i-a}{‘disassemble a trap’}{‘disassemble a trap with/for/on’}
\ex\label{ex:ngonyaningowa:4c}
\tabme{pang-u-a}{pang-ul-i-a}{‘disarrange’}{‘disarrange with/for/on’}
\ex\label{ex:ngonyaningowa:4d}
\tabme{chom-o-a}{chom-ol-e-a}{‘draw out’}{‘draw out with/for/on’}
\ex\label{ex:ngonyaningowa:4e}
\tabme{fung-u-a}{fung-ul-i-a}{‘open’}{‘open with/for/on’}
\z
\z

The allomorph -\textit{ol} appears on verbs whose roots have the vowel /o/. The applicative suffix in this case surfaces as -\textit{e}. The other verbs in \REF{ex:ngonyaningowa:4} have -\textit{ul} for the reversive and -\textit{i} for the applicative extension. The liquid /l/ is part of the reversive extension rather than the applicative. This analysis is based on the fact that other extensions that are known to have no /l/ also appear to trigger the -\textit{ul/-ol} reversive allomorphs, as shown in the examples below. In example \REF{ex:ngonyaningowa:4c}, the additional extension is causative -\textit{ish}, while in \REF{ex:ngonyaningowa:4d} and \REF{ex:ngonyaningowa:4e}, the passive -\textit{iw/-ew} is added.

\ea\label{ex:ngonyaningowa:5}
\ea\label{ex:ngonyaningowa:5a}
\tabme{zib-u-a}{zib-ul-ish-a}{‘unplug’}{‘cause to unplug’}
\ex\label{ex:ngonyaningowa:5b}
\tabme{pang-u-a}{pang-ul-ish-a}{‘disarrange’}{‘cause to disarrange’}
\ex\label{ex:ngonyaningowa:5c}
\tabme{fung-u-a}{fung-ul-ish-a}{‘open’}{‘cause to open’}
\ex\label{ex:ngonyaningowa:5d}
\tabme{chom-o-a}{chom-ol-ew-a}{‘pull out’}{‘be pulled out’}
\ex\label{ex:ngonyaningowa:5e}
\tabme{zib-u-a}{zib-ul-iw-a}{‘unplug’}{‘be unplugged’} 
\z
\z 

Furthermore, historical comparative evidence indicates that the reversive in Proto-Bantu may have been \textit{*-ʊd} \citep{Schadeberg2003,Meeussen1967}. \citet[370]{NurseHinnebusch1993} also identify \textit{*-ul} as the proto-form in Swahili’s immediate family of North East Coast Bantu. 

The form \textit{-uk} is sometimes cited as an intransitive reversive, in contrast to the transitive reversive described in the foregoing paragraphs \citep[239]{Ashton1947}. Consider the examples below.

\ea\label{ex:ngonyaningowa:6}
\ea\label{ex:ngonyaningowa:6a}
\tabme{zib-u-a}{zib-uk-a}{‘unplug’}{‘get unplugged’}
\ex\label{ex:ngonyaningowa:6b}
\tabme{teg-u-a}{teg-uk-a}{‘set a trap’}{‘get unset’}
\ex\label{ex:ngonyaningowa:6c}
\tabme{pang-u-a}{pang-uk-a}{‘disarrange’}{‘get disarranged’}
\ex\label{ex:ngonyaningowa:6d}
\tabme{chom-o-a}{chom-ok-a}{‘draw out, pull out’}{‘get pulled out’}
\ex\label{ex:ngonyaningowa:6e}
\tabme{fung-u-a}{fung-uk-a}{‘open’}{‘get opened’}
\z
\z

However, \textit{-uk} may be regarded as a fusion of the reversive -\textit{ul} and the stative -\textit{ik}. The two are merged with attendant loss of /l/ and /i/. The examples in \REF{ex:ngonyaningowa:6} have \textit{-uk} and -\textit{ok} as the suffix complex forms. They are formed by the fusion of the suffix allomorphs -\textit{u, -o} and the stative suffix \textit{-ik (pang-ul-ik }→ \textit{panguk).} 

  Semantically, the reversive mostly expresses the undoing of an action denoted by the root. We may characterize this as ‘undo X’ or ‘reverse X’ where X is the meaning of the base.

\ea\label{ex:ngonyaningowa:7}
\ea\label{ex:ngonyaningowa:7a}
\tabme{panga}{pang-u-a}{‘arrange’}{‘disarrange’}
\ex\label{ex:ngonyaningowa:7b}
\tabme{simba}{simb-u-a}{‘encode in symbols’}{‘decode’}
\ex\label{ex:ngonyaningowa:7c}
\tabme{tanza}{tanz-u-a}{‘complicate’}{‘solve, clarify’}
\ex\label{ex:ngonyaningowa:7d}
\tabme{tata}{tat-u-a}{‘tangle’}{‘disentangle’}
\ex\label{ex:ngonyaningowa:7e}
\tabme{vaa}{v-u-a}{‘dress’}{‘undress’}
\z
\z

One basic constraint on the reversive derivation, therefore, is that the root must denote an action that is reversible. For example, with \textit{simba} ‘encode’, the reversive \textit{simbua} leads to ‘undoing the code’ or ‘decoding’. Verbs denoting events that cannot be undone cannot receive the reversive derivation. For example, \textit{sema} ‘say’ → \textit{*semua} ‘unsay’ is not possible. Indeed, the verbs with meaning that can be undone are few. 

The ‘undo’ or ‘reverse an action’ reading is not the only meaning for the reversive extension. Other meanings include intensification (\citealt[239]{Ashton1947}; \citealt[90]{Polome1967}; \citealt[78]{Schadeberg2003}).   

\ea\label{ex:ngonyaningowa:8}
\ea\label{ex:ngonyaningowa:8a}   
\tabme{kama}{kamua}{‘squeeze’}{‘squeeze out’}
\ex\label{ex:ngonyaningowa:8b}
\tabme{songa}{songoa}{‘press’}{‘wring’}
\ex\label{ex:ngonyaningowa:8c}
\tabme{mega}{megua}{‘cut off a piece’}{‘cut off a piece’}
\z 
\z 

While \textit{kama} ‘squeeze’ is an underived stem form, the affixation of -\textit{u} yields \textit{kamua} ‘squeeze out,’ which is more intensive in meaning. Likewise, \textit{songa} ‘press’ is intensified into \textit{songoa }‘wring’. From these, we read some notion of removal. 

There exist words in the vocabulary that appear to be derived by the reversive suffix, but whose meaning is neither clearly ‘reversive’ nor ‘intensive’. The reputation of the reversive as unproductive is mainly due to such forms in which the meaning of the reversive derivation is not clear.

\ea\label{ex:ngonyaningowa:9}
\ea\label{ex:ngonyaningowa:9a}
\tabme{tamba}{tambua}{‘narrate’}{‘recognize’}
\ex\label{ex:ngonyaningowa:9b}
\tabme{remba}{rembua}{‘decorate’}{‘make eyes at’}
\ex\label{ex:ngonyaningowa:9c}
\tabme{enga}{engua}{‘watch’}{‘skim (e.g. cream off of milk)’}
\ex\label{ex:ngonyaningowa:9d}
\tabme{kosa}{kosoa}{‘err’}{‘criticize’}
\ex\label{ex:ngonyaningowa:9e}
\tabme{koma}{komoa}{‘stop’}{‘do something deliberately to hurt someone’}
\z 
\z 

The meanings of the apparently derived verbs in \REF{ex:ngonyaningowa:9} are not from the result of ‘reversing’ the action of the base, nor are they cases of ‘intensification’ of the bases as we saw earlier. Rather, these meanings appear to be idiosyncratic and fossilized.

\citet[103]{Shepardson1986} notes that most occurrences of the reversive form are in fact fossilized. Although we can discern the reversive affix on many bisyllabic roots/stems, the meaning of the base without the supposed reversive suffix is not clear, because the base cannot stand as a verb without this extension.  Examples of this appear in \REF{ex:ngonyaningowa:10}.

\ea\label{ex:ngonyaningowa:10}
\ea\label{ex:ngonyaningowa:10a}
\tabme{kagua}{*kaga}{‘inspect’}{}
\ex\label{ex:ngonyaningowa:10b}
\tabme{fufua}{*fufa}{‘resurrect’}{}
\ex\label{ex:ngonyaningowa:10c}
\tabme{kwangua}{*kwanga}{‘scrape off’}{}
\ex\label{ex:ngonyaningowa:10d}
\tabme{zindua}{*zinda}{‘inaugurate’}{}
\ex\label{ex:ngonyaningowa:10e}
\tabme{chukua}{*chuka}{‘take’}{}
\ex\label{ex:ngonyaningowa:10f}
\tabme{nyofoa}{*nyofa}{‘nibble’}{}
\z 
\z

Although \textit{kagua} ‘inspect’ is a recognizable verb stem, removal of the /u/ yields *\textit{kaga,} which is not recognizable. The same thing is true with the other five verbs exemplified in \REF{ex:ngonyaningowa:10}. However, the stems are all polysyllabic, a sign that they are at least historically complex.

To sum up, the reversive is part of the system of Swahili verb derivations. It is realized in four forms \textit{-u, -o, -ul} and \textit{-ol}. This affix does not change the argument structure of the root verb. Two meanings are commonly associated with the reversive: (a) ‘undo X,’ and (b) ‘intensive’. This suffix also appears in numerous fossilized verbs.

\section{Elicitation experiments}\label{sec:ngonyaningowa:4}

In this section, we report on two elicitation experiments attempting to determine the availability of the reversive derivation to speakers of Standard Swahili. One experiment was a coinage test, while the second experiment was an acceptability judgment test. 

\subsection{Methods}\label{sec:ngonyaningowa:4.1}

For subjects, we recruited 56 undergraduate students at Pwani University in Kenya. All of them were students in the Swahili program. They were from diverse geographical backgrounds and therefore may speak different regional dialects. Most reported that they were bilingual in Swahili and other ethnic languages. 

For the coinage test, we created two word lists of 10 nonsense words each. They were either disyllabic or trisyllabic, the most common forms of the roots in Swahili. We also ensured that the final consonant or syllable in the nonsense words would not be confused with actually-existing verb extensions. Each wordlist had words such that all five vowels in Swahili were represented.  None of the 20 words were found in dictionaries. All words were presented in the infinitive form to ensure they are viewed as verbs. The infinitive is characterized by the infinitive prefix \textit{ku-}, the root, and the final vowel \textit{-a}.

List A \tabref{tab:ngonyaningowa:1} contains non-derived verbs. Subjects (N=20) were told to assume that these words exist in Swahili and that if they did not know them, it might be due to the fact that one cannot know every word in the language. With that assumption, they were to write the verb that denoted “undoing” of these actions. 

\begin{table}
\begin{tabularx}{.66\textwidth}{lX}
\lsptoprule
 Form presented &  Predicted subject response for   “undoing” sense\\
\midrule
  {kugola} &  {kugoloa}\\
 {kupeza} &  {kupezua}\\
 {kujida} &  {kujidua}\\
 {kumuka} &  {kumukua}\\
 {kunaba} &  {kunabua}\\
 {kukweta} &  {kukwetua}\\ 
 {kuika} &  {kuikua}\\
  {kuoha} &  {kuohoa}\\
  {kubada} &  {kubadua}\\
  {kuvupa} &  {kuvupua}\\
\lspbottomrule
\end{tabularx}
\caption{Nonsense words $-$ List A}
\label{tab:ngonyaningowa:1}
\end{table}

List B \tabref{tab:ngonyaningowa:2} was given to another group of university students (N=16). These were also nonsense words, but contained forms corresponding to the reversive extension. Subjects were instructed to write verbs expressing “reverse actions” from what the presented “verbs” would mean.

\begin{table}
\begin{tabularx}{.66\textwidth}{lX}
\lsptoprule
Form presented & Predicted subject response for  “reverse action”\\
\midrule
{kukemua} & {kukema}\\
{kutubua} & {kutuba}\\
{kubadua} & {kubada}\\
{kuzepua} & {kuzepa}\\
{kulotoa} & {kulota}\\
{kuisua} & {kuisa}\\
{kuholoa} & {kuhola}\\
{kuyunua} & {kuyuna}\\
{kupapua} & {kupapa}\\
{kulikua} & {kulika}\\
\lspbottomrule
\end{tabularx}
\caption{Nonsense words $-$ List B}
\label{tab:ngonyaningowa:2}
\end{table}

For the second elicitation test, a questionnaire of 20 verb roots was administered to another group of university students (N=20) at the same institution, inviting their judgments. The verbs were selected from among the most frequent Swahili verbs identified in the SALAMA of Helsinki Corpus of Swahili (HCS). The corpus is made up of over 20 million words from news texts, books containing prose texts of fiction, education, and science from the second half of the 20th century and the 21\textsuperscript{st} century \citep{Hurskainen2008,Hurskainen2009}. The information that is parsed by its morphological analyzer includes, among other things, parts of speech, inflections, derivations, and etymology. Using Lemmie2.0, a web-based tool for working with a language corpus in the Language Bank of Finland \citep{CSC2003}, we searched HCS for frequencies of verbs. The search generated a list of verbs ranked from the most frequent. From the list, we created reversive derivations for each of the top 20 verbs by attaching the suffix -\textit{ul}. No glosses were presented for the roots. These are roots that all subjects know. The subjects were asked to make a judgment on each derived verb as to whether it was: (a) an existing word, (b) a possible word, or (c) an impossible word. The verbs and the derived forms presented to the students are in \tabref{tab:ngonyaningowa:3}.

\begin{table}
\begin{tabularx}{\textwidth}{lXXclXp{2.3cm}}
\lsptoprule
 &  Root &  Derived Verb &  &  &  Root &  Derived Verb\\
\midrule
 1. & kuwa\newline  ‘to be’ &  {kuwua} &  &  11. & kula\newline  ‘to eat’ & kulua \\
\tablevspace
 2. & kusema\newline  ‘to speak’ &  kusemua &  &  12. & kudai\newline  ‘to claim’ &  kudiua \\
\tablevspace
 3. & kufanya\newline  ‘to do’ &  kufanyua &  &  13. & kuenda\newline  ‘to go’ &  kuendua \\
\tablevspace
 4. & kutoa\newline  ‘to remove’ &  kutoloa &  &  14. & kupa\newline  ‘to give’ &  kupua\\
\tablevspace
 5. & kuweza\newline  ‘to be able’ &  kuwezua &  &  15. & kupita\newline  ‘to pass’ &  kupitua \\
\tablevspace
 6. & kutaka\newline  ‘to want’ &  kutakua &  &  16. & kuja\newline  ‘to come’ &  kujua \\
\tablevspace
 7. & kupata\newline  ‘to get’ &  kupatua &  &  17. & kujua\newline  ‘to know’ &  kujulua \\
\tablevspace
 8. & kufunga\newline  ‘to shut’ & kufungua\newline  ‘to open’ &  &  18. & kuchoma\newline  ‘to stab’ & kuchomoa\newline  ‘to pull out’\\
\tablevspace
 9. & kuanza\newline  ‘to start’ &  kuanzua &  &  19. & kuomba\newline  ‘to beg’ &  kuomboa \\
\tablevspace
 10. & kuona\newline  ‘to see’ & kuonua &  &  20. &  kupanga \newline  ‘to arrange’ & kupangua
\newline  ‘to disarrange’\\
\lspbottomrule
\end{tabularx}
\caption{Verbs for elicitation test 2}
\label{tab:ngonyaningowa:3}
\end{table}

\subsection{Results}\label{sec:ngonyaningowa:4.2}
\subsubsection{Coinage test}\label{sec:ngonyaningowa:4.2.1}

For List A, out of 200 tokens, 60\% of the predicted responses were rendered with the reversive form, observing the appropriate vowel harmony for nonce roots that had /o/. For List B, out of the total number of 160 tokens, 65\% of the responses removed the supposed reversive suffix, thus producing the predicted non-reversive forms.

The results of both directions of word formation (adding and subtracting a suffixal piece) show that the reversive is active in the speakers’ competence or repertoire. They can create new words with it. The idea of working with nonsense words is consistent with the notion that speakers’ competence enables them to understand and produce novel utterances they may have never encountered before \citep{AronoffSchvaneveldt1978}. However, the domain on which the reversive can be attached is very restricted. The domain is revealed in Elicitation Test 2, to which we now turn.

\subsubsection{Judgment test}\label{sec:ngonyaningowa:4.2.2}

Only three derived words are recognized by all as words found in the language. They are \textit{kufunga} ‘to shut’ → \textit{kufungua} ‘to open’, \textit{kuchoma} ‘to stab’ → \textit{kuchomoa} ‘to pull out’, and \textit{kupanga} ‘to arrange’ → \textit{kupangua} ‘to disarrange.’ The constraint for productivity in this case is very clear, namely, since the reversive denotes ‘undo X’, the only verbs that can be derived with this affix are verbs naming actions that can be undone.

%\begin{table}
%\begin{tabularx}{\textwidth}{lXXXXlll} 
%\lsptoprule 
% Root & Gloss &  Derived &  Gloss &  Existing &  Possible &  Impossible\\
%& & verb & & word & word & \\
%\midrule
%
%{ \textit{kuwa}} &
%
% ‘to be’ &  \textit{kuwua} & &   0 &  0 &  16\\
%{ \textit{kusema}} &
%
% ‘to speak’ &  \textit{kusemua} & &  1 &  1 &  14\\
%{ \textit{kufanya}} &
%
% ‘to do’ &  \textit{kufanyua} & &  6 &  4 &  6\\
%{ \textit{kutoa}} &
%
% ‘to remove’ &  \textit{kutoloa} & &  1 &  1 &  14\\
%{ \textit{kuweza}} &
%
% ‘to be able’ &  \textit{kuwezua} & &  3 &  5 &  8\\
%{ \textit{kutaka}} &
%
% ‘to want’ &  \textit{kutakua} & &  1 &  1 &  14\\
%{ \textit{kupata}} &
%
% ‘to get’ &  \textit{kupatua} & &  0 &  0 &  16\\
%{ \textit{kufunga}} &
%
% ‘to shut’ & { \textit{kufungua}} &
%
% ‘to open’ &  16 &  0 &  0\\
%{ \textit{kuanza}} &
%
% ‘to start’ &  \textit{kuanzua} & &  3 &  2 &  11\\
%{ \textit{kuona}} &
%
% ‘to see’ &  \textit{kuonua} & &  0 &  1 &  15\\
%{ \textit{kula}} &
%
% ‘to eat’ &  \textit{kulua} & &  0 &  1 &  15\\
%{ \textit{kudai}} &
%
% ‘to claim’ &  \textit{kudaiua} & &  0 &  1 &  15\\
%{ \textit{kuenda}} &
%
% ‘to go’ &  \textit{kuendua} & &  1 &  0 &  15\\
%{ \textit{kupa}} &
%
% ‘to give’ &  \textit{kupua} & &  0 &  2 &  14\\
%{ \textit{kupita}} &
%
% ‘to pass’ &  \textit{kupitua} & &  2 &  0 &  14\\
%{ \textit{kuja}} &
%
% ‘to come’ &  \textit{kujua} & &  1 &  1 &  14\\
%{ \textit{kujua}} &
%
% ‘to know’ &  \textit{kujulua} & &  0 &  0 &  16\\
%{ \textit{kuchoma}} &
%
% ‘to stab’ & { \textit{kuchomoa}} &
%
% ‘to pull out’ &  16 &  0 &  0\\
%{ \textit{kuomba}} &
%
% ‘to beg’ &  \textit{kuomboa} & &  1 &  0 &  15\\
% \textit{kupanga} &   ‘to arrange’ & { \textit{kupangua}} &
%
% ‘to disarrange’ &  16 &  0 &  0\\
%\lspbottomrule
%\end{tabularx}
%\caption{Acceptability judgments}
%\label{tab:ngonyaningowa:4}
%\end{table}

\begin{table} 
\small
\begin{tabularx}{\textwidth}{lp{2cm}Xrrr} 
\lsptoprule 
&  Root &  Derived verb&  Existing word&  Possible word&  Impossible\\ 
\midrule

 1. & kuwa

 ‘to be’ &  kuwua &   0 &  0 &  16\\
 2. & kusema

 ‘to speak’ &  kusemua &  1 &  1 &  14\\
 3. & kufanya

 ‘to do’ &  kufanyua &  6 &  4 &  6\\
 4. & kutoa

 ‘to remove’ &  kutoloa &  1 &  1 &  14\\
 5. & kuweza

 ‘to be able’ &  kuwezua &  3 &  5 &  8\\
 6. & kutaka

 ‘to want’ &  kutakua &  1 &  1 &  14\\
 7. & kupata

 ‘to get’ &  kupatua &  0 &  0 &  16\\
 8. & kufunga

 ‘to shut’ & kufungua

 ‘to open’ &  16 &  0 &  0\\
 9. & kuanza

 ‘to start’ &  kuanzua &  3 &  2 &  11\\
 10. & kuona

 ‘to see’ &  kuonua &  0 &  1 &  15\\
 11. & kula

 ‘to eat’ & kulua &  0 &  1 &  15\\
 12. & kudai

 ‘to claim’ & kudaiua &  0 &  1 &  15\\
 13. & kuenda

 ‘to go’ & kuendua &  1 &  0 &  15\\
 14. & kupa

 ‘to give’ &  kupua &  0 &  2 &  14\\
 15. & kupita

 ‘to pass’ &  kupitua &  2 &  0 &  14\\
 16. & kuja

 ‘to come’ &  kujua &  1 &  1 &  14\\
 17. & kujua

 ‘to know’ & kujulua &  0 &  0 &  16\\
 18. & kuchoma

 ‘to stab’ & kuchomoa

 ‘to pull out’ &  16 &  0 &  0\\
 19. & kuomba

 ‘to beg’ &  kuomboa &  1 &  0 &  15\\
 20. &  kupanga   

 ‘to arrange’ & kupangua

 ‘to disarrange’ &  16 &  0 &  0\\
\lspbottomrule 
\end{tabularx}
\caption{Acceptability judgments}
\label{tab:ngonyaningowa:4}
\end{table}


\section{Discussion}\label{sec:ngonyaningowa:5}

The domain or set of bases to which the reversive can be attached \citep{HaspelmathSims2010} is restricted. It is also very significant that 10 subjects identified \textit{kufanyua} ‘to undo’ from \textit{kufanya} ‘to do’ as a possible word. None of the dictionaries consulted have this as an entry. However, a general sense of ‘undo’ is quite possible, it seems. It can be argued that the reason subjects positively indicated that this is a word in the language reflects their intuition that the reversive is still productive. 

Earlier on, we noted that the characterization of the reversive as non-productive was, to a great extent, due to the abundance of lexicalized forms associated with it, i.e.\ forms which are non-predictable \citep{Shepardson1986}. However, lexicalization is not exclusively found with the reversive and the non-valency-changing suffixes. Consider the following three examples with apparent valency-changing affixes, namely the applicative, reciprocal, and stative, respectively. 

\ea\label{ex:ngonyaningowa:11}
\ea\label{ex:ngonyaningowa:11a}
{shika}

  ‘hold, catch, apprehend’

\ex\label{ex:ngonyaningowa:11b} 
shikilia

  ‘hold on tight’
\z
\z 

\ea\label{ex:ngonyaningowa:12}
\ea\label{ex:ngonyaningowa:12a}
ona

  ‘see’

\ex\label{ex:ngonyaningowa:12b}  
onekana

  ‘be seen’
\z 
\z 

\ea\label{ex:ngonyaningowa:13}
\ea\label{ex:ngonyaningowa:13a}
tanda

  ‘spread over’

\ex\label{ex:ngonyaningowa:13b}  
tandika

 ‘lay a cover, make (a bed)’
\z 
\z

These examples reveal that lexicalization is not an exclusive feature of the reversive. Even the most productive verbal suffixes have their share of lexicalized non-compositional forms. In \REF{ex:ngonyaningowa:11b}, the \textit{-ilia} portion has nothing to do with the meaning or function of the applicative or any prepositional meaning, although it does look phonologically like the applicative. However, it is a form of an intensifier. In \REF{ex:ngonyaningowa:12b}, the polysyllabic verb appears as if it bears the stative -\textit{ek} and the reciprocal -\textit{an}. One reading of the stative is ‘possibility’. While the stative reading is discernible in the meaning of the word in the sense of ‘be visible,’ the reciprocal reading is completely absent. What looks like the stative suffix is also found in \REF{ex:ngonyaningowa:13b}. However, the relationship between the base \REF{ex:ngonyaningowa:13a} and the seemingly derived form \REF{ex:ngonyaningowa:13b} is not compositional. The stative in its two readings is a valence-reducing affix. Nonetheless, in \REF{ex:ngonyaningowa:13b}, it appears to increase the valence by introducing a new object; therefore, it is not the stative we know. 

To summarize, in this section we have presented results from two elicitation tests. In the first test, a group of speakers was instructed to coin verbs from nonsense bases. The test revealed that the reversive derivation is productively available to contemporary Swahili speakers. In the second test, speakers were asked to make acceptability judgments for reversive verb forms derived from the most frequently used verbs. Only three verbs were acceptable to all the respondents. This points to the semantically restricted domain of application of the rule: it can apply only to verbs expressing actions that can conceivably be reversed.

\section{Conclusion}\label{sec:ngonyaningowa:6}

In this article, we set out to describe the reversive derivation in Swahili and to determine its productivity. We identified the suffix and its allomorphs \textit{-u, -o, -ul} and  -\textit{ol}. The prototypical meaning of the reversive is ‘undo X.’ However, several words were identified bearing this suffix either without the compositional meaning involving reversion or separative, or without an identifiable base. Those are lexicalized forms. The paper reported on tests of productivity of the extension. It demonstrated that the affix, which causes the new verb to read ‘undo X,’ where X is the meaning of any root, can only be attached to roots designating reversible actions. Its productivity is semantically constrained, but it is still available for the creation of new verbs. This was demonstrated by the elicitation tests in which subjects coined nonsense words with this affix. 

\section*{Abbreviations} 

Numbers in the glosses refer to the conventional Bantu noun classes.\\

\begin{tabularx}{.45\textwidth}{lX}
\textsc{ap} & applicative  \\
\textsc{caus} &  causative  \\ 
\textsc{fv} &  final vowel \\
\textsc{om} & object marker \\ 
\textsc{pass} & passive  \\
\end{tabularx}
\begin{tabularx}{.45\textwidth}{lX}
\textsc{pt} & past tense \\
\textsc{rec} & reciprocal  \\
\textsc{rev} & reversive \\ 
\textsc{sm} & subject marker \\
\textsc{st} & stative \\
\end{tabularx}       

\section*{Acknowledgments }

The work on which this paper is based was made possible by the first author’s Fulbright fellowship for research and teaching at Pwani University in 2014. We acknowledge also the students who participated in the study. We express our many thanks to the participants of the Morphophonology panel at ACAL 46 for their very insightful comments and questions. Special thanks to two anonymous reviewers and the editors, Doris Payne and B. Mokaya Bosire, for helping us clarify issues and modify the paper. 

\printbibliography[heading=subbibliography,notkeyword=this]


\end{document}