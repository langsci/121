\documentclass[output=paper]{langsci/langscibook} 
\title{The productivity of the reversive extension in Swahili} 
\author{% 
 Ngonyani\affiliation{}\lastand 
 Ngowa \affiliation{}
}
% \chapterDOI{} %will be filled in at production


\abstract{
Lorem ipsum dolor sit amet, consectetur adipiscing elit. Aenean nec fermentum risus, vehicula gravida magna. Sed dapibus venenatis scelerisque. In elementum dui bibendum, ultricies sem ac, placerat odio. Vivamus rutrum lacus eros, interdum scelerisque est euismod eget. Class aptent taciti sociosqu ad litora torquent per conubia nostra, per inceptos himenaeos.
}

\maketitle
\begin{document}

\section{Introduction} 
\newcommand{\tabme}[5][4cm]{%
  \begin{tabbing}
    \hspace{#1} \= \hspace{#1} \kill
    #2	\>		#3 \\
    #4	\>		#5 \\
  \end{tabbing}
}

\ea
\ea
\tabme{zib-a}{zib-a}{‘plug’}{‘unplug’}
\ex
\tabme{teg-a}{teg-u-a }{‘set a trap’}{‘disassemble a trap’}
\ex
\tabme{pang-a}{pang-u-a }{‘arrange’}{‘disarrange’}
\ex
\tabme{fung-a}{fung-u-a}{‘shut’ }{‘open’} 
\ex
\tabme{chom-a}{chom-o-a}{‘stab’}{‘pull out’}
\z
\z
 
\ea
\ea 
\tabme[2cm]{zib-u-a}{zib-ul-i-a}{‘unplug’}{‘unplug with/for’}
\ex
\tabme[5cm]{teg-u-a}{teg-ul-i-a}{‘disassemble a trap’}{‘disassemble a trap with/for/on’}
\ex
\tabme[4cm]{pang-u-a}{pang-ul-i-a}{‘disarrange’}{‘disarrange with/for/on’}
\ex
\tabme[5cm]{chom-o-a}{chom-ol-e-a}{‘draw out’}{‘draw out with/for/on’}
\ex
\tabme[3cm]{fung-u-a}{fung-ul-i-a}{‘open’}{‘open with/for/on’}
\z
\z


\section*{Abbreviations}
\section*{Acknowledgements}

\printbibliography[heading=subbibliography,notkeyword=this]

\end{document}