\documentclass[output=paper]{langsci/langscibook} 
\title{Remote past and phonological processes in Kaonde} 
\author{%
Boniface Kawasha\affiliation{Savannah State University}
}
% \sectionDOI{} %will be filled in at production



\abstract{}

\maketitle
\begin{document}


\begin{stylelsAbstract}
This paper examines the various morphophonological processes triggered by the remote past suffix -\emph{ile} and the environments in which they occur in Kaonde (Bantu L41), a language spoken in Zambia. The phenomena include vowel height harmony, partial nasal consonant harmony, complete consonant assimilation, alveolar palatalization, and imbrication. Depending on the verb structure and phonological environment, the phenomena affect either the final consonant of the verb root or base, the remote past suffixal consonant, or both the verb stem final consonant and consonant of the suffix. The remote past suffix \emph{{}-ile} exhibits at least ten morphophonemic alternations. The final consonant of certain verb roots or bases also alternates between alveolar and palatalized affricate pronunciations \textstyleipa{when in contact with the high front vowel of the suffix. The imbrication  process displays different patterns, and may be accompanied by segment deletion, vowel coalescence, vowel lengthening, and glide formation. }
\end{stylelsAbstract}

Key words: \textit{morphophonology,} \emph{harmony, assimilation, palatalization, }

\emph{                   imbrication}

\begin{stylelsSectioni}
1 Introduction
\end{stylelsSectioni}

This paper examines the various types of morphophonological processes triggered by the remote past inflectional suffix -\emph{ile} and the environments in which they occur when the suffix combines with a verb root or base in Kaonde (Bantu L41; Guthrie 1967-71), one of the languages spoken in northwest Zambia. The processes include vowel height harmony, partial nasal consonant harmony, complete consonant assimilation, alveolar palatalization, and imbrication. Depending on the phonological properties of the verb, the phenomena affect either the final consonant of the verb root or verb base, the remote past suffix lateral consonant, or both the verb stem final consonant and the suffix. The remote past suffix shows at least ten morphophonemic alternations which are \emph{{}-ile, -ile, -ele, -ine, -ene, -ishe, -eshe, -izhe, -ezhe, -inye}\textstyleipa{ and -}\emph{enye}.\footnote{The phonemes /\textstyleipa{ʃ/}, /\textstyleipa{ʒ/, }/t\textstyleipa{ʃ/}, /d\textstyleipa{ʒ/} and \textstyleipa{/}\href{http://en.wikipedia.org/wiki/Ɲ}{\textstyleipa{ɲ}}\textstyleipa{/} are orthographically represented by \textit{sh, zh, ch, j,} and \textit{ny}, respectively in Kaonde.} 
In addition, the final consonant of a number of verb roots or bases alternates between the voiceless alveolar stop /t/ and the voiceless affricate /t\textstyleipa{ʃ}/, and between the voiced alveolar stop /d/, or the lateral /l/ and the voiced affricate /d\textstyleipa{ʒ/ when it comes into contact with the high front vowel of the suffix. With respect to imbrication, the process displays different patterns, and may be accompanied by other processes that include segment deletion, vowel coalescence, vowel lengthening, and glide formation. }

\begin{styleBodyTextIndent}
In Kaonde, the remote past tense is expressed by means of the inflectional suffix -\emph{ile} in conjunction with the prefix tense maker \emph{{}-a-}\textit{.} Though the morpheme is referred to as \emph{perfect} or \emph{perfective} in many Bantu languages, it codes only the remote past tense and cannot be used to express any other meaning in Kaonde \citep[137]{Wright1977}; it is also attested with the same function in other neighboring languages, namely Chokwe, Luvale (White 1947; 1949; 196...; Horton 1949; Yukawa 1987a) and Lunda (Fisher 1981; Kawasha 2003).\footnote{ White (196-{}-) is a manuscript without an exact date. The perfective or perfect meaning “does not seem to fit its use in Kaonde” \citep{Wright1997}.} \citet[6]{White1947} points out that “this tense is used to denote a past action prior to yesterday” in Chokwe, Luchazi, Lunda, and Luvale. The same applies to Kaonde. Verb forms in the remote past are shown in (1).\footnote{ Some of the data used in this paper are based on my own knowledge of the language, while others are drawn from various written sources. I verified them with speakers of Kaonde.}
\end{styleBodyTextIndent}

%\setcounter{itemize}{0}
\begin{itemize}
\item \end{itemize}
\begin{stylelsSourceline}
a.  n-a-kay-ile
\end{stylelsSourceline}

\begin{stylelsIMT}
  1SG-TNS{}-play-RP
\end{stylelsIMT}

\begin{stylelsTranslationSubexample}
  ‘I played.’
\end{stylelsTranslationSubexample}

\begin{stylelsSourceline}
b.  w-a-tang-ile
\end{stylelsSourceline}

\begin{stylelsIMT}
  she{}-TNS{}-read-RP
\end{stylelsIMT}

\begin{stylelsTranslationSubexample}
  ‘He read.’
\end{stylelsTranslationSubexample}

The sentences in (1a-b) consist of the first person singular subject prefix \emph{n-} and third person \emph{w-} in initial position followed by the tense marker \emph{{}-a-} attached directly to the verb roots \emph{kay} ‘play’ and \textit{tang} ‘read’ with the remote past suffix -\emph{ile} as the last verbal element. \tabref{tab:1} presents examples of simple verb stems on the left, and their remote past forms in the right-hand column. In the examples in the left-hand column of \tabref{tab:1}, the verb stem is composed of the root and the final vowel \textit{a}. On the right is the verb form in the remote past consisting of the verb root and the suffix -\textit{ile}.


 \tabref{tab:1}: Simple and remote past verb stems


\begin{tabular}{llll}
\lsptoprule
\mdseries Stem &  & \mdseries Root-remote past & \\
{\mdseries \emph{ja }}

{\mdseries \emph{fwa }}

{\mdseries \emph{fika }}

{\mdseries \emph{amba }}

{\mdseries \emph{enda }}

\mdseries \emph{imba} & {\mdseries ‘eat’}

{\mdseries ‘die’}

{\mdseries ‘arrive’  }

{\mdseries ‘say’}

{\mdseries ‘move’  }

\mdseries ‘sing’ & {\mdseries \emph{j-ile}}

{\mdseries \emph{fw-ile}}

{\mdseries \emph{fik-ile}}

{\mdseries \emph{amb-ile}}

{\mdseries \emph{end-ele  }}

\mdseries \emph{imb-ile  } & {\mdseries ‘ate’}

{\mdseries ‘died’}

{\mdseries ‘arrive’}

{\mdseries ‘said’}

{\mdseries ‘moved’}

\mdseries ‘sang’\\
\lspbottomrule
\end{tabular}
\begin{styleBodyTextIndent}
This paper is organized as follows: §2 discusses the various phonological processes that affect either the vowel /i/ or the consonant /l/ of the remote past suffix -\emph{ile} in Kaonde. §3 examines palatalization which only affects the final consonant of the verb root or base. §4 presents the imbrication process and the concomitant phonological changes it brings about. §5 concludes the paper.
\end{styleBodyTextIndent}

\begin{stylelsSectioni}
2 Harmony 
\end{stylelsSectioni}

Kaonde has three types of assimilatory phonological processes whereby segments become identical or more or less identical. They include vowel height harmony, partial nasal consonant harmony, and coronal consonant assimilation. They all target either the consonantal segment /l/ or the initial vowel of the suffix by spreading a feature or features rightward from the verb base to the suffix.

\begin{stylelsSectionii}
2.1 Vowel height harmony
\end{stylelsSectionii}

Like most of the languages spoken in Zambia, the process of vowel height harmony involves assimilation of the suffixal vowel /i/ in Kaonde. In other words, a verb root vowel induces assimilation in the vowel of the suffix. If the verb root contains the vowel /i/, /u/ or /a/, the first vowel of the suffix is a high front vowel. However, mid vowels /e/ and /o/ trigger a left-to-right direction of harmony targeting the initial high vowel of the remote past suffix, which assimilates completely to the height of the preceding mid vowel. It becomes mid front /e/ if it is preceded by a mid-vowel. This type of height harmony occurs across an intervening consonant.

\begin{itemize}
\item \begin{stylelsLanginfo}
i → e/ mid-vowel (C) \_\_\_
\end{stylelsLanginfo}\end{itemize}

Compare the examples in \tabref{tab:2} and \tabref{tab:3}. Verb roots with non-mid vowels in \tabref{tab:2} have an inflectional suffix with an initial high front vowel /i/, while those with mid vowels in \tabref{tab:3} contain a suffix with an initial mi vowel /e/. 

%%please move \begin{table} just above \begin{tabular
\begin{table}
\caption{No height harmony in remote past suffix}
\label{tab:2}
\end{table}

\begin{tabular}{llll}
\lsptoprule
\mdseries Root-FV &  & \mdseries Root-remote past & \\
{\mdseries \emph{amb-a}}

{\mdseries \emph{lamb-a}}

{\mdseries \emph{zh-a}}

{\mdseries \emph{languluk-a}}

{\mdseries \emph{itab-a}}

{\mdseries \emph{bijik-a}}

{\mdseries \emph{jimb-a}}

{\mdseries \emph{imb-a}}

{\mdseries \emph{zhiik-a}}

{\mdseries \emph{ub-a}}

\mdseries \emph{yuk-a} & {\mdseries ‘speak’}

{\mdseries ‘be smooth’}

{\mdseries ‘dance’}

{\mdseries ‘think’}

{\mdseries ‘answer’}

{\mdseries ‘shout’}

{\mdseries ‘deceive’}

{\mdseries ‘sing’}

{\mdseries ‘bury’}

{\mdseries ‘do’}

\mdseries ‘know’ & {\mdseries \emph{amb-ile}}

{\mdseries \emph{lamb-ile}}

{\mdseries \emph{zh-ile}}

{\mdseries \emph{languluk-ile}}

{\mdseries \emph{itab-ile}}

{\mdseries \emph{bijik-ile}}

{\mdseries \emph{jimb-ile}}

{\mdseries \emph{imb-ile}}

{\mdseries \emph{zhiik-ile}}

{\mdseries \emph{ub-ile}}

\mdseries \emph{yuk-ile} & {\mdseries ‘spoke’}

{\mdseries ‘was smooth’}

{\mdseries ‘danced’}

{\mdseries ‘thought’}

{\mdseries ‘answered’}

{\mdseries ‘shouted’}

{\mdseries ‘deceived’}

{\mdseries ‘sang’}

{\mdseries ‘buried’}

{\mdseries ‘did’}

\mdseries ‘knew’\\
\lspbottomrule
\end{tabular}
%%please move \begin{table} just above \begin{tabular
\begin{table}
\caption{Mid-vowel height harmony in remote past suffix}
\label{tab:3}
\end{table}

\begin{tabular}{llll}
\lsptoprule
{\mdseries \emph{keb-a}}

{\mdseries \emph{fwenk-a}}

{\mdseries \emph{kenket-a}}

{\mdseries \emph{leet-a}}

{\mdseries \emph{nemb-a}}

{\mdseries \emph{lob-a}}

{\mdseries \emph{lomb-a}}

{\mdseries \emph{owv-a}}

\mdseries \emph{sok-a} & {\mdseries ‘look for’}

{\mdseries ‘swim’}

{\mdseries ‘look around’}

{\mdseries ‘bring’}

{\mdseries ‘write’}

{\mdseries ‘catch fish’}

{\mdseries ‘ask’}

{\mdseries ‘wash’}

\mdseries ‘burn’ & {\mdseries \emph{keb-ele}}

{\mdseries \emph{fwenk-ele}}

{\mdseries \emph{kenket-ele}}

{\mdseries \emph{leet-ele}}

{\mdseries \emph{nemb-ele}}

{\mdseries \emph{lob-ele}}

{\mdseries \emph{lomb-ele}}

{\mdseries \emph{ovw-ele}}

\mdseries \emph{sok-ele} & {\mdseries ‘looked for’}

{\mdseries ‘swam’}

{\mdseries ‘looked around’}

{\mdseries ‘brought’}

{\mdseries ‘wrote’}

{\mdseries ‘caught fish’}

{\mdseries ‘asked’}

{\mdseries ‘washed’}

\mdseries ‘burned’\\
\lspbottomrule
\end{tabular}
\begin{stylelsSectionii}
2.2 Nasal consonant harmony 
\end{stylelsSectionii}

\begin{styleBodyTextIndent}
Kaonde, like a many other Bantu languages \citep{Greenberg1951}, presents partial nasal consonant harmony that operates on suffixes if a verb root ends in a non-palatal or non-velar nasal /m/ or /n/. This process assimilates the segment /l/ of the suffix to an alveolar nasal by harmonizing it with the preceding nasal segment of the root in a single feature. That is, the base nasal triggers left-to-right harmony targeting the lateral consonant in suffix.\footnote{ Unlike some Bantu languages such as Kikongo, Kiyaka, Chokwe, etc. nasalization does not operate in Kaonde if the input in NVC-\textit{ile} or CVNVC-\textit{ile}, where the final consonant of the verb root is not a nasal.  } Notice that nasal consonants induce nasalization of underlying /l/ of the remote past suffix in \tabref{tab:4}. 
\end{styleBodyTextIndent}

\begin{styleBodyTextIndent}
Harmony does not normally operate on consonant clusters consisting of a nasal and an oral consonant. The final example in part a of \tabref{tab:4} shows nasal harmony even though the consonant /v/ intervenes between the high vowel and the verb stem nasal. This could be attributed to the fact that the root ends in a vowel. 
\end{styleBodyTextIndent}

%%please move \begin{table} just above \begin{tabular
\begin{table}
\caption{Nasal consonant harmony in remote past suffix}
\label{tab:4}
\end{table}

\begin{tabular}{lllll} & \multicolumn{2}{l}{\mdseries Root-FV} & \multicolumn{2}{l}{\mdseries Root-remote past}\\
\lsptoprule
\mdseries a. & {\mdseries \emph{lam-a}}

{\mdseries \emph{kaan-a}}

{\mdseries \emph{jim-zhindam-a}}

{\mdseries \emph{pum-a}}

{\mdseries \emph{nun-a}}

\mdseries \emph{unvw-a} & {\mdseries ‘keep’}

{\mdseries ‘refuse’}

{\mdseries ‘cultivate’}

{\mdseries ‘be quiet’}

{\mdseries ‘beat’}

{\mdseries ‘become fat’}

{\mdseries ‘understand, hear,}

\mdseries feel’ & {\mdseries \emph{lam-ine}}

{\mdseries \emph{kaan-ine}}

{\mdseries \emph{jim-ine}}

{\mdseries \emph{zhindam-ine}}

{\mdseries \emph{pum-ine}}

{\mdseries \emph{nun-ine}}

\mdseries \emph{unvw-ine} & {\mdseries ‘kept’}

{\mdseries ‘refused’}

{\mdseries ‘cultivated’}

{\mdseries ‘was/were quiet’}

{\mdseries ‘beat’}

{\mdseries ‘became fat’}

\mdseries ‘heard’\\
\mdseries b. & {\mdseries \emph{nyem-a }}

{\mdseries \emph{nem-a }}

{\mdseries \emph{sem-a}}

{\mdseries \emph{ten-a}}

{\mdseries \emph{tom-a}}

\mdseries \emph{pon-a} & {\mdseries ‘run (away from)’}

{\mdseries ‘be heavy’}

{\mdseries ‘give birth’}

{\mdseries ‘mention’}

{\mdseries ‘drink’}

\mdseries ‘fall’ & {\mdseries \emph{nyem-ene}}

{\mdseries \emph{nem-ene}}

{\mdseries \emph{sem-ene}}

{\mdseries \emph{ten-ene  }}

{\mdseries \emph{tom-ene }}

\mdseries \emph{pon-ene} & {\mdseries ‘ran, ran way from’}

{\mdseries ‘was/were important’}

{\mdseries ‘gave birth’}

{\mdseries ‘mentioned’}

{\mdseries ‘drank’}

\mdseries ‘fell’\\
\lspbottomrule
\end{tabular}
\begin{stylelsSectionii}
2.3 Complete consonant assimilation
\end{stylelsSectionii}

In addition to vowel height harmony and nasal consonant harmony, Kaonde also shows complete consonant assimilation across a vowel. This is the type of progressive assimilation that affects the consonant /l/ of the remote past suffix when it attaches to verb stems that contain the alveopalatal fricative consonants /\textstyleipa{ʃ/,}\textstyleipa{ /ʒ/.} The verb can be an underived monomorphemic root or a derived causative. The segment assimilates totally to the preceding verb root final consonant becoming \textit{sh}. Thus, the remote past suffix -\emph{ile}\textit{ }surfaces as [i\textstyleipa{ʃe],}\textstyleipa{ and [iʒe]}.\footnote{ In Bukusu, an /l/ assimilates to a root final consonant preceding /r/ across a vowel \citep{Hyman2003}: \textit{bír-il} (‘pass + applicative’) → \textit{bír-ir}.} The changes in Kaonde are schematized in (3):

\begin{itemize}
\item \begin{stylelsLanginfo}
/il/ → [i\textstyleipa{ʃ],} [i\textstyleipa{ʒ] after verb stems ending in }\textstyleipa{\textit{sh}}\textstyleipa{\textit{ }}\textstyleipa{and }\textstyleipa{\textit{zh}}\textit{.}
\end{stylelsLanginfo}\end{itemize}

Some examples of lateral /l/ assimilation to alveopalatal fricative consonants are in \tabref{tab:5}. In \tabref{tab:5}, part a, the root final consonant spreads rightward all its features to the suffixal consonant /l/, while in part b the consonant of the suffix assimilates completely to the causative suffix of the verb stem. 

%%please move \begin{table} just above \begin{tabular
\begin{table}
\caption{Alveopalatal fricative assimilation in remote past}
\label{tab:5}
\end{table}

\begin{tabular}{lllll} & \mdseries Root-FV &  & \multicolumn{2}{l}{\mdseries Root-remote past}\\
\lsptoprule
\mdseries a. & {\mdseries \emph{konsh-a}}

{\mdseries \emph{baanzh-a}}

\mdseries \emph{kozh-a  } & {\mdseries ‘can’}

{\mdseries ‘delay’  }

\mdseries ‘make sick, hurt’ & {\mdseries \emph{konsh-esh-e banzh-izh-e}}

\mdseries \emph{kozh-ezh-e} & {\mdseries ‘could’}

{\mdseries ‘delayed’}

\mdseries ‘made sick, hurt’\\
\mdseries b. & \mdseries \emph{jiish-a} & \mdseries ‘feed’ & \mdseries \emph{jiish-ish-e} & \mdseries ‘fed’\\
\lspbottomrule
\end{tabular}
\begin{styleBodyTextIndent}
This same process also affects a set of monosyllabic verb roots ending in [l]. The root final consonant palatalizes to \emph{zh} when the causative suffix -\emph{ish} is added.\footnote{ The data in \tabref{tab:6} involve imbrication, discussed in §4. The type of imbrication that       turns the voiceless consonant \emph{sh} of the causative suffix into a voiced one is also\par        attested in Chiluba \citep[73]{Lukusa1993}.\par } Compare the underived verb stem in the first column of \tabref{tab:6} with their causativized verb forms in the second column.
\end{styleBodyTextIndent}

%%please move \begin{table} just above \begin{tabular
\begin{table}
\caption{Palatalization of root-final /l/ in causative formation}
\label{tab:6}
\end{table}

\begin{tabular}{llll}
\lsptoprule
\mdseries Root-FV &  & \mdseries Root-caus- FV & \\
{\mdseries \emph{jil-a }}

{\mdseries \emph{bweel-a}}

\mdseries \emph{tweel-a} & {\mdseries ‘make cry’}

{\mdseries ‘bring back’}

\mdseries ‘enter’ & {\mdseries \emph{j-izh-a }}

{\mdseries \emph{bw-eezh-a }}

\mdseries \emph{tw-eezh-a} & {\mdseries ‘make cry’}

{\mdseries ‘bring back’}

\mdseries ‘cause to enter, insert’\\
\lspbottomrule
\end{tabular}
The /l/ of the remote past suffix assimilates completely to the causative consonant when the two suffixes occur together (\tabref{tab:7}).

\textbf{\tabref{tab:7}:  Alveopalatal fricative assimilation in remote past causative}

\begin{tabular}{llll}
\lsptoprule
\mdseries Causative verb-FV &  & \mdseries Causativized verb-remote past & \\
{\mdseries \emph{jizh-a }}

{\mdseries \emph{bweezh-a }}

\mdseries \emph{tweezha} & {\mdseries ‘make cry’}

{\mdseries ‘bring back’}

\mdseries ‘bring in’ & {\mdseries \emph{jizh-izhe }}

{\mdseries \emph{bwezh-ezhe }}

\mdseries \emph{twezh-ezhe} & {\mdseries ‘made cry’}

{\mdseries ‘brought back’}

\mdseries ‘brought in’\\
\lspbottomrule
\end{tabular}
The imbricated verb form in (4) displays this type of assimilation when it occurs in the remote past tense. 

\begin{itemize}
\item \end{itemize}
\begin{stylelsSourceline}
\emph{mweesh-a}    \emph{mwesh-eshe}  
\end{stylelsSourceline}

\begin{stylelsTranslation}
‘show’    ‘showed’
\end{stylelsTranslation}

Unlike languages such as Bemba and Tonga which show double frication as a result of suffixation of the causative morpheme \emph{{}-i-} (Hyman 1995; 2003) in conjunction with either the applicative suffix or the perfective suffix, the phenomenon also affects certain underived verbs.\footnote{ An anonymous reviewer has suggested that this may be a case of cyclic derivation      induced by an underlying causative /i/. However, the Kaonde assimilatory process\par       affects both underived and derived verb forms, suggesting it is no due to an\par       underlying causative.\par }

\begin{styleBodyTextIndent}
The /l/ of the remote past also assimilates to a stem final palatalized alveolar nasal consonant \emph{ny}. First, as in many Bantu languages, palatalization of the verb stem consonant is triggered by the affixation of a second type of causative suffix \emph{{}-y-}\emph{\textup{,}} though this is not productive in Kaonde where it only attaches to some verb stems ending in the alveolar nasal /n/ and to some extent to the bilabial nasal /m/. However, this palatalization affects only the alveolar nasal and not /m/. Assimilation of the remote past /l/ to the stem consonant is illustrated by the data in \tabref{tab:8}, where the consonant of the remote past tense suffix -\emph{ile } turns into \emph{ny}.  
\end{styleBodyTextIndent}

\begin{styleBodyTextIndent}
\textbf{\tabref{tab:8}: Palatalized nasal assimilation in remote past}
\end{styleBodyTextIndent}

\begin{tabular}{llll}
\lsptoprule
\mdseries Root-FV &  & \mdseries Root-remote past & \\
{\mdseries \emph{kany-a }}

{\mdseries \emph{fweny-a }}

\mdseries \emph{chiny-a} & {\mdseries ‘forbid’}

{\mdseries ‘approach, move near’}

\mdseries ‘scare’ & {\mdseries \emph{kany-inye}}

{\mdseries \emph{fweny-enye}}

\mdseries \emph{chiny-inye} & {\mdseries ‘forbade’}

{\mdseries ‘approached’}

\mdseries ‘scared’\\
\lspbottomrule
\end{tabular}
\begin{stylelsSectioni}
3 Palatalization 
\end{stylelsSectioni}

Palatalization is another phonological change that alveolar consonants are susceptible to as a result of suffixation of the remote past morpheme in Kaonde. The suffix triggers this phenomenon by palatalizing the root/stem final consonant.\footnote{ This applies also to the causative -\textit{ish}, applicative -\textit{il}, and neuter -\textit{ik}.} This morphophonological process changes either their place of articulation from alveolar to alveopalatal or both place and manner of articulation. Subject to palatalization are coronal consonants, namely sibilant fricatives /s/ and /z/, the voiceless stop /t/, and the lateral /l/.\footnote{ Palatalization is also very productive in other languages such as Chokwe and Luvale spoken in the same region in  Zambia. In these languages, /t/ followed by /i/ becomes [tʃ], /nd/ changes to /dʒ/, /n/ to /\href{http://en.wikipedia.org/wiki/Ɲ}{\textstyleipa{ɲ}}\textstyleipa{/}, /s/ to /ʃ/, and /z/ becomes [ʒ]. } 

\begin{styleBodyTextIndent}
\textstyleipa{The verb stem final coronal sibilant fricative consonants }/s/, and /z/ change only their anterior feature from [+anterior, + coronal] to [-anterior, + coronal]. The two segments palatalize to /\textstyleipa{ʃ}/\textstyleipa{ and /}ʒ\textstyleipa{/,} respectively, if they occur in front of the high front vowel of the remote past suffix.\footnote{ In Nyamwezi (Maganga \& Schadeberg 1992), the alveolar fricative /s/ and the lateral /l/ palatalize to [\textstyleipa{ʃ] and [dʒ}], respectively, like in Kaonde. However, in Mashi, if the verb stem final consonant is an alveolar fricative /s/ or /z/, it changes to velar stop [k] and [g], respectively, preceding the applicative extension \textit{iz} or \textit{ez} (Bashi Murhi-Orhakube 2008): \emph{ku-yuus-a }\emph{\textup{‘to finish’}}\emph{\textup{→}} \emph{ku-yuuk-iz-a} ‘\emph{\textup{to finish for’.}}  \par } Compare the root final consonant of the infinitive verbs in the left column of \tabref{tab:9} and those of the verb forms of the remote past in the right hand column.
\end{styleBodyTextIndent}

%%please move \begin{table} just above \begin{tabular
\begin{table}
\caption{Palatalization of root-final coronals}
\label{tab:9}
\end{table}

\begin{tabular}{lllll} & \multicolumn{2}{l}{\mdseries Root-FV} & \multicolumn{2}{l}{\mdseries Root-remote past}\\
\lsptoprule
\mdseries a. & {\mdseries \emph{as-a}}

{\mdseries \emph{fis-a}}

\mdseries \emph{sans-a} & {\mdseries ‘shoot’}

{\mdseries ‘hide’}

\mdseries ‘sprinkle’ & {\mdseries \emph{ash-ile}}

{\mdseries \emph{fish-ile}}

\mdseries \emph{sansh-ile} & {\mdseries ‘shot’}

{\mdseries ‘hid’}

\mdseries ‘sprinkled’\\
\mdseries b. & \mdseries \emph{baanz-a} & \mdseries ‘make a fire’ & \mdseries \emph{baanzh-ile} & \mdseries ‘made a fire’\\
\lspbottomrule
\end{tabular}
The coronal voiceless stop /t/ changes in manner of articulation to become an alveopalatal affricate /t\textstyleipa{ʃ/. Thus, v}erb root/base final consonants alternate between [+cor, +ant] and [+cor, -ant] (\tabref{tab:10}).

%%please move \begin{table} just above \begin{tabular
\begin{table}
\caption{Palatalization of root-final /t/}
\label{tab:10}
\end{table}

\begin{tabular}{llll}
\lsptoprule
\mdseries Root/stem-FV &  & \mdseries Root-remote past & \\
{\mdseries \emph{angat-a}}

{\mdseries \emph{it-a}}

{\mdseries \emph{kwat-a}}

{\mdseries \emph{pit-a}}

\mdseries \emph{sant-a} & {\mdseries ‘take from s.o.’}

{\mdseries ‘call’}

{\mdseries ‘arrest, hold’}

{\mdseries ‘pass’}

\mdseries ‘thank’ & {\mdseries \emph{angach-ile}}

{\mdseries \emph{pich-ile}}

{\mdseries \emph{kwach-ile}}

{\mdseries \emph{pich-ile }}

\mdseries \emph{sanch-ile} & {\mdseries ‘took from s.o.’}

{\mdseries ‘called’}

{\mdseries ‘arrested, held’}

{\mdseries ‘passed’}

\mdseries ‘thanked’\\
\lspbottomrule
\end{tabular}
The voiced counterpart of /t/ only occurs after the alveolar nasal /n/, realized as [nd]. It palatalizes to /dʒ/. This can be seen through the alternation of the root final consonant in \tabref{tab:11}.

%%please move \begin{table} just above \begin{tabular
\begin{table}
\caption{Palatalization of root-final /nt/ [nd]}
\label{tab:11}
\end{table}

\begin{tabular}{llll}
\lsptoprule
\multicolumn{2}{l}{\mdseries Root-FV} & \multicolumn{2}{l}{\mdseries Root-remote past}\\
{\mdseries \emph{fund-a }}

\mdseries \emph{yand-a} & {\mdseries ‘teach’}

\mdseries ‘suffer’ & {\mdseries \emph{funj-ile }}

\mdseries \emph{yanj-ile} & {\mdseries ‘taught’}

\mdseries ‘suffered’\\
\lspbottomrule
\end{tabular}
Also susceptible to palatalization are verb roots/bases that end in the lateral /l/ which becomes a voiced alveopalatal affricate /d\textstyleipa{ʒ/ }before \textit{{}-ile}. In short, this consonant, which varies between [l] and a central flap \textstyleipa{[ɾ]} or a lateral flap \textstyleipa{[ɺ], }alternates with \textstyleipa{/}d\textstyleipa{ʒ/}. Consider mutation of the verb final consonant /l/ to the alveopalatal affricate \textstyleipa{/}d\textstyleipa{ʒ/ in \tabref{tab:12}}.\footnote{ In Lunda the cognates to \textit{shaala} ‘remain’ and t\textit{waala }‘take’ in \tabref{tab:12} undergo imbrication to become \textit{sheeli} and \textit{tweeli} respectively.}

%%please move \begin{table} just above \begin{tabular
\begin{table}
\caption{Voiced palatalization of root-final /l/}
\label{tab:12}
\end{table}

\begin{tabular}{llll}
\lsptoprule
\multicolumn{2}{l}{\mdseries Root/stem-FV} & \multicolumn{2}{l}{\mdseries Root-remote past}\\
{\mdseries \emph{bul-a}}

{\mdseries \emph{lala}}

{\mdseries \emph{vul-a}}

{\mdseries \emph{vuul-a}}

{\mdseries \emph{vwal-a}}

{\mdseries \emph{shal-a}}

{\mdseries \emph{twal-a}}

{\mdseries \emph{ingil-a}}

\mdseries \emph{jil-a} & {\mdseries ‘lack’}

{\mdseries ‘break’}

{\mdseries ‘be plenty’}

{\mdseries ‘undress’}

{\mdseries ‘wear’}

{\mdseries ‘remain’}

{\mdseries ‘take’}

{\mdseries ‘work’}

\mdseries ‘cry’ & {\mdseries \emph{buj-ile}}

{\mdseries \emph{laj-ile}}

{\mdseries \emph{vuj-ile}}

{\mdseries \emph{vuuj-ile}}

{\mdseries \emph{vwaj-ile}}

{\mdseries \emph{shaj-ile}}

{\mdseries \emph{twaj-ile} }

{\mdseries \emph{ingij-ile}}

\mdseries \emph{jij-ile} & {\mdseries ‘lacked’}

{\mdseries ‘broke’}

{\mdseries ‘was plenty’}

{\mdseries ‘undressed’}

{\mdseries ‘wore’}

{\mdseries ‘remained’}

{\mdseries ‘took’}

{\mdseries ‘worked’}

\mdseries ‘cried’\\
\lspbottomrule
\end{tabular}
The applicative suffix -\emph{il} changes to -\emph{ij}\textit{ }when it precedes the remote past suffix.\footnote{ Like the remote past, the causative, applicative, and perfective suffixes also condition palatalization. For example, based on the root \textit{kas} ‘tie’, Kaonde has the intensive form \textit{kash-ish-a} ‘tie securely’, \textit{kash-il-a} ‘tie for’, and \textit{kash-ij-il-a} ‘tie completely for’. \par   \par } This is illustrated by the examples in the right column of \tabref{tab:13} where the inflectional remote past morpheme follows the applicative. The last example in \tabref{tab:13} possesses two palatalized segments, i.e. the root final consonant /\textstyleipa{ʃ}/ and /d\textstyleipa{ʒ}/ of the applicative extension. 

%%please move \begin{table} just above \begin{tabular
\begin{table}
\caption{Voiced palatalization of applicative /l/}
\label{tab:13}
\end{table}

\begin{tabular}{llll}
\lsptoprule
\multicolumn{2}{l}{\mdseries Root-applicative-FV } & \multicolumn{2}{l}{\mdseries Root-applicative-remote past}\\
{\mdseries \emph{fw-il-a }}

{\mdseries \emph{twaj-il-a }}

{\mdseries \emph{kajip-il-a}}

{\mdseries \emph{tambw-il-a}}

{\mdseries \emph{pum-in-a}}

\mdseries \emph{kasil-a} & {\mdseries ‘die for’}

{\mdseries ‘take for’}

{\mdseries ‘be angry with’}

{\mdseries ‘receive for’}

{\mdseries ‘hit at’}

\mdseries ‘tie for’ & {\mdseries \emph{fw-ij-ile }}

{\mdseries \emph{twaj-ij-ile }}

{\mdseries \emph{kajip-ij-ile }}

{\mdseries \emph{tambw-ij-ile}}

{\mdseries \emph{pum-ij-ile}}

\mdseries \emph{kash-ij-ile} & {\mdseries ‘died for’}

{\mdseries ‘took for’}

{\mdseries ‘was angry with’}

{\mdseries ‘received for’}

{\mdseries ‘hit at’}

\mdseries ‘tied for’\\
\lspbottomrule
\end{tabular}
\begin{styleBodyTextIndent}
It is worth noting that the vowel of the remote past tense suffix does not harmonize in height with the mid vowel of the verb root if the former is preceded by an applicative extension. In other words, vowel harmony only affects the suffix closest to the root. Compare the remote past examples containing one suffix in the first row of \tabref{tab:14} with those containing two suffixes in the second row. The remote past suffix -\textit{ele} in the right-hand column has harmonized with the vowel of the verb root; while if the applicative occurs immediately after the verb, it assimilates in height to the root  vowel, and not that of the remote past which maintains its underlying form -\textit{ile}, as it is not in the immediate post-root position.
\end{styleBodyTextIndent}

\begin{styleBodyTextIndent}
\textbf{\tabref{tab:14}: Height harmony in remote past}
\end{styleBodyTextIndent}

\begin{tabular}{lllll} & \multicolumn{2}{l}{\mdseries Root/Stem-FV} & \multicolumn{2}{l}{\mdseries Root/Stem-remote past}\\
\lsptoprule
\mdseries harmony after root & {\mdseries \emph{lek-a}}

{\mdseries \emph{leng-a}}

{\mdseries \emph{nemb-a  }}

{\mdseries \emph{komb}{}-a} & {\mdseries ‘to leave off’}

{\mdseries ‘to create’}

{\mdseries ‘to write’}

{\mdseries ‘to sweep’} & {\mdseries \emph{lek-ele}}

{\mdseries \emph{leng-ele}}

{\mdseries \emph{nemb-ele}}

{\mdseries \emph{komb-ele}} & {\mdseries ‘left off’}

{\mdseries ‘created’}

{\mdseries ‘wrote’}

{\mdseries ‘swept’}\\
\mdseries no harmony after applicative & {\mdseries \emph{ketekel-a}}

{\mdseries \emph{lekel-a}}

{\mdseries \emph{lengel-a}}

{\mdseries \emph{mwekel-a}}

{\mdseries \emph{nembel-a}}

{\mdseries \emph{pembel-a}}

{\mdseries \emph{sekel-a}}

{\mdseries \emph{kombel-a}}

\mdseries \emph{sokel-a} & {\mdseries ‘trust’}

{\mdseries ‘to let off’}

{\mdseries ‘to create for’}

{\mdseries ‘appear to/at’}

{\mdseries ‘to write to/for’}

{\mdseries ‘wait for’}

{\mdseries ‘rejoice’}

{\mdseries ‘to sweep for’}

\mdseries ‘burn for’ & {\mdseries \emph{ketekej-ile}}

{\mdseries \emph{lekej-ile}}

{\mdseries \emph{lengej-ile}}

{\mdseries \emph{mwekej-ile}}

{\mdseries \emph{nembej-ile}}

{\mdseries \emph{pembej-ile}}

{\mdseries \emph{sekej-ile}}

{\mdseries \emph{kombej-ile}}

\mdseries \emph{sokej-ila} & {\mdseries ‘trusted’}

{\mdseries ‘let off’}

{\mdseries ‘created for’}

{\mdseries ‘appeared to/at’}

{\mdseries ‘wrote to/for’}

{\mdseries ‘waited for’}

{\mdseries ‘rejoiced’}

{\mdseries ‘swept for’}

\mdseries ‘burned for’\\
\lspbottomrule
\end{tabular}
\begin{styleBodyTextIndent}
Nasal harmony in the remote past does not occur if a root carries the applicative before the remote past. In this case, neither the consonant of the applicative extension nor that of the remote past nasalize. Instead, the applicative -\emph{il} palatalizes to -\emph{ij} before the remote past suffix. 
\end{styleBodyTextIndent}

\begin{styleBodyTextIndent}
\textbf{\tabref{tab:15}: Palatalization of applicative /l/ (without nasal harmony)  }
\end{styleBodyTextIndent}

\begin{tabular}{llll}
\lsptoprule
\multicolumn{2}{l}{\mdseries Root-applicative-FV } & \multicolumn{2}{l}{\mdseries palatalization of applicative /l/ before remote past}\\
{\mdseries \emph{tam-in-a}}

{\mdseries \emph{pum-in-a}}

{\mdseries \emph{tum-in-a}}

{\mdseries \emph{nyem-en-a}}

\mdseries \emph{sem-en-a} & {\mdseries ‘be bad for’}

{\mdseries ‘beat/hit at’}

{\mdseries ‘send to’}

{\mdseries ‘run to/for’}

\mdseries ‘give birth in’ & {\mdseries \emph{tamij-ile}}

{\mdseries \emph{pumij-ile}}

{\mdseries \emph{tumij-ile}}

{\mdseries \emph{nyemej-ile}}

\mdseries \emph{sem-ej-ile} & {\mdseries ‘was bad for’}

{\mdseries ‘beat/hit at’}

{\mdseries ‘sent to’}

{\mdseries ‘ran to/for’}

\mdseries ‘gave birth in’\\
\lspbottomrule
\end{tabular}
As seen in the first column of \tabref{tab:15}, the underlying lateral consonant of the applicative suffix of the verb stems harmonizes with the nasal consonant of the preceding the verb root. The examples in the right-hand column show the applicative /l/ (or /n/) does not change in the applicative if the remote past suffix occurs. Nevertheless, the applicative consonant /l/ close to the root palatalizes to /d\textstyleipa{ʒ}/. One might argue that the applicative /l/ has become /n/, which then turns into /d\textstyleipa{ʒ}/. 

\begin{styleBodyTextIndent}
It is to be noted that the remote past suffix does not trigger palatalization of the root final consonant if the suffix is attached directly to a verb root containing a mid-vowel /e/ or /o/. This behavior suggests that vowel height harmony applies before palatalization. Consider the examples in \tabref{tab:16}, where palatalization fails to apply to verb roots containing a mid-vowel. 
\end{styleBodyTextIndent}

%%please move \begin{table} just above \begin{tabular
\begin{table}
\caption{No palatalization of root-final consonant after mid-vowel}
\label{tab:16}
\end{table}

\begin{tabular}{llll}
\lsptoprule
\multicolumn{2}{l}{\mdseries Root-FV } & \multicolumn{2}{l}{\mdseries Root-remote past}\\
{\mdseries \emph{kwend-a}}

{\mdseries \emph{send-a}}

{\mdseries \emph{pel-a}}

{\mdseries \emph{loot-a}}

{\mdseries \emph{leet-a}}

\mdseries \emph{kos-a} & {\mdseries ‘walk’}

{\mdseries ‘carry’}

{\mdseries ‘grind’}

{\mdseries ‘dream’}

{\mdseries ‘bring’}

\mdseries ‘be strong’ & {\mdseries \emph{end-ele}}

{\mdseries \emph{send-ele}}

{\mdseries \emph{pel-ele}}

{\mdseries \emph{loot-ele}}

{\mdseries \emph{leet-ele}}

\mdseries \emph{kos-ele} & {\mdseries ‘walked’}

{\mdseries ‘carried’}

{\mdseries ‘ground’}

{\mdseries ‘dreamed’}

{\mdseries ‘brought’}

\mdseries ‘was strong’\\
\lspbottomrule
\end{tabular}
However, palatalization of a stem final /l/ takes place before the remote past suffix if that /l/ in the stem is part of the suffix or extension, even if the stem has a mid-vowel. Most of the extensions have lost meanings and verbs carrying them are lexicalized. The examples in \tabref{tab:17} show how the consonant /l/ palatalizes to an alveopalatal affricate \textit{j }before the remote past suffix.

%%please move \begin{table} just above \begin{tabular
\begin{table}
\caption{Palatalization of stem-final consonant after mid-vowel}
\label{tab:17}
\end{table}

\begin{tabular}{llll}
\lsptoprule
\multicolumn{2}{l}{\mdseries Stem-FV } & \multicolumn{2}{l}{\mdseries Stem-remote past}\\
{\mdseries \emph{londel-a}}

{\mdseries \emph{imeen-a}}

\mdseries \emph{mwekel-a} & {\mdseries ‘follow’}

{\mdseries ‘stand for/at’}

\mdseries ‘appear at’ & {\mdseries \emph{londej-ile}}

{\mdseries \emph{imeneej-ile}}

\mdseries \emph{mwekej-ile} & {\mdseries ‘followed’}

\mdseries ‘stood for/at/’ ‘appeared at’\\
\lspbottomrule
\end{tabular}
\begin{stylelsSectioni}
4 Imbrication
\end{stylelsSectioni}

Imbrication is a phonological pattern in some Bantu languages whereby the perfective or remote past suffix is overlaid and fuses with a preceding root or base. Depending on the language, the suffix either splits into two separate parts or the verb root or base loses its last consonant. Normally, suffixation of the remote past suffix -\emph{ile} to a verb root results in the elongation of the verb by at least one syllable, as exemplified by the data in §2. For clarification, consider the uninflected verb forms on the left, and those with regular suffixation of the remote past morpheme on the right in \tabref{tab:18}. 

%%please move \begin{table} just above \begin{tabular
\begin{table}
\caption{Simple and remote past stems}
\label{tab:18}
\end{table}

\begin{tabular}{llll}
\lsptoprule
\multicolumn{2}{l}{\mdseries Root-FV } & \multicolumn{2}{l}{\mdseries Root-remote past}\\
{\mdseries \emph{sak-a}}

{\mdseries \emph{imb-a}}

{\mdseries \emph{keb-a}}

{\mdseries \emph{pot-a}}

\mdseries \emph{tum-a} & {\mdseries ‘want’}

{\mdseries ‘sing’}

{\mdseries ‘look for’}

{\mdseries ‘buy’}

\mdseries ‘send’ & {\mdseries \emph{sak-ile}}

{\mdseries \emph{imb-ile}}

{\mdseries \emph{keb-ele}}

{\mdseries \emph{pot-ele}}

\mdseries \emph{tum-ine} & {\mdseries ‘wanted’}

{\mdseries ‘sang’}

{\mdseries ‘looked for’}

{\mdseries ‘bought’}

{\mdseries ‘sent’}\\
\lspbottomrule
\end{tabular}
However, this is not the case with several verbs that bear certain types of derivational suffixes, frozen suffixes, and extensions. The suffixation of the remote past morpheme to such verb bases results in a phonological phenomenon termed \emph{imbrication} \citep{Bastin1983}. This process usually brings about sound changes which alter the resulting verb size without increasing its number of syllables. The concomitant effects of the imbrication process, which depend on the property of the verbal suffix or extension may include segmental deletion, segmental insertion, gliding, and vowel coalescence. Consider the examples in \tabref{tab:19}. 

%%please move \begin{table} just above \begin{tabular
\begin{table}
\caption{Imbrication in remote past formations}
\label{tab:19}
\end{table}

\begin{tabular}{llll}
\lsptoprule
\multicolumn{2}{l}{\mdseries Base-FV } & \multicolumn{2}{l}{\mdseries Remote past}\\
{\mdseries \emph{iman-a}}

\mdseries \emph{ikal-a} & {\mdseries ‘stand’}

\mdseries ‘sit, stay’ & {\mdseries \emph{im-eene}}

\mdseries \emph{ik-eele} & {\mdseries ‘stood’}

\mdseries ‘sat, stayed’\\
\lspbottomrule
\end{tabular}
As the two examples in \tabref{tab:19} show, the verb forms in the right column contain the same number of syllables as those on the left column despite the addition of the two-syllable remote past morpheme. This suffixation causes phonological changes in the verb, namely the fusion of the expansion (underlying) -\emph{al} and the inflectional remote past suffix -\emph{ile}\textit{ }into a single chunk resulting in consonantal deletion, which in turn triggers vowel coalescence between /a/ and /i/ accompanied by vowel lengthening. 

\begin{styleBodyTextIndent}
It is worthwhile noting that in Kaonde imbrication is not uniform as it comes in different patterns. The first pattern affects a class of verb stems of more than one syllable of the structure /…C-aC-/ or /…C-aCaC-/. These verbs change their form to /…CeeC-e/ when they attach the remote past suffix. Verbs that belong to this group contain derivational extensions ending in the lateral or alveolar nasal, such as the reciprocal \emph{{}-añan}, the associative -\emph{akan}\textit{,} causativized verbs, and those with a frozen stative -\emph{al} or -\emph{an}.\footnote{ \citet[90]{Meeussen1967} points out that the extension -\emph{ad}{}- “appears partly as an expansion, partly as a suffix with ill-defined meaning”. Though the extensions are identifiable, the seeming roots to which they attach do not have independent meaning. This is attested in Bemba \citep{Hyman1995}.} Most of what appear to be verb roots “suffixed” with -\textit{al} do not occur on their own and have no independent meaning without /al/. The “suffix” forms a lexicalized unit with the root. When these stems attach the remote past suffix, the initial vowel of the suffix is inserted before the final consonant of the verb stem, and the remote past consonant deletes. This segmental loss results in the fusion of the verb stem low vowel /a/ and the vowel of the suffix /i/ into a single front mid vowel /e/ accompanied by compensatory lengthening. That is, the two adjacent vowels merge into a single vowel that combines properties of both source vowels. This imbrication process can be schematically described as in (5).
\end{styleBodyTextIndent}

\ea
 …C-VC-ile  →  …C-ViCe  →  C-eeCe   \\
\gll ik-al-ile        ik-aile    ik-eele\\
 im-an-ile        im-aine    im-eene\\
\z

Listed in \tabref{tab:20} are examples of verb bases with frozen suffixes, reciprocal extensions, and associative extensions.

%%please move \begin{table} just above \begin{tabular
\begin{table}
\caption{Imbrication}
\label{tab:20}
\end{table}

\begin{tabular}{lllll} & \multicolumn{2}{l}{\mdseries Stem-FV} & \multicolumn{2}{l}{\mdseries Stem-remote past}\\
\lsptoprule
\mdseries a. \\ & {\mdseries \emph{kankaman-a}}

\mdseries \emph{unvwañan-a} & {\mdseries ‘tie together’}

\mdseries ‘understand\\
  each other’ & {\mdseries \emph{kankam-eene}}

\mdseries \emph{unvw-añeene} & {\mdseries ‘tied together’}

\mdseries ‘understood\\
  each other\\
\mdseries b.\\ & {\mdseries \emph{esekan-a}}

{\mdseries \emph{kasankan-a}}

{\mdseries \emph{kwakan-a}}

{\mdseries \emph{sambakan-a}}

\mdseries \emph{tentekan-a} & {\mdseries ‘be equal in size’}

{\mdseries ‘tie together’}

{\mdseries ‘unite’}

{\mdseries ‘meet’}

\mdseries ‘shake’ & {\mdseries \emph{esek-eene}}

{\mdseries \emph{kasank-eene}}

{\mdseries \emph{kwatak-eene}}

{\mdseries \emph{sambak-eene}}

\mdseries \emph{tentek-eene} & {\mdseries ‘was equal in size’}

{\mdseries ‘tied together’}

{\mdseries ‘united’}

{\mdseries ‘met’}

\mdseries ‘shook’\\
\mdseries c. & {\mdseries \emph{abany-a}}

\mdseries \emph{esekany-a} & {\mdseries ‘share’}

\mdseries ‘compare’ & {\mdseries \emph{abe-eny-e}}

\mdseries \emph{esek-eenye} & {\mdseries ‘shared’}

\mdseries ‘compared’\\
\mdseries d. & {\mdseries \emph{ikal-a}}

{\mdseries \emph{sangalal-a}}

{\mdseries \emph{iman-a}}

\mdseries \emph{pusan-a} & {\mdseries ‘sit, stay’}

{\mdseries ‘rejoice’}

{\mdseries ‘stand, stop’}

\mdseries ‘differ’ & {\mdseries \emph{ik-eele}}

{\mdseries \emph{sangal-eele}}

{\mdseries \emph{im-eene}}

\mdseries \emph{pus-eene} & {\mdseries ‘sat, stayed’}

{\mdseries ‘rejoiced’}

{\mdseries ‘stood, stopped’}

{\mdseries ‘differed’}\\
\lspbottomrule
\end{tabular}
A very small set of monosyllabic verb bases of the form CV:C- or CwV:C- with the frozen suffix -\emph{an} is also subject to imbrication. However, though these verb forms imbricate, they behave slightly different than the verb stems in \tabref{tab:18} and 19. This is because consonant truncation does not result in fusion between the two adjacent vowels as would normally be expected in that specific context. Instead, the two juxtaposed non-identical vowels remain separate segments as they are pronounced individually. This is illustrated schematically in (6). 

\begin{itemize}
\item \begin{stylelsLanginfo}
C(w)V:C + -\emph{ile}  → CVVCV
\end{stylelsLanginfo}\end{itemize}

Examples of monosyllabic verb roots that have the pattern C(w)V:C- are in \tabref{tab:21}. Observe that the sequence consisting of the root /a/ and the suffixal vowel /i/ does not induce vowel fusion. 

%%please move \begin{table} just above \begin{tabular
\begin{table}
\caption{Imbrication in monosyllabic C(w)V:C- roots}
\label{tab:21}
\end{table}

\begin{tabular}{lllll} & \multicolumn{2}{l}{\mdseries Stem-FV} & \multicolumn{2}{l}{\mdseries Stem-remote past}\\
\lsptoprule
\mdseries CV:C- & {\mdseries \emph{paan-a}}

{\mdseries \emph{taan-a}}

\mdseries \emph{kaan-a} & {\mdseries ‘share, distribute’}

{\mdseries ‘meet, find’}

\mdseries ‘refuse’ & {\mdseries \emph{pa-i-ne}}

{\mdseries \emph{ta-i-ne}}

\mdseries \emph{ka-i-ne} & {\mdseries ‘shared, distributed’}

{\mdseries ‘met, found’}

\mdseries ‘refused’\\
\mdseries CwV:C- & {\mdseries \emph{fwaan-a}}

\mdseries \emph{swaan-a} & {\mdseries ‘be deadly’}

\mdseries ‘inherit, succeed’ & {\mdseries \emph{fwa-i-ne}}

\mdseries \emph{swa-i-ne} & {\mdseries ‘was deadly’}

\mdseries ‘inherited, succeeded’\\
\lspbottomrule
\end{tabular}
Another group of verbs that undergoes imbrication consists of verb stems that are extended with the transitive suffix -\textit{ul }and -\textit{ulul}.\footnote{ The suffix is commonly known as \emph{separative} in Bantu studies though it has other functions. I prefer to call it \emph{transitive}.} Before we discuss these verbs, it is important to look at their intransitive counterparts. The remote past of verb stems with the intransitive counterpart -\textit{uk} and its long form -\textit{uluk} is formed through regular suffixation, as exemplified in \tabref{tab:22}.

%%please move \begin{table} just above \begin{tabular
\begin{table}
\caption{Intransitive verbs with \textit{{}-uk}, and \textit{{}-uluk}}
\label{tab:22}
\end{table}

\begin{tabular}{llll}
\lsptoprule
\multicolumn{2}{l}{\mdseries Stem-FV } & \multicolumn{2}{l}{\mdseries Stem-remote past}\\
\emph{jimuk-a}

\emph{lumbuluk-a}

\emph{tabuk-a}

\emph{shinkuk-a}

\emph{chimuk-a}

\emph{languluk-a}

\emph{kasuluk-a}

\emph{chimauk-a} & ‘be alert’

‘be clear’

‘be torn’

‘open’

‘be snapped’

‘think’

‘be untied’

‘be sewn’ & \emph{jimuk-ile}

\emph{lumbuluk-ile}

\emph{tabuk-ile}

\emph{shinkuk-ile}

\emph{chimuk-ile}

\emph{languluk-ile}

\emph{kasuluk-ile}

\emph{chimauk-ile} & ‘was, were alert’

‘was clear’

‘was/were torn’

‘opened’

‘was/were snapped’

‘thought’

‘was/were untied’

‘was sewn’\\
\lspbottomrule
\end{tabular}
On the other hand, the remote past suffix fuses with derived transitive verbs carrying the transitive suffix -\textit{ul,} inducing imbrication and other phonological modifications. The initial vowel of the remote past suffix is placed before the final consonant of the transitive verb stem, and the lateral consonant of the inflectional tense morpheme deletes resulting in gliding of the final vowel /u/ or /o/ into /w/ due to adjacency with the front high vowel of the triggering suffix. This process is accompanied by compensatory lengthening of the front high vowel of the suffix. In addition, vowel height harmony occurs with verb root stems if the first vowel is mid-back /o/. That is, the vowel /i/ of the suffix lowers to the front mid vowel /e/ by partially assimilating a single feature from the preceding vowel. This imbrication process is schematically described in (7).

\begin{itemize}
\item \begin{stylelsLanginfo}
…C-VC-ile  →  …C-Vile  →  …Cw-iile/…Cw-eele
\end{stylelsLanginfo}\end{itemize}
\begin{stylelsSourceline}
  tamb-ul-a        tamb-uile    tamw-iile
\end{stylelsSourceline}

\begin{stylelsTranslation}
‘receive’          ‘received’
\end{stylelsTranslation}

The examples in \tabref{tab:23} show imbrication of extended verb stems that contain the transitive suffix \textit{{}-}\emph{ul}.

%%please move \begin{table} just above \begin{tabular
\begin{table}
\caption{Imbrication in transitive verbs with \textit{{}-}\emph{ul} and \textit{{}-}\emph{ulul}}
\label{tab:23}
\end{table}

\begin{tabular}{lllll} & \multicolumn{2}{l}{\mdseries Stem-FV} & \multicolumn{2}{l}{\mdseries Stem-remote past}\\
\lsptoprule
\mdseries a. & {\mdseries \emph{kasulul-a}}

{\mdseries \emph{lumbulul-a}}

{\mdseries \emph{shinkul-a}}

{\mdseries \emph{tabul-a}}

\mdseries \emph{tatul-a} & {\mdseries ‘untie’}

{\mdseries ‘explain’}

{\mdseries ‘open’}

{\mdseries ‘tear’}

\mdseries ‘begin’ & {\mdseries \emph{kasulw-iile}}

{\mdseries \emph{lumbulw-iile}}

{\mdseries \emph{shinkw-iile}}

{\mdseries \emph{tabw-iile}}

\mdseries \emph{tatw-iile} & {\mdseries ‘untied’}

{\mdseries ‘explained’}

{\mdseries ‘opened’}

{\mdseries ‘tore’}

\mdseries ‘began’\\
\mdseries b. & {\mdseries \emph{kongol-a}}

{\mdseries \emph{solol-a}}

{\mdseries \emph{songol-a}}

\mdseries \emph{tongol-a} & {\mdseries ‘borrow’}

{\mdseries ‘reveal’}

{\mdseries ‘marry’}

\mdseries ‘choose’ & {\mdseries \emph{kongw-eele}}

{\mdseries \emph{solw-eele}}

{\mdseries \emph{songw-eele}}

\mdseries \emph{tongw-eele} & {\mdseries ‘borrowed’}

{\mdseries ‘revealed’}

{\mdseries ‘married’}

\mdseries ‘chose’\\
\mdseries c. & {\mdseries \emph{chimun-a}}

{\mdseries \emph{jimun-a}}

{\mdseries \emph{chimaul-a}}

\mdseries \emph{onaun-a} & {\mdseries ‘snap’}

{\mdseries ‘alert’}

{\mdseries ‘sew’}

\mdseries ‘destroy’ & {\mdseries \emph{chimw-iine}}

{\mdseries \emph{jimw-iine}}

{\mdseries \emph{chimaw-iine}}

\mdseries \emph{onaw-iine} & {\mdseries ‘snapped’}

{\mdseries ‘alerted’}

{\mdseries ‘sewed’}

\mdseries ‘destroyed’\\
\lspbottomrule
\end{tabular}
As in most of the Bantu languages, imbrication is also observed with the verb root \emph{món} ‘see’ even though it is monosyllabic without an extension. 

\begin{itemize}
\item \begin{stylelsLanginfo}
   Stem          Remote past
\end{stylelsLanginfo}\end{itemize}
\begin{stylelsSourceline}
    mon-a          mw-eene  
\end{stylelsSourceline}

\begin{stylelsTranslation}
‘see’          ‘saw’
\end{stylelsTranslation}

Imbrication also affects verb stems extended with the causative suffix-\emph{ish} and other verb roots ending in the voiced alveopalatal fricative /\textstyleipa{ʒ}/. The imbricated verb formation does not show any sign of the remote past tense suffix on the surface. The suffixal segments -\emph{il} are completely truncated and the presence of the final vowel /e/ is the only indication that they are imbricated verb forms in the remote past. This phenomenon is also observed in isiZulu and Sesotho in which certain verbs exhibit imbrication in the perfect without making “changes to the verb stem” \citep{Monich2015}. Furthermore, this phenomenon does not yield an expected vowel lengthening in the derivational extension domain, as is the case of most of the verbs with irregular suffixation. The form CVC-\emph{ish-ile} changes to CVC-\emph{ish-e} with zero suffixation, as exemplified in \tabref{tab:24}.

%%please move \begin{table} just above \begin{tabular
\begin{table}
\caption{Imbrication in verbs with causative -\emph{ish}}
\label{tab:24}
\end{table}

\begin{tabular}{lllll} & \multicolumn{2}{l}{\mdseries Root-caus-FV} & \multicolumn{2}{l}{\mdseries Remote past}\\
\lsptoprule
\mdseries a. & {\mdseries \emph{kos-esh-a}}

{\mdseries \emph{pos-esh-a}}

{\mdseries \emph{pot-esh-a}}

{\mdseries \emph{keb-esh-a}}

{\mdseries \emph{pulush-a}}

{\mdseries \emph{unvw-ish-a}}

\mdseries \emph{took-esh-a} & {\mdseries ‘make strong’}

{\mdseries ‘cure’}

{\mdseries ‘sell’}

{\mdseries ‘cause to search’}

{\mdseries ‘save’}

{\mdseries ‘listen attentively’}

\mdseries ‘whiten’ & {\mdseries \emph{kos-eshe}}

{\mdseries \emph{pos-eshe}}

{\mdseries \emph{pot-eshe}}

{\mdseries \emph{keb-eshe}}

{\mdseries \emph{pulw-ishe}}

{\mdseries \emph{unvw-ishe}}

\mdseries \emph{took-eshe} & {\mdseries ‘made strong’}

{\mdseries ‘cured’}

{\mdseries ‘sold’}

{\mdseries ‘made to search’}

{\mdseries ‘saved’}

{\mdseries ‘listened attentively’}

\mdseries ‘whitened’\\
\mdseries b. & {\mdseries \emph{ipwiizh-a}}

{\mdseries \emph{kamb-izh-a}}

{\mdseries \emph{katezh-a}}

\mdseries \emph{pitaizh-a} & {\mdseries ‘ask’}

{\mdseries ‘command’}

{\mdseries ‘be difficult’}

\mdseries ‘pass beyond’ & {\mdseries \emph{ipw-iizhe}}

{\mdseries \emph{kamb-izhe}}

{\mdseries \emph{kat-ezhe}}

\mdseries \emph{pita-izhe} & {\mdseries ‘asked’}

{\mdseries ‘commanded’}

{\mdseries ‘was difficult’}

\mdseries ‘passed beyond’\\
\lspbottomrule
\end{tabular}
As seen in \tabref{tab:24}, the verb base on the left and the imbricated verb on the right have the same forms CVCVCV-, VCCGVCV- and VCGVCV-. What distinguishes verb forms in the two columns is only the final vowel \emph{{}-a} versus \emph{{}-e}.

\begin{styleBodyTextIndent}
Verb bases with the passive suffix -\emph{w} or -\emph{iw }imbricate as well. Some verbs with the passive morphology have developed lexicalized or non-compositional meaning, as the -\emph{w} is frozen and has no semantic content. In addition, what appears as a root does not have independent meaning. When attached to such verbs, the -\emph{il} of the remote past suffix splits from the final vowel and is inserted before the verb stem final consonant. Thus, the passive suffix appears between the two parts of the remote past morpheme. However, since there is no segment deletion if the stem is monosyllabic, there is an increase in the number of syllables. This is also attested in many other Bantu languages such as Bemba (Chebanne 1993; Hyman 1995; Kula 2002). Consider the remote past verb forms in \tabref{tab:25}.
\end{styleBodyTextIndent}

%%please move \begin{table} just above \begin{tabular
\begin{table}
\caption{Imbrication in remote past passive}
\label{tab:25}
\end{table}

\begin{tabular}{llll}
\lsptoprule
\multicolumn{2}{l}{\mdseries Root-passive-FV} & \multicolumn{2}{l}{\mdseries Root-remote past-passive}\\
{\mdseries \emph{kas-w-a}}

{\mdseries \emph{kelw-a}}

{\mdseries \emph{kolw-a}}

{\mdseries \emph{leng-w-a}}

{\mdseries \emph{pum-w-a}}

\mdseries \emph{temw-a  } & {\mdseries ‘be tied’}

{\mdseries ‘be late’}

{\mdseries ‘be sick’}

{\mdseries ‘be created’}

{\mdseries ‘be hit’}

\mdseries ‘love’ & {\mdseries \emph{kash-il-w-e}}

{\mdseries \emph{kel-el-w-e}}

{\mdseries \emph{kol-el-w-e}}

{\mdseries \emph{leng-el-w-e}}

{\mdseries \emph{pum-in-w-e}}

\mdseries \emph{tem-en-w-e} & {\mdseries ‘was tied’}

{\mdseries ‘was late’}

{\mdseries ‘was sick’}

{\mdseries ‘was created’}

{\mdseries ‘was hit’}

\mdseries ‘was loved’\\
\lspbottomrule
\end{tabular}
Consonant deletion occurs in longer verb bases displaying the passive morphology. When the remote past suffix is inserted to the immediate left of -\textit{w}, the /l/ of the applicative or other relevant verb stem deletes, inducing either a long vowel [ii] or vowel coalescence of [ai] into [ee], depending on the property of the verb. Imbrication in the examples in \tabref{tab:26}, part a, induces a sequence of identical long vowels [ii]; while in part b it results in vowel coalescence. The same thing is noted with verb stems already suffixed with the applicative morpheme in the passive, as shown in part c of \tabref{tab:26}.

%%please move \begin{table} just above \begin{tabular
\begin{table}
\caption{Imbrication with vowel lengthening in remote past passive}
\label{tab:26}
\end{table}

\begin{tabular}{lllll} & \multicolumn{2}{l}{\mdseries Root-extension/applicative-passive-FV} & \multicolumn{2}{l}{\mdseries Remote past}\\
\lsptoprule
\mdseries \emph{\textup{a.}} & {\mdseries \emph{fwainw-a}}

{\mdseries \emph{ambiw-a}}

{\mdseries \emph{itabiw-a}}

{\mdseries \emph{paanyiiw-a}}

\mdseries \emph{ubiw-a} & {\mdseries ‘have to’}

{\mdseries ‘be told’}

{\mdseries ‘be accepted’}

{\mdseries ‘be given’}

\mdseries ‘be done’ & {\mdseries \emph{fwaiinw-e}}

{\mdseries \emph{ambiiw-e}}

{\mdseries \emph{itabiiw-e}}

{\mdseries \emph{panyiiw-e}}

\mdseries \emph{ub-iiw-e} & {\mdseries ‘had to’}

{\mdseries ‘was/were told’}

{\mdseries ‘was/were accepted’}

{\mdseries ‘was/were given’}

\mdseries ‘was done’\\
\mdseries \emph{\textup{b.}} & \mdseries \emph{kankalw-a} & \mdseries ‘be unable to’ & \mdseries \emph{kankeelw-e} & \mdseries ‘was unable to’\\
\mdseries \emph{\textup{c.}} & {\mdseries \emph{nemb-el-w-a}}

\mdseries \emph{twaj-il-w-a} & {\mdseries ‘be written to’}

\mdseries ‘be taken for’ & {\mdseries \emph{nemb-eel-w-e}}

\mdseries \emph{twaj-iil-w-e} & {\mdseries ‘was/were written to’}

\mdseries ‘was/were taken for’\\
\lspbottomrule
\end{tabular}
Certain verb bases already affected by the process also imbricate for a second time. In other words, the same verb may undergo double imbrication. The applicative extension triggers the first instance of imbrication, while the inflectional remote past suffix induces the second one (\tabref{tab:27}).

%%please move \begin{table} just above \begin{tabular
\begin{table}
\caption{Double imbrication in applied remote past passive}
\label{tab:27}
\end{table}

\begin{tabular}{llll}
\lsptoprule
\multicolumn{2}{l}{\mdseries Root-applicative-passive-FV} & \multicolumn{2}{l}{\mdseries applied passive in remote past}\\
{\mdseries \emph{zhikw-il-w-a}}

\mdseries \emph{sangw-il-w-a} & {\mdseries ‘be unearthed for’}

\mdseries ‘be resurrected’ & {\mdseries \emph{zhikw-iil-w-e}}

\mdseries \emph{sangw-iil-w-e} & {\mdseries ‘was unearthed for’}

\mdseries ‘was resurrected’\\
\lspbottomrule
\end{tabular}
\begin{stylelsSectioni}
5 Conclusion
\end{stylelsSectioni}

This paper has discussed the phonological processes triggered by the suffixation of the remote past tense suffix to verb roots and stems. It shows that the phenomena include vowel height harmony, nasal consonant harmony, complete consonant assimilation, and vowel harmony when preceded by mid vowels in the root. Nasal consonant harmony is a process that changes the lateral consonant of the suffix to a nasal /n/ following a bilabial or alveolar nasal in the root. Consonant assimilation is a long-distance phenomenon that affects the consonant of the suffix when it is attached to verb stems ending in the alveopalatal fricatives \textstyleipa{or the palatal nasal}\textstyleipa{. }Palatalization affects alveolar stops and the lateral which turns into the alveopalatal affricate. Regarding imbrication, the process exhibits different patterns depending on the extension and suffix that precedes the triggering remote past suffix. 

\begin{stylelsSectioni}
Acknowledgements
\end{stylelsSectioni}

I am grateful to an anonymous reviewer, Phil Mukanzu and Kelvin M. Mambwe for their help with the data. 

\begin{stylelsSectioni}
Abbreviations
\end{stylelsSectioni}

\textsc{1sg}  first person singular\textsc{ }

\textsc{caus}  causative 

\textsc{fv}  final vowel

\textsc{rp}  remote past

\textsc{tns}  tense

\end{document}%%remove this line and move all lines below to localbibliography.bib

\begin{verbatim}
@book{Bashi2008,
	address = {Réflexions préliminaires sur la rédaction d’un dictionnaire bilingue Mashi-français.\textbf{ }Lille},
	author = {Bashi Murhi-Orhakube, Constantin.},
	publisher = {Université de Lille III. (Doctoral dissertation.)},
	year = {2008}
}

@book{Bastin1983,
	address = {Tervuren},
	author = {Bastin, Yvonne.},
	publisher = {Musée royal de l’Afrique central},
	title = {\textit{La finale verbale -ide et l’imbrication en Bantou. }(Annales, série in-8, sciences humaines 114)},
	year = {1983}
}

Chebanne, Andy M. 1993. Imbrication of Suffixes in Setswana. (Paper presented at the 24th Annual Conference on African Linguistics, Columbus, OH, 23-25 July 1993.)

@book{Fisher1984,
	address = {Lusaka},
	author = {Fisher, M. K.},
	publisher = {Neczam},
	title = {\textit{Lunda}\textit{{}-}\textit{Ndembu handbook}},
	year = {1984}
}

@article{Greenberg1951,
	author = {Greenberg, Joseph Harold},
	journal = {\textit{Revue Congolaise }},
	number = {8},
	pages = {813-820},
	title = {Vowel and nasal harmony in Bantu languages},
	year = {1951}
}

\begin{styleTextbodyuser}
Guthrie, Malcolm. 1967-71. \textit{Comparative Bantu. An introduction to the comparative linguistics and prehistory of the Bantu languages.} (4 vols.) Farnborough: Gregg International.
\end{styleTextbodyuser}

@book{Horton1949,
	address = {Johannesburg},
	author = {Horton, Alonzo E.},
	publisher = {Witwatersrand University Press},
	title = {\textit{A grammar of Luvale}},
	year = {1949}
}

\begin{styleHTMLPreformatted}
@book{\textstyleauthors{Hyman1999,
	address = {Emergent templates},
	author = {\textstyleauthors{Hyman, Larry M.,  and  Sharon Inkelas.},
	publisher = {The unusual case of Tiene. (Paper presented at the Hopkins Optimality Workshop / Maryland Mayfest, Baltimore.)}},
	year = {1999}
}
\end{styleHTMLPreformatted}

@article{\textstyleauthors{Hyman1995,
	author = {\textstyleauthors{Hyman, Larry M},
	journal = {\textit{Journal of African Languages and Linguistics }},
	number = {16},
	pages = {3-39},
	title = {}Minimality and the prosodic morphology of Cibemba imbrication},
	year = {1995}
}

\begin{styleHTMLPreformatted}
@article{\textstyleauthors{Hyman1998,
	author = {\textstyleauthors{Hyman, Larry M},
	journal = {}\textstyleauthors{\textit{Phonology}}\textstyleauthors{},
	note = {}},
	number = {15},
	pages = {41-75},
	title = {Positional prominence and the ‘prosodic trough’ in Yaka},
	year = {1998}
}
\end{styleHTMLPreformatted}

\begin{styleHTMLPreformatted}
@article{\textstyleauthors{Hyman2003,
	author = {\textstyleauthors{Hyman, Larry M. }},
	journal = {\textit{Journal of African Languages and Linguistics}},
	number = {24},
	pages = {55-90},
	title = {Sound change, misanalysis, and analogy in the Bantu causative},
	volume = {1},
	year = {2003}
}
\end{styleHTMLPreformatted}

Hyman, Larry M. 2003. Suffix ordering in Bantu: A morphocentric approach. In Booij, G. \& van Marle, J. (eds.). \textit{Yearbook of morphology 2002}, 245–281. Dordrecht: Kluwer.

@book{Kawasha2003,
	address = {Eugene},
	author = {Kawasha, Boniface.},
	publisher = {University of Oregon. (Doctoral dissertation.)},
	title = {\textit{Lunda grammar: A morphosyntactic and semantic Analysis}},
	year = {2003}
}

@book{Kula2002,
	address = {Leiden},
	author = {Kula, Nancy.},
	publisher = {Leiden University. (Doctoral dissertation.)},
	title = {\textit{The phonology of verbal derivation in Bemba}},
	year = {2002}
}

@article{Lukusa1993,
	author = {Lukusa, Stephen T. C},
	journal = {\textit{Afrikanistische Arbeitspapiere }},
	number = {35},
	pages = {55–75},
	title = {Imbrication of extensions in Ciluba},
	year = {1993}
}

@book{Maganga1992,
	address = {}Cologne},
	author = {Maganga, Clement,  and  Schadeberg, Thilo C.},
	publisher = {Rüdiger Köppe Verlag},
	title = {\textit{Kinyamwezi: Grammar, texts, vocabulary},
	year = {1992}
}

@article{Meeussen1967,
	author = {Meeussen, Achille E},
	journal = {\textit{Africana Linguistica }},
	number = {3},
	pages = {79-121},
	title = {Bantu grammatical reconstructions},
	year = {1967}
}

@article{\textstyleauthorname{Monich2015,
	author = {\textstyleauthorname{Monich, Irina. }},
	journal = {Comparative morphological analysis of the perfect form in Sesotho and isiZulu.\textstyleHeadingiiiChar{ }\textstylejournaltitle{\textit{Natural Language \& Linguistic Theory}}},
	note = {. DOI: 10.1007/s11049-014-9280-6.},
	number = {33},
	pages = {1271-1291},
	year = {2015}
}

@incollection{\textstyleai{Yukawa1987,
	address = {Tokyo},
	author = {\textstyleai{Yukawa, Yasutoshi},
	booktitle = {}\textstyleaii{\textit{Studies in Zambian languages}}\textstyleai{,},
	editor = {Yukawa, Yasutoshi, Hachipola, Simooya Jerome \& Kagaya, Ryohei},
	note = { }},
	pages = {1-34},
	publisher = {ILCAA},
	title = {A tonal study of Luvale verbs},
	year = {1987}
}

@article{White1947,
	author = {White, Charles M. N},
	journal = {\textit{African Studies}},
	note = {. Johannesburg: Witwatersrand University Press.},
	number = {6},
	pages = {1-20},
	title = {A comparative survey of the verb forms in four Bantu languages of the west central Bantu group},
	volume = {1},
	year = {1947}
}

@book{\textstylest{White1949,
	address = {London, Cape Town and New York},
	author = {\textstylest{White, Charles }M.N.},
	publisher = {Longmans, Green \& Co},
	title = {\textit{A Short Lwena grammar}},
	year = {1949}
}

White, Charles M.N. 196.... \textit{A Chokwe grammar}. Lusaka: White Fathers and Northern Rhodesia and Nyasaland Publications Bureau.

Wright, John L. 1977. Kaonde: A brief description of the language. In Kashoki, Mubanga E. \& Mann, Michael (eds.). \textit{Language in Zambia:} \textit{Grammatical sketches}, 109-150. Lusaka: Institute for African Studies, University of Zambia.



\section*{Abbreviations}
\section*{Acknowledgements}

{\sloppy
\printbibliography[heading=subbibliography,notkeyword=this]
}
\end{verbatim}